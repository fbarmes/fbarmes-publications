\documentclass[aps,10pt,twocolumn]{revtex4}
%%\documentclass{elsart}


%============================================================================
%       packages
%============================================================================
\usepackage[final]{graphics}    % for graphics
\usepackage[final]{graphicx}    % for graphics
\usepackage{subfigure}      % allow subfigure
\usepackage{amssymb}        % font
\usepackage{amstext}        % ams latex package
\usepackage{amsmath}        % font
\usepackage{amsfonts}       % font
\usepackage{amsbsy}     % font
\usepackage{hhline}     % nice lines is tables
\usepackage{xspace}     % add space at the end of macros
\usepackage{color}


%============================================================================
%       new commands
%============================================================================
\newcommand{\remark}[1]{\textbf{\textcolor{red}{#1}}}
\newcommand{\remove}[1]{}
\newcommand{\SmATwo}{smectic A\ensuremath{_2}\xspace}

\newcommand{\linespacing}[1]{\renewcommand{\baselinestretch}{#1}\large\normalsize}
\newcommand{\mrm}[1]{\ensuremath{\mathrm{#1}}}
\newcommand{\vect}[1]{ \mathbf{#1} }
\newcommand{\vecth}[1]{ \mathbf{\hat{#1} } }

\newcommand{\gvect}[1]{\forcebold{#1}}
\newcommand{\gvecth}[1]{\forcebold{\hat{#1}}}

\newcommand{\dotproduct}[2]{\vect{#1} \cdot \vect{#2}   }


\newcommand{\lp}{\left(}
\newcommand{\rp}{\right)}


%============================================================================
%       Contact distance related macros
%============================================================================

\newcommand{\so}{\sigma_0}
\newcommand{\sel}{\sigma_\ell}

\newcommand{\rij}{\vecth{r}_{ij}}
\newcommand{\ui}{\vecth{u}_i}
\newcommand{\uj}{\vecth{u}_j}
\newcommand{\sijr}{\sigma(\ui,\uj,\rij)}
\newcommand{\dotProdP}[2]{ \left( #1 \cdot #2 \right) }
\newcommand{\dotProd}[2]{ #1 \cdot #2 }

\newcommand{\sir}{\sigma(\ui,\vect{r}_{ij})}

\newcommand{\ijr}{\ui,\uj,\rij}
\newcommand{\kS}{k_S}
\newcommand{\kSp}{k_S^{\prime}}

%%\newcommand{\etal}{\emph{et al.}\@\xspace}
\newcommand{\ie}{\emph{i.e.}\@\xspace}
\newcommand{\eg}{\emph{e.g.}\@\xspace}
\newcommand{\etc}{etc.\@\xspace}


%============================================================================
%       Macros - figures
%============================================================================
\pagebreak
\newlength{\picH}   % picture height
\newlength{\picW}   % picture width
\newcommand{\picA}{270} % picture angle

\picW = 10cm
\newcommand{\picB}[1]{\fbox{\includegraphics[width=\picW]{#1}}}
\newcommand{\picLB}[1]{\fbox{\includegraphics[height=\picW, angle=\picA]{#1}}}

\newcommand{\pic}[1]{\includegraphics[width=\picW]{#1}}
\newcommand{\picL}[1]{\includegraphics[height=\picW, angle=\picA]{#1}}


%%==============================================================================================
%%==============================================================================================
%%==============================================================================================
\begin{document}

\graphicspath
{
{../imgs/}
}

\title{Computer simulation of bistable switching in a \\nematic
device containing pear-shaped particles.}

\author{F. Barmes}
\affiliation{Centre Europ\'een de Calcul Atomique et Mol\'eculaire, 46, all\'ee d'Italie, 69007 Lyon, France.}


\author{D.J. Cleaver}

\affiliation{Materials and Engineering Research Institute, Sheffield Hallam
University, Sheffield S1 1WB, United Kingdom.}



\begin{abstract}
We study the microscopic basis of bistable switching of a confined liquid crystal via Monte Carlo simulations of
hard pear-shaped particles. Using both dielectric and dipolar field couplings to this intrinsically flexoelectric
fluid, it is shown that pulsed fields of opposing polarity can be used to switch between the vertical and hybrid
aligned states. Further, it is shown that the field-susceptibility of the surface polarisation, rather than the
bulk flexoelectricity, is the main driver of this switching behaviour.
\end{abstract}

%%\begin{keyword}
%% PACS codes here, in the form: \PACS code \sep code
%%\PACS 61.30.-v \sep 64.70Md \sep 61.30.Cz \sep 68.08.-p
%%\end{keyword}





%%\date{\today}


%%\pacs{\\
%%61.30.-v : liquid crystals\\
%%64.70Md : transitions in liquid crystals\\
%%61.30.Cz : molecular and microscopic models and theories of liquid crystals\\
%%68.08.-p : liquid-solid interfaces}




%%====================================================
%%  Abstract
%%====================================================


%%====================================================
%%  Do the title
%%====================================================
\maketitle

Flexoelectricity is an effect by which an applied bulk field can induce polar director field gradients in a
mesophase such as a nematic liquid crystal (LC)~\cite{deGennes}. Essentially, it provides a mechanism by which to
impose splay and bend distortions of a given polarity on a mesophase which is otherwise centrosymmetric.
Microscopically, this phase polarisation arises due to a combination of the action of the field on molecular
dipoles and some steric or electrostatic asymmetry in the molecule-molecule
interactions~\cite{Meyer69,ProstMarcerou77,MarcerouProst80}. Interest in flexoelectricity, particularly its
microscopic origins, has been reignited recently due to the development of liquid crystal displays (LCD's) which
exploit bistable surface anchoring of the nematic phase~\cite{ZBD,PSBD}. Several switching mechanisms have been
proposed for these devices such as the use of competing dielectric and flexoelectric couplings to the applied
field~\cite{BarberiGiocondo92}, the inclusion of  electrochiral ions~\cite{BarberiDurand91} or self-assembled
hydrogen-bonded fibers~\cite{MizoshitaSuzuki05} or, finally, the use of field-induced surface
defects~\cite{BarberiGiocondo97} whose density can even be used to achieve a grey-scale device.

Following the lead of Barberi and co-workers~\cite{BarberiGiocondo92,BarberiGiocondo97}, Davidson and Mottram
recently showed that a flexoelectric nematic subject to directional field pulses can execute two-way switching in
an LC cell with one monostable homeotropic substrate and one bistable planar/homeotropic
substrate~\cite{DavidsonMottram02}. Specifically, by considering both dielectric and flexoelectric couplings to
homogeneous applied fields, within an Ericksen-Leslie theory approximation, they identified a narrow parameter
window for two-way switching between vertical (V) and hybrid aligned nematic (HAN) states.

However, the model used in this study was somewhat simplified when one considers the full range of polar
contributions present in real systems: the flexoelectric polarisation can itself be resolved into
director-gradient and order parameter gradient terms (the latter being termed the `order electric' polarisation);
ion migration effects are known to be significant for the LC mixtures and substrate materials used in prototype
devices~\cite{Cliff}; and molecular polarisation (or ferroelectric order) in the substrate region is known to be
present in many confined LC systems~\cite{Jerome:1991.391,Blinov:2001.183}. The simplification associated with
Davidson and Mottram's approach is echoed in the recent debate on how best to partition between these different
polar terms when using HAN cells to measure flexoelectric coefficients. For HAN cells subject to applied fields,
measured director profile distortions are poorly described by director flexoelectricity
alone~\cite{Mazzulla:2001:021708}; strong cases have been made to compensate for this with either surface
polarisation~\cite{Mazzulla:2003:023702} or ion-distribution terms~\cite{Barbero:2003:023701}, whereas
Lattice-Boltzmann simulations indicate significant near-surface order electricity~\cite{Tim}. In practice, then,
there is evidence to suggest that a combination of all three of these effects commonly operates in concert with
director flexoelectricity.

Here, we consider the microscopic basis of directional-pulsed-field switching in such systems using Monte Carlo
simulations of hard pear-shaped particles confined in slab geometry. This particle shape is used since, following
the original arguments of Meyer~\cite{Meyer69}, its mesophases exhibit splay flexoelectricity. The interparticle
interactions are implemented using the parametric hard gaussian overlap (PHGO) approach introduced
in~\cite{BarmesRicci03}; we use the parameterisation with elongation $k=5$ shown, in Ref.~\cite{BarmesRicci03}, to
possess both nematic and \SmATwo phases, the latter being a bilayer smectic phase with layers formed perpendicular
to the director. Being based on a purely steric single-site interaction, this is a particularly efficient model
for use here, though similar behaviour can be expected for any of the generic flexoelectric particle models
introduced in recent years~\cite{Ricci,Pelcovits,Stelzer}. Investigation of two-way HAN-V switching also requires
a bistable particle-substrate interaction potential and a particle-field coupling which induces competition
between the dielectric and flexoelectric terms. The latter is achieved here by taking each pear to interact with
an applied field through both dielectric and dipolar terms. Thus
\begin{equation}
    U_i^\mrm{field} =
    -\frac{1}{2}\Delta\chi\lp \dotProd{\vect{E}}{\ui} \rp^2
    - \mu \dotProdP{\vect{E}}{\ui}
    \label{eqn:field}
\end{equation}
where $\Delta\chi$ is the anisotropy in the dielectric susceptibility, $\mu$ is the dipole moment, $\ui$ the
particle orientation and $\vect{E} = E\vecth{z}$ the applied electric field. Note that electrostatic
particle-particle interactions are neglected in this study on the basis that, as in Ref.~\cite{Ricci}, they play a
minor role. Similarly, in all simulated systems, we follow Davidson and Mottram in adopting the common
approximation that \textbf{E} is constant rather than imposing a fixed voltage drop and calculating $\bf{E}(z)$
based on instantaneous dielectric profiles and an assumption of constant electric displacement.

Following the approach taken in our previous studies of liquid crystal
anchoring~\cite{BarmesCleaver04a,BarmesCleaver04b}, the pear-shaped particles do not interact directly with the
constraining substrates. Instead, particle-substrate interactions are mediated through objects embedded within the
mesogenic particles (see Fig.~\ref{fig:PSUConfig}). The benefit of this approach is that changing the shapes of
such embedded objects provides a microscopic mechanism by which to control the orientation (i.e. homeotropic,
tilted or planar) and nature (monostable or bistable; strong or weak) of the resultant anchoring. Here we use
centrosymmetric (i.e. ellipsoid-like) HGO particles for the embedded objects, so as to take advantage of the
strong homeotropic-planar bistability region previously determined for the HGO-planar substrate surface
potential~\cite{BarmesCleaver04b}. To prevent the HGO particles from protruding outside their host particles, the
former are shifted along the particle axes (towards the bulky end of each pear) so as to make the endpoints of the
embedded and host particles coincide (see Fig.~\ref{fig:PSUConfig}). This results in a polar particle-substrate
interaction, since the points of the pears are left free to penetrate the substrate plane; the extent of this
allowed penetration and, hence, the relative stability of the homeotropic anchoring state are controlled by $k_S$,
the length-to-breadth ratio of the embedded object. The particle-substrate interactions are, then, given by
\begin{equation}
\label{eqn:Pear-Surface}
    \mathcal{V}_i^{\rm{Pear-Surface}} = \left\{ %}
    \begin{array}{ccc}
        0   &\mathrm{if}    &|z_{\mrm{obj},i} - z_0| \geq \sigma^{\rm{HGO-Surface}}_w    \\
        \infty  &\mathrm{if}    &|z_{\mrm{obj},i} - z_0| < \sigma^{\rm{HGO-Surface}}_w
    \end{array}
    \right.
\end{equation}
where $z_0$ defines the substrate plane and $z_{\mrm{obj},i}$ is the $z$-coordinate of the object embedded in pear
$i$:
\begin{equation}
    z_{\mrm{obj},i} = z_i - \frac{\sigma_0}{2}\lp k - k_S\rp\cos\theta_i .
\end{equation}
Here $z_i$ and $\theta_i$ are, respectively, the $z$-coordinate and zenithal angle of pear $i$ and $\sigma_0$ sets
the particle width. $\sigma^{\rm{HGO-Surface}}_w=\so\left[ (1-\chi_S\sin^2\theta)/(1-\chi_S) \right]^{1/2}$ is the
HGO-surface contact distance~\cite{BarmesCleaver04b} and $\chi_S=(k^2_S-1)/(k^2_S+1)$ is the shape anisotropy of
the embedded HGO particles. The ratio $k_S/k$ is restricted to the range $[0.4:0.8]$.

\picW = 8cm
\begin{figure}
    \centering
    \pic{fig_01.ps}
    \caption{Schematic representation of the interaction between a pear shaped particle of elongation $k$
    and a planar substrate (shown as a horizontal bar). The interaction is mediated by an ellipsoidal HGO particle of elongation $k_S$ embedded
    in the pear and shifted along the molecular axis to make the ends of the two objects
    coincident.}
    \label{fig:PSUConfig}
\end{figure}

Before HAN-V switching was investigated, the surface anchoring properties of the potential
Eqn.(\ref{eqn:Pear-Surface}) were characterised by means of preliminary simulations using systems of $N=1000$
particles confined symmetrically between identical substrates. A slab geometry of height $L_z=4k\sigma_0$ along
$\vecth{z}$ was used, periodic boundary conditions being applied in the $\vecth{x}$ and $\vecth{y}$ directions.
The surfaces located at $z=L_z$ and $z=-L_z$ will be subsequently referred to as the top and
bottom surfaces respectively.\\
The simulations were performed at constant nematic number density $\rho^{*}=0.15$ and in series of increasing and
decreasing $k_S$. From these runs, a homeotropic to planar anchoring transition was identified from the behaviour
of $\overline{Q^{Su}_{zz}}(\rho^{*},k_S/k)$ (see~\cite{BarmesCleaver04a} for a definition), the
density-profile-weighted average of the order tensor element $Q_{zz}(z)$ in the interfacial regions. These data,
shown in Fig.~\ref{fig:QzzWa_SH}, indicate a discontinuous transition between the homeotropic and planar states,
corresponding, respectively, to positive and negative $\overline{Q^{Su}_{zz}}(\rho^{*},k_S/k)$, for $0.64\leq
k_S/k \leq 0.74$. The hysteresis identified from series of simulations performed with, respectively, increasing
and decreasing $k_S/k$ suggests a region of bistability which is maximal for $k_S/k=0.7$. Though not apparent from
the data shown in Fig.~\ref{fig:QzzWa_SH}, the homeotropic state observed here exhibits considerable surface
polarisation since the interaction potential (\ref{eqn:Pear-Surface}) only allows the points of the pears to
penetrate the substrate plane.
%%
%%
\picW = 8cm
\begin{figure}
    \centering
    \picL{fig_02.ps}
    \caption{Behaviour of $\overline{Q^{Su}_{zz}}$ as a function of the reduced embedded object
    elongation $k_S/k$ for pear shaped PHGO particles in symmetric and hybrid confining
    geometries. In the case of hybrid anchoring, only the particles near the bistable substrate are
    considered. $\blacktriangle$ indicate simulations performed in series with increasing $k_S/k$ while
    $\blacktriangledown$ indicate simulations performed in series with decreasing $k_S/k$.}
    \label{fig:QzzWa_SH}
\end{figure}
%%
%%

When attempting to transfer this behaviour to a hybrid anchored system, it was found that imposing strong,
monostable homeotropic anchoring (i.e. small $k_S$) at the top surface resulted in the loss of bistability at the
bottom surface. This situation is known to prevail in very thin films when one anchoring coefficient is
significantly larger than the other ~\cite{Cleaver:2001.1,Sarlah} since the dominant surface extrapolation length
is then able to exceed the film thickness. To recover bistability at the bottom surface, therefore, the slab
thickness was increased and the top surface anchoring strength reduced (by increasing the $k_S$ value used). As
shown in Fig.~\ref{fig:QzzWa_SH}, an acceptable degree of bistability was obtained for a double-thickness slab
(i.e. $L_z=8k\sigma_0$, $N=2000$) with $k_{S}/k=0.6$ and $\sim 0.7$ at the top and bottom surfaces, respectively.
The use here of different values for $k_S/k$ at the top and bottom surfaces simply results in the particles
interacting with each substrate in a different way, just as if each substrate had a different surface treatment.

\picW = 6cm
\begin{figure}
    \centering
    \subfigure[]{\picL{fig_03a.ps}}
    \subfigure[]{\picL{fig_03b.ps}}

    \subfigure[]{\picL{fig_03c.ps}}
    \subfigure[]{\picL{fig_03d.ps}}
    \caption{$Q_{zz}(z)$ profiles corresponding to each stage of the two way switching between
    the HAN to V states using $\mu=2.5$ and $\vect{E}=\pm0.2\vecth{z}$.
    Subfigures (a) and (b) show the results obtained using
    $\Delta\chi=-1.0$ while subfigures (c) and (d) show those for $\Delta\chi=+1.0$.
    The dotted lines show the starting configurations and the dashed lines
    typical target HAN and V profiles.}
    \label{fig:QzzSwitch}
\end{figure}
%%
%%
\picW = 12cm
\begin{figure}
    \centering
    %
    \pic{fig_04.ps}
    %
    \caption{Configuration snapshots illustrating reversible switching of a system with
    $\vect{E}=\pm0.2\vecth{z}$, $\Delta\chi = -1.0$
    and $\mu = 2.5$.}
    \label{fig:fullSwitchSnaps}
\end{figure}

The two-way switching identified by Davidson and Mottram requires an appropriate balance of the dielectric and
dipolar field-coupling terms. The dipolar term needs to be sufficiently strong to latch the lower region of the
cell into a configuration that will equilibrate into the vertical state on removal of the field. However, the
$E^2$ coupling of the dielectric term dictates that \emph{it} dominates (re-establishing the HAN state) at high
field values. In initial assessment of HAN to V switching, runs performed with $k_BT=1.0$, $\Delta\chi=-1.0$ and
$\mu=1.0$, and a range of $E$ values, fields more negative than $E=-0.4$ led to domination by the dielectric term.
However, a field $E=-0.2$ was found to induce significant but unsaturated deviation of the director profile away
from its initial state. None of these systems relaxed to the V state on removal of the field, so further runs were
performed with $E=-0.2$ and gradually increased dipolar coupling strengths $\mu$ in the range [1.0-3.5]. For each
of these cases, the slab was subjected to consecutive extended runs of $2.5\times 10^5$ sweeps (where one sweep is
one attempted move per particle) with the field applied and, then, removed. The $Q_{zz}(z)$ profiles obtained
indicate that the $\mu=2.0$ system exhibited substantial director deviation throughout the bulk region of the cell
but no switching. For $\mu=2.5$, by contrast (Fig.~\ref{fig:QzzSwitch}(a)), the director deviation was
concentrated in the lower half of the slab, a configuration which, on removal of the field, successfully switched
into the vertical state. Configuration snapshots showing the HAN to V switching behaviour of this $\mu=2.5$ system
are given in Figs.~\ref{fig:fullSwitchSnaps}(a)-(c).

Taking the final configuration from this sequence and then performing a similar sequence with an equal but
opposite applied field led to the $Q_{zz}(z)$ profiles shown in Fig.~\ref{fig:QzzSwitch}(b). Here, upon
application of the field, most of the vertically-aligned slab remained undistorted but a narrow region near the
bottom surface developed features compatible with planar anchoring. Upon removal of the field, this small region
proved sufficient to seed its planar orientation into the bulk part of the cell, leading to recovery of the HAN
state. Configuration snapshots illustrating this reverse (V to HAN) switching are given in
Figs.~\ref{fig:fullSwitchSnaps}(c)-(e). Note that this reverse switching process became less accessible for higher
dipole couplings due to the consequent reduction in the thickness of the planar surface region. In practice, no
reverse switching was observed for $\mu > 3.5$.

While our simulations have successfully reproduced the reversible pulse-field-induced switching predicted by
Davidson and Mottram~\cite{DavidsonMottram02}, the $Q_{zz}(z)$ profiles shown in Fig.~\ref{fig:QzzSwitch} are
significantly different from those obtained using the director-flexoelectricity-based approach. Specifically, the
``E on" profile of Fig.~\ref{fig:QzzSwitch}(b) is homeotropically aligned for positive $z$ and has a bend from
homeotropic-to-planar for negative $z$. In comparison, the corresponding profiles (marked with positive $E$
values) in Davidson and Mottram's Fig. 5 are largely planar for low $z$ but bent in the high $z$ region. To assess
this further, we have repeated our simulation procedure for the alternative parameterization $\Delta\chi=1.0$,
$\mu=2.5$ and $E=\pm 0.2$, \ie the same numerical values but with the opposite sign for the dielectric anisotropy;
see Figs.~\ref{fig:QzzSwitch}(c-d). Here, we have again observed reversible switching between the HAN and V
states. However, the $Q_{zz}(z)$ profiles obtained from these positive $\Delta\chi$ switching runs are
qualitatively identical to those given in Figs.~\ref{fig:QzzSwitch}(a-b), whereas Davidson and Mottram's
calculations predict a marked dependence on the sign of the dielectric coupling term. These qualitative
differences raise the possibility that competition between bulk dielectric and flexoelectric couplings is
\emph{not} the main driver for the reversible switching observed in our simulations.

In seeking to examine this suggestion, we note that the bulk flexoelectric coupling considered by Davidson and
Mottram is, in their constant $E$ approximation of Ericksen-Leslie theory, equivalent to a \emph{centrosymmetric}
effective surface term proportional to $\cos(2 \theta)$\cite{DavidsonMottram02}. Closer examination of our
simulation results, conversely, indicates that the field dependence of the \emph{surface polarisation} plays the
dominant role. Thus, for $\vect{E}=-0.2\vecth{z}$ (Figs.~\ref{fig:QzzSwitch}(a) and \ref{fig:fullSwitchSnaps}(b))
the top surface polarisation (and, consequently, anchoring strength) is reduced by the applied field while that at
the bottom surface shows an even stronger field-susceptibility: the negative applied field leads to a highly polar
homeotropic surface layer, with the pear points embedded in the substrate. Equivalently, the positive applied
field destabilises this polar homeotropic state at the bottom substrate, promoting, instead, planar surface
alignment (Figs.~\ref{fig:QzzSwitch}(b) and \ref{fig:fullSwitchSnaps}(d)). As noted above, the effective surface
term associated with bulk-region flexoelectricity has no polar component and so cannot have driven all of this
behaviour. We conclude, therefore, that the switching observed in this system was induced by the combined action
of the applied field and the particle-substrate interaction to produce a \emph{polar} surface-region anchoring
field.

Clear temperature susceptibility of the surface polarisation has been observed experimentally for 5CB at both
planar and homeotropic substrates~\cite{Blinov:2001.183}. While no field-dependence was noted for the surface
polarisations inferred from isothermal measurements on HAN cells~\cite{Mazzulla:2001:021708}, differences between
the monostable substrates used in these cells and the weakly-anchored bistable substrate employed in our
simulations is not unexpected. The dominance of surface polarisation over bulk flexoelectric effects in our
simulated systems is likely related to the thinness of the film employed; the influence of flexoelectric
distortions is effectively integrated across a LC film and, so, is inevitably reduced for the film thicknesses
accessible to molecular simulation (the slab thickness employed here corresponds to approximately 50nm).
Notwithstanding this proviso, our results demonstrate that surface polarisation and, in particular, its field
susceptibility, can play a significant role in HAN-V switching through its influence on surface anchoring strength
and stability. Consequently, such polarisation changes appear able to affect the orientational structure of the
confined films, as well as providing a mechanism for generating voltage offsets.

%%====================================================
%%  Bibliography
%%====================================================
\begin{thebibliography}{10}
\expandafter\ifx\csname url\endcsname\relax
  \def\url#1{\texttt{#1}}\fi
\expandafter\ifx\csname urlprefix\endcsname\relax\def\urlprefix{URL }\fi

\bibitem{deGennes}
{P.G. de Gennes}, The Physics of Liquid Crystals, 2nd ed., Clarendon press,
  1993.

\bibitem{Meyer69}
{R.B. Meyer}, Piezoelectric effects in liquid crystals, Phys. Rev. Letts. 22
  (1969) 918.

\bibitem{ProstMarcerou77}
{J. Prost and J.P. Marcerou}, On the microscopic interpretation of
  flexoelectricity, J. de Phys. 38 (1977) 315.

\bibitem{MarcerouProst80}
{J.P. Mearcerou and J. Prost}, The different aspects of flexoelectricity in
  nematics, Mol. Cryst. Liq. Cryst. 58 (1980) 259.

\bibitem{ZBD}
{G.P. Bryan-Brown, C.V. Brown, I.C. Sage and V.C. Hui}, {Voltage-dependent
  anchoring of a nematic liquid crystal on a grating surface}, Nature 392
  (1998) 365.

\bibitem{PSBD}
S.~Kitson, A.~Geisow, Controllable alignment of nematic liquid crystals around
  microscopic posts: Stabilization of multiple states, App. Phys. Letts.
  80~(19) (2002) 3635--3637.

\bibitem{BarberiGiocondo92}
R.~Barberi, M.~Giocondo, D.~Durand, Flexoelectrically controlled surface
  bistable switching in nematic liquid crystals, App. Phys. Letts. 60 (1992)
  1085.

\bibitem{BarberiDurand91}
R.~Barberi, D.~Durand, Electrochirally controlled bistable surface switching in
  nematic liquid crystals, App. Phys. Letts. 58 (1991) 2907.

\bibitem{MizoshitaSuzuki05}
N.~Mizoshita, Y.~Suzuki, K.~Hanabusa, T.~Kato, Bistable nematic liquid crystals
  with self-assembled fibers, Adv. Mat. 17 (2005) 692.

\bibitem{BarberiGiocondo97}
{R. Barberi, M. Giocondo, J. Li, R. Bartolino, I. Dozov and G. Durand}, {Fast
  bistable nematic display with grey scale}, App. Phys. Letts. 71 (1997) 3495.

\bibitem{DavidsonMottram02}
{A.J Davidson and N.J. Mottram}, Flexoelectric switching in a bistable nematic
  device, Phys. Rev. E 65 (2002) 051710.

\bibitem{Cliff}
J.~C. Jones, S.~Beldon, P.~Brett, M.~Francis, M.~Goulding, Low voltage zenithal
  bistable devices with wide operating windows, SID 03 (2003) 26.3.

\bibitem{Jerome:1991.391}
B.~Jerome, Surface effects and anchoring in liquid-crystals, Rep. Prog. Phys.
  54~(3) (1991) 391.

\bibitem{Blinov:2001.183}
L.~M. Blinov, M.~I. Barnik, H.~Ohaka, N.~M. Shtykov, K.~Yoshino, {Surface and
  flexoelectric polarization in a nematic liquid crystal 5CB}, Eur. Phys. J. E
  4 (2001) 183.

\bibitem{Mazzulla:2001:021708}
A.~Mazzulla, F.~Ciuchi, J.~R. Sambles, Optical determination of flexoelectric
  coefficients and surface polarization in a hybrid aligned nematic cell, Phys.
  Rev. E 64~(2) (2001) 021708.

\bibitem{Mazzulla:2003:023702}
A.~Mazzulla, F.~Ciuchi, J.~R. Sambles, Reply to "comment on 'optical
  determination of flexoelectric coefficients and surface polarization in a
  hybrid aligned nematic cell' ", Phys. Rev. E 68~(2) (2003) 023702.

\bibitem{Barbero:2003:023701}
G.~Barbero, L.~R. Evangelista, Comment on "optical determination of
  flexoelectric coefficients and surface polarization in a hybrid aligned
  nematic cell", Phys. Rev. E 68~(2) (2003) 023701.

\bibitem{Tim}
T.~J. Spencer, Lattice boltzmann method for q-tensor nemato-dynamics in liquid
  crystal display devices, Ph.D. thesis, Sheffield Hallam University (April
  2005).

\bibitem{BarmesRicci03}
{F. Barmes, M. Ricci, C. Zannoni and D.J. Cleaver }, Computer simulations of
  pear shaped particles, Phys. Rev. E 68 (2003) 021708.

\bibitem{Ricci}
R.~Berardi, M.~Ricci, C.~Zannoni, Ferroelectric and structured phases from
  polar tapered mesogens, Ferroelectrics 309 (2004) 3.

\bibitem{Pelcovits}
J.~L. Billeter, R.~A. Pelcovits, Molecular shape and flexoelectricity, Liq.
  Cryst. 27~(9) (2000) 1151--1160.

\bibitem{Stelzer}
J.~Stelzer, R.~Berardi, C.~Zannoni, Flexoelectric coefficients for model pear
  shaped molecules from monte carlo simulations, Mol. Cryst. Liq. Cryst. 352
  (2000) 621--628.

\bibitem{BarmesCleaver04a}
{F. Barmes and D.J. Cleaver }, {Computer simulations of a liquid crystal
  anchoring transition}, Phys. Rev. E 69 (2004) 061705.

\bibitem{BarmesCleaver04b}
{F. Barmes and D.J. Cleaver }, {Using particle shape to induce tilted and
  bistable liquid crystal anchoring}, Phys. Rev. E 71 (2005) 021705.

\bibitem{Cleaver:2001.1}
D.~J. Cleaver, P.~I.~C. Teixeira, Discontinuous structural transition in a thin
  hybrid liquid crystal film, Chem. Phys. Lett. 338 (2001) 1.

\bibitem{Sarlah}
A.~\v{S}arlah, S.~\v{Z}umer, Equilibrium structures and pretransitional
  fluctuations in a very thin hybrid nematic film, Phys.\ Rev.\ E 60 (1999)
  1821.

\end{thebibliography}

\end{document}
%%==============================================================================================
%%==============================================================================================
%%==============================================================================================
