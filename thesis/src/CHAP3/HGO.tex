
\section{The Hard Gaussian Overlap Model.}

The HGO model is a steric model in which the contact distance is the shape parameter determined 
by Berne and Pechukas~\cite{BernePechukas72}. The HGO model can be seen as an
equivalent of the Gay-Berne model stripped of its attractive interactions.\\
The hard Gaussian overlap potential $\mathcal{V}^\mrm{HGO}$ between two particles $i$ and $j$
with respective orientations $\ui$ and $uj$ and intermolecular vector $\vect{r}_{ij} = r_{ij}\rij$ is
defined as~:
%
\begin{equation}
	\mathcal{V}^\mrm{HGO} = \left\{  %}
	\begin{array}{ccc}
	0	&\mrm{if}	&r_{ij} \geq \sijr	\\
	\infty	&\mrm{if}	&r_{ij}	< \sijr
	\end{array}
	\right.
\end{equation}

where $\sijr$ is the contact distance~:
\begin{equation}
	\sijr = \so\left\{
	1 - \frac{1}{2}\chi\left[ 
	\frac{ \lp\dotProd{\rij}{\ui} + \dotProd{\rij}{\uj}\rp^2 }{1 + \chi(\dotProd{\ui}{\uj})}
      + \frac{ \lp\dotProd{\rij}{\ui} - \dotProd{\rij}{\uj}\rp^2 }{1 - \chi(\dotProd{\ui}{\uj})}
	\right] \right\}^{-\frac{1}{2}}.
\end{equation}

Here $\so$ is the unit of distance and $\chi$ is the shape anisotropy parameter defined using $k$ the
length to breadth ratio as~:
\begin{equation}
	\chi = \frac{k^2-1}{k^2+1}
\end{equation}

Although this model was originally derived using geometrical considerations, the hard
Gaussian overlap molecule can not be represented by a solid shape~\cite{Rigby89}, rather it 
is a mathematical abstraction of an interaction surface between two non-spherical objects. 
The shape of an HGO molecule can,
however, be taken to be very close to that of an ellipsoid of revolution. For example one can
assume  the contact distance of the interactions between two ellipsoids and two HGOs in the
arrangements shown in Figure~\ref{fig:HGO_CD}. The two models agree for end to end and 
side by side configurations. However, in the case of a T-geometry, 
the HGO contact distance is $\sigma=\frac{\so}{(1-\chi)^{\frac{1}{2}}}$ rather than
$\frac{\sigma}{2} = \so(1+k)$ for the hard ellipsoid of revolution (HER) model. 
Due to this similarity, the volume of an HGO molecule is often taken to be that of 
the equivalent ellipsoid~\cite{Rigby89,DeMiguelDelRio01}, that is~:
\begin{equation}
	V_\mrm{HGO} = \frac{\pi}{6}k\so^3.
\end{equation}

\picW = 4cm
\begin{figure}
	\centering
	\subfigure[$\sigma_\mrm{HGO}=\sigma_\mrm{HER}=\so$]{\fbox{\pic{HGOConfig.ss.ps}}}
	\subfigure[$\sigma_\mrm{HGO}=\sigma_\mrm{HER} = \so k$]{\fbox{\pic{HGOConfig.ee.ps}}}
	\subfigure
	[\mbox{$\sigma_\mrm{HGO} = \frac{\so}{(1-\chi)^{\frac{1}{2}}}$} $\sigma_\mrm{HER}=
	\frac{\so}{2}(1+k)$]
	{\fbox{\pic{HGOConfig.T.ps}}}
	\caption{Comparison between the contact distances for the HGO and HER models.}
	\label{fig:HGO_CD}
\end{figure}

A more comprehensive comparison between the HGO and HER models has been performed by
Bhethanabotla and Steele~\cite{BhethanabotlaSteele87} who computed the virial coefficients $B_2$
to $B_5$ for both models with $k\in[1.5:3.0]$. They found that the differences between
equivalent coefficients of the two models are insignificant for this range of elongation. 
Since the HGO model is computationally cheaper than the HER would, and, due to the similarities
just outlined, a similar phase diagram is to be expected for both models. This explains for
the HGO model being increasingly used.\\

An extensive amount of theoretical work has been performed on the HGO
fluid through which the model's virial  coefficients~\cite{Rigby70,BhethanabotlaSteele87,Rigby89} 
and equation of  state~\cite{BoublikPena90,MaesoSolana93} have been obtained. 
The first molecular simulation of the HGO
fluid were performed by Padilla and Velasco~\cite{PadillaVelasco97} on systems of $N=256$
and $512$ particles with elongation $k=3$ and $5$ using the Monte Carlo method in the
isothermal-isobaric ensemble. Here the authors found both isotropic and nematic phases, the
transition densities and pressures $(\rho^*, P^*)$ being approximately  $(0.295, 4.5)$ for $k=3$
and $(0.116,0.88)$ for $k=5$. This work was later refined by de Miguel and Mart\'{\i}n
del R\'{\i}o~\cite{DeMiguelDelRio01} who accurately located the isotropic-nematic transition
regions for systems of $N=500$ molecules with $k \in [3:10]$. Comparison of these results with
the transition properties of the HER showed quantitative differences which have been explained
to be a consequence of the larger excluded volume of a pair of HGO.\\ 
The latest study on the HGO fluid was performed by de Miguel and Mart\'{\i}n 
del R\'{\i}o~\cite{DeMiguelDelRio03}
where the equation  of state of the model was computed and compared to several theoretical 
approaches~; the best  agreement being found with the Parsons-Lee density functional theory.















