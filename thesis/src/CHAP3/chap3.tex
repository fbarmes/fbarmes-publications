
%===============================================================================================
%			NOTES
%===============================================================================================
%
%===============================================================================================


\chapter{Computer simulations of Hard Gaussian Overlaps}
\label{chap:three}


\introduction

In the previous Chapter it was shown that the Gay-Berne potential is one of the most versatile 
molecular models for liquid crystal simulations. Depending on the chosen shape and energy 
parameterisation, it can be used to model many liquid crystalline phases. 
However, when the nematic phase only is of interest, the use of
hard particle models is a preferable option, being computationally easier and faster to
implement than the soft models. Also hard particles have proven to be a very good test-bed
for perturbation theories.\\
In this Chapter the computer simulation of such a model, the hard Gaussian overlap (HGO) 
model is discussed.  First a literature review of the computer simulations and theoretical 
work performed on  this model is provided. The techniques used for the computation of the most
relevant observables in computer simulations are then presented. Finally, some preliminary
results from bulk simulations of prolate and oblate HGO particles 
are presented.  A comparison with existing results for calamitic molecules is also included.

%=================================================================

\section{The Hard Gaussian Overlap Model.}

The HGO model is a steric model in which the contact distance is the shape parameter determined 
by Berne and Pechukas~\cite{BernePechukas72}. The HGO model can be seen as an
equivalent of the Gay-Berne model stripped of its attractive interactions.\\
The hard Gaussian overlap potential $\mathcal{V}^\mrm{HGO}$ between two particles $i$ and $j$
with respective orientations $\ui$ and $uj$ and intermolecular vector $\vect{r}_{ij} = r_{ij}\rij$ is
defined as~:
%
\begin{equation}
	\mathcal{V}^\mrm{HGO} = \left\{  %}
	\begin{array}{ccc}
	0	&\mrm{if}	&r_{ij} \geq \sijr	\\
	\infty	&\mrm{if}	&r_{ij}	< \sijr
	\end{array}
	\right.
\end{equation}

where $\sijr$ is the contact distance~:
\begin{equation}
	\sijr = \so\left\{
	1 - \frac{1}{2}\chi\left[ 
	\frac{ \lp\dotProd{\rij}{\ui} + \dotProd{\rij}{\uj}\rp^2 }{1 + \chi(\dotProd{\ui}{\uj})}
      + \frac{ \lp\dotProd{\rij}{\ui} - \dotProd{\rij}{\uj}\rp^2 }{1 - \chi(\dotProd{\ui}{\uj})}
	\right] \right\}^{-\frac{1}{2}}.
\end{equation}

Here $\so$ is the unit of distance and $\chi$ is the shape anisotropy parameter defined using $k$ the
length to breadth ratio as~:
\begin{equation}
	\chi = \frac{k^2-1}{k^2+1}
\end{equation}

Although this model was originally derived using geometrical considerations, the hard
Gaussian overlap molecule can not be represented by a solid shape~\cite{Rigby89}, rather it 
is a mathematical abstraction of an interaction surface between two non-spherical objects. 
The shape of an HGO molecule can,
however, be taken to be very close to that of an ellipsoid of revolution. For example one can
assume  the contact distance of the interactions between two ellipsoids and two HGOs in the
arrangements shown in Figure~\ref{fig:HGO_CD}. The two models agree for end to end and 
side by side configurations. However, in the case of a T-geometry, 
the HGO contact distance is $\sigma=\frac{\so}{(1-\chi)^{\frac{1}{2}}}$ rather than
$\frac{\sigma}{2} = \so(1+k)$ for the hard ellipsoid of revolution (HER) model. 
Due to this similarity, the volume of an HGO molecule is often taken to be that of 
the equivalent ellipsoid~\cite{Rigby89,DeMiguelDelRio01}, that is~:
\begin{equation}
	V_\mrm{HGO} = \frac{\pi}{6}k\so^3.
\end{equation}

\picW = 4cm
\begin{figure}
	\centering
	\subfigure[$\sigma_\mrm{HGO}=\sigma_\mrm{HER}=\so$]{\fbox{\pic{HGOConfig.ss.ps}}}
	\subfigure[$\sigma_\mrm{HGO}=\sigma_\mrm{HER} = \so k$]{\fbox{\pic{HGOConfig.ee.ps}}}
	\subfigure
	[\mbox{$\sigma_\mrm{HGO} = \frac{\so}{(1-\chi)^{\frac{1}{2}}}$} $\sigma_\mrm{HER}=
	\frac{\so}{2}(1+k)$]
	{\fbox{\pic{HGOConfig.T.ps}}}
	\caption{Comparison between the contact distances for the HGO and HER models.}
	\label{fig:HGO_CD}
\end{figure}

A more comprehensive comparison between the HGO and HER models has been performed by
Bhethanabotla and Steele~\cite{BhethanabotlaSteele87} who computed the virial coefficients $B_2$
to $B_5$ for both models with $k\in[1.5:3.0]$. They found that the differences between
equivalent coefficients of the two models are insignificant for this range of elongation. 
Since the HGO model is computationally cheaper than the HER would, and, due to the similarities
just outlined, a similar phase diagram is to be expected for both models. This explains for
the HGO model being increasingly used.\\

An extensive amount of theoretical work has been performed on the HGO
fluid through which the model's virial  coefficients~\cite{Rigby70,BhethanabotlaSteele87,Rigby89} 
and equation of  state~\cite{BoublikPena90,MaesoSolana93} have been obtained. 
The first molecular simulation of the HGO
fluid were performed by Padilla and Velasco~\cite{PadillaVelasco97} on systems of $N=256$
and $512$ particles with elongation $k=3$ and $5$ using the Monte Carlo method in the
isothermal-isobaric ensemble. Here the authors found both isotropic and nematic phases, the
transition densities and pressures $(\rho^*, P^*)$ being approximately  $(0.295, 4.5)$ for $k=3$
and $(0.116,0.88)$ for $k=5$. This work was later refined by de Miguel and Mart\'{\i}n
del R\'{\i}o~\cite{DeMiguelDelRio01} who accurately located the isotropic-nematic transition
regions for systems of $N=500$ molecules with $k \in [3:10]$. Comparison of these results with
the transition properties of the HER showed quantitative differences which have been explained
to be a consequence of the larger excluded volume of a pair of HGO.\\ 
The latest study on the HGO fluid was performed by de Miguel and Mart\'{\i}n 
del R\'{\i}o~\cite{DeMiguelDelRio03}
where the equation  of state of the model was computed and compared to several theoretical 
approaches~; the best  agreement being found with the Parsons-Lee density functional theory.


















\section{Observables computation.}


In this Section, the computational details are given for the calculation of the most important
observables considered in this thesis. First, the observables used for the measurement of positional 
order are considered, followed by those related to orientational order. This section closes on the
computation of observables profiles for the study of confined systems.

%==================================================================================
%==================================================================================
\subsection{Positional order}


%-------------------------------------------
%-------------------------------------------
\subsubsection{The radial pair distribution function~: $g(r)$}


The pair distribution function $g(r)$ is of great importance in the molecular simulations of
fluids as it provides detailed insight into the structure of the
studied phase. $g(r)$ represents the probability of finding a pair of particles $i$ and $j$ with
an intermolecular separation $r_{ij}$.  As a result, quantitative insight into the the nature 
of the studied phase (gas-liquid-solid) and the  positional correlations of the particles
can be obtained using $g(r)$. This function can be expressed as~\cite{AandT}~:
\begin{equation}
        g(r) = \frac{V}{N^2} \left< \sum_{i}\sum_{j\neq i} \delta(r - r_{ij}) \right>
\end{equation}
%
Where $\delta(r-r_{ij})$ is a function which is non zero over a given interval, $V$ is the volume and 
$N$ the number of particles.\\

\picW = 10cm
\begin{figure}
	\centering
	\pic{grSphericalShells.ps}
	\caption{Representation of the volume corresponding to a spherical shell between the
	distances $r$ and $r+\delta r$ as used for the computation of $g(r)$. For clarity
	purposes, part of the volume has been excised.}
	\label{fig:g(r)SphericalShells}
\end{figure}

Within the course of a simulation, $g(r)$ is constructed by computing an histogram of all pair
separations $r_{ij}\in[r: r+\delta r]$ where $r \in [0:\frac{L_\mrm{min}}{2}]$ and 
$L_\mrm{min}$ is the shortest simulation box length. The histogram bin heights represent 
the average particle occupancies in concentric spherical shells around any particle taken as 
the reference (see Figure~\ref{fig:g(r)SphericalShells}). In order to obtain $g(r)$, the 
histogram must be normalized  by the average expected occupancy of an ideal gas at 
the same density. This implies that the histogram bin corresponding to a distance $r$, must 
be normalized by $\rho^{*}V_\mrm{shell}(r)$  where $\rho^{*}$ is the number density of 
the fluid and $V_\mrm{shell}(r)$ is the volume between two spheres of radius $r$ and $r+\delta
r$. The volume $V_\mrm{shell}(r)$ is shown on Figure~\ref{fig:g(r)SphericalShells} 
and is given by~:
%
\begin{eqnarray}
	V_\mrm{shell} &=& \frac{4}{3}\pi\left[ (r+\delta r)^3 - r^3 \right]	\nonumber \\
	V_\mrm{shell} &=& \frac{4}{3}\pi\left[ (\delta r)^3 + 3r^2\delta r + 3r(\delta r)^2\right]
\end{eqnarray}

To obtain of smooth functions requires the computation of an average $g(r)$ from several
uncorrelated configurations. This in turns, implies that the histogram must also be normalized by
$N_\mrm{call}$, the number of configurations used. Also the histogram must be normalized by $N$,
the number of particles used so as to make it system-size independent. As a result the 
total normalization coefficient is given by~:
\begin{eqnarray}
	f_\mrm{norm} &=& \lp N N_\mrm{call}\rho^{*} V_\mrm{shell} \rp^{-1}	\nonumber \\
	f_\mrm{norm} &=& (3V)
	\left\{ 4N^2N_\mrm{call}\pi\left[ (\delta r)^3 + 3r^2\delta r + 3r(\delta
	r)^2\right] \right\}^{-1}
\end{eqnarray}


%-------------------------------------------
%-------------------------------------------
\subsubsection{Projections of $g(r)$~: $g_\parallel(r_\parallel)$ and $g_\perp(r_\perp)$}

In the case of liquid crystalline phases, the anisotropic nature of the fluid can make it
necessary to consider 
different distribution functions in different directions of space. For instance, the 
distributions functions resolved parallel and perpendicular to the director, namely 
$g_\parallel(r_\parallel)$ and $g_\perp(r_\perp)$, are of great utility in the study of 
smectic phases. The former measures the degree of layering in the sample while the latter
measures the intra-layer positional order.\\

\picW = 7.2cm
\begin{figure}
	\centering
	\subfigure[$g_\parallel(r_\parallel)$]{\pic{grParallelShells.ps}}
	\subfigure[$g_\perp(r_\perp)$]{\pic{grPerpShells.ps}}
	\caption{Representation of the volumes corresponding to a cylindrical shell between the
	distances $r$ and $r+\delta r$ as used for the computation of 
	$g_\parallel(r_\parallel)$(a) and $g_\perp(r_\perp)$(b). For clarity
	purposes, part of the volume has been excised.}
	\label{fig:gp(r)CylinderShells}
\end{figure}

The approach used in the computation of $g_\parallel(r_\parallel)$ and $g_\perp(r_\perp)$ is 
very similar to that used for
$g(r)$. Here histograms of the projection of $\vect{r_{ij}}$
parallel ($r_\parallel = \dotProd{\vecth{n}}{\vect{r_{ij}}}$) and perpendicular 
($r_\perp = \sqrt{r_{ij}^2 - r_\parallel^2 }$) to the director $\vecth{n}$ are considered. In
order to simplify the normalization process, the histograms are computed within a cylindrical
geometry as shown on Figure~\ref{fig:gp(r)CylinderShells}. Again, the histograms are normalized
by $\rho^{*}V_\mrm{shell}$, the expected average occupancy of a shell of an ideal gas, 
$N_\mrm{call}$ and $N$. However, because a cylinder is considered, the expression for
$V_\mrm{shell}$ is different. This is given for each of the functions as~:
%
\begin{equation}
V_\mrm{shell} = \left\{ %}
\begin{array}{ccc}
	\pi r^2 \delta r	 &\mrm{\ for\ } &g_\parallel(r_\parallel)	\\
	h_\mrm{cyl} \pi \left[ (\delta r)^2 + 2r\delta r  \right] &\mrm{\ for\ } &g_\perp(r_\perp)	
\end{array}
\right.
\end{equation}

%------------------------------
\picW = 7cm
\begin{figure}
	\subfigure[]{\pic{grCylConfig.ps}}
	\subfigure[]{\pic{grCylConfig2.ps}}
	\caption{Representation of the geometry used for the calculation of the size of the
	cylinder used for the computation of $g_\parallel(r_\parallel)$ and $g_\perp(r_\perp)$. 
	(a) shows a three dimensional view and (b) is the projection  of (a) in the plane 
	P taking $h_\mrm{cyl}=L_\mrm{min}$.}
	\label{fig:cylGeometry}
\end{figure}
%------------------------------

where $h_\mrm{cyl}$ represents the height of the cylinder in which the computation is performed.
The size of the cylinder must be chosen so as to be smaller than the simulation box but large
enough to consider as wide a region as possible. The chosen method sets the cylinder height
to $h_\mrm{cyl} = 0.8L_\mrm{min}$. The cylinder radius $r_\mrm{cyl}$ is then chosen so that 
the cylinder would just fit in a cubic box of size $L_\mrm{min}$ if $h_\mrm{cyl} = L_\mrm{min}$
(See Figure~\ref{fig:cylGeometry}). Therefore the size of the cylinder is~:
\begin{eqnarray}
	h_\mrm{cyl} &=& 0.8 L_\mrm{min}	\\
	r_\mrm{cyl} &=& b\tan\alpha	
\end{eqnarray}

where $b$ and $\alpha$ are shown on Figure~\ref{fig:cylGeometry} and are given by~:
\begin{eqnarray*}
	b &=& \frac{L_\mrm{min} \lp \sqrt{3}-1 \rp}{2}	\\
	\alpha &=& \arccos\sqrt{\frac{2}{3}}
\end{eqnarray*}


%-------------------------------------------
%-------------------------------------------
\subsubsection{Molecule-based projections of $g(r)$~: 
$g^\mrm{mol}_\parallel(r_\parallel)$ and $g^\mrm{mol}_\perp(r_\perp)$}

In ordered systems where the particles form layers which are not parallel with one another, taking
the director $\vecth{n}$ as the reference for the computation of the pair correlation functions
becomes irrelevant. Rather an alternative scheme which allows to `follow' the layers is needed.
This is obtained by the use of $g^\mrm{mol}_\parallel(r_\parallel)$ and
$g^\mrm{mol}_\perp(r_\perp)$ which give the pair correlation functions parallel and
perpendicular to the molecular orientation rather than $\vecth{n}$. In practice, these histograms 
are computed for every pair of particles $i$ and $j$ taking $\ui$ as the reference. 
$r_\parallel$ and $r_\perp$ are then defined as~:
%
\begin{eqnarray}
	r_\parallel &=& \dotProd{\ui}{\vect{r_{ij}}}	\\
	r_\perp &=& \sqrt{r_{ij}^2 - r_\parallel^2 }
\end{eqnarray}
%
The same cylinder geometry is used as for the computation of $g^\mrm{mol}_\parallel(r_\parallel)$ and
$g^\mrm{mol}_\perp(r_\perp)$, the difference being that its orientation changes according to which
particle $i$ is being considered.





%==================================================================================
%==================================================================================
\subsection{Orientational order}
\label{ss:P2}

%------------------------------------------------------
%------------------------------------------------------
\subsubsection{Nematic order parameter.}

The liquid crystalline phase can be characterized partly through the long range orientational
order of the molecules; this triggers the need for an appropriate order parameter so as to
quantify the degree of order in a given phase. Ideally this order parameter should have a value
of zero for a phase with an isotropic distribution of molecular orientations and a value of 
one for a phase with perfect alignment.\\

Experimentally, an appropriate definition for this is the so called nematic order parameter
$P_2$~\cite{Intro_LC} which is the average over all particles of the second order Legendre polynomial 
in $\cos\alpha$, where $\alpha$ is the angle between every molecule and the 
director $\vecth{n}$~\cite{Intro_LC}.
\begin{eqnarray}
	P_2 &=& \left< P_2(\cos\alpha)  \right>_\mrm{particles}	\\
	P_2 &=& \left< \frac{3}{2}\cos^2\alpha - \frac{1}{2} \right>_\mrm{particles}
\end{eqnarray}
Also, the nature of $P_2(\cos\alpha)$ involving $\cos^2\alpha$ implies that the nematic order
parameter does not differentiate particles with orientations $\ui$ and $-\ui$.\\

Within the scope of computer simulations, the computation of $P_2$ and, thus, its run average
\Ptwo is not trivial. However it can be shown~\cite{EppengaFrenkel84,AdvCpuSimsLiqCryst2}
that the problem can be reduced to the diagonalisation of the ordering matrix 
$Q_{\alpha\beta}$, a traceless second order tensor defined as~:
%
\begin{equation}
	Q_{\alpha\beta} = \frac{1}{2N}\sum_{i=1}^{N} \left\{ 
		3u_{i,\alpha}u_{i,\beta} - \delta_{\alpha\beta} \right\}
\end{equation}
%
where $\delta_{\alpha\beta}$ is the Kroeneker function. The order parameter $P_2$ is defined 
by $\lambda_+$, the maximum Eigen value of $Q_{\alpha\beta}$~\cite{EppengaFrenkel84}. 
The director is, then, the Eigen vector associated with $\lambda_+$.\\
%
In a simulation, the nematic order parameter \Ptwo is obtained by averaging the values of 
$P_2$ obtained from a significant number of uncorrelated configurations.\\

\picW = 10cm
\begin{figure}
	\centering
	\picL{P2recal_n.ps}
	\caption{Variation of the value of $\lambda_+(n)$ computed using the $\vect{Q}$ matrix method 
	for a system of $n$ particles with an isotropic distribution of orientations.}
	\label{fig:P2(n)}
\end{figure}


Eppenga and Frenkel~\cite{EppengaFrenkel84} showed that, while this method is very accurate 
for the description  of well ordered phases, the case of the less ordered phases is 
more problematic, especially for small systems. Indeed, in the case of a phase with an isotropic 
distribution of orientations, $P_2$ should be zero whereas the value of the computed 
$\lambda_+$ decays to zero  as the number of particles increases (Figure~\ref{fig:P2(n)}). 
The difference between  the computed and expected values become negligible for a particle 
numbers $N\geq\mathcal{O}(10^3)$.
%

%------------------------------------------------------
%------------------------------------------------------
\subsubsection{Polar order parameter.}

In the case of molecules with permanent dipole moments, both the order in the system, and the
direction of $\vecth{n}$ become important. The task of differentiating $\ui$ and $-\ui$ is
achieved using the first order Legendre polynomial $P_1(\cos\alpha)=\cos\alpha$. The polar
order parameter is therefore referred to as \Pone.\\
The computation of \Pone requires the knowledge of the polar director $\vecth{n}_{P1} =
\frac{\vect{n}_{P_1}}{|\vect{n}_{P_1}|}$ with~:
\begin{equation}
	\vect{n}_{P1} = \frac{1}{N}\sum_{i=1}^N \ui.
\end{equation}
%
and the instantaneous value of $P_1$ is given by~:
\begin{equation}
	P_1 = |\vect{n}_{P_1}|
\end{equation}

The simulation averaged polar order parameter \Pone is obtained by taking the
average of the instantaneous $P_1$ values from a high enough number of uncorrelated
configurations.\\
A low polar order parameter does not necessarily indicate a disordered phase as non-polar
nematic phases usually have an equal proportion of particles with $\ui \sim \vecth{n}$ and
$\ui \sim -\vecth{n}$~; however a high value of $P_1$ does imply a high value of $P_2$.



%==================================================================================
%==================================================================================
\subsection{Observables profiles}

In confined systems, the presence of interfaces introduces structural changes
which, for a given set of thermodynamic and surface parameters, are functions of the distance 
from the substrate. As a result, a great deal of insight into surface induced effects can be
obtained through computation of observables profiles. Since it is common to consider the case of 
a slab geometry in the $\vecth{z}$ direction (such as that used in Chapter~\ref{chap:four}),
such profiles are referred to as the $z$-profiles.\\

The computation of most profiles is very straightforward; the simulation box is
divided into $N_\mrm{slice}$ virtual slices, of width $w_\mrm{slice}$ parallel to the
substrates, in which the  observables are computed independently.
The computation of the profile $\mathcal{A}(z)$ of a property $\mathcal{A}$ 
requires the computation of $\mathcal{A}$ in each slice. $\mathcal{A}(z_0)$ is obtained by
computing $\mathcal{A}$ considering only those particles whose $z$ coordinates $z_i$ are such that 
$z_i \in [z_0 - \frac{w_\mrm{slice}}{2} : z_0 + \frac{w_\mrm{slice}}{2}]$.
In the course of a simulation, smooth profiles are obtained through averaging a significant
number of instantaneous profiles (typically $500$) obtained from uncorrelated configurations.\\

The most commonly computed profiles are $\rho^{*}_\ell(z)$ and $Q_{zz}(z)$. The former
measures the variation of density across the simulation box. Its computation requires simple
division of the number of particles in a given slice by the volume of that slice.\\
%
$Q_{zz}(z)$ represents the variation of the $zz$ element of $Q_{\alpha\beta}$ across the slab.
$Q_{zz}$ measures the degree of order with respect to $\vecth{z}$, the surface normal. 
$Q_{zz} = -0.5$ for perfect
order perpendicular to $\vecth{z}$ (planar arrangement) and $Q_{zz} = 1.0$ for perfect 
order parallel to $\vecth{z}$ (homeotropic arrangement.) Again the computation of $Q_{zz}(z)$ 
is straightforward as it can be performed by considering only those particles whose centres of 
mass are located within the slice of interest.\\

In some cases however, the computation of the profiles is not straightforward. A good example
is the computation of $P_2(z)$. Here, the reduced number of particles in each slice
introduces a lack of accuracy of the computed profiles because of the $n$ dependence of
$\lambda_+$ 
as presented in Section~\ref{ss:P2}. As a result, the value of $P_2$ in a given slice at position $z_0$ 
can not be computed simply by applying the $\vect{Q}$ matrix method to those particles 
whose centres of mass lie within the slice. Rather, $P_2(z_0)$ is computed using the approach
proposed by Wall and Cleaver~\cite{WallCleaver97} and based on the original expression 
for $\lambda_+$ from~\cite{EppengaFrenkel84}~:
%
\begin{equation}
	\lambda_+ - \frac{3\lambda_+}{4n}\left[ 1 + P_2^2(n-1) \right] - \frac{P_2^3}{4}
	- \frac{3}{4n}\lp P_2^2 - P_2^3\rp - \frac{1-3P_2^2+2P_2^3}{4n^2}
\end{equation}
%
where $\lambda_+$ is the maximum Eigen value of $Q_{\alpha\beta}$ and $P_2$ denotes the true order
parameter in the slice. This can be rearranged so as to give a polynomial in $P_2$ as~:
%
\begin{equation}
	aP_2^3 + bP_2^2 + c P_2 + d = 0
	\label{eqn:P2ProfPoly}
\end{equation}
%
with~:
\begin{eqnarray*}
	a &=& -n^2 + 3n + 2			\\
	b &=& -3\lambda_+ n (n-1) - 3(n-1)	\\
	c &=& 0					\\
	d &=& 4n^2\lambda_+^3 - 3n\lambda_+ - 1
\end{eqnarray*}

$P_2(z_0)$ is then obtained by solving Equation~\ref{eqn:P2ProfPoly} taking into account the $n$
particles which belong to the slice centred at $z=z_0$. It should be noted that some special
cases must be considered. The computation is skipped if $n=0$ or $1$ as this would lead,
respectively, to a trivial solution or an incorrect value of $P_2=1$. In the case $n=2$,
Equation~\ref{eqn:P2ProfPoly} reduces to a second order polynomial with roots 
$P_2 = \pm\sqrt{-\frac{d}{b}}$.\\
%
If the roots of Equation~\ref{eqn:P2ProfPoly} are complex then an alternative scheme is 
used where $P_2 = P_2^\mrm{recal}$ with $P_2^\mrm{recal} = \lambda_+ - 
\left< \lambda^\mrm{rd}(n)\right>$. Here $\left<\lambda^\mrm{rd}(n)\right>$ is the average
$\lambda_+$ obtained from applying the $\vect{Q}$ matrix method to a high number (\eg $10^5$) of
configuration of $n$ particles with an isotropic distribution of orientations. If
$P_2^\mrm{recal} < 0$ then the computation of $P_2$ for this slice is skipped.










\section{Computer simulations}

In this Section, results from Monte Carlo computer simulations of bulk systems of hard Gaussian
overlap particles are presented. Although most of
these simulations do not lead to new results (as the phase diagram of the model is already
known,) they do provide a good test-bed for the simulation code used to produce the 
novel results given in Chapter~\ref{chap:four} to~\ref{chap:six}.\\
Two sets of results are presented here. First the bulk simulation of calamitic particles is
considered using two elongations $k=3$ and $5$. The phase diagrams from these simulations are
compared with those extracted from the literature~\cite{PadillaVelasco97,DeMiguelDelRio01}
in order to validate the simulation code. Results from the simulation of discotic particles are
then presented using the elongations $k=1/3$ and $1/5$.



%=================================================================================================
%=================================================================================================
\subsection{Calamitic particles}
\label{HGO_cal_res}
Calamitic mesogens have been simulated using the hard Gaussian overlap model in the canonical
and isothermal-isobaric ensembles and in compression sequences. 
Systems of $N=1000$ particles with elongation $k=3$ and $5$  were used. Typical runs consisted 
of $5.10^5$ to $1.10^6$ sweeps (\ie attempted moves per particle) for both equilibration and
production. The phase diagrams of the model were generated by computing $P^{*}(\rho^{*})$ 
from the constant $NPT$ runs and $\Ptwo(\rho^{*})$ from both sets of runs. Those are shown on
Figure~\ref{fig:HGO_phaseDia_k3_5} respectively for $k=3$ and $5$.\\


\picW = 7cm
\begin{figure}
	\centering
	\subfigure[]{\picL{HGO_P-rho_k3.ps}\picL{HGO_P2-rho_k3.ps}}
	\subfigure[]{\picL{HGO_P-rho_k5.ps}\picL{HGO_P2-rho_k5.ps}}
	\caption{Phase diagrams obtained from Monte Carlo simulations in the constant $NPT$ and
	constant $NVT$ ensembles of systems of $N=1000$ hard Gaussian overlap particles with
	$k=3$(a) and $k=5$(b)}
	\label{fig:HGO_phaseDia_k3_5}
\end{figure}




\picW = 5cm
\begin{figure}
	\centering
	\subfigure[$k=3$, $P^{*}=3.5$]{\pic{HGO_box_k3_iso.ps}}
	\hspace*{1cm}
	\subfigure[$k=3$, $P^{*}=8.0$]{\pic{HGO_box_k3_nem.ps}}
	
	\subfigure[$k=5$, $P^{*}=0.8$]{\pic{HGO_box_k5_iso.ps}}
	\hspace*{1cm}
	\subfigure[$k=5$, $P^{*}=1.5$]{\pic{HGO_box_k5_nem.ps}}
	\caption{Typical configuration snapshots obtained from constant pressure Monte Carlo
	simulation of systems of $N=1000$ HGO particles with elongation $k=3$(a,b) and $5$(c,d).}
	\label{fig:HGOSnaps_k3_5}
\end{figure}


For both elongations the $P^{*}(\rho^{*})$ curves show a `plateau' characteristic of a first order
transition. These correspond on $\Ptwo(\rho^{*})$ to a sharp, `S'-shaped increase in the nematic
order parameter from values corresponding to an isotropic phase to those consistent with a
nematic phase. Observation of the configuration snapshots (\eg Figure~\ref{fig:HGOSnaps_k3_5}) 
confirms this. From these data $P^{*}_{in}$, the pressure at the isotropic-nematic coexistence and
$\rho^{*}_{i}$ and $\rho^{*}_{n}$ respectively the density at coexistence of the isotropic and
nematic phases can be estimated. Since the isotropic to nematic transition was not of specific
interest in this thesis, those have not been determined with the great accuracy that
techniques such as thermodynamic integration allow~\cite{CampMason96,CampAllen97}. Rather, 
$\rho^{*}_{i}$ and $\rho^{*}_{n}$ are taken to be the densities corresponding respectively to the
beginning and end of the `plateau' in $P^{*}(\rho^{*})$. $P^{*}_{in}$ is taken to be the
average of the pressures corresponding to $\rho^{*}_{i}$ and $\rho^{*}_{n}$.
The coexistence data are shown in Table~\ref{tble:HGOTransitionData_k3_5} 
along with the results from Padilla and Velasco~\cite{PadillaVelasco97} and de Miguel 
and Mart\'{\i}n del R\'{\i}o~\cite{DeMiguelDelRio01} for comparison.

\begin{table}
\centering
\begin{tabular}{||c||c||c||c||c||c||c||}
\hhline{|t:=:t:=:t:=:t:=:t:=:t:=:t:=:t|}
Source		&$k$	&System size	&$P^{*}_{in}$	&$\rho^{*}_i$	&$\rho^{*}_n$	&$\frac{\rho^{*}_n -\rho^{*}_i}{\rho^{*}_i}$	\\
\hhline{|:=::=::=::=::=::=::=:|}
\cite{PadillaVelasco97}	&3	&$256$	&$4.50$		&$0.290$	&$0.299$	&$0.031$	\\
\cite{DeMiguelDelRio01}	&3	&$500$	&$4.92$		&$0.299$	&$0.304$	&$0.019$	\\
this study		&3	&$1000$	&$4.975$	&$0.299$	&$0.309$	&$0.033$	\\
\hhline{|:=::=::=::=::=::=::=:|}
\cite{PadillaVelasco97}	&5	&$256$	&$0.880$	&$0.113$	&$0.120$	&$0.062$	\\
\cite{DeMiguelDelRio01}	&5	&$500$	&$0.996$	&$0.119$	&$0.127$	&$0.067$	\\
this study		&5	&$1000$	&$1.025$	&$0.122$	&$0.127$	&$0.040$	\\
\hhline{|b:=:b:=:b:=:b:=:b:=:b:=:b:=:b|}
\end{tabular}
\caption{Comparison of the isotropic-nematic transition data for the HGO model and $k=3$ and $5$
with existing results.}
\label{tble:HGOTransitionData_k3_5}
\end{table}

The comparison shows that the results obtained here are fully compatible with those obtained 
in~\cite{PadillaVelasco97} and~\cite{DeMiguelDelRio01}. The slight numerical differences can be
attributed to the differences in system sizes. 
de Miguel~\cite{DeMiguel92} has already shown that the effect of increasing system sizes on the
isotropic-nematic transition of system of Gay-Berne particles is to shift the transition to
higher densities or lower temperatures. This shift can be observed through the shift of \Ptwo
and $\rho^{*}_{i}$ and $\rho^{*}_{n}$ to higher densities or lower temperature with increased
system sizes. Another effect of system-size noticeable in de Miguel's results is a slight
strengthening of the IN transition with bigger systems.
The good agreement shown here validates the accuracy of the Monte Carlo simulation code 
used in this study.


%=================================================================================================
%=================================================================================================
\subsection{Discotic particles}

The Monte Carlo simulation code was subsequently applied to the modeling of discotic particles 
which have not been considered in the literature. These systems were simulated using 
methods adopted with the calamitic
particles, that is using Monte Carlo simulations in the canonical and isothermal-isobaric
ensembles. Systems of $N=1000$ particles with elongations $k=1/3$ and $1/5$ were simulated 
in compression sequences, using similar run lengths to those employed with the prolate elongaions. 
The phase diagrams obtained from these simulations are shown on Figure~\ref{fig:HGO_phaseDia_k0.33_0.2} 
for $k=1/3$ and $k=1/5$.\\

\picW = 7cm
\begin{figure}
	\centering
	\subfigure[$k=1/3$]{\picL{HGO_P-rho_k0.33.ps}\picL{HGO_P2-rho_k0.33.ps}}
	\subfigure[$k=1/5$]{\picL{HGO_P-rho_k0.2.ps}\picL{HGO_P2-rho_k0.2.ps}}
	\caption{Equation of states obtained from Monte Carlo simulations in the constant $NPT$ and
	$NVT$ ensembles of systems of $N=1000$ hard Gaussian overlap particles with
	$k=1/3$(a) and $k=1/5$(b).}
	\label{fig:HGO_phaseDia_k0.33_0.2}
\end{figure}



\picW = 5cm
\begin{figure}
	\centering
	\subfigure[$k=1/3$, $P^{*}=25.0$]{\pic{HGO_box_k0.33_iso.ps}}
	\hspace*{1cm}
	\subfigure[$k=1/3$, $P^{*}=59.0$]{\pic{HGO_box_k0.33_nem.ps}}
	
	\subfigure[$k=1/5$, $P^{*}=14.0$]{\pic{HGO_box_k0.2_iso.ps}}
	\hspace*{1cm}
	\subfigure[$k=1/5$, $P^{*}=30.0$]{\pic{HGO_box_k0.2_nem.ps}}
	\caption{Typical configuration snapshots obtained from constant pressure Monte Carlo
	simulations of systems of $N=1000$ HGO particles with elongation $k=1/3$(a,b) and $1/5$(c,d).}
	\label{fig:HGOSnaps_k0.33_0.2}
\end{figure}

These results show a similar behaviour to that observed with the calamitic particles. Both
$P^{*}(\rho^{*})$ curves show a `plateau' characteristic of a first order phase
transition which corresponds to the typical sharp `S'-shaped increase in $\Ptwo(\rho^{*})$
indicating an isotropic to discotic-nematic phase transition. This is further confirmed by
observation of configuration snapshots, \eg Figure~\ref{fig:HGOSnaps_k0.33_0.2}.
Despite the high pressures used
here, no signs of a transition to a columnar phase have been observed. This is consistent with
the lack of smectic phase for the hard Gaussian overlap model with $k>1$.







%=================================================================
