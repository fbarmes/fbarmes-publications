

\chapter{More on confined geometries}
\label{chap:five}
%===============================================================================
%===============================================================================

\introduction

Simulations of confined systems of hard Gaussian overlap particles interacting with the
substrates through the hard needle wall potential have shown that, despite the simplicity 
of the model, a range of surface induced behaviour can be observed. Using this simple setup 
a thorough and systematic study of the surface induced structural changes has been performed 
and an anchoring transition has been identified. In this Chapter, the focus is 
brought to bear on the study of alternative confined system configurations.\\
%
First, more realistic surface potentials are studied and their anchoring phase diagrams computed 
so as to investigate their suitability for the modeling of anchoring transitions. Two such 
potentials are of interest, namely the rod-sphere potential and the rod-surface potential. For
each of these, the the surface induced arrangements are studied and the possibility of 
bistable regions is explored.\\
%
The second part of this Chapter contains a study of hybrid anchored systems performed
using the hard needle wall potential; following this, the possibility of simulating 
easy switching between the hybrid aligned nematic and vertical states of the cell is investigated.


%===============================================================================
%===============================================================================

\section{Realistic surface potentials : the RSP.}
\label{s:realisticPotentials_RSP}


%===============================================================================================
%===============================================================================================
\subsection{The rod-sphere potential.}
\label{ss:RSP_potential}


%====================================================================
%		POTENTIAL DESCRIPTION AND LITTERATURE REVIEW
%====================================================================

The rod-sphere surface potential (RSP) describes the interaction between a Gaussian ellipsoid and a
sphere located in the surface plane and with the same $x$ and $y$ coordinates as the ellipsoid. 
Again, the particles (HGO) do not interact directly with the substrate, rather another
HGO ellipsoid is inserted in each particle (\eg Figure~\ref{fig:RSPConfig}). 
This inner ellipsoid interacts with the surface through $\mathcal{V}^{RSP}$ as~:
%
\begin{equation}
	\mathcal{V}^{RSP} = \left\{	%}
	\begin{array}{ccc}
		0	&\mathrm{if}	&|z_i - z_0| \geq \sigma^{RSP}_w	\\
		\infty	&\mathrm{if}	&|z_i - z_0| < \sigma^{RSP}_w	
	\end{array}
	\right.
\end{equation}
%
where $\sigma^{RSP}_w$ is the contact distance for the interaction between a hard Gaussian 
overlap particle  with length $\sigma_\parallel$ and breadth $\sigma_\perp$ and a sphere of diameter 
$\sigma_j$. The contact distance for such an interaction is given by Equation 
(4) of~\cite{BernePechukas72}~:
%
\begin{equation}
	\sigma^{RSP}_w = \sqrt{
	\frac{\sigma^2_\perp + \sigma^2_j}
	{1 - \chi(\dotproduct{u_i}{r_{ij}})^2 }
	}
	\label{eqn:sw_RSP_BP}
\end{equation}

For implementation into a simulation code, Equation~\ref{eqn:sw_RSP_BP} is best written
in terms of $\theta$ and $\sigma_0$~; recalling that the unit of distance 
$\so = \sigma_\perp\sqrt{2}$ and for convenience, imposing $\so = \sigma_j\sqrt{2}$. 
Also to enable
comparison with the hard needle wall potential, the sphere is taken to be tangent with the
substrate so as to keep it out of the simulation box. This leads to the final expression for
$\sigma^{RSP}_w$ as used in the simulations~:
%
\begin{equation}
	\sigma^{RSP}_w = \sigma_0\lp \frac{1}{\sqrt{1 - \chi_S\cos^2\theta}} - \frac{1}{2} \rp
	\label{eqn:sigma_w_RSP}
\end{equation}
%
with~:
%
\begin{equation}
	\chi_S = \frac{k^2_S - 1}{k^2_S + 1}
\end{equation}
%
$k_S$ being the length to breadth ratio of the inner ellipsoid. A graphical representation of
this contact distance  as a function of $k_S$ and $\theta$ is shown on
Figure~\ref{fig:sw_RSP}.\\

%------------------------------------------
\picW = 10cm
\begin{figure}
	\centering
	\pic{RSPConfig.ps}
	\caption{Representation of the geometry used for the interaction between the inner HGO
	particle and the sphere representing the substrate in the RSP surface potential.}
	\label{fig:RSPConfig}
\end{figure}

%------------------------------------------

%------------------------------------------
\picW = 10cm
\begin{figure}
	\centering
	\pic{sw_RSP.ps}
	\caption{Representation of $\sigma^{RSP}_w(k_S, \cos\theta)$ for the RSP surface potential.}
	\label{fig:sw_RSP}
\end{figure}
%------------------------------------------

The rod-sphere model has already been used as the contact distance for a soft surface potential in 
studies of confined Gay-Berne particles in single component systems~\cite{ZhangChakrabarti96, 
WallCleaver97,TeixeiraChrzanowska01},  binary mixtures~\cite{LathamCleaver00} 
and switching situations~\cite{rwThesis}.  In all of these, full particles (\ie $k_S = k$) 
were used and tilted layers were observed in the interfacial regions.
In~\cite{WallCleaver97,TeixeiraChrzanowska01} the tilt was explained to be a consequence of 
the competition 
between packing constraints and the form of the surface potential. In other words, the attractive 
part of the potential was thought to be responsible for the tilt, the authors noticing that increasing 
the  particle-surface coupling $\alpha$ (see equation (9) of~\cite{WallCleaver97}) induced a tighter 
distribution of the particle orientations about the optimal tilt angle. It is interesting to
note however, that using the same surface potential, but a molecular elongation $k=2$ instead of
$3$, Wall and Cleaver~\cite{WallCleaver03} found that the surface anchoring changed from tilted
to planar.\\
Another case of surface tilted arrangement was obtained in the study by Lange and 
Schmid~\cite{LangeSchmid02,LangeSchmid02a,LangeSchmid02c} where 
Gay-Berne particles were confined between two structureless walls. There, the surface potential 
used was
similar to the RSP, but a different $\sigma_w$ function was used so as to describe the
interaction between a surface and an ellipsoid of revolution; also, they used the common $12-6$
Gay-Berne potential rather the $9-3$ version of~\cite{WallCleaver97}. The particles simulated were found
to exhibit planar anchoring, but tilted phases were obtained by inclusion of polymer chains
grafted on to the surface. The tilt in that case resulted from competition between the planar
orientation favoured by the solvent particles and the homeotropic alignment preferred by the 
polymer chains (since it reduced their bond energy). Again, in that study, tilting behaviour could 
be explained in terms
of the attractive parts of the potential since an anchoring transition between planar and tilted
arrangements was obtained by varying the number of grafted polymer chains.\\
Ascribing the tilt to the attractive part of the potential in~\cite{WallCleaver97} was also 
consistent with the many theoretical treatments of confined hard particle 
systems~\cite{HolystPoniewierski88,Chrzanowska_Teixera_01}~: 
none of these predict tilted surface alignment, planar and homeotropic alignments being the only
arrangement predicted.\\
It is interesting to note, however, that when using a surface made of fixed atoms
and a similar surface potential to that used in~\cite{WallCleaver97},  
Palermo~\etal\cite{PalermoBiscarini98} 
did not find any tilt arrangements, rather the natural planar arrangement was adopted at 
the substrates. This discrepancy would suggest that the interaction described by the rod-surface
potential where the particles and substrate sphere have always the same $x$ and $y$ coordinate
might be responsible for the tilting behaviour. However, this argument can not be proved just by
comparing the results from these existing studies as they employed different forms of the 
Gay-Berne type surface interaction potential, a 9-3 type in~\cite{WallCleaver97} as opposed to 12-6
in~\cite{PalermoBiscarini98}.\\


The case of short $k_S$, that is the case of particles absorbing the substrates, 
has not been considered in these studies. This case can, however,  be readily
understood. If the amount of volume absorbed is great enough to induce a significant 
reduction in the free energy, then a homeotropic arrangement should be more stable, as borne out
by the simulation studies of Allen~\cite{Allen99}.

%===============================================================================================
%===============================================================================================
\subsection{Simulation results using the RSP.}
\label{ss:RSPresults}
%====================================================================
%			RESULTS - PHASE DIAGRAMS
%====================================================================

Further investigations of the
surface induced structural changes obtained using the RSP surface potential were performed using
Monte Carlo computer simulations in the canonical ensemble. Systems of $N=1000$ hard Gaussian overlap
particles with elongation $k=3$ confined in an infinitely wide slab geometry of fixed height 
$L_z=4k\sigma_0$ were considered, the walls being situated at $z_0 = \pm\frac{L_z}{2}$ and
symmetric anchoring conditions applied. Sequence of
simulations were performed at constant number density $\rho^{*}$ and decreasing 
$k_S$ for several values of $\rho^{*}$ and the surface induced structural changes were studied 
using the observable 
profiles ($\rho^{*}_\ell(z)$, $Q_{zz}(z)$ and $P_2(z)$) introduced in the previous Chapter. 
From these profiles, the anchoring and order phase diagrams were computed in the 
interfacial and bulk regions. These diagrams are shown on Figures~\ref{fig:RSP_QzzWa} 
and~\ref{fig:RSP_P2Wa} and were computed using a similar method to that given in the previous
Chapter. The difference here was that
definition of the boundary between the interfacial and bulk regions was changed
such that the interfacial region was taken to extend from the surface to the second 
maximum of $\rho^{*}_\ell$
regardless of the surface arrangement obtained for this model. The reason for this definition 
change lies in the similarity
between density profiles obtained for this model at different values of $k_S$:
the primary peaks in $\rho^{*}_\ell(z)$ were always situated at $|z-z_i|> 0.0$ 
(see \eg Figures~\ref{fig:RSP_typeProf_k3_homeo} and~\ref{fig:RSP_typeProf_k3_planar}). 
Further details regarding these profiles are given later in this Section.\\

\picW = 7cm
\begin{figure}
	\centering
	\subfigure[Interfacial region.]{\pic{QzzWaSu_RSPphaseDia_k3_kSD.ps}}
	\subfigure[Bulk region.]{\pic{QzzWaBu_RSPphaseDia_k3_kSD.ps}}
	\caption{Anchoring phase diagrams obtained from series of simulations of $N=1000$
	confined HGO particles with $k=3$ at constant density and decreasing $k_S$ 
	and using the RSP surface potential.}
	\label{fig:RSP_QzzWa}
\end{figure}

\begin{figure}
	\centering
	\subfigure[Interfacial region.]{\pic{P2WaSu_RSPphaseDia_k3_kSD.ps}}
	\subfigure[Bulk region.]{\pic{P2WaBu_RSPphaseDia_k3_kSD.ps}}
	\caption{Order phase diagrams obtained from series of simulations of $N=1000$
	confined HGO particles with $k=3$ at constant density and decreasing $k_S$ 
	and using the RSP surface potential.}
	\label{fig:RSP_P2Wa}
\end{figure}

\picW = 7cm
\begin{figure}
	\centering
	\subfigure[$k^{'}_S = 0.0$]
	{\pic{HGO_box_NVT_RSconf_k3_N1000_kS000_d0.3500_0.50M.ps}}
	%
	\subfigure[$k^{'}_S = 1.0$]
	{\pic{HGO_box_NVT_RSconf_k3_N1000_kS100_d0.3500_0.50M.ps}}
	\caption{Typical configuration snapshots showing the surface induced homeotropic (a) and 
	tilted (b)  surface induced arrangements for confined systems of $N=1000$ HGO particles 
	using $\mathcal{V}^{RSP}$ for surface interactions and $\rho^{*} = 0.35$.}
	\label{fig:RSP_snaps}
\end{figure}
%---------------------------------------------------------------------------

Observation of the order and anchoring phase diagrams reveals that the $\overline{P}_2$ diagrams 
are qualitatively
similar to those obtained with the HNW models, whereas there are some  qualitative 
differences in the $\overline{Q}_{zz}$ diagrams.\\
In the case of short $k_S$, the $\overline{Q}_{zz}$ behaviour for the RSP model 
is not unlike that of the HNW model and confirms the predictions made at the end of the  
last Section. Throughout the
density range considered here and for short $k_S$, the system adopts an homeotropic arrangement
where order increases with increased density. This is further confirmed by 
configuration snapshots (\eg Figure~\ref{fig:RSP_snaps}(a) for the state point
$\rho^{*}=0.35$, $k_S/k=0.0$).\\
In the case of long $k_S$ (\ie $k_S/k>0.6$), however, there is a qualitative difference between 
the diagrams shown in  Figure~\ref{fig:RSP_QzzWa} and their equivalent for the HNW model. 
Here, throughout the density range considered, the value of $\overline{Q}_{zz}$ does not reach 
that expected for planar ordering, remaining, instead, low at about $0.2$. Although such
values can be understood in the low density regime, where $\overline{P}_2$ is compatible with
an isotropic phase, this behaviour is rather more surprising in the case of high densities
where the corresponding $\overline{P}_2$ diagram shows nematic order. In these latter regions, 
the low values of
$\overline{Q}_{zz}$ are, however, compatible with a tilted arrangement. 
Observation of typical snapshots for these high densities 
(\eg Figure~\ref{fig:RSP_snaps}(b) ) confirms the presence of a tilt, showing a
phase where the average surface alignment is of about $\pi/4$ radians.\\


%=================================================================================================
%		RESULTS - PROFILES
%=================================================================================================

\picW = 12cm
\begin{figure}
	\centering
	\pic{RSP_typeProf_k3_homeo.ps}
	\caption{Typical $z$-profiles for confined systems of HGO particles with 
	$k=3.0$ and $k^{'}_S = 0.0$ using the RSP potential.}
	\label{fig:RSP_typeProf_k3_homeo}
\end{figure}

\begin{figure}
	\centering
	\pic{RSP_typeProf_k3_planar.ps}
	\caption{Typical $z$-profiles for confined systems of HGO particles with 
	$k=3.0$ and $k^{'}_S = 1.0$ using the RSP potential.}
	\label{fig:RSP_typeProf_k3_planar}
\end{figure}

Further details of the surface induced structural changes obtained using the RSP potential 
can be obtained from appropriate z-profiles. These are shown for two states points 
corresponding to homeotropic and tilted arrangements in 
Figures~\ref{fig:RSP_typeProf_k3_homeo} and~\ref{fig:RSP_typeProf_k3_planar} respectively.
These profiles, share some of the features of the equivalent profiles obtained with the HNW
potential.
%%
The case $k^\prime_S = 0.0$ corresponds to a  homeotropic arrangement. This is characterized by
positive values of $Q_{zz}(z)$ and peak separations in the oscillations of $\rho^{*}_\ell(z)$ 
of about $\sel$. The quality of in-plane ordering is similar to that observed with the previous
surface potential.\\
%
For $k^\prime_S = 1.0$, however, the situation is very different, and there is little similarity 
between the profiles at $\kSp = 1$ for the RSP and HNW potentials.
The layering shown in Figure~\ref{fig:RSP_typeProf_k3_planar} is not as well defined and only 
two peaks  can  be clearly observed in $\rho^{*}_\ell(z)$. Moreover, the peak separation is much 
larger than $\so$.  Finally, $Q_{zz}(z)$ fails to display the negative values associated with 
planar ordering,  even at high density or which the corresponding $P_2(z)$ profile indicates 
an ordered phase. Those features correspond to a tilted arrangement.\\
%
%



%===============================================================================================
%===============================================================================================
\subsection{Origin of the tilt}


%====================================================================
%			ANALYTICAL ANALYSIS
%====================================================================
Here, the origins of this tilting behaviour are revisited by studying the form of the RSP 
as  a function of $k_S$. In Appendix~\ref{chap:A}, it is shown that for a particle with 
elongation $k$ whose inner 
ellipsoid is in contact with the substrate surface, the volume $Ve(k_S,\theta)$ absorbed into 
the surface is given by~:
%
\begin{equation}
	Ve=\frac{1}{3}\pi\lp \frac{1}{2}-\sqrt{\frac{\sigma^{RSP}_w}{k^2\cos^2\theta+\sin^2\theta}}\rp^2 
	\lp 1 + \sqrt{\frac{\sigma^{RSP}_w}{k^2\cos^2\theta+\sin^2\theta}} \rp
	\label{eqn:Ve_RSP}
\end{equation}
%
A graphical representation of this absorbed volume is given in Figure~\ref{fig:Ve_RSP_fkS} 
for a particle of
elongation $k=3$.  The preferred surface induced arrangements can be associated with the 
maxima in $Ve(k_S,\theta)$. For short $k_S$, $Ve(k_S,\theta)$ is maximal at $\theta = 0$ 
and, therefore, the most stable arrangement is homeotropic. In the limit of $k_S=k$, however
$Ve(k_S,\theta)$ is maximal  for intermediate $\theta$, which suggests that a tilted 
arrangement may be most stable.\\
%

More insight into this result can be found in the expression of the surface potential
(Equation~\ref{eqn:sigma_w_RSP}). In the case $k_S=k$, $\sigma^{HNW}_w$ 
represents the distance from the substrate to the particle's centre of mass when one of its
needle's ends
is in contact with the surface plane.  $\sigma^{RSP}_w$, in contrast, 
indicates whether or not the HGO particle overlaps a sphere embedded within the substrate.\\
The difference between the two shape parameter (Figure~\ref{fig:cmpHNW_RSP})
shows that there are some tilt angle for which $\sigma^{RSP}_w$ is smaller than $\sigma^{HNW}_w$, 
that is the particle ends are able to overlap the surface plane. This region of reduced 
$\sigma^{RSP}_w$  coincides with the maximum in $Ve(k_s,\theta)$ and, therefore, can be 
associated with the tilt behaviour.\\

%=================================
\picW = 10cm
\begin{figure}
	\centering
	\pic{Ve_RSP_fkS.ps}
	\caption{Representation of $Ve(kS,\theta)$ for the RSP potential and $k=3$.}
	\label{fig:Ve_RSP_fkS}
\end{figure}
%=================================

%=================================
\begin{figure}
	\centering
	\picL{cmpHNW_RSP.ps}
	\caption{Comparison between $\sigma^{HNW}_w$ (solid line) and $\sigma^{RSP}_w$ (dashed
	line). The dotted line represents the difference between the two 
	($\sigma^{RSP}_w-\sigma^{HNW}_w$.)}
	\label{fig:cmpHNW_RSP}
\end{figure}
%=================================


%=================================
\picW = 10cm
\begin{figure}
	\centering
	\pic{Ve_RSP_fk.ps}
	\caption{Representation of $Ve(k,\theta)$ for the RSP potential and $k_S=k$.}
	\label{fig:Ve_RSP_fk}
\end{figure}
%=================================

The optimum tilt angle $\theta_{tilt}$ for which the absorbed volume of a single particle is 
maximal can be
calculated for different values of k.  By considering the absorbed volume given by 
Equation~\ref{eqn:Ve_RSP} and setting $k_S=k$, an expression
for $Ve(k,\theta)$ (Figure~\ref{fig:Ve_RSP_fk}) can be obtained.
$\theta_{tilt}$, the angle which maximizes $Ve(k,\theta)$, is the angle that
solves~:
\begin{equation}
	\frac{d}{d\theta}Ve(k,\theta) = 0
	\label{eqn:dVeEq}
\end{equation}
%
where $\frac{d}{d\theta}Ve(k,\theta)$ is given by~:
%
\begin{equation}
	\frac{d}{d\theta}Ve(k,\theta) = k\pi \lp\frac{A_0\lp B_0 + C_0 \rp}{D_0} 
	-\frac{A_0 F_0 \lp B_0 + C_0 \rp}{I_0} 
	\rp.
\end{equation}
%
Here~:
%
\begin{eqnarray*}
	A_0 &=& \lp \frac{1}{2} - \sqrt{\frac{A^2}{B}} \rp^2	\\
	B_0 &=& -\frac{A^2\lp2\cos\theta\sin\theta - 2k^2\cos\theta\sin\theta\rp}{B^2}	\\
	C_0 &=& -\frac{2A(k^2-1)\cos\theta \sin\theta}{(k^2+1)BC^\frac{3}{2}}	\\
	D_0 &=& 6\sqrt{\frac{A^2}{B}}	\\
	F_0 &=& 1+\sqrt{\frac{A^2}{B}}	\\
	I_0 &=& 3\sqrt{\frac{A^2}{B}}
\end{eqnarray*}
%
and~:
%
\begin{eqnarray*}
	A &=& \frac{1}{\sqrt{C}} - \frac{1}{2}	\\
	B &=& k^2\cos\theta + \sin^2\theta	\\
	C &=& 1 - \frac{(k^2-1)\cos^2\theta}{1+k^2}.\\
\end{eqnarray*}

%=================================
\picW = 10cm
\begin{figure}
	\centering
	\pic{dVeContour_RSP_fk.ps}
	\caption{Representation of $\theta_{tilt}(k)$ using the RSP potential and $k_S=k$.}
	\label{fig:dVeContour_RSP_fk}
\end{figure}
%=================================


Equation~\ref{eqn:dVeEq} has been solved numerically by computation of the contour of 
$\frac{d}{d\theta}Ve(k,\theta)$ at level $0$, as shown on Figure~\ref{fig:dVeContour_RSP_fk}. 
This shows that $\theta_{tilt}$ is fairly constant at about 0.9 radians, 
that is about $50$ degrees.
As a result, the above treatment suggest that configurations with a tilt angle of about $50$ 
degrees are expected to be favoured from simulations of full HGO particles confined 
with the RSP potential.
However, as many body effects  have not been considered here, the existence of 
such a tilt in a bulk system is not assumed by this result.\\

That said, the simulations presented in Section~\ref{ss:RSPresults} clearly show that such a 
tilt does develop when
using the RSP as a surface potential. Although, the tilt angle was not directly available from
observation of the profiles, it can be estimated by use of the definition of
$Q_{\alpha\beta}$. At the state point $\rho^{*} = 0.35$ and $k_S/k = 1.0$ and at the z location
where $\rho^{*}_\ell(z)$ is maximal, $Q_{zz} = 0.209$, this latter value corresponds to the 
the simulation average. This value of $Q_{zz}$ corresponds to an average tilt angle 
$\theta = 0.812$ radians, that is  $46.6^\circ$. This value is consistent 
with the angle observable on the configuration snapshots.\\
The difference between the observed and predicted tilt angles (of $9.7\%$) can be understood from the 
packing constraints. The packing improves with lower tilt angles which can, in turn, 
increase system's total absorbed volume. However the absorbed volume of each particle decreases as the
difference between its optimal and actual tilt angles increases. This creates a competition 
between the amount of absorbed volume that can be obtained with a higher packing fraction 
but lower average tilt angle and that obtained with an average tilt angle closer to the 
optimal single particle angle but lower packing fraction.\\

The simulations and the theoretical treatment described in this Section
have shown that a tilted phase can be both predicted and obtained with a purely steric model.
As a result, it appears that the tilted phases obtained 
in~\cite{ZhangChakrabarti96, WallCleaver97,TeixeiraChrzanowska01, LathamCleaver00} arise due to
the geometrical characteristics of the rod-sphere potential, rather than competition between
packing density and attractive particle-particle and particle-wall interactions. 
This explanation is consistent with the
change from tilted to planar surface alignment observed by Wall and 
Cleaver~\cite{WallCleaver97,WallCleaver03}
when they reduced the molecular elongation from $k=3$ to $k=2$. In the latter
case, the molecules were too short to significantly absorb at the surface and therefore adopt the
planar state. Observation of Figure~\ref{fig:Ve_RSP_fk} at $k=2$ confirms this, as for this
elongation, the absorbed volume is virtually independent of molecular orientation and, therefore,
does not form a tilted arrangement.\\
In the light of this explanation, it seems
reasonable to assume that a planar surface arrangement would have been obtained if the
simulations of~\cite{WallCleaver97,TeixeiraChrzanowska01, LathamCleaver00} had been performed using
a lattice of fixed spheres to represent the surface, as was done in~\cite{PalermoBiscarini98}.\\






\section{Realistic surface potentials~: the RSUP.}
\label{s:realisticPotentials_RSUP}



%===============================================================================================
%===============================================================================================
\subsection{The rod-surface potential.}

The rod-surface potential (RSUP) represents an alternative interaction between a Gaussian 
ellipsoid and a plane and is given by~:
\begin{equation}
	\mathcal{V}^{RSUP} = \left\{	%}
	\begin{array}{ccc}
		0	&\mathrm{if}	&|z_i - z_0| \geq \sigma^{RSUP}_w	\\
		\infty	&\mathrm{if}	&|z_i - z_0| < \sigma^{RSUP}_w	
	\end{array}
	\right.
\end{equation}
This time, the contact distance for this is obtained by integration of the rod-sphere potential 
(without the $\frac{\so}{2}$ shift) over the x-y plane leading to~\cite{rwThesis}~:
%
\begin{equation}
	\sigma^{RSUP}_w = \sigma_0\sqrt{\frac{1-\chi_S\sin^2\theta}{1-\chi_S}}
\end{equation}
%
with the same definition for $\chi_S$ as with the RSP potential. This potential can be thought 
of as being equivalent to the RSP but with the important difference that each particle
effectively interacts  with an infinity 
of spheres as opposed to just one. Again a shift is introduced so as to remove the virtual
spheres from the simulation box. The contact distance used in the simulation is therefore given by~:
\begin{equation}
	\sigma^{RSUP}_w = \sigma_0\lp \sqrt{\frac{1-\chi_S\sin^2\theta}{1-\chi_S}} -
	\frac{1}{2}\rp.
\end{equation}


%------------------------------------------
\picW = 10cm
\begin{figure}
	\centering
	\pic{RSUPConfig.ps}
	\caption[Representation of the geometry used for the interaction between the inner HGO
	particle and the surface in the RSUP potential.]
	{Representation of the geometry used for the interaction between the inner HGO
	particle and the surface in the RSUP potential. The three spheres represent the
	substrate which is really made of an infinity of such spheres located between the horizontal lines 
	which effectively mark the substrate location.}
	\label{fig:RSUPConfig}
\end{figure}

%------------------------------------------


%------------------------------------------
\picW = 10cm
\begin{figure}
	\centering
	\pic{sigmaRSUP.ps}
	\caption{Representation of $\sigma^{RSUP}(k_S,\theta)$ for $k=3$ using the RSUP
	potential.}
	\label{fig:sigma_RSUP}
\end{figure}


%------------------------------------------

A representation of $\sigma^{RSUP}_w(k_S,\theta)$ is given in Figure~\ref{fig:sigma_RSUP} for
$k=3$. Again, the expression for the absorbed volume into the surface can be used to predict the
surface behaviour of this model. In the case of the RSUP potential, this volume reads~:
%
\begin{equation}
	V_e^{RSUP}=\frac{1}{3}\pi\lp \frac{1}{2}-
	\sqrt{\frac{\sigma^{RSUP}_w}{k^2\cos^2\theta+\sin^2\theta}}\rp^2 
	\lp 1 + \sqrt{\frac{\sigma^{RSUP}_w}{k^2\cos^2\theta+\sin^2\theta}} \rp
	\label{eqn:Ve_RSUP}
\end{equation}
%
A graphical representation of this volume is shown on Figure~\ref{fig:Ve_RSUP_fks}. 
In the limit $k_S = 0$, $Ve^{RSUP}(k_S,\theta)$ has its maximum at $\theta = 0$ thus indicating 
an homeotropic arrangement.\\
%
In the limit $k_S = k$, $V_e^{RSUP}$ is close to zero for all $\theta$ and has a small maximum at 
$\theta = 0$. However, by design, $\sigma_w^{RSUP}$ forbids any particle adsorption into the
substrate if $k_S=k$; this is even further illustrated by the value of
$\sigma_w^{RSUP}(k_S=k,\theta=0)$ which is equal to the contact distance between a HGO particle
with $\theta=0$ and a sphere. As a result, this small maximum in $V_e^{RSUP}$ can be explained
to be a result of the approximation of using ellipsoidal shaped particles 
in Appendix~\ref{chap:A} when deriving
the expression for $V_e^{RSUP}$ and, therefore, $V_e^{RSUP}(k_S=k),\theta$ should, in fact,
be zero for all
values of $\theta$. A consequence of this, the $\theta=0$ peak in Figure~\ref{eqn:Ve_RSUP} 
does not represent a 
stable surface arrangement of the RSUP model in the case $k_S=k$, as this is not 
absorption driven. Rather, it can be safely assumed that the stable arrangement for this system 
is planar, in common with the findings of previous 
theoretical and simulation work on rod-shaped objects absorbed at planar 
surfaces~\cite{HolystPoniewierski88,Chrzanowska_Teixera_01,VanRoijDijkstra00,DijkstraVanRoij01}.


%------------------------------------------
\picW = 10cm
\begin{figure}
	\centering
	\pic{Ve_RSUP_fkS.ps}
	\caption{Representation of $Ve(k_s,\theta)$ for the RSUP potential and $k=3$.}
	\label{fig:Ve_RSUP_fks}
\end{figure}
%------------------------------------------

The mechanism expected to drive an anchoring transition with the RSUP potential is slightly
different from that seen with the HNW potential. With the latter, the surface rearrangement 
is mainly driven
by the molecular volume that can be absorbed into the surface, and the transition
from homeotropic to planar arrangements occurs when the volume absorbed by the
latter is greater than with the former arrangement. In the case of the RSUP potential, however, 
there is
no absorption in the case of the planar arrangement. However, this is the base state of
any rod-shaped object in contact with a surface. Thus, as $k_S$ is decreased from  $k_S=k$ to $k_S=0$, 
the volume that can be absorbed in a homeotropic arrangement gradually increases. In this case,
therefore, an anchoring transition from planar to homeotropic arrangement 
is expected when the volume that can be absorbed by an homeotropic surface induces a total free energy
lower than in that of the planar base state.


%===============================================================================================
%===============================================================================================
\subsection{Simulation results obtained using the rod-surface potential.}

The surface induced structural changes obtained from the rod-surface potential have been studied using
Monte Carlo simulations in the canonical ensemble on systems of $N=1000$ HGO particles with 
elongation $k=3$. The simulation slab was the same as that used previously, with the walls situated 
on the top and bottom of the cell with constant height $L_z = 4k\sigma_0$ and symmetric 
anchoring conditions. Two series of simulations at each chosen density were performed 
with,respectively, increasing and decreasing $k_S$.
Typical $z$-profiles for this model are shown on Figures~\ref{fig:RSUP_typeProf_k3_homeo} 
and~\ref{fig:RSUP_typeProf_k3_planar} respectively for $k_S=0.0$ and $k_S = k$.\\

%------------------------------
\picW = 12cm
\begin{figure}
	\centering
	\pic{RSUP_typeProf_k3_homeo.ps}
	\caption{Typical $z$-profiles for confined systems of $N=1000$ HGO particles with 
	$k=3.0$ and $k^{'}_S = 0.0$ using the RSUP potential.}
	\label{fig:RSUP_typeProf_k3_homeo}
\end{figure}
%------------------------------


%------------------------------
\picW = 12cm
\begin{figure}
	\centering
	\pic{RSUP_typeProf_k3_planar.ps}
	\caption{Typical $z$-profiles for confined systems of $N=1000$ HGO particles with 
	$k=3.0$ and $k^{'}_S = 1.0$ using the RSUP potential.}
	\label{fig:RSUP_typeProf_k3_planar}
\end{figure}
%------------------------------

%===================================================
\picW = 6cm
\begin{figure}
	\centering
	\subfigure[Simulations with decreasing $k_S$]{\pic{QzzWa_HGO_RSU_k3_fkS_S1_Su.ps}\pic{QzzWa_HGO_RSU_k3_fkS_S1_Bu.ps}}
	\subfigure[Simulations with increasing $k_S$]{\pic{QzzWa_HGO_RSU_k3_fkS_S3_Su.ps}\pic{QzzWa_HGO_RSU_k3_fkS_S3_Bu.ps}}
	\subfigure[Bistability diagrams]{\pic{QzzWa_HGO_RSU_k3_fkS_SuBist.ps}\pic{QzzWa_HGO_RSU_k3_fkS_BuBist.ps}}
	\caption{Anchoring phase diagrams obtained from series of simulations of $N=1000$
	confined HGO particles with $k=3$ at constant density and decreasing $k_S$ 
	using the RSUP surface potential. Diagrams on the l.h.s are relative to the interfacial
	region and those on the r.h.s are relative to the bulk region.}
	\label{fig:QzzWaPhaseDia_k3_RSUP}
\end{figure}
%===================================================



%===================================================
\picW = 7cm
\begin{figure}
	\centering
	\subfigure[Simulations with decreasing $k_S$]{\pic{SWa_HGO_RSU_k3_fkS_S1_Su.ps}\pic{SWa_HGO_RSU_k3_fkS_S1_Bu.ps}}
	\subfigure[Simulations with increasing $k_S$]{\pic{SWa_HGO_RSU_k3_fkS_S3_Su.ps}\pic{SWa_HGO_RSU_k3_fkS_S3_Bu.ps}}
	\caption{Order phase diagrams obtained from series of simulations of $N=1000$
	confined HGO particles with $k=3$ at constant density and decreasing $k_S$ 
	using the RSUP surface potential. Diagrams on the l.h.s are relative to the interfacial
	region and those on the r.h.s are relative to the bulk region. The bistability is
	negligible.}
	\label{fig:SWaPhaseDia_k3_RSUP}
\end{figure}
%===================================================



In the limit of $k_S=0$, the surface induced structural changes for this model are very
similar to their counterparts with the RSP model (Figures~\ref{fig:RSP_typeProf_k3_homeo} 
and~\ref{fig:RSP_typeProf_k3_planar}); the two sets of profiles
are virtually indistinguishable. The surface arrangement is homeotropic which explains the
very strong similarities between the two sets; with $\theta\sim0$, 
both models induce the same geometry between the particles and the substrate.\\
%
In the limit $k_S=k$, the surface induced effects for $\mathcal{V}^{RSP}$ and $\mathcal{V}^{RSUP}$
are very different.
The short peak separation in $\rho^{*}_\ell(z)$ and the negative values in $Q_{zz}(z)$, coupled with 
the high values of $P_2(z)$, indicate an induced planar surface arrangement very much in
agreement with the predictions made earlier. This is further confirmed by the
similarity of the planar arrangement profile features for the HNW and RSUP models. The main difference
between the two arises because with the RSUP model, planar particles are not allowed to 
absorb at the surface. 
This leads to the regions of zero $\rho^{*}_\ell(z)$ with a width of $0.5\sigma_0$ close to 
each substrate.\\

The full surface induced behaviour of this system has been computed as a function of $k_S$ and 
$\theta$ using the anchoring and order phase diagrams shown in 
Figures~\ref{fig:QzzWaPhaseDia_k3_RSUP} and~\ref{fig:SWaPhaseDia_k3_RSUP}. 
The convention adopted to distinguish the interfacial 
from the bulk region was the same as that used with the RSP model.
%
The anchoring phase diagrams are given in Figure~\ref{fig:QzzWaPhaseDia_k3_RSUP}(a) 
and (b) for, respectively, decreasing and increasing $k_S$. From those
a strong difference between the two sets can be observed. The corresponding bistability 
phase diagrams (Figure~\ref{fig:QzzWaPhaseDia_k3_RSUP}c)  reports a very wide and strong 
bistability behaviour for this surface potential.  The region of bistability is much greater
here than that obtained using the HNW potential, extending over a wider range of density 
and $k_S$. Also for a given state point, this potential induces larger bistability values. 
This makes the RSUP model a very good candidate for the modeling of switching between the 
two arrangements on a bistable surface.\\
%
The improved bistability of the RSUP model when compared with the HNW
model lies in the difference between the mechanisms driving the surface-induced anchoring. For the
HNW, the competition between the planar and homeotropic alignment is driven by the
amount of volume that can be absorbed into the surface for each alignment. 
This is slightly different from the RSUP case where the planar alignment is the natural 
state of the system and does not rely on the particles absorbing the surface; homeotropic 
alignment is introduced as an alternative to this natural states by increasing the possibility 
of absorption when reducing $k_S$. 
As a result the free energy minima corresponding to the two alignments for the two potentials 
are subtly different. Although free energy data were not determined in this study, 
the stronger bistability obtained for the RSUP model suggest it has a higher 
and wider energy barrier between the two locally stable  
alignment states.\\


The order phase diagrams for the RSUP model (Fig~\ref{fig:SWaPhaseDia_k3_RSUP}) show the 
same general features as
the corresponding diagrams calculated for the HNW model. In the bulk region, the two data
sets are very similar. However, the diagrams for the interfacial region present some differences
in that the high symmetry around the transition line and the strong disordering of
the particles at the transition are diminished somewhat. This can be attributed to the
difference in the $\rho^{*}_\ell$ profiles for the two potentials, the profiles for the 
RSUP potential lacking the disorder-related
double peak behaviour. As a result, in the case of competing alignment, the local surface
order was not reduced due to particles diffusing between the two regions corresponding to the two
density peaks.\\





\section{Hybrid systems}


The study of hybrid anchored systems of HGO particles presented in Chapter~\ref{chap:five} 
showed that the presence of a top surface with strong homeotropic anchoring can cause a 
bottom surface with competing anchoring to lose the bistability of its homeotropic and planar surface
arrangements. The bistability behaviour can, however, be recovered by two complementary means. The
first of these is to increase the slab height, which has the effect of
creating a smoother transition between the homeotropic and planar arrangements. The second
solution is to reduce the anchoring strength at the top surface in order to reduce the elastic
forces this imposes on the particles anchored at the bottom surface.\\

\picW = 13.5cm
\begin{figure}
	\centering
	\pic{GBP_RSUP_QzzWa_hyb_d0.15_3g.ps}
	\caption{Comparison between $\overline{Q}^{Sb}_{zz}(k^\prime_{Sb})$ (dashed lines) from
	simulations of hybrid anchored systems of PHGO particles and
	$\overline{Q}^{Su}_{zz}(k^\prime_{S})$ from simulations of symmetric systems described
	in
	Section~\ref{s:GBP_RSUP_symmetric}(solid lines). The arrows indicate whether the
	simulations have been performed with increasing ($\bigtriangleup$) or 
	decreasing ($\bigtriangledown$) values of $k^\prime_S$}
	\label{fig:GBP_RSUP_hybQzzWaSb}
\end{figure}

Here, therefore, we
study the effect of the top surface anchoring parameterisation on the
bistability behaviour of the bottom surface in an hybrid anchored system of pear shaped
particles interacting with the surfaces through the $PSU$ potential. For this, systems of
$N=1000$ PHGO particles of density $\rho^{*}=0.15$ with $k=5$ confined in an hybrid anchored slab 
have been studied using Monte Carlo simulations in
the canonical ensemble. Series of increasing and decreasing $k^\prime_{Sb}$ in the range
$[0.56:0.74]$ have been used; this range corresponding to the bistability region of $k^\prime_S$
for symmetric systems.
Three values for the top anchoring strength were considered~: 
$k^\prime_{St} = 0.4$, $0.5$ and $0.6$. The transition between the homeotropic and
planar arrangements at the bottom surface was studied through the computation of $\overline{Q}^{Sb}_{zz}$
as a function of $k^\prime_{Sb}$.
A comparison between the $\overline{Q}^{Sb}_{zz}(k^\prime_{Sb})$ data obtained for these hybrid 
systems and the data obtained from equivalent symmetric systems in
Section~\ref{s:GBP_RSUP_symmetric} is shown on Figure~\ref{fig:GBP_RSUP_hybQzzWaSb}.
Configuration snapshots from those simulations with $k^\prime_{Sb} = 0.64$ and $0.74$ are shown
on Figure~\ref{fig:GBP_RSUP_hyb_typSnaps}. We note that, as in the case of the equivalent HGO
systems, the HAN state found for this film thickness suggest discontinuous director profiles.\\

\picW = 4.5cm
\begin{figure}
	\centering
	\subfigure[$k^\prime_{St}=0.4$, $k^\prime_{Sb}=0.64$]{\pic{GBP_box_hyb_kSt40_kSb064_d0.15.ps}}
	\subfigure[$k^\prime_{St}=0.4$, $k^\prime_{Sb}=0.74$]{\pic{GBP_box_hyb_kSt40_kSb074_d0.15.ps}}
	
	\subfigure[$k^\prime_{St}=0.5$, $k^\prime_{Sb}=0.64$]{\pic{GBP_box_hyb_kSt50_kSb064_d0.15.ps}}
	\subfigure[$k^\prime_{St}=0.5$, $k^\prime_{Sb}=0.74$]{\pic{GBP_box_hyb_kSt50_kSb074_d0.15.ps}}
	
	\subfigure[$k^\prime_{St}=0.6$, $k^\prime_{Sb}=0.64$]{\pic{GBP_box_hyb_kSt60_kSb064_d0.15.ps}}
	\subfigure[$k^\prime_{St}=0.6$, $k^\prime_{Sb}=0.74$]{\pic{GBP_box_hyb_kSt60_kSb074_d0.15.ps}}
	\caption{Typical configuration snapshots obtained from simulations of confined systems of
	$N=1000$ PHGO particles with $k=5$ at $\rho^{*}=0.15$, hybrid anchoring and different 
	values of $k^\prime_{St}$ and $k^\prime_{Sb}$.}
	\label{fig:GBP_RSUP_hyb_typSnaps}
\end{figure}


\picW = 4.8cm
\begin{figure}
	\centering
	\subfigure[increasing $k^\prime_{Sb}$]{\pic{GBP_box_hyb_kSt60_kSb070_d0.15_S3.0.ps}}
	\subfigure[decreasing $k^\prime_{Sb}$]{\pic{GBP_box_hyb_kSt60_kSb070_d0.15_S1.0.ps}}
	\caption{Configuration snapshots showing the HAN and V states of a hybrid anchored slab
	of $N=1000$ PHGO particles with $k=5$ at $\rho^{*} = 0.15$ with $k^\prime_{St} = 0.6$ 
	and $k^\prime_{Sb} = 0.7$. Those configuration have been obtained from series of 
	simulations with 
	increasing (a) and decreasing (b) values of $k^\prime_{Sb}$}
	\label{fig:GBP_RSUP_hyb_kSt060_kSt070_snaps}
\end{figure}


For the two lower values of $k^\prime_{St}$ (representing stronger homeotropic anchoring), 
although the two 
series of simulations lead to hysteresis in the values of $\overline{Q}^{Sb}_{zz}(k^\prime_{Sb})$, no
bistability can be observed. None of the state points considered here correspond to a situation 
where the values of $\overline{Q}^{Sb}_{zz}$ obtained from the two series are both significantly
different from zero and of opposite signs.\\
%
With $k^\prime_{St} = 0.6$, however, a small bistable region  is recovered around
$k^\prime_{Sb}= 0.7$; 
the values of $\overline{Q}^{Sb}_{zz}$ are different and of opposite signs. The bistability value for 
$k^\prime_{Sb} = 0.7$ is $0.914$ which is very close to that obtained with symmetric anchored 
systems. Configuration snapshots corresponding to the HAN and V states
of the cell at this state point are given on Figure~\ref{fig:GBP_RSUP_hyb_kSt060_kSt070_snaps}.\\
%
These results show that reducing the strength of the anchoring at the top surface allows to
recover the bistability region by increasing the hysteresis in $\overline{Q}^{Sb}_{zz}$.
The value $k^\prime_{St} = 0.6$ seems to be the highest reasonable that can be used,
as according to $\overline{Q}^{Su}_{zz}(k^\prime_S)$, the use of a higher value would not lead to homeotropic
anchoring at the top surface.\\


The results from these simulations are reasonably encouraging for the application of the model to
the HAN to V switching since, despite the very narrow bistability region for $k^\prime_{Sb}$, 
the difference
between the $\overline{Q}^{Sb}_{zz}$ values obtained from the two series with $k^\prime_{St}=0.6$ 
appears sufficient for the model to be used in the display modeling. Also, the snapshots show 
encouraging  HAN and V states which should be further improved by the use of wider systems.











%===============================================================================
%===============================================================================
%\section{Conclusions}
%\begin{itemize}
%	\item structural transition in hybrid systems
%	\item simple theory on surface arrangements
%	\item found an entropy based explanation for tilted phases (RSP)
%	\item a fairly realistic contact distance that gives H and P phase (RSUP
%\end{itemize}


\conclusion

In this Chapter, two issues have been addressed. In the first part of the Chapter, two
more realistic surface potentials have been studied, namely the rod-sphere and the rod-surface
potentials. The aim of this work was to find a potential which has a more realistic basis
than the hard needle wall potential but which also displays planar and homeotropic surface
arrangements. A region of bistability between the two arrangement was also required for future
applications relating the modeling of display cells.\\
The rod-sphere potential was found to be unsuitable as the planar arrangement was replaced by a
tilted structure. However, the results obtained using this model proved to be interesting
since they showed that a tilted phase can be obtained from purely steric interactions. The
rod-surface potential, meanwhile, proved a better candidate for the aim stated above, as it
not only recovered the surface behaviour of the HNW potential but actually displayed stronger
and wider bistability regions.\\
In the second part of this Chapter, hybrid anchored systems of HGO particles confined between a
homeotropic top surface and a bottom surface with competing anchoring have been studied using
the HNW potential for the surface interactions. These simulations showed that the
bistability behaviour of the model can be lost if high anchoring strength is used at the top
surface or if the slab is too narrow. Using moderate homeotropic anchoring at the top surfaces
and systems sizes of $N=2000$ particles, however, bistability was established and a cell
was successfully switched between the HAN and V states if only the easy switching was
considered. Achievement of the reverse switching requires the use of flexoelectric particles and
a surface potential allowing bistability between planar and homeotropic arrangements for such
particles. Those two problems are addressed, respectively, in Chapters~\ref{chap:six}
and~\ref{chap:seven}.






