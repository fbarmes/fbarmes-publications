

\section{Hybrid anchored systems.}
\label{s:hybridSystems}


In this Section the study of hybrid anchored confined systems is addressed using particles
confined in a slab geometry but with different anchoring conditions at each of the two surfaces.
This study is performed using Monte Carlo simulations of hard Gaussian overlap particles
confined in a slab geometry and interacting with the surfaces through the hard needle wall
potential. The aim here is to achieve switching between the Hybrid Aligned Nematic (HAN) and
Vertical (V) states using an electrical field as described in~\cite{DavidsonMottram02}. The
difference between the switching investigated here and that in
Reference~\cite{DavidsonMottram02} is that due the absence of flexoelectricity
in the HGO model, two-way switching is attempted by changing the sign of the particles'
dielectric anisotropy (as was done in Chapter~\ref{chap:four}) rather than the sign of the
applied field.

%===================================================================
%===================================================================
\subsection{Effect of hybrid anchoring.}
\label{ss:hybridEffect}

Here the effects of hybrid anchoring on a confined system are studied~; more
specifically the case of different arrangements (\ie planar and homeotropic at two substrates)
is of interest.

Cleaver and Teixeira~\cite{Cleaver_Teixeira_01} have already studied the structural transition
between the arrangements seeded at the surfaces of an hybrid anchored cell of hard Gaussian
overlap particles with $k=5$. They showed that the cell can exhibit either a continuous or
discontinuous transition between the homeotropic and planar arrangements according to the values
of the surface parameters. This implies that the observation of an HAN state requires an
appropriate choice of surface anchoring strengths so as to avoid any director discontinuities.
Also, achieving electric-field induced switching requires that the substrate parameters used
are compatible with those corresponding to a bistable surface.\\
%
Here, the case of hybrid anchored slabs has been investigated using Monte Carlo simulations
of systems of $N=1000$ hard Gaussian overlap particles of elongation $k=3$ and $5$
confined in a slab geometry and interacting with  the surfaces  using the hard needle
wall potential. The anchoring at the top was kept constant at $k^\prime_S=0.0$ so as to
induce strong homeotropic anchoring. The anchoring at the bottom surface was allowed to
vary using sequences of simulations with increasing and decreasing $k^\prime_S$ in the range
$[0:1]$.\\


%===========================================
%	TYPICAL PROFILES FIGS
%===========================================
\picW = 14cm
\begin{figure}
	\centering
	\pic{HGO_HNW_Hconf_typeProf_k3_homeo.ps}
	\caption{Typical profiles for a hybrid anchored slab of $N=1000$ HGO particles using the
	HNW surface potential with an homeotropic top surface
	($k^\prime_{St}=0.0$) and an homeotropic bottom surface ($k^\prime_{St}=0.2$).}
	\label{fig:HGO_Hconf_typeProfHomeo}
\end{figure}


\picW = 14cm
\begin{figure}
	\centering
	\pic{HGO_HNW_Hconf_typeProf_k3_bist.ps}
	\caption{Typical profiles for a hybrid anchored slab of $N=1000$ HGO particles using the
	HNW surface potential with an homeotropic top surface ($k^\prime_{St}=0.0$)
	and competing anchoring at the bottom surface ($k^\prime_{St}=0.5$).}
	\label{fig:HGO_Hconf_typeProfBist}
\end{figure}


\picW = 14cm
\begin{figure}
	\centering
	\pic{HGO_HNW_Hconf_typeProf_k3_planar.ps}
	\caption{Typical profiles for a hybrid anchored slab of $N=1000$ HGO particles using the
	HNW surface potential with an homeotropic top surface ($k^\prime_{St}=0.0$) and a
	planar bottom surface ($k^\prime_{St}=0.8$).}
	\label{fig:HGO_Hconf_typeProfPlanar}
\end{figure}
%===========================================

Typical profiles for systems with parameterisations at the bottom surface
corresponding, respectively, to homeotropic ($k^\prime_{Sb} = 0.2$), competing ($k^\prime_{Sb} =
0.5$) and planar anchoring ($k^\prime_{Sb} = 0.8$) are shown on
Figures~\ref{fig:HGO_Hconf_typeProfHomeo} to~\ref{fig:HGO_Hconf_typeProfPlanar}. These profiles
were obtained from the simulation sequences performed with decreasing $k^\prime_{Sb}$. Only
these results are shown because even in the case of the competing anchoring parameterisation,
both series gave very similar results. Also, the profiles or configuration snapshots obtained
from simulations with particles of elongation $k=5$ are not shown as they are very similar to
those obtained with $k=3$.\\
For all profiles, the top interfacial regions exhibit the features
typical of strong homeotropic anchoring as $k_{St}$ was kept constant at $0.0$. The bottom
surface profiles exhibit features corresponding to the values of the needle lengths used~; they
have the same characteristics as were observed in the equivalent cases with symmetric
surfaces.\\

The structural transition between the two surface arrangements, that is the change in molecular
orientation from one surface to the other can also be observed on the profiles.
At isotropic densities, the surface effects do not extend into the bulk part of the
slab and, therefore, this region remains disordered. As a result, both interfacial regions are
free of any influence from each other. As the number density is increased to values
corresponding to a bulk nematic phase, the surface induced structural changes extend much
further into the cell. As a result, the bulk region comes under the competing influences of both
surfaces. In all three cases considered here, the profiles seem to indicate a smooth transition
between the two surface arrangements, as indicated by the almost linear changes in
$\rho^{*}_\ell(z)$ and $Q_{zz}(z)$ between the surface features.\\

%The results obtained with a planar bottom surface are in agreement with the observation
%of a continuous transition in~\cite{Cleaver_Teixeira_01} for a similarly anchored slab. However
%the authors have noticed a discontinuous transition from planar to homeotropic for a situation
%which, in the light of the results obtained in Chapter~\ref{chap:four} corresponds
%to a slab with planar anchoring on one side and competing anchoring on the other. There, the
%discontinuous transition was indicated by the fluctuations in $Q_{zz}$ between positive and
%negative values. However those fluctuations at the surface with competing anchoring have not
%been observed here.\\




%===========================================
%	TYPICAL SNAPS FIGS
%===========================================
\picW = 6cm
\begin{figure}
	\centering
	\subfigure[$k^\prime_{Sb}=0.2$, $\rho^{*}=0.28$]{\pic{HGO_box_k3_Hconf_homeo_iso.ps}}
	\subfigure[$k^\prime_{Sb}=0.2$, $\rho^{*}=0.34$]{\pic{HGO_box_k3_Hconf_homeo_nem.ps}}

	\subfigure[$k^\prime_{Sb}=0.5$, $\rho^{*}=0.28$]{\pic{HGO_box_k3_Hconf_bist_iso.ps}}
	\subfigure[$k^\prime_{Sb}=0.5$, $\rho^{*}=0.34$]{\pic{HGO_box_k3_Hconf_bist_nem.ps}}

	\subfigure[$k^\prime_{Sb}=0.8$, $\rho^{*}=0.28$]{\pic{HGO_box_k3_Hconf_planar_iso.ps}}
	\subfigure[$k^\prime_{Sb}=0.8$, $\rho^{*}=0.34$]{\pic{HGO_box_k3_Hconf_planar_nem.ps}}
	\caption{Configuration snapshots for hybrid anchored slabs with a strong
	homeotropic anchoring at the top surface ($k^\prime_{St}=0.0$) and different values for
	$k^\prime_{Sb}$. Snapshots for two densities corresponding to isotropic (left) and
	nematic (right) are shown.}
	\label{fig:typeSnap_HGO_Hconf_k3}
\end{figure}

The structural transition between the surface arrangements can also be observed using
configuration snapshots (\eg Figure~\ref{fig:typeSnap_HGO_Hconf_k3}). Specifically the case of a
slab with planar and homeotropic anchoring is of interest as this corresponds to the geometry
where the electric switching is to be performed. For this situation, the snapshots suggest that a
slight discontinuity in the planar to homeotropic structural transition can be observed~; the
molecular orientation changes rapidly from that corresponding to homeotropic anchoring to that
of planar anchoring. This is confirmed by the $<P_2>$ profile
(Figure~\ref{fig:HGO_Hconf_typeProfPlanar}) which shows a low value in the bulk part of the cell
whereas the snapshots clearly indicate good order throughout the cell. These low values can be
understood by the presence of particles with very different orientations in the same slice which
in turn lowers the value of $<P_2>$. This behaviour is not apparent on $Q_{zz}(z)$ as similar
values could be obtained from a slice of $n$ particles with $\theta\sim\pi/4$ and a slice of
equal number of particles with $\theta\sim 0$ and $\theta\sim\pi/2$.\\
In the case of a bottom surface with homeotropic or competing anchoring, the profiles and
snapshots agree in indicating a continuous structural transition between the two surface induced
arrangements. These observation are consistent with the simulation of Cleaver and
Teixeira~\cite{Cleaver_Teixeira_01} who found a discontinuous structural transition between the
two surface arrangement provided the anchoring conditions of the two surfaces are made
sufficiently different.\\

%===========================================
%	ANCHORING PHASE DIA
%===========================================
\picW = 7cm
\begin{figure}
	\centering
	\subfigure[simulations with decreasing $k^\prime_{Sb}$]{\pic{QzzWa_HGO_HNW_Hconf_k3_Sb_S1.0.ps}}
	\subfigure[simulations with increasing $k^\prime_{Sb}$]{\pic{QzzWa_HGO_HNW_Hconf_k3_Sb_S3.0.ps}}
	\caption{Anchoring phase diagram  showing the evolution of~$\overline{Q}^{Sb}_{zz}$
	of a hybrid anchored slab with $k^\prime_{St}=0.0$ as a function
	of number density  $\rho^{*}$ and $k^\prime_{Sb}$. Systems of $N=1000$ HGO particles
	of elongation $k=3$
	and the HNW surface potential have been used. }
	\label{fig:anchPhaseDia_HGO_Hconf_k3}
\end{figure}


%\picW = 7cm
%\begin{figure}
%	\centering
%	\subfigure[simulations with decreasing $k^\prime_{Sb}$]
%	        {\pic{QzzWa_HGO_k5_HNW_Hconf_Sb_S1.0.ps}}
%	\subfigure[simulations with increasing $k^\prime_{Sb}$]
%	        {\pic{QzzWa_HGO_k5_HNW_Hconf_Sb_S3.0.ps}}
%	\caption{Anchoring phase diagram  showing the evolution of~$\overline{Q}_{zz}^{Sb}$
%	of a hybrid anchored slab
%	with $k^\prime_{St}=0.0$ as a function of number density
%	$\rho^{*}$ and $k^\prime_{Sb}$. Systems of $N=1000$ HGO particles of elongation $k=5$
%	and the HNW surface potential have been used. }
%	\label{fig:anchPhaseDia_HGO_Hconf_k5}
%\end{figure}
%===========================================

The combined effects of density and needle length are shown on the anchoring phase diagrams
computed for the bottom surfaces ($\overline{Q}^{Sb}_{zz}$). These were computed using the
approach adopted with symmetric systems and are shown in
Figure~\ref{fig:anchPhaseDia_HGO_Hconf_k3}.\\
%
The behaviour of $\overline{Q}_{zz}^{Sb}$ for these systems is very similar to that of
$\overline{Q}_{zz}^{Su}$ for the symmetric systems and the same remarks apply. However, one
striking difference is that, in the case of hybrid systems, the diagrams from series with
increasing and decreasing $k^\prime_{Sb}$ are very similar. This means that the hysteresis
used to establish bistable regions is not seen in those systems.\\
This change can be ascribed to the combined effects of the presence of the top surface with
strong homeotropic anchoring and the small height of the slab. As a result of these, the elastic
forces imposed on the particles at the bottom surface by those on the top surface prevent the
former from adopting a planar orientation for parameterisations corresponding to weak
competing anchoring. The consequence of this is that a planar orientation is only observed at
the bottom surface if the corresponding anchoring is strong. But this removes the
possibility of bistability.\\

In order to recover the bistable regions, it is necessary to reduce the elastic forces imposed
at the bottom surface by the homeotropic anchoring at the top surface. There are two approaches
by which to achieve this: to use a weaker homeotropic anchoring at the top surface; or to
increase the height of the slab, (and, therefore, the number of particles in the simulation
box). In the next section, the second solution is used in an attempt to regain surface
bistability.


%===================================================================
%===================================================================
\subsection{System size effect.}
\label{ss:sizeEffect}

Here the influence of the height of the slab on the surface bistability in hybrid system is
investigated. This has been performed by considering three slabs of hard Gaussian overlap
particles with respective height $L_z=4k\so$, $L_z=6k\so$ and $L_z=8k\so$ respectively . In
order to keep the width of the slabs big enough so as to avoid interactions between
particles and their own images, increase in the slab height was accompanied with an
increase in the system sizes, and the height $L_z=4k\so$, $L_z=6k\so$ and $L_z=8k\so$
correspond respectively $N=1000$, $N=1250$ and $N=2000$. Although the
cross section surface of the slabs was not equal for the three systems, the short positional
correlation of the systems used should imply that the slabs were wide enough so that only the
slab height has an effect on the observed planar to homeotropic surface transition.\\
These systems were studied using Monte Carlo simulations in the canonical ensemble and
using the hard needle wall potential for surface interactions. Extreme hybrid anchoring
conditions were considered using $k^\prime_{Sb}=1.0$ and $k^\prime_{St}=0.0$. Typical
profiles at $\rho^{*}=0.35$ are shown on Figure~\ref{fig:HGO_Hconf_typeProfile_Size}. Here, for
comparison purposes, the $z$ coordinates have been renormalized by $L_z$.\\


%=============================================
\picW = 14cm
\begin{figure}
	\centering
	\pic{systSizeEffectsProfiles.ps}
	\caption{Profiles corresponding to hybrid anchored systems of hard Gaussian overlap
	particles with $k=3$ at $\rho^{*}=0.35$ and different slab heigh and system sizes. 
	The surface potential is the HNW with parameterisation $k^\prime_{St}=0.0$ and 
	$k^\prime_{Sb}=1.0$.}
	\label{fig:HGO_Hconf_typeProfile_Size}
\end{figure}
%=============================================

\begin{figure}
	\centering
	\picW = 6cm
	\subfigure[$N=1000$, $L_z=4k\so$]{\pic{HGO_box_k3_Hconf_Size_N1000.ps}}
	\picW = 7cm
	\subfigure[$N=1250$, $L_z=6k\so$]{\pic{HGO_box_k3_Hconf_Size_N1250.ps}}

	\picW = 9cm
	\subfigure[$N=2000$, $L_z=8k\so$]{\pic{HGO_box_k3_Hconf_Size_N2000.ps}}
	\caption{Typical snapshots for hybrid anchored systems of $N=1000$(a), $1250$(b) and
	$2000$(c) hard Gaussian overlap particles with $k=3$ using the HNW surface potential
	with $k^\prime_{St}=0.0$ and $k^\prime_{Sb}=1.0$.}
	\label{fig:HGO_Hconf_snapsSize}
\end{figure}
%==============================================


On those profiles, similar interfacial behaviour can be observed for all three systems, the
profiles displaying features typical of homeotropic anchoring on the top surface and of planar
anchoring on the bottom surface.\\
%
The $\rho^{*}_\ell(z/L_z)$ and $Q_{zz}(z/L_z)$ profiles also show similar behaviour for all
three systems in the bulk region. All three show a linear increase in $Q_{zz}$ from
the bottom to the top surface. The main difference between these is that the regions of linear
behaviour extend over larger portions of the cell with increase in $L_z$, thus
indicating bigger `buffer regions' between the two surfaces.\\

More important differences can, however, be observed on the $P_2(z/L_z)$ profiles. For the
smallest system ($N=1000$), the $P_2(z/L_z)$ profiles show low values in the bulk regions
identified previously. Again, this effect is ascribed to the presence of particles with
significantly different orientations in the same analysis slice (Section~\ref{ss:hybridEffect}.)
On the other hand, the two bigger systems show very different behaviour, the corresponding
$P_2(L_z)$ profiles maintaining high values throughout the cell.\\
%
Further insight into this can be obtained from the corresponding configuration snapshots (\eg
Figure~\ref{fig:HGO_Hconf_snapsSize}). From these, the smallest system clearly shows a
discontinuity in the structural transition from planar to homeotropic as can be observed by the
rapid change from parallel to perpendicular orientations. The two bigger systems show a
different behaviour, however, the transition between the two arrangements being continuous and
smooth.\\
The modest differences between the structures and profiles for the bigger systems with
$L_z=6k\so$ and $L_z=8k\so$ suggest that there is a critical height at which the transition
between the two arrangements becomes continuous. The amount of data obtained here only allows to
conclude that this %[
critical slab height is in $]4k\so : 6k\so[$.\\ %]

The results found in this section proved to be fully compatible with the theoretical results of
\v{S}arlah and \v{Z}ummer~\cite{SarlahZummer99} who found that hybrid anchored films with a
thickness of only a few molecular lengths do not show a continuous bent-director structure.
Also, the experimental observation of Vanderbrouck~\etal\cite{VandenbrouckValignat99} confirm
the observation made in this Section as they observed that a thin film of 5CB molecules spun cast
onto silicon wafer, and thus having planar hybrid anchoring condition at respectively the solid
and free surfaces, are stable only if their thickness is greater than 20nm.

%===================================================================
%===================================================================
%\clearpage
\subsection{HAN to V states switching.}

The anchoring phase diagrams presented in Section~\ref{ss:hybridEffect} have shown no
bistability, possibly  due to the combination of too strong a homeotropic anchoring at
the top surface and the small height used. The smooth structural transitions obtained using
larger slab height in Section~\ref{ss:sizeEffect} suggest, however, that bistable behaviour may
be achievable.\\ %
As a result switching between the HAN and V states has been attempted. In order to recover the
bistability at the bottom surface, a slab of height $L_z=8k\so$ and a system size of $N=2000$
particles have been used and the bottom surface needle length has been set to $k_{Sb}/k=0.5$
which correspond to a good bistability in an equivalent symmetric anchored system. The first
simulations performed used a top surface needle length $k_{St}/k = 0.0$, but that proved to
induce too strong an homeotropic anchoring and no bistability at the bottom surface could be
observed. By gradually reducing the top surface anchoring strength, the bistability at the
bottom surface could be regained using $k^\prime_{St}=0.4$.
Using this latter value of $k_{St}$, the switching between the HAN and vertical states has been
performed using a similar sequence of simulations to that employed in Chapter~\ref{chap:four}.
The evolution of $\overline{Q}_{zz}^{St}$ and $\overline{Q}_{zz}^{Sb}$ as a function of the
number of sweeps are shown, respectively, on Figure~\ref{fig:QzzWaEvol_HGOHconfSwitch}(a) and
(b). Configuration snapshots corresponding to the last configuration of each phase
are shown on Figure~\ref{fig:snaps_HGOHconfSwitch}.\\

\picW = 10cm
\begin{figure}
	\centering
	\subfigure[$\overline{Q}_{zz}^{St}(n)$]{\picL{HGO_Hconf_N2000_Eswitch_QzzWaSt.ps}}
	\subfigure[$\overline{Q}_{zz}^{Sb}(n)$]{\picL{HGO_Hconf_N2000_Eswitch_QzzWaSb.ps}}
	\caption{Evolution of $\overline{Q}_{zz}(n)$ as a function of the number of sweeps $n$ for the
	top(a) and bottom(b) surface regions while switching a hybrid anchored system of
	$N=2000$ HGO particles with $k=3$ between the HAN and V states.}
	\label{fig:QzzWaEvol_HGOHconfSwitch}
\end{figure}

\picW = 4.7cm
\begin{figure}
	\centering
	\subfigure[$0.50.10^6$ sweeps]{\pic{HGO_box_k3_Hconf_N2000_Eswitch_01.ps}}
	\subfigure[$0.75.10^6$ sweeps]{\pic{HGO_box_k3_Hconf_N2000_Eswitch_02.ps}}
	\subfigure[$2.25.10^6$ sweeps]{\pic{HGO_box_k3_Hconf_N2000_Eswitch_03.ps}}
	\subfigure[$2.50.10^6$ sweeps]{\pic{HGO_box_k3_Hconf_N2000_Eswitch_04.ps}}
	\subfigure[$4.50.10^6$ sweeps]{\pic{HGO_box_k3_Hconf_N2000_Eswitch_05.ps}}
	\caption{Configuration snapshots of a system of $N=2000$ HGO particles with $k=3$ at
	different stages of switching between the HAN and V states.}
	\label{fig:snaps_HGOHconfSwitch}
\end{figure}

The sequence was started using previously equilibrated configuration with a HAN alignment. After
$0.5.10^6$ sweeps, an electric field $\vect{E}=E\vecth{z}$ with $E=6.0$ was applied  during
$0.25.10^6$ sweeps using $\delta\epsilon>0$ so as to align the particles along $\vecth{z}$. Upon
removal of the field, equilibrium in the vertical state was well established after $1.5.10^6$
sweeps. The field was then reapplied for $0.25.10^6$ sweeps using $\delta\epsilon<0$ so as to
align the particles perpendicular to $\vecth{z}$. Upon removal of the field, equilibrium in the
HAN state was achieved after $2.0.10^6$ sweeps.\\

This sequence shows successful switching between the HAN and vertical states of an hybrid
aligned cell corresponding to that considered in~\cite{DavidsonMottram02}. The model used here
did not include flexoelectricity and, therefore, only the easy switching namely, HAN to V if
$\delta\epsilon>0$ and V to HAN if $\delta\epsilon<0$ could be modeled. This is however very
encouraging as achievement of the easy switching implies that the reverse (`hard') switching
could, in principle, be achieved easily with an appropriate electrical parameterisation of the
model.









