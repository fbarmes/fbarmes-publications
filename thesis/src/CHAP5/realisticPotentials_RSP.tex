
\section{Realistic surface potentials : the RSP.}
\label{s:realisticPotentials_RSP}


%===============================================================================================
%===============================================================================================
\subsection{The rod-sphere potential.}
\label{ss:RSP_potential}


%====================================================================
%		POTENTIAL DESCRIPTION AND LITTERATURE REVIEW
%====================================================================

The rod-sphere surface potential (RSP) describes the interaction between a Gaussian ellipsoid and a
sphere located in the surface plane and with the same $x$ and $y$ coordinates as the ellipsoid. 
Again, the particles (HGO) do not interact directly with the substrate, rather another
HGO ellipsoid is inserted in each particle (\eg Figure~\ref{fig:RSPConfig}). 
This inner ellipsoid interacts with the surface through $\mathcal{V}^{RSP}$ as~:
%
\begin{equation}
	\mathcal{V}^{RSP} = \left\{	%}
	\begin{array}{ccc}
		0	&\mathrm{if}	&|z_i - z_0| \geq \sigma^{RSP}_w	\\
		\infty	&\mathrm{if}	&|z_i - z_0| < \sigma^{RSP}_w	
	\end{array}
	\right.
\end{equation}
%
where $\sigma^{RSP}_w$ is the contact distance for the interaction between a hard Gaussian 
overlap particle  with length $\sigma_\parallel$ and breadth $\sigma_\perp$ and a sphere of diameter 
$\sigma_j$. The contact distance for such an interaction is given by Equation 
(4) of~\cite{BernePechukas72}~:
%
\begin{equation}
	\sigma^{RSP}_w = \sqrt{
	\frac{\sigma^2_\perp + \sigma^2_j}
	{1 - \chi(\dotproduct{u_i}{r_{ij}})^2 }
	}
	\label{eqn:sw_RSP_BP}
\end{equation}

For implementation into a simulation code, Equation~\ref{eqn:sw_RSP_BP} is best written
in terms of $\theta$ and $\sigma_0$~; recalling that the unit of distance 
$\so = \sigma_\perp\sqrt{2}$ and for convenience, imposing $\so = \sigma_j\sqrt{2}$. 
Also to enable
comparison with the hard needle wall potential, the sphere is taken to be tangent with the
substrate so as to keep it out of the simulation box. This leads to the final expression for
$\sigma^{RSP}_w$ as used in the simulations~:
%
\begin{equation}
	\sigma^{RSP}_w = \sigma_0\lp \frac{1}{\sqrt{1 - \chi_S\cos^2\theta}} - \frac{1}{2} \rp
	\label{eqn:sigma_w_RSP}
\end{equation}
%
with~:
%
\begin{equation}
	\chi_S = \frac{k^2_S - 1}{k^2_S + 1}
\end{equation}
%
$k_S$ being the length to breadth ratio of the inner ellipsoid. A graphical representation of
this contact distance  as a function of $k_S$ and $\theta$ is shown on
Figure~\ref{fig:sw_RSP}.\\

%------------------------------------------
\picW = 10cm
\begin{figure}
	\centering
	\pic{RSPConfig.ps}
	\caption{Representation of the geometry used for the interaction between the inner HGO
	particle and the sphere representing the substrate in the RSP surface potential.}
	\label{fig:RSPConfig}
\end{figure}

%------------------------------------------

%------------------------------------------
\picW = 10cm
\begin{figure}
	\centering
	\pic{sw_RSP.ps}
	\caption{Representation of $\sigma^{RSP}_w(k_S, \cos\theta)$ for the RSP surface potential.}
	\label{fig:sw_RSP}
\end{figure}
%------------------------------------------

The rod-sphere model has already been used as the contact distance for a soft surface potential in 
studies of confined Gay-Berne particles in single component systems~\cite{ZhangChakrabarti96, 
WallCleaver97,TeixeiraChrzanowska01},  binary mixtures~\cite{LathamCleaver00} 
and switching situations~\cite{rwThesis}.  In all of these, full particles (\ie $k_S = k$) 
were used and tilted layers were observed in the interfacial regions.
In~\cite{WallCleaver97,TeixeiraChrzanowska01} the tilt was explained to be a consequence of 
the competition 
between packing constraints and the form of the surface potential. In other words, the attractive 
part of the potential was thought to be responsible for the tilt, the authors noticing that increasing 
the  particle-surface coupling $\alpha$ (see equation (9) of~\cite{WallCleaver97}) induced a tighter 
distribution of the particle orientations about the optimal tilt angle. It is interesting to
note however, that using the same surface potential, but a molecular elongation $k=2$ instead of
$3$, Wall and Cleaver~\cite{WallCleaver03} found that the surface anchoring changed from tilted
to planar.\\
Another case of surface tilted arrangement was obtained in the study by Lange and 
Schmid~\cite{LangeSchmid02,LangeSchmid02a,LangeSchmid02c} where 
Gay-Berne particles were confined between two structureless walls. There, the surface potential 
used was
similar to the RSP, but a different $\sigma_w$ function was used so as to describe the
interaction between a surface and an ellipsoid of revolution; also, they used the common $12-6$
Gay-Berne potential rather the $9-3$ version of~\cite{WallCleaver97}. The particles simulated were found
to exhibit planar anchoring, but tilted phases were obtained by inclusion of polymer chains
grafted on to the surface. The tilt in that case resulted from competition between the planar
orientation favoured by the solvent particles and the homeotropic alignment preferred by the 
polymer chains (since it reduced their bond energy). Again, in that study, tilting behaviour could 
be explained in terms
of the attractive parts of the potential since an anchoring transition between planar and tilted
arrangements was obtained by varying the number of grafted polymer chains.\\
Ascribing the tilt to the attractive part of the potential in~\cite{WallCleaver97} was also 
consistent with the many theoretical treatments of confined hard particle 
systems~\cite{HolystPoniewierski88,Chrzanowska_Teixera_01}~: 
none of these predict tilted surface alignment, planar and homeotropic alignments being the only
arrangement predicted.\\
It is interesting to note, however, that when using a surface made of fixed atoms
and a similar surface potential to that used in~\cite{WallCleaver97},  
Palermo~\etal\cite{PalermoBiscarini98} 
did not find any tilt arrangements, rather the natural planar arrangement was adopted at 
the substrates. This discrepancy would suggest that the interaction described by the rod-surface
potential where the particles and substrate sphere have always the same $x$ and $y$ coordinate
might be responsible for the tilting behaviour. However, this argument can not be proved just by
comparing the results from these existing studies as they employed different forms of the 
Gay-Berne type surface interaction potential, a 9-3 type in~\cite{WallCleaver97} as opposed to 12-6
in~\cite{PalermoBiscarini98}.\\


The case of short $k_S$, that is the case of particles absorbing the substrates, 
has not been considered in these studies. This case can, however,  be readily
understood. If the amount of volume absorbed is great enough to induce a significant 
reduction in the free energy, then a homeotropic arrangement should be more stable, as borne out
by the simulation studies of Allen~\cite{Allen99}.

%===============================================================================================
%===============================================================================================
\subsection{Simulation results using the RSP.}
\label{ss:RSPresults}
%====================================================================
%			RESULTS - PHASE DIAGRAMS
%====================================================================

Further investigations of the
surface induced structural changes obtained using the RSP surface potential were performed using
Monte Carlo computer simulations in the canonical ensemble. Systems of $N=1000$ hard Gaussian overlap
particles with elongation $k=3$ confined in an infinitely wide slab geometry of fixed height 
$L_z=4k\sigma_0$ were considered, the walls being situated at $z_0 = \pm\frac{L_z}{2}$ and
symmetric anchoring conditions applied. Sequence of
simulations were performed at constant number density $\rho^{*}$ and decreasing 
$k_S$ for several values of $\rho^{*}$ and the surface induced structural changes were studied 
using the observable 
profiles ($\rho^{*}_\ell(z)$, $Q_{zz}(z)$ and $P_2(z)$) introduced in the previous Chapter. 
From these profiles, the anchoring and order phase diagrams were computed in the 
interfacial and bulk regions. These diagrams are shown on Figures~\ref{fig:RSP_QzzWa} 
and~\ref{fig:RSP_P2Wa} and were computed using a similar method to that given in the previous
Chapter. The difference here was that
definition of the boundary between the interfacial and bulk regions was changed
such that the interfacial region was taken to extend from the surface to the second 
maximum of $\rho^{*}_\ell$
regardless of the surface arrangement obtained for this model. The reason for this definition 
change lies in the similarity
between density profiles obtained for this model at different values of $k_S$:
the primary peaks in $\rho^{*}_\ell(z)$ were always situated at $|z-z_i|> 0.0$ 
(see \eg Figures~\ref{fig:RSP_typeProf_k3_homeo} and~\ref{fig:RSP_typeProf_k3_planar}). 
Further details regarding these profiles are given later in this Section.\\

\picW = 7cm
\begin{figure}
	\centering
	\subfigure[Interfacial region.]{\pic{QzzWaSu_RSPphaseDia_k3_kSD.ps}}
	\subfigure[Bulk region.]{\pic{QzzWaBu_RSPphaseDia_k3_kSD.ps}}
	\caption{Anchoring phase diagrams obtained from series of simulations of $N=1000$
	confined HGO particles with $k=3$ at constant density and decreasing $k_S$ 
	and using the RSP surface potential.}
	\label{fig:RSP_QzzWa}
\end{figure}

\begin{figure}
	\centering
	\subfigure[Interfacial region.]{\pic{P2WaSu_RSPphaseDia_k3_kSD.ps}}
	\subfigure[Bulk region.]{\pic{P2WaBu_RSPphaseDia_k3_kSD.ps}}
	\caption{Order phase diagrams obtained from series of simulations of $N=1000$
	confined HGO particles with $k=3$ at constant density and decreasing $k_S$ 
	and using the RSP surface potential.}
	\label{fig:RSP_P2Wa}
\end{figure}

\picW = 7cm
\begin{figure}
	\centering
	\subfigure[$k^{'}_S = 0.0$]
	{\pic{HGO_box_NVT_RSconf_k3_N1000_kS000_d0.3500_0.50M.ps}}
	%
	\subfigure[$k^{'}_S = 1.0$]
	{\pic{HGO_box_NVT_RSconf_k3_N1000_kS100_d0.3500_0.50M.ps}}
	\caption{Typical configuration snapshots showing the surface induced homeotropic (a) and 
	tilted (b)  surface induced arrangements for confined systems of $N=1000$ HGO particles 
	using $\mathcal{V}^{RSP}$ for surface interactions and $\rho^{*} = 0.35$.}
	\label{fig:RSP_snaps}
\end{figure}
%---------------------------------------------------------------------------

Observation of the order and anchoring phase diagrams reveals that the $\overline{P}_2$ diagrams 
are qualitatively
similar to those obtained with the HNW models, whereas there are some  qualitative 
differences in the $\overline{Q}_{zz}$ diagrams.\\
In the case of short $k_S$, the $\overline{Q}_{zz}$ behaviour for the RSP model 
is not unlike that of the HNW model and confirms the predictions made at the end of the  
last Section. Throughout the
density range considered here and for short $k_S$, the system adopts an homeotropic arrangement
where order increases with increased density. This is further confirmed by 
configuration snapshots (\eg Figure~\ref{fig:RSP_snaps}(a) for the state point
$\rho^{*}=0.35$, $k_S/k=0.0$).\\
In the case of long $k_S$ (\ie $k_S/k>0.6$), however, there is a qualitative difference between 
the diagrams shown in  Figure~\ref{fig:RSP_QzzWa} and their equivalent for the HNW model. 
Here, throughout the density range considered, the value of $\overline{Q}_{zz}$ does not reach 
that expected for planar ordering, remaining, instead, low at about $0.2$. Although such
values can be understood in the low density regime, where $\overline{P}_2$ is compatible with
an isotropic phase, this behaviour is rather more surprising in the case of high densities
where the corresponding $\overline{P}_2$ diagram shows nematic order. In these latter regions, 
the low values of
$\overline{Q}_{zz}$ are, however, compatible with a tilted arrangement. 
Observation of typical snapshots for these high densities 
(\eg Figure~\ref{fig:RSP_snaps}(b) ) confirms the presence of a tilt, showing a
phase where the average surface alignment is of about $\pi/4$ radians.\\


%=================================================================================================
%		RESULTS - PROFILES
%=================================================================================================

\picW = 12cm
\begin{figure}
	\centering
	\pic{RSP_typeProf_k3_homeo.ps}
	\caption{Typical $z$-profiles for confined systems of HGO particles with 
	$k=3.0$ and $k^{'}_S = 0.0$ using the RSP potential.}
	\label{fig:RSP_typeProf_k3_homeo}
\end{figure}

\begin{figure}
	\centering
	\pic{RSP_typeProf_k3_planar.ps}
	\caption{Typical $z$-profiles for confined systems of HGO particles with 
	$k=3.0$ and $k^{'}_S = 1.0$ using the RSP potential.}
	\label{fig:RSP_typeProf_k3_planar}
\end{figure}

Further details of the surface induced structural changes obtained using the RSP potential 
can be obtained from appropriate z-profiles. These are shown for two states points 
corresponding to homeotropic and tilted arrangements in 
Figures~\ref{fig:RSP_typeProf_k3_homeo} and~\ref{fig:RSP_typeProf_k3_planar} respectively.
These profiles, share some of the features of the equivalent profiles obtained with the HNW
potential.
%%
The case $k^\prime_S = 0.0$ corresponds to a  homeotropic arrangement. This is characterized by
positive values of $Q_{zz}(z)$ and peak separations in the oscillations of $\rho^{*}_\ell(z)$ 
of about $\sel$. The quality of in-plane ordering is similar to that observed with the previous
surface potential.\\
%
For $k^\prime_S = 1.0$, however, the situation is very different, and there is little similarity 
between the profiles at $\kSp = 1$ for the RSP and HNW potentials.
The layering shown in Figure~\ref{fig:RSP_typeProf_k3_planar} is not as well defined and only 
two peaks  can  be clearly observed in $\rho^{*}_\ell(z)$. Moreover, the peak separation is much 
larger than $\so$.  Finally, $Q_{zz}(z)$ fails to display the negative values associated with 
planar ordering,  even at high density or which the corresponding $P_2(z)$ profile indicates 
an ordered phase. Those features correspond to a tilted arrangement.\\
%
%



%===============================================================================================
%===============================================================================================
\subsection{Origin of the tilt}


%====================================================================
%			ANALYTICAL ANALYSIS
%====================================================================
Here, the origins of this tilting behaviour are revisited by studying the form of the RSP 
as  a function of $k_S$. In Appendix~\ref{chap:A}, it is shown that for a particle with 
elongation $k$ whose inner 
ellipsoid is in contact with the substrate surface, the volume $Ve(k_S,\theta)$ absorbed into 
the surface is given by~:
%
\begin{equation}
	Ve=\frac{1}{3}\pi\lp \frac{1}{2}-\sqrt{\frac{\sigma^{RSP}_w}{k^2\cos^2\theta+\sin^2\theta}}\rp^2 
	\lp 1 + \sqrt{\frac{\sigma^{RSP}_w}{k^2\cos^2\theta+\sin^2\theta}} \rp
	\label{eqn:Ve_RSP}
\end{equation}
%
A graphical representation of this absorbed volume is given in Figure~\ref{fig:Ve_RSP_fkS} 
for a particle of
elongation $k=3$.  The preferred surface induced arrangements can be associated with the 
maxima in $Ve(k_S,\theta)$. For short $k_S$, $Ve(k_S,\theta)$ is maximal at $\theta = 0$ 
and, therefore, the most stable arrangement is homeotropic. In the limit of $k_S=k$, however
$Ve(k_S,\theta)$ is maximal  for intermediate $\theta$, which suggests that a tilted 
arrangement may be most stable.\\
%

More insight into this result can be found in the expression of the surface potential
(Equation~\ref{eqn:sigma_w_RSP}). In the case $k_S=k$, $\sigma^{HNW}_w$ 
represents the distance from the substrate to the particle's centre of mass when one of its
needle's ends
is in contact with the surface plane.  $\sigma^{RSP}_w$, in contrast, 
indicates whether or not the HGO particle overlaps a sphere embedded within the substrate.\\
The difference between the two shape parameter (Figure~\ref{fig:cmpHNW_RSP})
shows that there are some tilt angle for which $\sigma^{RSP}_w$ is smaller than $\sigma^{HNW}_w$, 
that is the particle ends are able to overlap the surface plane. This region of reduced 
$\sigma^{RSP}_w$  coincides with the maximum in $Ve(k_s,\theta)$ and, therefore, can be 
associated with the tilt behaviour.\\

%=================================
\picW = 10cm
\begin{figure}
	\centering
	\pic{Ve_RSP_fkS.ps}
	\caption{Representation of $Ve(kS,\theta)$ for the RSP potential and $k=3$.}
	\label{fig:Ve_RSP_fkS}
\end{figure}
%=================================

%=================================
\begin{figure}
	\centering
	\picL{cmpHNW_RSP.ps}
	\caption{Comparison between $\sigma^{HNW}_w$ (solid line) and $\sigma^{RSP}_w$ (dashed
	line). The dotted line represents the difference between the two 
	($\sigma^{RSP}_w-\sigma^{HNW}_w$.)}
	\label{fig:cmpHNW_RSP}
\end{figure}
%=================================


%=================================
\picW = 10cm
\begin{figure}
	\centering
	\pic{Ve_RSP_fk.ps}
	\caption{Representation of $Ve(k,\theta)$ for the RSP potential and $k_S=k$.}
	\label{fig:Ve_RSP_fk}
\end{figure}
%=================================

The optimum tilt angle $\theta_{tilt}$ for which the absorbed volume of a single particle is 
maximal can be
calculated for different values of k.  By considering the absorbed volume given by 
Equation~\ref{eqn:Ve_RSP} and setting $k_S=k$, an expression
for $Ve(k,\theta)$ (Figure~\ref{fig:Ve_RSP_fk}) can be obtained.
$\theta_{tilt}$, the angle which maximizes $Ve(k,\theta)$, is the angle that
solves~:
\begin{equation}
	\frac{d}{d\theta}Ve(k,\theta) = 0
	\label{eqn:dVeEq}
\end{equation}
%
where $\frac{d}{d\theta}Ve(k,\theta)$ is given by~:
%
\begin{equation}
	\frac{d}{d\theta}Ve(k,\theta) = k\pi \lp\frac{A_0\lp B_0 + C_0 \rp}{D_0} 
	-\frac{A_0 F_0 \lp B_0 + C_0 \rp}{I_0} 
	\rp.
\end{equation}
%
Here~:
%
\begin{eqnarray*}
	A_0 &=& \lp \frac{1}{2} - \sqrt{\frac{A^2}{B}} \rp^2	\\
	B_0 &=& -\frac{A^2\lp2\cos\theta\sin\theta - 2k^2\cos\theta\sin\theta\rp}{B^2}	\\
	C_0 &=& -\frac{2A(k^2-1)\cos\theta \sin\theta}{(k^2+1)BC^\frac{3}{2}}	\\
	D_0 &=& 6\sqrt{\frac{A^2}{B}}	\\
	F_0 &=& 1+\sqrt{\frac{A^2}{B}}	\\
	I_0 &=& 3\sqrt{\frac{A^2}{B}}
\end{eqnarray*}
%
and~:
%
\begin{eqnarray*}
	A &=& \frac{1}{\sqrt{C}} - \frac{1}{2}	\\
	B &=& k^2\cos\theta + \sin^2\theta	\\
	C &=& 1 - \frac{(k^2-1)\cos^2\theta}{1+k^2}.\\
\end{eqnarray*}

%=================================
\picW = 10cm
\begin{figure}
	\centering
	\pic{dVeContour_RSP_fk.ps}
	\caption{Representation of $\theta_{tilt}(k)$ using the RSP potential and $k_S=k$.}
	\label{fig:dVeContour_RSP_fk}
\end{figure}
%=================================


Equation~\ref{eqn:dVeEq} has been solved numerically by computation of the contour of 
$\frac{d}{d\theta}Ve(k,\theta)$ at level $0$, as shown on Figure~\ref{fig:dVeContour_RSP_fk}. 
This shows that $\theta_{tilt}$ is fairly constant at about 0.9 radians, 
that is about $50$ degrees.
As a result, the above treatment suggest that configurations with a tilt angle of about $50$ 
degrees are expected to be favoured from simulations of full HGO particles confined 
with the RSP potential.
However, as many body effects  have not been considered here, the existence of 
such a tilt in a bulk system is not assumed by this result.\\

That said, the simulations presented in Section~\ref{ss:RSPresults} clearly show that such a 
tilt does develop when
using the RSP as a surface potential. Although, the tilt angle was not directly available from
observation of the profiles, it can be estimated by use of the definition of
$Q_{\alpha\beta}$. At the state point $\rho^{*} = 0.35$ and $k_S/k = 1.0$ and at the z location
where $\rho^{*}_\ell(z)$ is maximal, $Q_{zz} = 0.209$, this latter value corresponds to the 
the simulation average. This value of $Q_{zz}$ corresponds to an average tilt angle 
$\theta = 0.812$ radians, that is  $46.6^\circ$. This value is consistent 
with the angle observable on the configuration snapshots.\\
The difference between the observed and predicted tilt angles (of $9.7\%$) can be understood from the 
packing constraints. The packing improves with lower tilt angles which can, in turn, 
increase system's total absorbed volume. However the absorbed volume of each particle decreases as the
difference between its optimal and actual tilt angles increases. This creates a competition 
between the amount of absorbed volume that can be obtained with a higher packing fraction 
but lower average tilt angle and that obtained with an average tilt angle closer to the 
optimal single particle angle but lower packing fraction.\\

The simulations and the theoretical treatment described in this Section
have shown that a tilted phase can be both predicted and obtained with a purely steric model.
As a result, it appears that the tilted phases obtained 
in~\cite{ZhangChakrabarti96, WallCleaver97,TeixeiraChrzanowska01, LathamCleaver00} arise due to
the geometrical characteristics of the rod-sphere potential, rather than competition between
packing density and attractive particle-particle and particle-wall interactions. 
This explanation is consistent with the
change from tilted to planar surface alignment observed by Wall and 
Cleaver~\cite{WallCleaver97,WallCleaver03}
when they reduced the molecular elongation from $k=3$ to $k=2$. In the latter
case, the molecules were too short to significantly absorb at the surface and therefore adopt the
planar state. Observation of Figure~\ref{fig:Ve_RSP_fk} at $k=2$ confirms this, as for this
elongation, the absorbed volume is virtually independent of molecular orientation and, therefore,
does not form a tilted arrangement.\\
In the light of this explanation, it seems
reasonable to assume that a planar surface arrangement would have been obtained if the
simulations of~\cite{WallCleaver97,TeixeiraChrzanowska01, LathamCleaver00} had been performed using
a lattice of fixed spheres to represent the surface, as was done in~\cite{PalermoBiscarini98}.\\




