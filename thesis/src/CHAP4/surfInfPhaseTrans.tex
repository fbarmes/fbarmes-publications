

\section{Surface influence on phase transitions.}
\label{s:surfInfphaseTrans}



%===============================================================================================
\subsection{ The $\overline{Q}_{zz}$ and $\overline{P}_2$ observables.}


To determine the location of the planar to homeotropic anchoring transition more precisely 
requires the ability to
characterize quantitatively the nature of the arrangement displayed by a confined system. 
Section~\ref{ss:typicalProfiles} has shown that simple observation of the profiles is not
sufficient to determine the arrangement type as information from both $Q_{zz}(z)$ and 
$\rho^{*}_\ell(z)$ is required. Besides, quantitative information on the anchoring behaviour is
hard to obtain solely by observation of profiles.\\
As the transition point is a function of both the global number density and needle
length, a useful observable for characterising the surface arrangement would be a scalar
able to distinguish 
both the type and strength of the anchoring for a given $(\rho^{*}, \kSp)$ state point.\\
This need is fulfilled by the use of the novel observables $\overline{Q}_{zz}$ and 
$\overline{P}_2$. These are density-profile-weighted averages of,
respectively, $Q_{zz}(z)$ and $P_2(z)$, taken over a given
region of interest. In general $\overline{Q}_{zz}$  and $\overline{P}_2$ are defined as~:
%
\begin{eqnarray}
	\overline{Q}_{zz} &=& \frac{\sum_{z_i} Q^{n}_{zz}(z_i) \rho^{*}_\ell(z_i)}
			{\sum_{z_i} \rho^{*}_\ell(z_i)}		\\
	%
	\nonumber \\ 
	%
	\overline{P}_2 &=& \frac{\sum_{z_i} P_2(z_i) \rho^{*}_\ell(z_i)}
			{\sum_{z_i} \rho^{*}_\ell(z_i)}	
\end{eqnarray}
%
where the $z_i$ considered are restricted on the region of interest.
Here $Q^{n}_{zz}(z)$ is a rescaled version of $Q_{zz}$ such that $Q^{n}_{zz}\in[-1:1]$. Hence the
definition of $Q^{n}_{zz}(z)$~:
%
\begin{equation}
	Q^{n}_{zz} = \left\{	%}
	\begin{array}{ccc}
		Q_{zz}		&\text{if}	&Q_{zz} \geq 0	\\
		2.Q_{zz}	&\text{if}	&Q_{zz} < 0\\
	\end{array}
	\right.
\end{equation}


The computation of the $\overline{Q}_{zz}$ and $\overline{P}_2$ observables has been 
performed on regions of the cell corresponding to the interfacial and bulks domains. 
This has enabled the behaviour of the system to be studied in each region separately. 
The naming convention adopted for the observables
corresponding to each region is described in Table~\ref{tble:SuBuNamingCvention}.\\

\begin{table}[h]
\centering
\begin{tabular}{||c||c||c||}
\hhline{|t:=:t:=:t:=:t|}
key	&Description			&Associated observables		\\
\hhline{|:=::=::=:|}
Sb	&Bottom interfacial region	&$\overline{Q}^{Sb}_{zz}$, $\overline{P}^{Sb}_2$\\
Bu	&Bulk region			&$\overline{Q}^{Bu}_{zz}$, $\overline{P}^{Bu}_2$\\
St	&Top interfacial region		&$\overline{Q}^{St}_{zz}$, $\overline{P}^{St}_2$\\
Su	&Both interfacial region	&$\overline{Q}^{Su}_{zz}$, $\overline{P}^{Su}_2$\\
\hhline{|b:=:b:=:b:=:b|}
\end{tabular}
\caption{Naming convention for the simulation slab regions and associated observable.}
\label{tble:SuBuNamingCvention}
\end{table}

It is now necessary to define an 
appropriate boundary between the interfacial and bulk regions. 
This boundary needs to be located at a point where the surface has no direct influence on
the molecules; as a result the boundary $z_i$ could be chosen such that $|z_i-z_0| = \frac{L_n}{2}$
since at $z=z_i$, the particles can rotate  freely without direct interaction  with the surface. 
This approach fails, however, in the limit of zero needle length, as
it implies $|z_i-z_0| \sim 0$ whereas the $z$-profiles clearly show a non-zero interfacial
regions for all needle lengths.\\
A different approach was taken, therefore, in defining this boundary; the interfacial region
width was made a function of the needle length and density by making the boundary $z_i$ 
dependent on features of the density profiles.
The scheme used is illustrated on Figure~\ref{fig:SuBuDefs}. If the anchoring was found
to be planar (first local maximum of $\rho^{*}_\ell(z)$ at $|z_i-z_0|\sim 0$), 
the interfacial region  was taken to extend from  the surface to the distance corresponding 
to the second maximum in  $\rho^{*}_\ell(z)$. If however, the anchoring was homeotropic 
(first local maximum of $\rho^{*}_\ell(z)$ at $|z_i-z_0|\sim \frac{L_n}{2}$) then the 
interfacial region was taken to extend from
the surface to the first local minimum in $\rho^{*}_\ell(z)$. In those cases with ambiguous 
double peaked density profiles, the first scheme was adopted.

\picW = 7cm
\begin{figure}[h]
	\centering
	\subfigure[planar case]{\picL{planarSuBu.ps}}
	\subfigure[homeotropic case]{\picL{homeoSuBu.ps}}
	\caption{Definition of the slab interfacial and bulk regions.}
	\label{fig:SuBuDefs}
\end{figure}

%===============================================================================================
%===============================================================================================
\subsection{Anchoring transitions.}

Here, the planar to homeotropic anchoring transition is located from measurement of 
$\overline{Q}_{zz}$ as a function of $\rho^{*}$ and $\kSp$ which can be used to construct an 
anchoring phase diagram. Such diagrams have been determined for systems with elongations 
$k=3$ and $k=5$ using 
simulations performed at constant densities and increasing and decreasing needle lengths. 
Results for both series in the interfacial (Su) and bulk (Bu) regions are shown in 
Figures~\ref{fig:QzzPhaseDia_k3} and~\ref{fig:QzzPhaseDia_k5} for $k=3$ and $k=5$ respectively.

%---------------------------------------------------------
\picW = 7cm
\begin{figure}
	\centering
	\subfigure[Series with increasing $\kSp$.]{
	\pic{QzzWaSu_phaseDia_k3_LnU.ps}
	\pic{QzzWaBu_phaseDia_k3_LnU.ps}}
	
	\subfigure[Series with decreasing $\kSp$.]{
	\pic{QzzWaSu_phaseDia_k3_LnD.ps}
	\pic{QzzWaBu_phaseDia_k3_LnD.ps}}
	
	\subfigure[Bistability phase diagram (\ie difference between (a) and (b)).]{
	\pic{QzzWaSu_bistPhaseDia_k3.ps}
	\pic{QzzWaBu_bistPhaseDia_k3.ps}}
	\caption{Anchoring phase diagrams of $\overline{Q}_{zz}$ for $k=3$ for the surface 
	(left) and bulk (right) regions of the cell. }
	\label{fig:QzzPhaseDia_k3}
\end{figure}

\picW = 7cm
\begin{figure}
	\centering
	\subfigure[Series with increasing $\kSp$.]{
	\pic{QzzWaSu_phaseDia_k5_LnU.ps}
	\pic{QzzWaBu_phaseDia_k5_LnU.ps}}
	
	\subfigure[Series with decreasing $\kSp$.]{
	\pic{QzzWaSu_phaseDia_k5_LnD.ps}
	\pic{QzzWaBu_phaseDia_k5_LnD.ps}}
	
	\subfigure[Bistability phase diagram (\ie difference between (a) and (b)).]{
	\pic{QzzWaSu_bistPhaseDia_k5.ps}
	\pic{QzzWaBu_bistPhaseDia_k5.ps}}
	\caption{Anchoring phase diagrams of $\overline{Q}_{zz}$ for $k=5$ for the surface 
	(left) and bulk (right) regions of the cell. }
	\label{fig:QzzPhaseDia_k5}
\end{figure}
%---------------------------------------------------------

The results obtained for the two elongations are very similar. In the \textbf{interfacial
region}, the anchoring transitions occur at $k_S/k$ values close to those predicted in 
section~\ref{ss:PHstability}, as can be observed from the lines of constant $\overline{Q}^{Su}_{zz} =
0$. For higher densities, the agreement between  the simulation and theoretical result can be
seen to improve. 
Also the region around $\kSp$ becomes sharper with increasing density indicating a 
possible discontinuous transition between planar and homeotropic anchoring states.\\

In the \textbf{bulk region}, little surface influence can be observed at low density, as
the values of $\overline{Q}^{Bu}_{zz}$ remain close to zero due to the systems orientational
isotropy. As the number density is increased,  the local density in the bulk 
regions reaches values corresponding to bulk nematic densities. The surface influence then extends 
further into the cell and sharp anchoring transitions become apparent at needle lengths similar to
those suggested by the interfacial region anchoring diagram.
This, however, occurs for global number densities significantly greater than the isotropic to nematic
transition densities of the equivalent bulk system (see Section~\ref{HGO_cal_res}).
This indicates that the I-N transitions in the bulk regions
were shifted to higher number densities due to the presence of the surfaces. this is further
discussed in Section~\ref{ss:surfInflINTans}.\\

The anchoring phase diagrams are also found to be asymmetric in that bulk
planar ordering develops at lower densities than its homeotropic counterpart. 
As stated earlier, this is due, in part, to the increased absorbed volume in the case of
homeotropic anchoring which is sufficient to prevent the observation of uniformly 
ordered slabs of homeotropic arrangement in the limit $\kSp=0$ for the range of density 
considered here. That said, interpolation of the result determined here to higher densities
suggests that homeotropically ordered phases should exist for $\kSp < k^{T\prime}_{S}$.

%==============================================================================================
\subsection{Anchoring bistability}

Another interesting feature of Figures~\ref{fig:QzzPhaseDia_k3} and~\ref{fig:QzzPhaseDia_k5}
comes from the comparison of the diagrams for increasing and decreasing
needle lengths (\ie diagrams (a) and (b)). As the density is increased, the hysteresis between 
the two set of results also 
increases. This confirms an earlier observation that in conditions corresponding to 
competing  anchoring according the $z$-profiles observed can be dependent on 
sample's history (recall 
Figures~\ref{fig:typicalProfile_k3_bist} and~\ref{fig:typicalProfile_k5_bist}).\\
%
Since all of the data used to construct those diagrams were obtained from equilibrated systems,
these discrepancies suggest possible bistable behaviour for state points close to the
anchoring transition. This bistability has been measured more precisely by computing the
absolute value of the difference between results obtained with series of increasing and
decreasing needle lengths. This difference equals $0$ if the two diagram agree and $2$ for full
bistability. The results for both elongations in the interfacial and bulk regions of the slab,
shown on Figures~\ref{fig:QzzPhaseDia_k3}(c) and~\ref{fig:QzzPhaseDia_k5}(c)
indicate for both $k=3$ and $k=5$ distinct bistable regions at high densities.\\

In order to demonstrate the existence of this bistability, an attempt to
switch the cell from planar to homeotropic and back has been carried out. For this, a previously
equilibrated system of $N=1000$ particles with $k=3$, $\rho^{*}=0.34$ and $\kSp = 0.5$ was
taken. This configuration was extracted from a series performed with decreasing densities which 
showed planar anchoring at this state point. The switching was performed through the series of 
simulations $R_1$ to $R_5$ listed in Table~\ref{tble:bistFieldConditions}, \ie by 
applying and removing an electric field $\vect{E} = E\vecth{z}$ and
taking the dielectric constant $\chi_e$ to be, alternately, positive and negative.
The effect of the field is to align the  particles parallel or perpendicular with $\vect{E}$
respectively for positive and negative values of $\chi_e$, respectively.\\
While this setup is admittedly somewhat unrealistic, it can nevertheless be related to an
experimental system in which the mesogens employed can display different dielectric constant 
according to the frequency of an applied AC field.\\

\begin{table}
\centering
\begin{tabular}{||c||c||c||c||c||}
	\hhline{|t:=:t:=:t:=:t:=:t:=:t|}
	Run		&	$\vecth{E}$	&$E$	& $\chi_e$	&run length\\
	\hhline{|:=::=::=::=::=:|}
	$R_1$	&	(0,0,0)		&0.0	&0.0	&$0.25.10^6$\\
	$R_2$	&	(0,0,1)		&6.0	&0.5	&$0.25.10^6$\\
	$R_3$	&	(0,0,0)		&0.0	&0.0	&$1.00.10^6$\\
	$R_4$	&	(0,0,1)		&6.0	&-0.5	&$0.25.10^6$\\
	$R_5$	&	(0,0,0)		&0.0	&0.0	&$0.50.10^6$\\
	\hhline{|b:=:b:=:b:=:b:=:b:=:b|}
\end{tabular}
\caption{Electric parameterisation in the switching between the planar and
homeotropic states of the bistable system.}
\label{tble:bistFieldConditions}
\end{table}

If the considered state points correspond to a bistable region, both planar and homeotropic phases
should be obtained in the field-off runs provided the sample is prealigned suitably for each
arrangement. The purpose of applying the field with each value of $\chi_e$ is, thus, to perform 
this pre-alignment operation.


\picW = 6cm
\begin{figure}
	\centering
	\subfigure[start $R_1$]{\pic{HGO_box_init.ps}}
	\subfigure[end $R_1$]{\pic{HGO_box_Sconf_switch_01.ps}}
	\subfigure[end $R_2$]{\pic{HGO_box_Sconf_switch_02.ps}}
	\subfigure[end $R_3$]{\pic{HGO_box_Sconf_switch_03.ps}}
	\subfigure[end $R_4$]{\pic{HGO_box_Sconf_switch_04.ps}}
	\subfigure[end $R_5$]{\pic{HGO_box_Sconf_switch_05.ps}}
	\caption{Configuration snapshots corresponding to the initial (start) and final
	configurations of runs $R_1$ to $R_5$.}
	\label{fig:HGOk3BistSnaps}
\end{figure}


\picW = 12cm
\begin{figure}
	\centering
	\subfigure[$\overline{Q}_{zz}^{Su}(n)$]{\picL{QzzWaSu_k3_Bist.ps}}
	\subfigure[$\overline{Q}_{zz}^{Bu}(n)$]{\picL{QzzWaBu_k3_Bist.ps}}
	\caption{Evolution of $\overline{Q}_{zz}$ in the interfacial(a) and bulk(b) regions as a
	function of $n$, the number of sweeps.}
	\label{fig:QzzWak3Bist}
\end{figure}

The configuration snapshots corresponding to the initial and final states from each run are shown on 
Figure~\ref{fig:HGOk3BistSnaps}(a) to (f). The corresponding behaviour of
$\overline{Q}^{Su}_{zz}$ and $\overline{Q}^{Bu}_{zz}$, as a function of Monte Carlo sweeps, are
shown in Figures~\ref{fig:QzzWak3Bist}(a) and~\ref{fig:QzzWak3Bist}(b). Also, for comparison,
the values of $\overline{Q}_{zz}$ at this state point and corresponding to the two different surface
arrangements as obtained from the runs with increasing and decreasing $\kSp$ are shown as
horizontal lines.\\
%
The results confirm the existence of the bistable region. 
Run $R_1$ shows that the system remains stable in its initial planar arrangement; after 
reorientation of the particles along $\vecth{z}$ by the applied field (run $R_2$), 
the  system equilibrates naturally to a  homeotropic arrangement (run $R_3$) although the
final value of $\overline{Q}_{zz}$ is higher than that obtained from the previous runs performed
in the computation of the anchoring phase diagrams. This discrepancy might be induced by the
application of the strong value of the field which forced all interfacial particles 
to take an almost perfect homeotropic alignment. Upon removal of the field, 
the system equilibrated towards the stable homeotropic state, but due to packing constraints 
fewer particles were allowed to take a planar orientation close to the surface as in the case 
of previous simulations. This should not however question the equilibrium of the state obtained
here.
%
Upon changing the molecular dielectric susceptibility to $\chi_e<0$, reapplication of the field 
(run $R_4$) recreates a planar arrangement which relaxes to the original stable state 
upon field removal (run $R_5$). In this case, the system equilibrated to the same value of
$\overline{Q}_{zz}$ as that obtained previously as in the case of planar anchoring, there is no
instance of homeotropic alignment in the interfacial region.\\
%
It should be noted also, that the `response times' of the systems were different in the bulk and
interfacial region; however the Monte Carlo technique was used and this does not follow the time
evolution of the systems. An appropriate study of the dynamic behaviour of the system would have
required the use of the Molecular Dynamics techniques, but this was not of prime interest here.
The purpose of these simulations was to prove the existence of the bistability behaviour of the
model, and within the simulation run lengths available here, this has been fulfilled.


%==============================================================================================
\subsection{The I-N transition.}
\label{ss:surfInflINTans}

Here, the influence of confinement on the liquid crystalline phase behaviour of the model is
studied, specifically, the influence of confinement upon the location of the I-N transition is of
interest. To some extend, this issue has already been addressed for similar systems 
studied very recently by Zhou~\etal\cite{ZhouChen03}. This work was based on simulations of 
hard Gaussian overlap particles of elongation $k\in [2:3]$ confined between plane structureless
walls represented by a surface potential describing the interaction between an ellipsoid and 
a plane.
The authors found that the effect of confinement was to shift the location of the I-N transition
towards lower number densities. Another effect was enhancement of orientational order in that particles 
whose shape anisotropy was not sufficient for the formation of liquid crystalline phases in the
bulk (\ie 3d) displayed ordered phases with an order parameter consistent with a nematic 
phase in confined systems.\\

The effect of confinement upon the phase behaviour of the model studied here was assessed by
computing the variation of $\overline{P}_2$ and the average local density
$\overline{\rho^{*}_\ell}$ in bulk and surface regions as a function of the overall number density 
$\rho^{*}$ and reduced needle length $k_S/k$.
Here the  approach taken for the study of the anchoring transition was applied,
using $P_2(z)$ as the main observable.\\
The order phase diagrams of the system as a function of $\rho^{*}$ and $\kSp$ have been
computed from simulations with constant $\rho^{*}$ performed in series of increasing and decreasing
$\kSp$. The difference between this and the study of the anchoring transition is that now lines of
constant $\kSp$ on the diagram are of interest;
the results shown here have been computed using series of simulations with constant density 
rather than constant needle lengths as more data were available for the former. 
However comparison of data obtained from series with constant density and varying needle length 
show that ultimately both series agree. For the sake of completeness, 
Figure~\ref{fig:P2Wa-rho_fd} shows a sample of the results obtained for series of simulation 
with constant needle length and increasing densities.
The order phase diagrams (obtained from the series with constant number density) 
for $k=3$ in the interfacial and bulk regions are shown on 
Figure~\ref{fig:P2PhaseDia_k3}, and the evolution of $\overline{\rho^{*}_\ell}$ with 
$\rho^{*}$ and $k_S/k$ is shown on Figure~\ref{fig:rhoLPhaseDia_k3}. 
From these, the effects of confinement on the I-N transition can be assessed.\\

%---------------------------------------------------------
\picW = 10cm
\begin{figure}
	\centering
	\subfigure[interfacial region]{\picL{P2WaSu-rho_k3.ps}}
	\subfigure[bulk region]{\picL{P2WaBu-rho_k3.ps}}
	\caption{Evolution of $\overline{P}_2$ in the interfacial(a) and bulk(b) regions as
	obtained from simulations at constant $k_s/k$ and increasing densities.}
	\label{fig:P2Wa-rho_fd}
\end{figure}

%---------------------------------------------------------
\picW = 7cm
\begin{figure}
	\centering
	\subfigure[Series with increasing needle length]
	{\pic{P2WaSu_phaseDia_k3_LnU.ps}\pic{P2WaBu_phaseDia_k3_LnU.ps}}
	\subfigure[Series with decreasing needle length]
	{\pic{P2WaSu_phaseDia_k3_LnD.ps}\pic{P2WaBu_phaseDia_k3_LnD.ps}}
	\caption{Order phase diagrams for $k=3$ in the interfacial (left) and bulk (right) regions of the
	slab obtained from series with increasing and decreasing density.}
	\label{fig:P2PhaseDia_k3}
\end{figure}

%---------------------------------------------------------
\picW = 7cm
\begin{figure}
	\centering
	\subfigure[Series with increasing needle length]
	{\pic{rhoLSu_k3_S4.ps}\pic{rhoLBu_k3_S4.ps}}
	\subfigure[Series with decreasing needle length]
	{\pic{rhoLSu_k3_S2.ps}\pic{rhoLBu_k3_S2.ps}}
	\caption{average local density $\overline{\rho^{*}_\ell}$ for $k=3$ 
	in the interfacial (left) and bulk (right) regions of the
	slab obtained from series with increasing and decreasing density.}
	\label{fig:rhoLPhaseDia_k3}
\end{figure}


%---------------------------------------------------------
%\picW = 7cm
%\begin{figure}
%	\centering
%	\subfigure[Series with increasing needle lengths]
%	{\pic{P2WaSu_phaseDia_k5_LnU.ps}\pic{P2WaBu_phaseDia_k5_LnU.ps}}
%	
%	\subfigure[Series with decreasing needle lengths]
%	{\pic{P2WaSu_phaseDia_k5_LnD.ps}\pic{P2WaBu_phaseDia_k5_LnD.ps}}
%	\caption{Order phase diagrams for $k=5$ in the interfacial (left) and bulk (right) 
%	regions of the slab obtained from series with increasing and decreasing density.}
%	\label{fig:P2PhaseDia_k5}
%\end{figure}


%---------------------------------------------------------

Observation of these results reveals the main effects of confinement mentioned in
Chapter~\ref{chap:two} and at the beginning of this Chapter.
In the interfacial region, Figure~\ref{fig:rhoLPhaseDia_k3} shows an enhanced density
which, in turn, results in higher order as shown on Figure~\ref{fig:P2PhaseDia_k3}. As a result
of this, the average local density in the bulk region is reduced, and so is the degree of
ordering. Generally, therefore, the effect of confinement on these systems is to shift the 
isotropic-nematic transition to lower number densities in the interfacial region and to higher 
number densities in the bulk region.\\ 

Close observation of Figures~\ref{fig:P2PhaseDia_k3} and~\ref{fig:rhoLPhaseDia_k3}
shows, however, that the anchoring conditions have a strong influence on the surface induced 
shifts in the local
density and order parameter values. 
In particular and in the region corresponding to competing anchoring \ie $k_S \sim 0.5$,
$\overline{\rho^{*}_\ell}$ shows a sudden decrease in value which is accompanied by a region
of low orientational order. However, this effect seems to be much stronger on 
$\overline{P}_2$ than is suggested by the behaviour of $\overline{\rho^{*}_\ell}$. This is
because,
in addition to the reduced average local density, the double peaked nature of the $z$-profiles
indicates the particles to have competing preferred orientations, which reduces
the value of the order parameter. This leads to a higher shift in the number density of the 
I-N transition of competing anchoring cases than would be expected purely from local density
effects.\\

The results observed here are consistent with those obtained by Zhou~\etal\cite{ZhouChen03}. The
shifts in $\overline{\rho^{*}_\ell}$ and $\overline{P}_2$ with number density confirms 
that for non-competing anchoring conditions, 
the principal  effect of confinement is to enhance order in the systems and shift the 
location of the I-N transition towards lower number densities. 
Calculating the observables independently in the interfacial and bulk regions
shows that both regions exhibit a qualitatively but not quantitatively similar behaviour; this
was not, however, observed in~\cite{ZhouChen03}, where all observables were averaged
over the full samples, so that different shifts in the bulk and surface I-N 
transition densities were not monitored. 
Finally, we have shown that the  behaviour of the film is dependent on the type 
of anchoring applied; this issue was not addressed in~\cite{ZhouChen03} where only the 
case of strong planar anchoring was considered.


