

\section{A first surface potential}
\label{s:HNWmodel}

Here, surface induced structural changes are studied using ellipsoidal shaped particles interacting
through the hard Gaussian overlap potential. In this Chapter all
particle-surface interactions are performed using the Hard Needle Wall potential 
(HNW)~\cite{Chrzanowska_Teixera_01} as this provides a simple and intuitive steric interaction which
can be tuned so as to induce either homeotropic or planar arrangement. As a result surface
induced structural changes can be studied using one model along with parameters values
appropriate for both surface arrangements.
Also, this potential allows the study of the transition from one arrangement to the other.


%==============================================================================================
%==============================================================================================
\subsection{The Hard Needle Wall potential}
\label{ss:HNW}


With the Hard Needle Wall (HNW) potential, the particles do not interact directly 
with the surfaces,  rather the surface  interaction is achieved by a needle of length 
$k_S$ placed at the centre of each particle
(Figure~\ref{fig:HGO_wal}). As a result, the HNW potential can be viewed as an extension of the
potential used by Allen~\cite{Allen99} where only the centre of mass interacts with the surface.
The case considered in~\cite{Allen99}, thus, corresponds to the HNW potential with a zero needle
length. The interaction potential between the needle and the surface is defined by 
$\mathcal{V}^\mrm{HNW}$ as~:

\begin{equation}
\mathcal{V}^{HNW} = \left\{ %}
	\begin{array}{ccc}
	0               &\mrm{if}   &|z_i-z_0| \geq \sigma_w  \\
	\infty          &\mrm{if}   &|z_i-z_0| < \sigma_w
	\end{array}
	\right.
\end{equation}

with~:

\begin{equation}
        \sigma_w = \frac{1}{2}\so k_S\cos(\theta)
	\label{eqn:sigma_w_HNW}
\end{equation}

Here $\so$ defines the unit of distance, $k_S$ is the dimensionless needle length and 
$\theta$ the angle 
between the surface normal and the particle's orientation vector, which and also corresponds 
to the Euler zenithal angle.  The behaviour of $\sigma_w(\theta,k_S)$ is shown on 
Figure~\ref{fig:sw_HNW}\\

%----------------------------------------------------
\picW = 7cm
\begin{figure}
	\centering
	\pic{HNWConfig.ps}
	\caption{The HNW configuration}
	\label{fig:HGO_wal}
\end{figure}
%----------------------------------------------------

%------------------------------------------
\picW = 7cm
\begin{figure}
	\centering
	\pic{sw_HNW_2.ps}
	\caption{Evolution of $\sigma_w(k_S, \cos\theta)$ for the HNW surface potential.}
	\label{fig:sw_HNW}
\end{figure}
%------------------------------------------

At high densities, this potential drives the preferred
anchoring behaviour by dictating the volume of absorbed particles that can be subsumed 
into the surface. The only model parameter upon which the anchoring strength and 
angle are dependent is the needle length. According to this, there is a finite volume of 
any `surface particle' that can be absorbed on the surface. The effect of this absorbed 
volume is to reduce the system's free energy due to the increased free volume (and, thus,
entropy) it afford to rest of the system.
The greater the volume  absorbed, the lower the free energy. As a result the most stable
surface arrangement, \ie that which minimizes the free energy, is the one that maximizes 
the absorbed volume.\\

The surface behaviour of the HNW model has already been studied by Chrza\-now\-ska 
\etal\cite{Chrzanowska_Teixera_01} and Cleaver and Teixeira~\cite{Cleaver_Teixeira_01}. For
small $k_S$, the homeotropic arrangement is most stable, whereas the planar arrangement is 
favored for long $k_S$. Some more quantitative insight into this can be obtained by studying the
amount of volume that can be absorbed into the surface as a function of molecular orientation 
and needle length. Approximating the shape of a Gaussian ellipsoid to that of an ellipsoid of
revolution~\cite{Rigby89}, the volume $V_e$ of a single particle of elongation $k$ absorbed into 
the substrate as a function of $k_S$ and $\theta$ can be obtained as described in
Appendix~\ref{chap:A}~:
\begin{equation}
	Ve=\frac{1}{3}\pi
	\lp \frac{1}{2}-\sqrt{\frac{\sigma^\mrm{HNW}_w(k_S,\theta)}{k^2\cos^2\theta+\sin^2\theta}}\rp^2 
	\lp 1 + \sqrt{\frac{\sigma^{HNW}_w(k_S,\cos\theta)}{k^2\cos^2\theta+\sin^2\theta}} \rp
	\label{eqn:Ve_HNW}
\end{equation}

On replacing $\sigma^\mrm{HNW}_w(k_S,\theta)$ with the expression of
Equation~\ref{eqn:sigma_w_HNW}, $V_e$ reads~:
\begin{equation}
	V_e = \frac{1}{8}\pi
	\lp-1+\sqrt{\frac{k_S^2\cos^2\theta}{5+4\cos(2\theta)}}\rp^2
	\lp2+\sqrt{\frac{k_S^2\cos^2\theta}{5+4\cos(2\theta)}}\rp
\end{equation}
A graphical representation of this function is given in Figure~\ref{fig:Ve_HNW}. 

\picW = 8cm
\begin{figure}
	\centering
	\pic{Ve_HNW_fkS.ps}
	\caption{Representation of the absorbed volume $V_e$ of a single particle 
	for the HNW potential as a function of $k_S$ and $\theta$.}
	\label{fig:Ve_HNW}
\end{figure}

This shows that regardless the needle length, the absorbed volume of a single particle is maximal 
for $\theta=\pi/2$,
suggesting that the planar arrangement should be most stable for every case. One exception is the
case $k_S=0$ where both planar and homeotropic arrangement allow the adsorption of an equal
volume (half the ellipsoid volume.)
Clearly this disagrees with the simulation results of~\cite{Chrzanowska_Teixera_01,Cleaver_Teixeira_01} 
where intermediate, short needle lengths
also showed stable homeotropic alignment. This discrepancy can be resolved by also considering
the packing behaviour of the two arrangements. This is described in the next Section.




%==============================================================================================
%==============================================================================================
\subsection{Relative stability of anchoring orientations}
\label{ss:PHstability}

We now consider the behaviour of
a system of numerous hard Gaussian overlap molecules in a confined geometry, interacting with 
the surface through the HNW potential. The aim is to calculate the homeotropic to
planar transition needle length for this system. Each particle close to the surface is 
approximated to be an ellipsoid of revolution with elongation $k=\frac{\sel}{\so}$ and 
with semi axis $a=b=\frac{\so}{2}$ and $c=\frac{\sel}{2}$. The following quantities are also
defined~:

\begin{itemize}
        \item $Veps$~: full volume of an ellipsoid.
	\item $V_P(k_S)$~: volume absorbed by one ellipsoid with $\theta=\frac{\pi}{2}$
	\item $S_P$~: area of the substrate occupied by one ellipsoid with $\theta=\frac{\pi}{2}$
	\item $V_H(k_S)$~: volume absorbed by one ellipsoid with $\theta=0$
        \item $S_H$~: area of the substrate occupied by one ellipsoid with $\theta=0$
\end{itemize}

\picW = 7cm
\begin{figure}[h]
	\centering
	\pic{HPTransTheory_f3.ps}
	\caption{Schematic representation of the geometry considered for the 
	calculation of $V_H(k_S)$.}
	\label{fig:HPTransTheory_f3}
\end{figure}
%----------------------------------------------------------------

The homeotropic to planar transition can be understood by considering the ratio of the volume
absorbed into the surface to the surface occupied by the particle on the substrate (\ie the
projection of the particle onto the substrate) for the two key arrangements. The free energy  
for these will be equal when the two ratios are equal. 
As a result, the problem of finding the anchoring transition needle length reduces to solving~:
%
\begin{equation}
        \frac{V_H(k_S)}{S_H} = \frac{V_P(k_S)}{S_P}
	\label{eqn:P-Htransition}
\end{equation}

%-----------------------------------------------------------------------------------------------
%-----------------------------------------------------------------------------------------------
\subsubsection{expression for $V_P(k_S)$ and $S_P$}
The case of planar alignment is straightforward~:
%
\begin{eqnarray}
        S_P &=& \frac{\pi\so\sel}{4}\\
        V_P &=& \frac{\pi\so^2\sel}{12}
\end{eqnarray}


%-----------------------------------------------------------------------------------------------
%-----------------------------------------------------------------------------------------------
\subsubsection{expression for $S_H$}

In the case of Homeotropic arrangement, $\theta=0 \forall k_S$. As a result, although the
particle will be positioned at a different distances from the surface as a function of $k_S$, the
projection of the particle on to the surface is constant and thus~:
%
\begin{equation}
        S_H = \frac{\pi\so^2}{4}
\end{equation}
%
%-----------------------------------------------------------------------------------------------
%-----------------------------------------------------------------------------------------------
\subsubsection{expression for $V_H(k_S)$ }

The volume that can be absorbed if $\theta=0$, is a function of $k_S$, but is not obtained
trivially. The extreme cases are known~:
\begin{eqnarray*}
        V_H(0) &=& \frac{Veps}{2}\\
	V_H(k) &=& 0
\end{eqnarray*}

The set up shown in Figure~\ref{fig:HPTransTheory_f3} is considered where ultimately 
$V_H(k_S)$ corresponds to the solid volume. First an expression for $V_F$, the volume of 
a portion of an ellipsoid from the tip to a distance $z_0$ is  required as a function 
of $z_0$. From this $V_H(k_S)$ can be identified. Starting from the equation of an ellipsoid~:
\begin{equation}
	\frac{x^2}{a^2} + \frac{y^2}{b^2} + \frac{z^2}{c^2} = 1
	\label{eqn:ellipsoidEqn}
\end{equation}

$V_F$ is given by~;
\begin{equation}
	V_F = \int^{z_0}_{-c} \int^{y_\mrm{max}}_{y_\mrm{min}} \int^{x_\mrm{max}}_{x_\mrm{min}} dx.dy.dz
\end{equation}
%
Using~(\ref{eqn:ellipsoidEqn}), the triple integral transforms to~:
%
\begin{eqnarray}
	V_F &=& \int^{z_0}_{-c}
		\int^{ \frac{b}{c}\sqrt{c^2 - z_0^2} }_{-\frac{b}{c}\sqrt{c^2 - z_0^2}}
		\int^{\frac{a}{b}\sqrt{b^2K - y^2} }_{-\frac{a}{b}\sqrt{b^2K - y^2}}
		dx.dy.dz\\
	\mrm{with\ } K &=& 1 - \frac{z^2}{c^2} \nonumber
\end{eqnarray}
%
%
and thus~:
%
\begin{equation}
	V_F 	= \pi ab \left[  z_0\lp 1 - \frac{z_0^2}{3c^2}\rp + \frac{2c}{3} \right]
	\label{eqn:V_F(z_0)}
\end{equation}

This expression can be checked, considering two known limits.
If $z_0=c$, $V_F = \frac{4}{3} \pi abc$ which is the full volume of an ellipsoid.
If $z_0=-c$, $V_F = 0$, again giving the expected result.\\

$V_H(k_S)$ can then be obtained by identification of the parameters $a,b,c$ and $z_o$ with the
setup shown in Figure~\ref{fig:HPTransTheory_f3} using ~:
\begin{eqnarray*}
	a &=& b = \frac{\so}{2}\\
	c &=& \frac{\sel}{2}	\\
	z_0 &=& \frac{-k_S}{2}
\end{eqnarray*}
%
which leads to~:
%
\begin{equation}
	V_H = \frac{\pi\so^2}{4}
	\left[
	\frac{k_S}{2} \lp \frac{k_S^2}{3\sel^2} -1  \rp + \frac{\sel}{3}
	\right]
	\label{eqn:V_H(Ln)}
\end{equation}



%======================================================================================================
%======================================================================================================
\subsubsection{Transition needle length }

Having obtained expressions for $V_H(k_S)$, $S_H$, $V_P(k_S)$ and $S_P$,
Equation \ref{eqn:P-Htransition} can be solved.
%
\begin{gather}
        \frac{V_H(k_S)}{S_H} = \frac{V_P(k_S)}{S_P}                     \\
        \frac{1}{6k^2\so^2}k_S^3 - \frac{1}{2}k_S + \frac{\so}{3}\lp k-1\rp = 0
\end{gather}

Also, for clarity and generality purposes, the results are best expressed using the reduced 
needle length $k_S^{'} = \frac{k_S}{k}$. This leads to~:
\begin{equation}
	\frac{k}{6\so^2}k_S^{'3} - \frac{k}{2}k_S^{'} +\frac{\so}{3}\lp k-1\rp = 0
	\label{eqn:LnTfinalPoly}
\end{equation}

The transition needle length $k_S^{T\prime}$ is therefore given by the root 
of~\ref{eqn:LnTfinalPoly} satisfying $k_S^{T\prime} \in ]0:1]$
The result is $k-$dependent, the variation of the transition $k_S^{T\prime}$ as a function
of k is shown in Figure~\ref{fig:LnT-k}.

\picW = 10cm
\begin{figure}
        \centering
        \picL{LnT-k.ps}
	\caption{Variation of $k_S^{T\prime}(k)$}
	\label{fig:LnT-k}
\end{figure}


For the two elongations used in this study, that is $k=3$ and $5$, the transition needle length is :
\begin{center}
	\textbf{$k=3$~:} $k_S^{T\prime} \sim 0.4817$	\\
	\textbf{$k=5$~:} $k_S^{T\prime} \sim 0.6084$	\\
\end{center}
%
Thus with $k=3$, the anchoring transition should occur for a reduced needle length of about 50\% 
whereas with $k=5$, the transition should occur for a reduced needle length of about 60\%.





