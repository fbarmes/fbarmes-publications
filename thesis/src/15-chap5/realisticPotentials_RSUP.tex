
\section{Realistic surface potentials~: the RSUP.}
\label{s:realisticPotentials_RSUP}



%===============================================================================================
%===============================================================================================
\subsection{The rod-surface potential.}

The rod-surface potential (RSUP) represents an alternative interaction between a Gaussian 
ellipsoid and a plane and is given by~:
\begin{equation}
	\mathcal{V}^{RSUP} = \left\{	%}
	\begin{array}{ccc}
		0	&\mathrm{if}	&|z_i - z_0| \geq \sigma^{RSUP}_w	\\
		\infty	&\mathrm{if}	&|z_i - z_0| < \sigma^{RSUP}_w	
	\end{array}
	\right.
\end{equation}
This time, the contact distance for this is obtained by integration of the rod-sphere potential 
(without the $\frac{\so}{2}$ shift) over the x-y plane leading to~\cite{rwThesis}~:
%
\begin{equation}
	\sigma^{RSUP}_w = \sigma_0\sqrt{\frac{1-\chi_S\sin^2\theta}{1-\chi_S}}
\end{equation}
%
with the same definition for $\chi_S$ as with the RSP potential. This potential can be thought 
of as being equivalent to the RSP but with the important difference that each particle
effectively interacts  with an infinity 
of spheres as opposed to just one. Again a shift is introduced so as to remove the virtual
spheres from the simulation box. The contact distance used in the simulation is therefore given by~:
\begin{equation}
	\sigma^{RSUP}_w = \sigma_0\lp \sqrt{\frac{1-\chi_S\sin^2\theta}{1-\chi_S}} -
	\frac{1}{2}\rp.
\end{equation}


%------------------------------------------
\picW = 10cm
\begin{figure}
	\centering
	\pic{RSUPConfig.ps}
	\caption[Representation of the geometry used for the interaction between the inner HGO
	particle and the surface in the RSUP potential.]
	{Representation of the geometry used for the interaction between the inner HGO
	particle and the surface in the RSUP potential. The three spheres represent the
	substrate which is really made of an infinity of such spheres located between the horizontal lines 
	which effectively mark the substrate location.}
	\label{fig:RSUPConfig}
\end{figure}

%------------------------------------------


%------------------------------------------
\picW = 10cm
\begin{figure}
	\centering
	\pic{sigmaRSUP.ps}
	\caption{Representation of $\sigma^{RSUP}(k_S,\theta)$ for $k=3$ using the RSUP
	potential.}
	\label{fig:sigma_RSUP}
\end{figure}


%------------------------------------------

A representation of $\sigma^{RSUP}_w(k_S,\theta)$ is given in Figure~\ref{fig:sigma_RSUP} for
$k=3$. Again, the expression for the absorbed volume into the surface can be used to predict the
surface behaviour of this model. In the case of the RSUP potential, this volume reads~:
%
\begin{equation}
	V_e^{RSUP}=\frac{1}{3}\pi\lp \frac{1}{2}-
	\sqrt{\frac{\sigma^{RSUP}_w}{k^2\cos^2\theta+\sin^2\theta}}\rp^2 
	\lp 1 + \sqrt{\frac{\sigma^{RSUP}_w}{k^2\cos^2\theta+\sin^2\theta}} \rp
	\label{eqn:Ve_RSUP}
\end{equation}
%
A graphical representation of this volume is shown on Figure~\ref{fig:Ve_RSUP_fks}. 
In the limit $k_S = 0$, $Ve^{RSUP}(k_S,\theta)$ has its maximum at $\theta = 0$ thus indicating 
an homeotropic arrangement.\\
%
In the limit $k_S = k$, $V_e^{RSUP}$ is close to zero for all $\theta$ and has a small maximum at 
$\theta = 0$. However, by design, $\sigma_w^{RSUP}$ forbids any particle adsorption into the
substrate if $k_S=k$; this is even further illustrated by the value of
$\sigma_w^{RSUP}(k_S=k,\theta=0)$ which is equal to the contact distance between a HGO particle
with $\theta=0$ and a sphere. As a result, this small maximum in $V_e^{RSUP}$ can be explained
to be a result of the approximation of using ellipsoidal shaped particles 
in Appendix~\ref{chap:A} when deriving
the expression for $V_e^{RSUP}$ and, therefore, $V_e^{RSUP}(k_S=k),\theta$ should, in fact,
be zero for all
values of $\theta$. A consequence of this, the $\theta=0$ peak in Figure~\ref{eqn:Ve_RSUP} 
does not represent a 
stable surface arrangement of the RSUP model in the case $k_S=k$, as this is not 
absorption driven. Rather, it can be safely assumed that the stable arrangement for this system 
is planar, in common with the findings of previous 
theoretical and simulation work on rod-shaped objects absorbed at planar 
surfaces~\cite{HolystPoniewierski88,Chrzanowska_Teixera_01,VanRoijDijkstra00,DijkstraVanRoij01}.


%------------------------------------------
\picW = 10cm
\begin{figure}
	\centering
	\pic{Ve_RSUP_fkS.ps}
	\caption{Representation of $Ve(k_s,\theta)$ for the RSUP potential and $k=3$.}
	\label{fig:Ve_RSUP_fks}
\end{figure}
%------------------------------------------

The mechanism expected to drive an anchoring transition with the RSUP potential is slightly
different from that seen with the HNW potential. With the latter, the surface rearrangement 
is mainly driven
by the molecular volume that can be absorbed into the surface, and the transition
from homeotropic to planar arrangements occurs when the volume absorbed by the
latter is greater than with the former arrangement. In the case of the RSUP potential, however, 
there is
no absorption in the case of the planar arrangement. However, this is the base state of
any rod-shaped object in contact with a surface. Thus, as $k_S$ is decreased from  $k_S=k$ to $k_S=0$, 
the volume that can be absorbed in a homeotropic arrangement gradually increases. In this case,
therefore, an anchoring transition from planar to homeotropic arrangement 
is expected when the volume that can be absorbed by an homeotropic surface induces a total free energy
lower than in that of the planar base state.


%===============================================================================================
%===============================================================================================
\subsection{Simulation results obtained using the rod-surface potential.}

The surface induced structural changes obtained from the rod-surface potential have been studied using
Monte Carlo simulations in the canonical ensemble on systems of $N=1000$ HGO particles with 
elongation $k=3$. The simulation slab was the same as that used previously, with the walls situated 
on the top and bottom of the cell with constant height $L_z = 4k\sigma_0$ and symmetric 
anchoring conditions. Two series of simulations at each chosen density were performed 
with,respectively, increasing and decreasing $k_S$.
Typical $z$-profiles for this model are shown on Figures~\ref{fig:RSUP_typeProf_k3_homeo} 
and~\ref{fig:RSUP_typeProf_k3_planar} respectively for $k_S=0.0$ and $k_S = k$.\\

%------------------------------
\picW = 12cm
\begin{figure}
	\centering
	\pic{RSUP_typeProf_k3_homeo.ps}
	\caption{Typical $z$-profiles for confined systems of $N=1000$ HGO particles with 
	$k=3.0$ and $k^{'}_S = 0.0$ using the RSUP potential.}
	\label{fig:RSUP_typeProf_k3_homeo}
\end{figure}
%------------------------------


%------------------------------
\picW = 12cm
\begin{figure}
	\centering
	\pic{RSUP_typeProf_k3_planar.ps}
	\caption{Typical $z$-profiles for confined systems of $N=1000$ HGO particles with 
	$k=3.0$ and $k^{'}_S = 1.0$ using the RSUP potential.}
	\label{fig:RSUP_typeProf_k3_planar}
\end{figure}
%------------------------------

%===================================================
\picW = 6cm
\begin{figure}
	\centering
	\subfigure[Simulations with decreasing $k_S$]{\pic{QzzWa_HGO_RSU_k3_fkS_S1_Su.ps}\pic{QzzWa_HGO_RSU_k3_fkS_S1_Bu.ps}}
	\subfigure[Simulations with increasing $k_S$]{\pic{QzzWa_HGO_RSU_k3_fkS_S3_Su.ps}\pic{QzzWa_HGO_RSU_k3_fkS_S3_Bu.ps}}
	\subfigure[Bistability diagrams]{\pic{QzzWa_HGO_RSU_k3_fkS_SuBist.ps}\pic{QzzWa_HGO_RSU_k3_fkS_BuBist.ps}}
	\caption{Anchoring phase diagrams obtained from series of simulations of $N=1000$
	confined HGO particles with $k=3$ at constant density and decreasing $k_S$ 
	using the RSUP surface potential. Diagrams on the l.h.s are relative to the interfacial
	region and those on the r.h.s are relative to the bulk region.}
	\label{fig:QzzWaPhaseDia_k3_RSUP}
\end{figure}
%===================================================



%===================================================
\picW = 7cm
\begin{figure}
	\centering
	\subfigure[Simulations with decreasing $k_S$]{\pic{SWa_HGO_RSU_k3_fkS_S1_Su.ps}\pic{SWa_HGO_RSU_k3_fkS_S1_Bu.ps}}
	\subfigure[Simulations with increasing $k_S$]{\pic{SWa_HGO_RSU_k3_fkS_S3_Su.ps}\pic{SWa_HGO_RSU_k3_fkS_S3_Bu.ps}}
	\caption{Order phase diagrams obtained from series of simulations of $N=1000$
	confined HGO particles with $k=3$ at constant density and decreasing $k_S$ 
	using the RSUP surface potential. Diagrams on the l.h.s are relative to the interfacial
	region and those on the r.h.s are relative to the bulk region. The bistability is
	negligible.}
	\label{fig:SWaPhaseDia_k3_RSUP}
\end{figure}
%===================================================



In the limit of $k_S=0$, the surface induced structural changes for this model are very
similar to their counterparts with the RSP model (Figures~\ref{fig:RSP_typeProf_k3_homeo} 
and~\ref{fig:RSP_typeProf_k3_planar}); the two sets of profiles
are virtually indistinguishable. The surface arrangement is homeotropic which explains the
very strong similarities between the two sets; with $\theta\sim0$, 
both models induce the same geometry between the particles and the substrate.\\
%
In the limit $k_S=k$, the surface induced effects for $\mathcal{V}^{RSP}$ and $\mathcal{V}^{RSUP}$
are very different.
The short peak separation in $\rho^{*}_\ell(z)$ and the negative values in $Q_{zz}(z)$, coupled with 
the high values of $P_2(z)$, indicate an induced planar surface arrangement very much in
agreement with the predictions made earlier. This is further confirmed by the
similarity of the planar arrangement profile features for the HNW and RSUP models. The main difference
between the two arises because with the RSUP model, planar particles are not allowed to 
absorb at the surface. 
This leads to the regions of zero $\rho^{*}_\ell(z)$ with a width of $0.5\sigma_0$ close to 
each substrate.\\

The full surface induced behaviour of this system has been computed as a function of $k_S$ and 
$\theta$ using the anchoring and order phase diagrams shown in 
Figures~\ref{fig:QzzWaPhaseDia_k3_RSUP} and~\ref{fig:SWaPhaseDia_k3_RSUP}. 
The convention adopted to distinguish the interfacial 
from the bulk region was the same as that used with the RSP model.
%
The anchoring phase diagrams are given in Figure~\ref{fig:QzzWaPhaseDia_k3_RSUP}(a) 
and (b) for, respectively, decreasing and increasing $k_S$. From those
a strong difference between the two sets can be observed. The corresponding bistability 
phase diagrams (Figure~\ref{fig:QzzWaPhaseDia_k3_RSUP}c)  reports a very wide and strong 
bistability behaviour for this surface potential.  The region of bistability is much greater
here than that obtained using the HNW potential, extending over a wider range of density 
and $k_S$. Also for a given state point, this potential induces larger bistability values. 
This makes the RSUP model a very good candidate for the modeling of switching between the 
two arrangements on a bistable surface.\\
%
The improved bistability of the RSUP model when compared with the HNW
model lies in the difference between the mechanisms driving the surface-induced anchoring. For the
HNW, the competition between the planar and homeotropic alignment is driven by the
amount of volume that can be absorbed into the surface for each alignment. 
This is slightly different from the RSUP case where the planar alignment is the natural 
state of the system and does not rely on the particles absorbing the surface; homeotropic 
alignment is introduced as an alternative to this natural states by increasing the possibility 
of absorption when reducing $k_S$. 
As a result the free energy minima corresponding to the two alignments for the two potentials 
are subtly different. Although free energy data were not determined in this study, 
the stronger bistability obtained for the RSUP model suggest it has a higher 
and wider energy barrier between the two locally stable  
alignment states.\\


The order phase diagrams for the RSUP model (Fig~\ref{fig:SWaPhaseDia_k3_RSUP}) show the 
same general features as
the corresponding diagrams calculated for the HNW model. In the bulk region, the two data
sets are very similar. However, the diagrams for the interfacial region present some differences
in that the high symmetry around the transition line and the strong disordering of
the particles at the transition are diminished somewhat. This can be attributed to the
difference in the $\rho^{*}_\ell$ profiles for the two potentials, the profiles for the 
RSUP potential lacking the disorder-related
double peak behaviour. As a result, in the case of competing alignment, the local surface
order was not reduced due to particles diffusing between the two regions corresponding to the two
density peaks.\\


