

\chapter{More on confined geometries}
\label{chap:five}
%===============================================================================
%===============================================================================

\introduction

Simulations of confined systems of hard Gaussian overlap particles interacting with the
substrates through the hard needle wall potential have shown that, despite the simplicity 
of the model, a range of surface induced behaviour can be observed. Using this simple setup 
a thorough and systematic study of the surface induced structural changes has been performed 
and an anchoring transition has been identified. In this Chapter, the focus is 
brought to bear on the study of alternative confined system configurations.\\
%
First, more realistic surface potentials are studied and their anchoring phase diagrams computed 
so as to investigate their suitability for the modeling of anchoring transitions. Two such 
potentials are of interest, namely the rod-sphere potential and the rod-surface potential. For
each of these, the the surface induced arrangements are studied and the possibility of 
bistable regions is explored.\\
%
The second part of this Chapter contains a study of hybrid anchored systems performed
using the hard needle wall potential; following this, the possibility of simulating 
easy switching between the hybrid aligned nematic and vertical states of the cell is investigated.


%===============================================================================
%===============================================================================

\section{Realistic surface potentials : the RSP.}
\label{s:realisticPotentials_RSP}


%===============================================================================================
%===============================================================================================
\subsection{The rod-sphere potential.}
\label{ss:RSP_potential}


%====================================================================
%		POTENTIAL DESCRIPTION AND LITTERATURE REVIEW
%====================================================================

The rod-sphere surface potential (RSP) describes the interaction between a Gaussian ellipsoid and a
sphere located in the surface plane and with the same $x$ and $y$ coordinates as the ellipsoid. 
Again, the particles (HGO) do not interact directly with the substrate, rather another
HGO ellipsoid is inserted in each particle (\eg Figure~\ref{fig:RSPConfig}). 
This inner ellipsoid interacts with the surface through $\mathcal{V}^{RSP}$ as~:
%
\begin{equation}
	\mathcal{V}^{RSP} = \left\{	%}
	\begin{array}{ccc}
		0	&\mathrm{if}	&|z_i - z_0| \geq \sigma^{RSP}_w	\\
		\infty	&\mathrm{if}	&|z_i - z_0| < \sigma^{RSP}_w	
	\end{array}
	\right.
\end{equation}
%
where $\sigma^{RSP}_w$ is the contact distance for the interaction between a hard Gaussian 
overlap particle  with length $\sigma_\parallel$ and breadth $\sigma_\perp$ and a sphere of diameter 
$\sigma_j$. The contact distance for such an interaction is given by Equation 
(4) of~\cite{BernePechukas72}~:
%
\begin{equation}
	\sigma^{RSP}_w = \sqrt{
	\frac{\sigma^2_\perp + \sigma^2_j}
	{1 - \chi(\dotproduct{u_i}{r_{ij}})^2 }
	}
	\label{eqn:sw_RSP_BP}
\end{equation}

For implementation into a simulation code, Equation~\ref{eqn:sw_RSP_BP} is best written
in terms of $\theta$ and $\sigma_0$~; recalling that the unit of distance 
$\so = \sigma_\perp\sqrt{2}$ and for convenience, imposing $\so = \sigma_j\sqrt{2}$. 
Also to enable
comparison with the hard needle wall potential, the sphere is taken to be tangent with the
substrate so as to keep it out of the simulation box. This leads to the final expression for
$\sigma^{RSP}_w$ as used in the simulations~:
%
\begin{equation}
	\sigma^{RSP}_w = \sigma_0\lp \frac{1}{\sqrt{1 - \chi_S\cos^2\theta}} - \frac{1}{2} \rp
	\label{eqn:sigma_w_RSP}
\end{equation}
%
with~:
%
\begin{equation}
	\chi_S = \frac{k^2_S - 1}{k^2_S + 1}
\end{equation}
%
$k_S$ being the length to breadth ratio of the inner ellipsoid. A graphical representation of
this contact distance  as a function of $k_S$ and $\theta$ is shown on
Figure~\ref{fig:sw_RSP}.\\

%------------------------------------------
\picW = 10cm
\begin{figure}
	\centering
	\pic{RSPConfig.ps}
	\caption{Representation of the geometry used for the interaction between the inner HGO
	particle and the sphere representing the substrate in the RSP surface potential.}
	\label{fig:RSPConfig}
\end{figure}

%------------------------------------------

%------------------------------------------
\picW = 10cm
\begin{figure}
	\centering
	\pic{sw_RSP.ps}
	\caption{Representation of $\sigma^{RSP}_w(k_S, \cos\theta)$ for the RSP surface potential.}
	\label{fig:sw_RSP}
\end{figure}
%------------------------------------------

The rod-sphere model has already been used as the contact distance for a soft surface potential in 
studies of confined Gay-Berne particles in single component systems~\cite{ZhangChakrabarti96, 
WallCleaver97,TeixeiraChrzanowska01},  binary mixtures~\cite{LathamCleaver00} 
and switching situations~\cite{rwThesis}.  In all of these, full particles (\ie $k_S = k$) 
were used and tilted layers were observed in the interfacial regions.
In~\cite{WallCleaver97,TeixeiraChrzanowska01} the tilt was explained to be a consequence of 
the competition 
between packing constraints and the form of the surface potential. In other words, the attractive 
part of the potential was thought to be responsible for the tilt, the authors noticing that increasing 
the  particle-surface coupling $\alpha$ (see equation (9) of~\cite{WallCleaver97}) induced a tighter 
distribution of the particle orientations about the optimal tilt angle. It is interesting to
note however, that using the same surface potential, but a molecular elongation $k=2$ instead of
$3$, Wall and Cleaver~\cite{WallCleaver03} found that the surface anchoring changed from tilted
to planar.\\
Another case of surface tilted arrangement was obtained in the study by Lange and 
Schmid~\cite{LangeSchmid02,LangeSchmid02a,LangeSchmid02c} where 
Gay-Berne particles were confined between two structureless walls. There, the surface potential 
used was
similar to the RSP, but a different $\sigma_w$ function was used so as to describe the
interaction between a surface and an ellipsoid of revolution; also, they used the common $12-6$
Gay-Berne potential rather the $9-3$ version of~\cite{WallCleaver97}. The particles simulated were found
to exhibit planar anchoring, but tilted phases were obtained by inclusion of polymer chains
grafted on to the surface. The tilt in that case resulted from competition between the planar
orientation favoured by the solvent particles and the homeotropic alignment preferred by the 
polymer chains (since it reduced their bond energy). Again, in that study, tilting behaviour could 
be explained in terms
of the attractive parts of the potential since an anchoring transition between planar and tilted
arrangements was obtained by varying the number of grafted polymer chains.\\
Ascribing the tilt to the attractive part of the potential in~\cite{WallCleaver97} was also 
consistent with the many theoretical treatments of confined hard particle 
systems~\cite{HolystPoniewierski88,Chrzanowska_Teixera_01}~: 
none of these predict tilted surface alignment, planar and homeotropic alignments being the only
arrangement predicted.\\
It is interesting to note, however, that when using a surface made of fixed atoms
and a similar surface potential to that used in~\cite{WallCleaver97},  
Palermo~\etal\cite{PalermoBiscarini98} 
did not find any tilt arrangements, rather the natural planar arrangement was adopted at 
the substrates. This discrepancy would suggest that the interaction described by the rod-surface
potential where the particles and substrate sphere have always the same $x$ and $y$ coordinate
might be responsible for the tilting behaviour. However, this argument can not be proved just by
comparing the results from these existing studies as they employed different forms of the 
Gay-Berne type surface interaction potential, a 9-3 type in~\cite{WallCleaver97} as opposed to 12-6
in~\cite{PalermoBiscarini98}.\\


The case of short $k_S$, that is the case of particles absorbing the substrates, 
has not been considered in these studies. This case can, however,  be readily
understood. If the amount of volume absorbed is great enough to induce a significant 
reduction in the free energy, then a homeotropic arrangement should be more stable, as borne out
by the simulation studies of Allen~\cite{Allen99}.

%===============================================================================================
%===============================================================================================
\subsection{Simulation results using the RSP.}
\label{ss:RSPresults}
%====================================================================
%			RESULTS - PHASE DIAGRAMS
%====================================================================

Further investigations of the
surface induced structural changes obtained using the RSP surface potential were performed using
Monte Carlo computer simulations in the canonical ensemble. Systems of $N=1000$ hard Gaussian overlap
particles with elongation $k=3$ confined in an infinitely wide slab geometry of fixed height 
$L_z=4k\sigma_0$ were considered, the walls being situated at $z_0 = \pm\frac{L_z}{2}$ and
symmetric anchoring conditions applied. Sequence of
simulations were performed at constant number density $\rho^{*}$ and decreasing 
$k_S$ for several values of $\rho^{*}$ and the surface induced structural changes were studied 
using the observable 
profiles ($\rho^{*}_\ell(z)$, $Q_{zz}(z)$ and $P_2(z)$) introduced in the previous Chapter. 
From these profiles, the anchoring and order phase diagrams were computed in the 
interfacial and bulk regions. These diagrams are shown on Figures~\ref{fig:RSP_QzzWa} 
and~\ref{fig:RSP_P2Wa} and were computed using a similar method to that given in the previous
Chapter. The difference here was that
definition of the boundary between the interfacial and bulk regions was changed
such that the interfacial region was taken to extend from the surface to the second 
maximum of $\rho^{*}_\ell$
regardless of the surface arrangement obtained for this model. The reason for this definition 
change lies in the similarity
between density profiles obtained for this model at different values of $k_S$:
the primary peaks in $\rho^{*}_\ell(z)$ were always situated at $|z-z_i|> 0.0$ 
(see \eg Figures~\ref{fig:RSP_typeProf_k3_homeo} and~\ref{fig:RSP_typeProf_k3_planar}). 
Further details regarding these profiles are given later in this Section.\\

\picW = 7cm
\begin{figure}
	\centering
	\subfigure[Interfacial region.]{\pic{QzzWaSu_RSPphaseDia_k3_kSD.ps}}
	\subfigure[Bulk region.]{\pic{QzzWaBu_RSPphaseDia_k3_kSD.ps}}
	\caption{Anchoring phase diagrams obtained from series of simulations of $N=1000$
	confined HGO particles with $k=3$ at constant density and decreasing $k_S$ 
	and using the RSP surface potential.}
	\label{fig:RSP_QzzWa}
\end{figure}

\begin{figure}
	\centering
	\subfigure[Interfacial region.]{\pic{P2WaSu_RSPphaseDia_k3_kSD.ps}}
	\subfigure[Bulk region.]{\pic{P2WaBu_RSPphaseDia_k3_kSD.ps}}
	\caption{Order phase diagrams obtained from series of simulations of $N=1000$
	confined HGO particles with $k=3$ at constant density and decreasing $k_S$ 
	and using the RSP surface potential.}
	\label{fig:RSP_P2Wa}
\end{figure}

\picW = 7cm
\begin{figure}
	\centering
	\subfigure[$k^{'}_S = 0.0$]
	{\pic{HGO_box_NVT_RSconf_k3_N1000_kS000_d0.3500_0.50M.ps}}
	%
	\subfigure[$k^{'}_S = 1.0$]
	{\pic{HGO_box_NVT_RSconf_k3_N1000_kS100_d0.3500_0.50M.ps}}
	\caption{Typical configuration snapshots showing the surface induced homeotropic (a) and 
	tilted (b)  surface induced arrangements for confined systems of $N=1000$ HGO particles 
	using $\mathcal{V}^{RSP}$ for surface interactions and $\rho^{*} = 0.35$.}
	\label{fig:RSP_snaps}
\end{figure}
%---------------------------------------------------------------------------

Observation of the order and anchoring phase diagrams reveals that the $\overline{P}_2$ diagrams 
are qualitatively
similar to those obtained with the HNW models, whereas there are some  qualitative 
differences in the $\overline{Q}_{zz}$ diagrams.\\
In the case of short $k_S$, the $\overline{Q}_{zz}$ behaviour for the RSP model 
is not unlike that of the HNW model and confirms the predictions made at the end of the  
last Section. Throughout the
density range considered here and for short $k_S$, the system adopts an homeotropic arrangement
where order increases with increased density. This is further confirmed by 
configuration snapshots (\eg Figure~\ref{fig:RSP_snaps}(a) for the state point
$\rho^{*}=0.35$, $k_S/k=0.0$).\\
In the case of long $k_S$ (\ie $k_S/k>0.6$), however, there is a qualitative difference between 
the diagrams shown in  Figure~\ref{fig:RSP_QzzWa} and their equivalent for the HNW model. 
Here, throughout the density range considered, the value of $\overline{Q}_{zz}$ does not reach 
that expected for planar ordering, remaining, instead, low at about $0.2$. Although such
values can be understood in the low density regime, where $\overline{P}_2$ is compatible with
an isotropic phase, this behaviour is rather more surprising in the case of high densities
where the corresponding $\overline{P}_2$ diagram shows nematic order. In these latter regions, 
the low values of
$\overline{Q}_{zz}$ are, however, compatible with a tilted arrangement. 
Observation of typical snapshots for these high densities 
(\eg Figure~\ref{fig:RSP_snaps}(b) ) confirms the presence of a tilt, showing a
phase where the average surface alignment is of about $\pi/4$ radians.\\


%=================================================================================================
%		RESULTS - PROFILES
%=================================================================================================

\picW = 12cm
\begin{figure}
	\centering
	\pic{RSP_typeProf_k3_homeo.ps}
	\caption{Typical $z$-profiles for confined systems of HGO particles with 
	$k=3.0$ and $k^{'}_S = 0.0$ using the RSP potential.}
	\label{fig:RSP_typeProf_k3_homeo}
\end{figure}

\begin{figure}
	\centering
	\pic{RSP_typeProf_k3_planar.ps}
	\caption{Typical $z$-profiles for confined systems of HGO particles with 
	$k=3.0$ and $k^{'}_S = 1.0$ using the RSP potential.}
	\label{fig:RSP_typeProf_k3_planar}
\end{figure}

Further details of the surface induced structural changes obtained using the RSP potential 
can be obtained from appropriate z-profiles. These are shown for two states points 
corresponding to homeotropic and tilted arrangements in 
Figures~\ref{fig:RSP_typeProf_k3_homeo} and~\ref{fig:RSP_typeProf_k3_planar} respectively.
These profiles, share some of the features of the equivalent profiles obtained with the HNW
potential.
%%
The case $k^\prime_S = 0.0$ corresponds to a  homeotropic arrangement. This is characterized by
positive values of $Q_{zz}(z)$ and peak separations in the oscillations of $\rho^{*}_\ell(z)$ 
of about $\sel$. The quality of in-plane ordering is similar to that observed with the previous
surface potential.\\
%
For $k^\prime_S = 1.0$, however, the situation is very different, and there is little similarity 
between the profiles at $\kSp = 1$ for the RSP and HNW potentials.
The layering shown in Figure~\ref{fig:RSP_typeProf_k3_planar} is not as well defined and only 
two peaks  can  be clearly observed in $\rho^{*}_\ell(z)$. Moreover, the peak separation is much 
larger than $\so$.  Finally, $Q_{zz}(z)$ fails to display the negative values associated with 
planar ordering,  even at high density or which the corresponding $P_2(z)$ profile indicates 
an ordered phase. Those features correspond to a tilted arrangement.\\
%
%



%===============================================================================================
%===============================================================================================
\subsection{Origin of the tilt}


%====================================================================
%			ANALYTICAL ANALYSIS
%====================================================================
Here, the origins of this tilting behaviour are revisited by studying the form of the RSP 
as  a function of $k_S$. In Appendix~\ref{chap:A}, it is shown that for a particle with 
elongation $k$ whose inner 
ellipsoid is in contact with the substrate surface, the volume $Ve(k_S,\theta)$ absorbed into 
the surface is given by~:
%
\begin{equation}
	Ve=\frac{1}{3}\pi\lp \frac{1}{2}-\sqrt{\frac{\sigma^{RSP}_w}{k^2\cos^2\theta+\sin^2\theta}}\rp^2 
	\lp 1 + \sqrt{\frac{\sigma^{RSP}_w}{k^2\cos^2\theta+\sin^2\theta}} \rp
	\label{eqn:Ve_RSP}
\end{equation}
%
A graphical representation of this absorbed volume is given in Figure~\ref{fig:Ve_RSP_fkS} 
for a particle of
elongation $k=3$.  The preferred surface induced arrangements can be associated with the 
maxima in $Ve(k_S,\theta)$. For short $k_S$, $Ve(k_S,\theta)$ is maximal at $\theta = 0$ 
and, therefore, the most stable arrangement is homeotropic. In the limit of $k_S=k$, however
$Ve(k_S,\theta)$ is maximal  for intermediate $\theta$, which suggests that a tilted 
arrangement may be most stable.\\
%

More insight into this result can be found in the expression of the surface potential
(Equation~\ref{eqn:sigma_w_RSP}). In the case $k_S=k$, $\sigma^{HNW}_w$ 
represents the distance from the substrate to the particle's centre of mass when one of its
needle's ends
is in contact with the surface plane.  $\sigma^{RSP}_w$, in contrast, 
indicates whether or not the HGO particle overlaps a sphere embedded within the substrate.\\
The difference between the two shape parameter (Figure~\ref{fig:cmpHNW_RSP})
shows that there are some tilt angle for which $\sigma^{RSP}_w$ is smaller than $\sigma^{HNW}_w$, 
that is the particle ends are able to overlap the surface plane. This region of reduced 
$\sigma^{RSP}_w$  coincides with the maximum in $Ve(k_s,\theta)$ and, therefore, can be 
associated with the tilt behaviour.\\

%=================================
\picW = 10cm
\begin{figure}
	\centering
	\pic{Ve_RSP_fkS.ps}
	\caption{Representation of $Ve(kS,\theta)$ for the RSP potential and $k=3$.}
	\label{fig:Ve_RSP_fkS}
\end{figure}
%=================================

%=================================
\begin{figure}
	\centering
	\picL{cmpHNW_RSP.ps}
	\caption{Comparison between $\sigma^{HNW}_w$ (solid line) and $\sigma^{RSP}_w$ (dashed
	line). The dotted line represents the difference between the two 
	($\sigma^{RSP}_w-\sigma^{HNW}_w$.)}
	\label{fig:cmpHNW_RSP}
\end{figure}
%=================================


%=================================
\picW = 10cm
\begin{figure}
	\centering
	\pic{Ve_RSP_fk.ps}
	\caption{Representation of $Ve(k,\theta)$ for the RSP potential and $k_S=k$.}
	\label{fig:Ve_RSP_fk}
\end{figure}
%=================================

The optimum tilt angle $\theta_{tilt}$ for which the absorbed volume of a single particle is 
maximal can be
calculated for different values of k.  By considering the absorbed volume given by 
Equation~\ref{eqn:Ve_RSP} and setting $k_S=k$, an expression
for $Ve(k,\theta)$ (Figure~\ref{fig:Ve_RSP_fk}) can be obtained.
$\theta_{tilt}$, the angle which maximizes $Ve(k,\theta)$, is the angle that
solves~:
\begin{equation}
	\frac{d}{d\theta}Ve(k,\theta) = 0
	\label{eqn:dVeEq}
\end{equation}
%
where $\frac{d}{d\theta}Ve(k,\theta)$ is given by~:
%
\begin{equation}
	\frac{d}{d\theta}Ve(k,\theta) = k\pi \lp\frac{A_0\lp B_0 + C_0 \rp}{D_0} 
	-\frac{A_0 F_0 \lp B_0 + C_0 \rp}{I_0} 
	\rp.
\end{equation}
%
Here~:
%
\begin{eqnarray*}
	A_0 &=& \lp \frac{1}{2} - \sqrt{\frac{A^2}{B}} \rp^2	\\
	B_0 &=& -\frac{A^2\lp2\cos\theta\sin\theta - 2k^2\cos\theta\sin\theta\rp}{B^2}	\\
	C_0 &=& -\frac{2A(k^2-1)\cos\theta \sin\theta}{(k^2+1)BC^\frac{3}{2}}	\\
	D_0 &=& 6\sqrt{\frac{A^2}{B}}	\\
	F_0 &=& 1+\sqrt{\frac{A^2}{B}}	\\
	I_0 &=& 3\sqrt{\frac{A^2}{B}}
\end{eqnarray*}
%
and~:
%
\begin{eqnarray*}
	A &=& \frac{1}{\sqrt{C}} - \frac{1}{2}	\\
	B &=& k^2\cos\theta + \sin^2\theta	\\
	C &=& 1 - \frac{(k^2-1)\cos^2\theta}{1+k^2}.\\
\end{eqnarray*}

%=================================
\picW = 10cm
\begin{figure}
	\centering
	\pic{dVeContour_RSP_fk.ps}
	\caption{Representation of $\theta_{tilt}(k)$ using the RSP potential and $k_S=k$.}
	\label{fig:dVeContour_RSP_fk}
\end{figure}
%=================================


Equation~\ref{eqn:dVeEq} has been solved numerically by computation of the contour of 
$\frac{d}{d\theta}Ve(k,\theta)$ at level $0$, as shown on Figure~\ref{fig:dVeContour_RSP_fk}. 
This shows that $\theta_{tilt}$ is fairly constant at about 0.9 radians, 
that is about $50$ degrees.
As a result, the above treatment suggest that configurations with a tilt angle of about $50$ 
degrees are expected to be favoured from simulations of full HGO particles confined 
with the RSP potential.
However, as many body effects  have not been considered here, the existence of 
such a tilt in a bulk system is not assumed by this result.\\

That said, the simulations presented in Section~\ref{ss:RSPresults} clearly show that such a 
tilt does develop when
using the RSP as a surface potential. Although, the tilt angle was not directly available from
observation of the profiles, it can be estimated by use of the definition of
$Q_{\alpha\beta}$. At the state point $\rho^{*} = 0.35$ and $k_S/k = 1.0$ and at the z location
where $\rho^{*}_\ell(z)$ is maximal, $Q_{zz} = 0.209$, this latter value corresponds to the 
the simulation average. This value of $Q_{zz}$ corresponds to an average tilt angle 
$\theta = 0.812$ radians, that is  $46.6^\circ$. This value is consistent 
with the angle observable on the configuration snapshots.\\
The difference between the observed and predicted tilt angles (of $9.7\%$) can be understood from the 
packing constraints. The packing improves with lower tilt angles which can, in turn, 
increase system's total absorbed volume. However the absorbed volume of each particle decreases as the
difference between its optimal and actual tilt angles increases. This creates a competition 
between the amount of absorbed volume that can be obtained with a higher packing fraction 
but lower average tilt angle and that obtained with an average tilt angle closer to the 
optimal single particle angle but lower packing fraction.\\

The simulations and the theoretical treatment described in this Section
have shown that a tilted phase can be both predicted and obtained with a purely steric model.
As a result, it appears that the tilted phases obtained 
in~\cite{ZhangChakrabarti96, WallCleaver97,TeixeiraChrzanowska01, LathamCleaver00} arise due to
the geometrical characteristics of the rod-sphere potential, rather than competition between
packing density and attractive particle-particle and particle-wall interactions. 
This explanation is consistent with the
change from tilted to planar surface alignment observed by Wall and 
Cleaver~\cite{WallCleaver97,WallCleaver03}
when they reduced the molecular elongation from $k=3$ to $k=2$. In the latter
case, the molecules were too short to significantly absorb at the surface and therefore adopt the
planar state. Observation of Figure~\ref{fig:Ve_RSP_fk} at $k=2$ confirms this, as for this
elongation, the absorbed volume is virtually independent of molecular orientation and, therefore,
does not form a tilted arrangement.\\
In the light of this explanation, it seems
reasonable to assume that a planar surface arrangement would have been obtained if the
simulations of~\cite{WallCleaver97,TeixeiraChrzanowska01, LathamCleaver00} had been performed using
a lattice of fixed spheres to represent the surface, as was done in~\cite{PalermoBiscarini98}.\\






\section{Realistic surface potentials~: the RSUP.}
\label{s:realisticPotentials_RSUP}



%===============================================================================================
%===============================================================================================
\subsection{The rod-surface potential.}

The rod-surface potential (RSUP) represents an alternative interaction between a Gaussian 
ellipsoid and a plane and is given by~:
\begin{equation}
	\mathcal{V}^{RSUP} = \left\{	%}
	\begin{array}{ccc}
		0	&\mathrm{if}	&|z_i - z_0| \geq \sigma^{RSUP}_w	\\
		\infty	&\mathrm{if}	&|z_i - z_0| < \sigma^{RSUP}_w	
	\end{array}
	\right.
\end{equation}
This time, the contact distance for this is obtained by integration of the rod-sphere potential 
(without the $\frac{\so}{2}$ shift) over the x-y plane leading to~\cite{rwThesis}~:
%
\begin{equation}
	\sigma^{RSUP}_w = \sigma_0\sqrt{\frac{1-\chi_S\sin^2\theta}{1-\chi_S}}
\end{equation}
%
with the same definition for $\chi_S$ as with the RSP potential. This potential can be thought 
of as being equivalent to the RSP but with the important difference that each particle
effectively interacts  with an infinity 
of spheres as opposed to just one. Again a shift is introduced so as to remove the virtual
spheres from the simulation box. The contact distance used in the simulation is therefore given by~:
\begin{equation}
	\sigma^{RSUP}_w = \sigma_0\lp \sqrt{\frac{1-\chi_S\sin^2\theta}{1-\chi_S}} -
	\frac{1}{2}\rp.
\end{equation}


%------------------------------------------
\picW = 10cm
\begin{figure}
	\centering
	\pic{RSUPConfig.ps}
	\caption[Representation of the geometry used for the interaction between the inner HGO
	particle and the surface in the RSUP potential.]
	{Representation of the geometry used for the interaction between the inner HGO
	particle and the surface in the RSUP potential. The three spheres represent the
	substrate which is really made of an infinity of such spheres located between the horizontal lines 
	which effectively mark the substrate location.}
	\label{fig:RSUPConfig}
\end{figure}

%------------------------------------------


%------------------------------------------
\picW = 10cm
\begin{figure}
	\centering
	\pic{sigmaRSUP.ps}
	\caption{Representation of $\sigma^{RSUP}(k_S,\theta)$ for $k=3$ using the RSUP
	potential.}
	\label{fig:sigma_RSUP}
\end{figure}


%------------------------------------------

A representation of $\sigma^{RSUP}_w(k_S,\theta)$ is given in Figure~\ref{fig:sigma_RSUP} for
$k=3$. Again, the expression for the absorbed volume into the surface can be used to predict the
surface behaviour of this model. In the case of the RSUP potential, this volume reads~:
%
\begin{equation}
	V_e^{RSUP}=\frac{1}{3}\pi\lp \frac{1}{2}-
	\sqrt{\frac{\sigma^{RSUP}_w}{k^2\cos^2\theta+\sin^2\theta}}\rp^2 
	\lp 1 + \sqrt{\frac{\sigma^{RSUP}_w}{k^2\cos^2\theta+\sin^2\theta}} \rp
	\label{eqn:Ve_RSUP}
\end{equation}
%
A graphical representation of this volume is shown on Figure~\ref{fig:Ve_RSUP_fks}. 
In the limit $k_S = 0$, $Ve^{RSUP}(k_S,\theta)$ has its maximum at $\theta = 0$ thus indicating 
an homeotropic arrangement.\\
%
In the limit $k_S = k$, $V_e^{RSUP}$ is close to zero for all $\theta$ and has a small maximum at 
$\theta = 0$. However, by design, $\sigma_w^{RSUP}$ forbids any particle adsorption into the
substrate if $k_S=k$; this is even further illustrated by the value of
$\sigma_w^{RSUP}(k_S=k,\theta=0)$ which is equal to the contact distance between a HGO particle
with $\theta=0$ and a sphere. As a result, this small maximum in $V_e^{RSUP}$ can be explained
to be a result of the approximation of using ellipsoidal shaped particles 
in Appendix~\ref{chap:A} when deriving
the expression for $V_e^{RSUP}$ and, therefore, $V_e^{RSUP}(k_S=k),\theta$ should, in fact,
be zero for all
values of $\theta$. A consequence of this, the $\theta=0$ peak in Figure~\ref{eqn:Ve_RSUP} 
does not represent a 
stable surface arrangement of the RSUP model in the case $k_S=k$, as this is not 
absorption driven. Rather, it can be safely assumed that the stable arrangement for this system 
is planar, in common with the findings of previous 
theoretical and simulation work on rod-shaped objects absorbed at planar 
surfaces~\cite{HolystPoniewierski88,Chrzanowska_Teixera_01,VanRoijDijkstra00,DijkstraVanRoij01}.


%------------------------------------------
\picW = 10cm
\begin{figure}
	\centering
	\pic{Ve_RSUP_fkS.ps}
	\caption{Representation of $Ve(k_s,\theta)$ for the RSUP potential and $k=3$.}
	\label{fig:Ve_RSUP_fks}
\end{figure}
%------------------------------------------

The mechanism expected to drive an anchoring transition with the RSUP potential is slightly
different from that seen with the HNW potential. With the latter, the surface rearrangement 
is mainly driven
by the molecular volume that can be absorbed into the surface, and the transition
from homeotropic to planar arrangements occurs when the volume absorbed by the
latter is greater than with the former arrangement. In the case of the RSUP potential, however, 
there is
no absorption in the case of the planar arrangement. However, this is the base state of
any rod-shaped object in contact with a surface. Thus, as $k_S$ is decreased from  $k_S=k$ to $k_S=0$, 
the volume that can be absorbed in a homeotropic arrangement gradually increases. In this case,
therefore, an anchoring transition from planar to homeotropic arrangement 
is expected when the volume that can be absorbed by an homeotropic surface induces a total free energy
lower than in that of the planar base state.


%===============================================================================================
%===============================================================================================
\subsection{Simulation results obtained using the rod-surface potential.}

The surface induced structural changes obtained from the rod-surface potential have been studied using
Monte Carlo simulations in the canonical ensemble on systems of $N=1000$ HGO particles with 
elongation $k=3$. The simulation slab was the same as that used previously, with the walls situated 
on the top and bottom of the cell with constant height $L_z = 4k\sigma_0$ and symmetric 
anchoring conditions. Two series of simulations at each chosen density were performed 
with,respectively, increasing and decreasing $k_S$.
Typical $z$-profiles for this model are shown on Figures~\ref{fig:RSUP_typeProf_k3_homeo} 
and~\ref{fig:RSUP_typeProf_k3_planar} respectively for $k_S=0.0$ and $k_S = k$.\\

%------------------------------
\picW = 12cm
\begin{figure}
	\centering
	\pic{RSUP_typeProf_k3_homeo.ps}
	\caption{Typical $z$-profiles for confined systems of $N=1000$ HGO particles with 
	$k=3.0$ and $k^{'}_S = 0.0$ using the RSUP potential.}
	\label{fig:RSUP_typeProf_k3_homeo}
\end{figure}
%------------------------------


%------------------------------
\picW = 12cm
\begin{figure}
	\centering
	\pic{RSUP_typeProf_k3_planar.ps}
	\caption{Typical $z$-profiles for confined systems of $N=1000$ HGO particles with 
	$k=3.0$ and $k^{'}_S = 1.0$ using the RSUP potential.}
	\label{fig:RSUP_typeProf_k3_planar}
\end{figure}
%------------------------------

%===================================================
\picW = 6cm
\begin{figure}
	\centering
	\subfigure[Simulations with decreasing $k_S$]{\pic{QzzWa_HGO_RSU_k3_fkS_S1_Su.ps}\pic{QzzWa_HGO_RSU_k3_fkS_S1_Bu.ps}}
	\subfigure[Simulations with increasing $k_S$]{\pic{QzzWa_HGO_RSU_k3_fkS_S3_Su.ps}\pic{QzzWa_HGO_RSU_k3_fkS_S3_Bu.ps}}
	\subfigure[Bistability diagrams]{\pic{QzzWa_HGO_RSU_k3_fkS_SuBist.ps}\pic{QzzWa_HGO_RSU_k3_fkS_BuBist.ps}}
	\caption{Anchoring phase diagrams obtained from series of simulations of $N=1000$
	confined HGO particles with $k=3$ at constant density and decreasing $k_S$ 
	using the RSUP surface potential. Diagrams on the l.h.s are relative to the interfacial
	region and those on the r.h.s are relative to the bulk region.}
	\label{fig:QzzWaPhaseDia_k3_RSUP}
\end{figure}
%===================================================



%===================================================
\picW = 7cm
\begin{figure}
	\centering
	\subfigure[Simulations with decreasing $k_S$]{\pic{SWa_HGO_RSU_k3_fkS_S1_Su.ps}\pic{SWa_HGO_RSU_k3_fkS_S1_Bu.ps}}
	\subfigure[Simulations with increasing $k_S$]{\pic{SWa_HGO_RSU_k3_fkS_S3_Su.ps}\pic{SWa_HGO_RSU_k3_fkS_S3_Bu.ps}}
	\caption{Order phase diagrams obtained from series of simulations of $N=1000$
	confined HGO particles with $k=3$ at constant density and decreasing $k_S$ 
	using the RSUP surface potential. Diagrams on the l.h.s are relative to the interfacial
	region and those on the r.h.s are relative to the bulk region. The bistability is
	negligible.}
	\label{fig:SWaPhaseDia_k3_RSUP}
\end{figure}
%===================================================



In the limit of $k_S=0$, the surface induced structural changes for this model are very
similar to their counterparts with the RSP model (Figures~\ref{fig:RSP_typeProf_k3_homeo} 
and~\ref{fig:RSP_typeProf_k3_planar}); the two sets of profiles
are virtually indistinguishable. The surface arrangement is homeotropic which explains the
very strong similarities between the two sets; with $\theta\sim0$, 
both models induce the same geometry between the particles and the substrate.\\
%
In the limit $k_S=k$, the surface induced effects for $\mathcal{V}^{RSP}$ and $\mathcal{V}^{RSUP}$
are very different.
The short peak separation in $\rho^{*}_\ell(z)$ and the negative values in $Q_{zz}(z)$, coupled with 
the high values of $P_2(z)$, indicate an induced planar surface arrangement very much in
agreement with the predictions made earlier. This is further confirmed by the
similarity of the planar arrangement profile features for the HNW and RSUP models. The main difference
between the two arises because with the RSUP model, planar particles are not allowed to 
absorb at the surface. 
This leads to the regions of zero $\rho^{*}_\ell(z)$ with a width of $0.5\sigma_0$ close to 
each substrate.\\

The full surface induced behaviour of this system has been computed as a function of $k_S$ and 
$\theta$ using the anchoring and order phase diagrams shown in 
Figures~\ref{fig:QzzWaPhaseDia_k3_RSUP} and~\ref{fig:SWaPhaseDia_k3_RSUP}. 
The convention adopted to distinguish the interfacial 
from the bulk region was the same as that used with the RSP model.
%
The anchoring phase diagrams are given in Figure~\ref{fig:QzzWaPhaseDia_k3_RSUP}(a) 
and (b) for, respectively, decreasing and increasing $k_S$. From those
a strong difference between the two sets can be observed. The corresponding bistability 
phase diagrams (Figure~\ref{fig:QzzWaPhaseDia_k3_RSUP}c)  reports a very wide and strong 
bistability behaviour for this surface potential.  The region of bistability is much greater
here than that obtained using the HNW potential, extending over a wider range of density 
and $k_S$. Also for a given state point, this potential induces larger bistability values. 
This makes the RSUP model a very good candidate for the modeling of switching between the 
two arrangements on a bistable surface.\\
%
The improved bistability of the RSUP model when compared with the HNW
model lies in the difference between the mechanisms driving the surface-induced anchoring. For the
HNW, the competition between the planar and homeotropic alignment is driven by the
amount of volume that can be absorbed into the surface for each alignment. 
This is slightly different from the RSUP case where the planar alignment is the natural 
state of the system and does not rely on the particles absorbing the surface; homeotropic 
alignment is introduced as an alternative to this natural states by increasing the possibility 
of absorption when reducing $k_S$. 
As a result the free energy minima corresponding to the two alignments for the two potentials 
are subtly different. Although free energy data were not determined in this study, 
the stronger bistability obtained for the RSUP model suggest it has a higher 
and wider energy barrier between the two locally stable  
alignment states.\\


The order phase diagrams for the RSUP model (Fig~\ref{fig:SWaPhaseDia_k3_RSUP}) show the 
same general features as
the corresponding diagrams calculated for the HNW model. In the bulk region, the two data
sets are very similar. However, the diagrams for the interfacial region present some differences
in that the high symmetry around the transition line and the strong disordering of
the particles at the transition are diminished somewhat. This can be attributed to the
difference in the $\rho^{*}_\ell$ profiles for the two potentials, the profiles for the 
RSUP potential lacking the disorder-related
double peak behaviour. As a result, in the case of competing alignment, the local surface
order was not reduced due to particles diffusing between the two regions corresponding to the two
density peaks.\\





\section{Hybrid anchored systems.}
\label{s:hybridSystems}


In this Section the study of hybrid anchored confined systems is addressed using particles
confined in a slab geometry but with different anchoring conditions at each of the two surfaces.
This study is performed using Monte Carlo simulations of hard Gaussian overlap particles
confined in a slab geometry and interacting with the surfaces through the hard needle wall
potential. The aim here is to achieve switching between the Hybrid Aligned Nematic (HAN) and
Vertical (V) states using an electrical field as described in~\cite{DavidsonMottram02}. The
difference between the switching investigated here and that in
Reference~\cite{DavidsonMottram02} is that due the absence of flexoelectricity
in the HGO model, two-way switching is attempted by changing the sign of the particles'
dielectric anisotropy (as was done in Chapter~\ref{chap:four}) rather than the sign of the
applied field.

%===================================================================
%===================================================================
\subsection{Effect of hybrid anchoring.}
\label{ss:hybridEffect}

Here the effects of hybrid anchoring on a confined system are studied~; more
specifically the case of different arrangements (\ie planar and homeotropic at two substrates)
is of interest.

Cleaver and Teixeira~\cite{Cleaver_Teixeira_01} have already studied the structural transition
between the arrangements seeded at the surfaces of an hybrid anchored cell of hard Gaussian
overlap particles with $k=5$. They showed that the cell can exhibit either a continuous or
discontinuous transition between the homeotropic and planar arrangements according to the values
of the surface parameters. This implies that the observation of an HAN state requires an
appropriate choice of surface anchoring strengths so as to avoid any director discontinuities.
Also, achieving electric-field induced switching requires that the substrate parameters used
are compatible with those corresponding to a bistable surface.\\
%
Here, the case of hybrid anchored slabs has been investigated using Monte Carlo simulations
of systems of $N=1000$ hard Gaussian overlap particles of elongation $k=3$ and $5$
confined in a slab geometry and interacting with  the surfaces  using the hard needle
wall potential. The anchoring at the top was kept constant at $k^\prime_S=0.0$ so as to
induce strong homeotropic anchoring. The anchoring at the bottom surface was allowed to
vary using sequences of simulations with increasing and decreasing $k^\prime_S$ in the range
$[0:1]$.\\


%===========================================
%	TYPICAL PROFILES FIGS
%===========================================
\picW = 14cm
\begin{figure}
	\centering
	\pic{HGO_HNW_Hconf_typeProf_k3_homeo.ps}
	\caption{Typical profiles for a hybrid anchored slab of $N=1000$ HGO particles using the
	HNW surface potential with an homeotropic top surface
	($k^\prime_{St}=0.0$) and an homeotropic bottom surface ($k^\prime_{St}=0.2$).}
	\label{fig:HGO_Hconf_typeProfHomeo}
\end{figure}


\picW = 14cm
\begin{figure}
	\centering
	\pic{HGO_HNW_Hconf_typeProf_k3_bist.ps}
	\caption{Typical profiles for a hybrid anchored slab of $N=1000$ HGO particles using the
	HNW surface potential with an homeotropic top surface ($k^\prime_{St}=0.0$)
	and competing anchoring at the bottom surface ($k^\prime_{St}=0.5$).}
	\label{fig:HGO_Hconf_typeProfBist}
\end{figure}


\picW = 14cm
\begin{figure}
	\centering
	\pic{HGO_HNW_Hconf_typeProf_k3_planar.ps}
	\caption{Typical profiles for a hybrid anchored slab of $N=1000$ HGO particles using the
	HNW surface potential with an homeotropic top surface ($k^\prime_{St}=0.0$) and a
	planar bottom surface ($k^\prime_{St}=0.8$).}
	\label{fig:HGO_Hconf_typeProfPlanar}
\end{figure}
%===========================================

Typical profiles for systems with parameterisations at the bottom surface
corresponding, respectively, to homeotropic ($k^\prime_{Sb} = 0.2$), competing ($k^\prime_{Sb} =
0.5$) and planar anchoring ($k^\prime_{Sb} = 0.8$) are shown on
Figures~\ref{fig:HGO_Hconf_typeProfHomeo} to~\ref{fig:HGO_Hconf_typeProfPlanar}. These profiles
were obtained from the simulation sequences performed with decreasing $k^\prime_{Sb}$. Only
these results are shown because even in the case of the competing anchoring parameterisation,
both series gave very similar results. Also, the profiles or configuration snapshots obtained
from simulations with particles of elongation $k=5$ are not shown as they are very similar to
those obtained with $k=3$.\\
For all profiles, the top interfacial regions exhibit the features
typical of strong homeotropic anchoring as $k_{St}$ was kept constant at $0.0$. The bottom
surface profiles exhibit features corresponding to the values of the needle lengths used~; they
have the same characteristics as were observed in the equivalent cases with symmetric
surfaces.\\

The structural transition between the two surface arrangements, that is the change in molecular
orientation from one surface to the other can also be observed on the profiles.
At isotropic densities, the surface effects do not extend into the bulk part of the
slab and, therefore, this region remains disordered. As a result, both interfacial regions are
free of any influence from each other. As the number density is increased to values
corresponding to a bulk nematic phase, the surface induced structural changes extend much
further into the cell. As a result, the bulk region comes under the competing influences of both
surfaces. In all three cases considered here, the profiles seem to indicate a smooth transition
between the two surface arrangements, as indicated by the almost linear changes in
$\rho^{*}_\ell(z)$ and $Q_{zz}(z)$ between the surface features.\\

%The results obtained with a planar bottom surface are in agreement with the observation
%of a continuous transition in~\cite{Cleaver_Teixeira_01} for a similarly anchored slab. However
%the authors have noticed a discontinuous transition from planar to homeotropic for a situation
%which, in the light of the results obtained in Chapter~\ref{chap:four} corresponds
%to a slab with planar anchoring on one side and competing anchoring on the other. There, the
%discontinuous transition was indicated by the fluctuations in $Q_{zz}$ between positive and
%negative values. However those fluctuations at the surface with competing anchoring have not
%been observed here.\\




%===========================================
%	TYPICAL SNAPS FIGS
%===========================================
\picW = 6cm
\begin{figure}
	\centering
	\subfigure[$k^\prime_{Sb}=0.2$, $\rho^{*}=0.28$]{\pic{HGO_box_k3_Hconf_homeo_iso.ps}}
	\subfigure[$k^\prime_{Sb}=0.2$, $\rho^{*}=0.34$]{\pic{HGO_box_k3_Hconf_homeo_nem.ps}}

	\subfigure[$k^\prime_{Sb}=0.5$, $\rho^{*}=0.28$]{\pic{HGO_box_k3_Hconf_bist_iso.ps}}
	\subfigure[$k^\prime_{Sb}=0.5$, $\rho^{*}=0.34$]{\pic{HGO_box_k3_Hconf_bist_nem.ps}}

	\subfigure[$k^\prime_{Sb}=0.8$, $\rho^{*}=0.28$]{\pic{HGO_box_k3_Hconf_planar_iso.ps}}
	\subfigure[$k^\prime_{Sb}=0.8$, $\rho^{*}=0.34$]{\pic{HGO_box_k3_Hconf_planar_nem.ps}}
	\caption{Configuration snapshots for hybrid anchored slabs with a strong
	homeotropic anchoring at the top surface ($k^\prime_{St}=0.0$) and different values for
	$k^\prime_{Sb}$. Snapshots for two densities corresponding to isotropic (left) and
	nematic (right) are shown.}
	\label{fig:typeSnap_HGO_Hconf_k3}
\end{figure}

The structural transition between the surface arrangements can also be observed using
configuration snapshots (\eg Figure~\ref{fig:typeSnap_HGO_Hconf_k3}). Specifically the case of a
slab with planar and homeotropic anchoring is of interest as this corresponds to the geometry
where the electric switching is to be performed. For this situation, the snapshots suggest that a
slight discontinuity in the planar to homeotropic structural transition can be observed~; the
molecular orientation changes rapidly from that corresponding to homeotropic anchoring to that
of planar anchoring. This is confirmed by the $<P_2>$ profile
(Figure~\ref{fig:HGO_Hconf_typeProfPlanar}) which shows a low value in the bulk part of the cell
whereas the snapshots clearly indicate good order throughout the cell. These low values can be
understood by the presence of particles with very different orientations in the same slice which
in turn lowers the value of $<P_2>$. This behaviour is not apparent on $Q_{zz}(z)$ as similar
values could be obtained from a slice of $n$ particles with $\theta\sim\pi/4$ and a slice of
equal number of particles with $\theta\sim 0$ and $\theta\sim\pi/2$.\\
In the case of a bottom surface with homeotropic or competing anchoring, the profiles and
snapshots agree in indicating a continuous structural transition between the two surface induced
arrangements. These observation are consistent with the simulation of Cleaver and
Teixeira~\cite{Cleaver_Teixeira_01} who found a discontinuous structural transition between the
two surface arrangement provided the anchoring conditions of the two surfaces are made
sufficiently different.\\

%===========================================
%	ANCHORING PHASE DIA
%===========================================
\picW = 7cm
\begin{figure}
	\centering
	\subfigure[simulations with decreasing $k^\prime_{Sb}$]{\pic{QzzWa_HGO_HNW_Hconf_k3_Sb_S1.0.ps}}
	\subfigure[simulations with increasing $k^\prime_{Sb}$]{\pic{QzzWa_HGO_HNW_Hconf_k3_Sb_S3.0.ps}}
	\caption{Anchoring phase diagram  showing the evolution of~$\overline{Q}^{Sb}_{zz}$
	of a hybrid anchored slab with $k^\prime_{St}=0.0$ as a function
	of number density  $\rho^{*}$ and $k^\prime_{Sb}$. Systems of $N=1000$ HGO particles
	of elongation $k=3$
	and the HNW surface potential have been used. }
	\label{fig:anchPhaseDia_HGO_Hconf_k3}
\end{figure}


%\picW = 7cm
%\begin{figure}
%	\centering
%	\subfigure[simulations with decreasing $k^\prime_{Sb}$]
%	        {\pic{QzzWa_HGO_k5_HNW_Hconf_Sb_S1.0.ps}}
%	\subfigure[simulations with increasing $k^\prime_{Sb}$]
%	        {\pic{QzzWa_HGO_k5_HNW_Hconf_Sb_S3.0.ps}}
%	\caption{Anchoring phase diagram  showing the evolution of~$\overline{Q}_{zz}^{Sb}$
%	of a hybrid anchored slab
%	with $k^\prime_{St}=0.0$ as a function of number density
%	$\rho^{*}$ and $k^\prime_{Sb}$. Systems of $N=1000$ HGO particles of elongation $k=5$
%	and the HNW surface potential have been used. }
%	\label{fig:anchPhaseDia_HGO_Hconf_k5}
%\end{figure}
%===========================================

The combined effects of density and needle length are shown on the anchoring phase diagrams
computed for the bottom surfaces ($\overline{Q}^{Sb}_{zz}$). These were computed using the
approach adopted with symmetric systems and are shown in
Figure~\ref{fig:anchPhaseDia_HGO_Hconf_k3}.\\
%
The behaviour of $\overline{Q}_{zz}^{Sb}$ for these systems is very similar to that of
$\overline{Q}_{zz}^{Su}$ for the symmetric systems and the same remarks apply. However, one
striking difference is that, in the case of hybrid systems, the diagrams from series with
increasing and decreasing $k^\prime_{Sb}$ are very similar. This means that the hysteresis
used to establish bistable regions is not seen in those systems.\\
This change can be ascribed to the combined effects of the presence of the top surface with
strong homeotropic anchoring and the small height of the slab. As a result of these, the elastic
forces imposed on the particles at the bottom surface by those on the top surface prevent the
former from adopting a planar orientation for parameterisations corresponding to weak
competing anchoring. The consequence of this is that a planar orientation is only observed at
the bottom surface if the corresponding anchoring is strong. But this removes the
possibility of bistability.\\

In order to recover the bistable regions, it is necessary to reduce the elastic forces imposed
at the bottom surface by the homeotropic anchoring at the top surface. There are two approaches
by which to achieve this: to use a weaker homeotropic anchoring at the top surface; or to
increase the height of the slab, (and, therefore, the number of particles in the simulation
box). In the next section, the second solution is used in an attempt to regain surface
bistability.


%===================================================================
%===================================================================
\subsection{System size effect.}
\label{ss:sizeEffect}

Here the influence of the height of the slab on the surface bistability in hybrid system is
investigated. This has been performed by considering three slabs of hard Gaussian overlap
particles with respective height $L_z=4k\so$, $L_z=6k\so$ and $L_z=8k\so$ respectively . In
order to keep the width of the slabs big enough so as to avoid interactions between
particles and their own images, increase in the slab height was accompanied with an
increase in the system sizes, and the height $L_z=4k\so$, $L_z=6k\so$ and $L_z=8k\so$
correspond respectively $N=1000$, $N=1250$ and $N=2000$. Although the
cross section surface of the slabs was not equal for the three systems, the short positional
correlation of the systems used should imply that the slabs were wide enough so that only the
slab height has an effect on the observed planar to homeotropic surface transition.\\
These systems were studied using Monte Carlo simulations in the canonical ensemble and
using the hard needle wall potential for surface interactions. Extreme hybrid anchoring
conditions were considered using $k^\prime_{Sb}=1.0$ and $k^\prime_{St}=0.0$. Typical
profiles at $\rho^{*}=0.35$ are shown on Figure~\ref{fig:HGO_Hconf_typeProfile_Size}. Here, for
comparison purposes, the $z$ coordinates have been renormalized by $L_z$.\\


%=============================================
\picW = 14cm
\begin{figure}
	\centering
	\pic{systSizeEffectsProfiles.ps}
	\caption{Profiles corresponding to hybrid anchored systems of hard Gaussian overlap
	particles with $k=3$ at $\rho^{*}=0.35$ and different slab heigh and system sizes. 
	The surface potential is the HNW with parameterisation $k^\prime_{St}=0.0$ and 
	$k^\prime_{Sb}=1.0$.}
	\label{fig:HGO_Hconf_typeProfile_Size}
\end{figure}
%=============================================

\begin{figure}
	\centering
	\picW = 6cm
	\subfigure[$N=1000$, $L_z=4k\so$]{\pic{HGO_box_k3_Hconf_Size_N1000.ps}}
	\picW = 7cm
	\subfigure[$N=1250$, $L_z=6k\so$]{\pic{HGO_box_k3_Hconf_Size_N1250.ps}}

	\picW = 9cm
	\subfigure[$N=2000$, $L_z=8k\so$]{\pic{HGO_box_k3_Hconf_Size_N2000.ps}}
	\caption{Typical snapshots for hybrid anchored systems of $N=1000$(a), $1250$(b) and
	$2000$(c) hard Gaussian overlap particles with $k=3$ using the HNW surface potential
	with $k^\prime_{St}=0.0$ and $k^\prime_{Sb}=1.0$.}
	\label{fig:HGO_Hconf_snapsSize}
\end{figure}
%==============================================


On those profiles, similar interfacial behaviour can be observed for all three systems, the
profiles displaying features typical of homeotropic anchoring on the top surface and of planar
anchoring on the bottom surface.\\
%
The $\rho^{*}_\ell(z/L_z)$ and $Q_{zz}(z/L_z)$ profiles also show similar behaviour for all
three systems in the bulk region. All three show a linear increase in $Q_{zz}$ from
the bottom to the top surface. The main difference between these is that the regions of linear
behaviour extend over larger portions of the cell with increase in $L_z$, thus
indicating bigger `buffer regions' between the two surfaces.\\

More important differences can, however, be observed on the $P_2(z/L_z)$ profiles. For the
smallest system ($N=1000$), the $P_2(z/L_z)$ profiles show low values in the bulk regions
identified previously. Again, this effect is ascribed to the presence of particles with
significantly different orientations in the same analysis slice (Section~\ref{ss:hybridEffect}.)
On the other hand, the two bigger systems show very different behaviour, the corresponding
$P_2(L_z)$ profiles maintaining high values throughout the cell.\\
%
Further insight into this can be obtained from the corresponding configuration snapshots (\eg
Figure~\ref{fig:HGO_Hconf_snapsSize}). From these, the smallest system clearly shows a
discontinuity in the structural transition from planar to homeotropic as can be observed by the
rapid change from parallel to perpendicular orientations. The two bigger systems show a
different behaviour, however, the transition between the two arrangements being continuous and
smooth.\\
The modest differences between the structures and profiles for the bigger systems with
$L_z=6k\so$ and $L_z=8k\so$ suggest that there is a critical height at which the transition
between the two arrangements becomes continuous. The amount of data obtained here only allows to
conclude that this %[
critical slab height is in $]4k\so : 6k\so[$.\\ %]

The results found in this section proved to be fully compatible with the theoretical results of
\v{S}arlah and \v{Z}ummer~\cite{SarlahZummer99} who found that hybrid anchored films with a
thickness of only a few molecular lengths do not show a continuous bent-director structure.
Also, the experimental observation of Vanderbrouck~\etal\cite{VandenbrouckValignat99} confirm
the observation made in this Section as they observed that a thin film of 5CB molecules spun cast
onto silicon wafer, and thus having planar hybrid anchoring condition at respectively the solid
and free surfaces, are stable only if their thickness is greater than 20nm.

%===================================================================
%===================================================================
%\clearpage
\subsection{HAN to V states switching.}

The anchoring phase diagrams presented in Section~\ref{ss:hybridEffect} have shown no
bistability, possibly  due to the combination of too strong a homeotropic anchoring at
the top surface and the small height used. The smooth structural transitions obtained using
larger slab height in Section~\ref{ss:sizeEffect} suggest, however, that bistable behaviour may
be achievable.\\ %
As a result switching between the HAN and V states has been attempted. In order to recover the
bistability at the bottom surface, a slab of height $L_z=8k\so$ and a system size of $N=2000$
particles have been used and the bottom surface needle length has been set to $k_{Sb}/k=0.5$
which correspond to a good bistability in an equivalent symmetric anchored system. The first
simulations performed used a top surface needle length $k_{St}/k = 0.0$, but that proved to
induce too strong an homeotropic anchoring and no bistability at the bottom surface could be
observed. By gradually reducing the top surface anchoring strength, the bistability at the
bottom surface could be regained using $k^\prime_{St}=0.4$.
Using this latter value of $k_{St}$, the switching between the HAN and vertical states has been
performed using a similar sequence of simulations to that employed in Chapter~\ref{chap:four}.
The evolution of $\overline{Q}_{zz}^{St}$ and $\overline{Q}_{zz}^{Sb}$ as a function of the
number of sweeps are shown, respectively, on Figure~\ref{fig:QzzWaEvol_HGOHconfSwitch}(a) and
(b). Configuration snapshots corresponding to the last configuration of each phase
are shown on Figure~\ref{fig:snaps_HGOHconfSwitch}.\\

\picW = 10cm
\begin{figure}
	\centering
	\subfigure[$\overline{Q}_{zz}^{St}(n)$]{\picL{HGO_Hconf_N2000_Eswitch_QzzWaSt.ps}}
	\subfigure[$\overline{Q}_{zz}^{Sb}(n)$]{\picL{HGO_Hconf_N2000_Eswitch_QzzWaSb.ps}}
	\caption{Evolution of $\overline{Q}_{zz}(n)$ as a function of the number of sweeps $n$ for the
	top(a) and bottom(b) surface regions while switching a hybrid anchored system of
	$N=2000$ HGO particles with $k=3$ between the HAN and V states.}
	\label{fig:QzzWaEvol_HGOHconfSwitch}
\end{figure}

\picW = 4.7cm
\begin{figure}
	\centering
	\subfigure[$0.50.10^6$ sweeps]{\pic{HGO_box_k3_Hconf_N2000_Eswitch_01.ps}}
	\subfigure[$0.75.10^6$ sweeps]{\pic{HGO_box_k3_Hconf_N2000_Eswitch_02.ps}}
	\subfigure[$2.25.10^6$ sweeps]{\pic{HGO_box_k3_Hconf_N2000_Eswitch_03.ps}}
	\subfigure[$2.50.10^6$ sweeps]{\pic{HGO_box_k3_Hconf_N2000_Eswitch_04.ps}}
	\subfigure[$4.50.10^6$ sweeps]{\pic{HGO_box_k3_Hconf_N2000_Eswitch_05.ps}}
	\caption{Configuration snapshots of a system of $N=2000$ HGO particles with $k=3$ at
	different stages of switching between the HAN and V states.}
	\label{fig:snaps_HGOHconfSwitch}
\end{figure}

The sequence was started using previously equilibrated configuration with a HAN alignment. After
$0.5.10^6$ sweeps, an electric field $\vect{E}=E\vecth{z}$ with $E=6.0$ was applied  during
$0.25.10^6$ sweeps using $\delta\epsilon>0$ so as to align the particles along $\vecth{z}$. Upon
removal of the field, equilibrium in the vertical state was well established after $1.5.10^6$
sweeps. The field was then reapplied for $0.25.10^6$ sweeps using $\delta\epsilon<0$ so as to
align the particles perpendicular to $\vecth{z}$. Upon removal of the field, equilibrium in the
HAN state was achieved after $2.0.10^6$ sweeps.\\

This sequence shows successful switching between the HAN and vertical states of an hybrid
aligned cell corresponding to that considered in~\cite{DavidsonMottram02}. The model used here
did not include flexoelectricity and, therefore, only the easy switching namely, HAN to V if
$\delta\epsilon>0$ and V to HAN if $\delta\epsilon<0$ could be modeled. This is however very
encouraging as achievement of the easy switching implies that the reverse (`hard') switching
could, in principle, be achieved easily with an appropriate electrical parameterisation of the
model.











%===============================================================================

\conclusion

In this Chapter, two issues have been addressed. In the first part of the Chapter, two
more realistic surface potentials have been studied, namely the rod-sphere and the rod-surface
potentials. The aim of this work was to find a potential which has a more realistic basis
than the hard needle wall potential but which also displays planar and homeotropic surface
arrangements. A region of bistability between the two arrangement was also required for future
applications relating the modeling of display cells.\\
The rod-sphere potential was found to be unsuitable as the planar arrangement was replaced by a
tilted structure. However, the results obtained using this model proved to be interesting
since they showed that a tilted phase can be obtained from purely steric interactions. The
rod-surface potential, meanwhile, proved a better candidate for the aim stated above, as it
not only recovered the surface behaviour of the HNW potential but actually displayed stronger
and wider bistability regions.\\
In the second part of this Chapter, hybrid anchored systems of HGO particles confined between a
homeotropic top surface and a bottom surface with competing anchoring have been studied using
the HNW potential for the surface interactions. These simulations showed that the
bistability behaviour of the model can be lost if high anchoring strength is used at the top
surface or if the slab is too narrow. Using moderate homeotropic anchoring at the top surfaces
and systems sizes of $N=2000$ particles, however, bistability was established and a cell
was successfully switched between the HAN and V states if only the easy switching was
considered. Achievement of the reverse switching requires the use of flexoelectric particles and
a surface potential allowing bistability between planar and homeotropic arrangements for such
particles. Those two problems are addressed, respectively, in Chapters~\ref{chap:six}
and~\ref{chap:seven}.






