

\chapter{Surface influence on liquid crystalline systems}
\label{chap:four}

%===============================================================================================
%===============================================================================================
\introduction

When a Liquid Crystal is placed in contact with another phase, a surface is created which breaks
the symmetry of the system. There are two main effects to this symmetry breaking~\cite{Jerome91}.
The first is to introduce molecular layering close to the interface while the second  is to 
modify the surface orientational ordering in the interfacial region.\\
The purpose of this Chapter is twofold. First, the effects of confinement are studied on systems of
ellipsoidal particles represented by the hard Gaussian overlap model and using a simple 
particle-surface potential, the so called Hard Needle Wall potential. The effects of varying
density and particle-substrate interactions are studied.\\
The second aim of this Chapter is to study the anchoring transition from planar to homeotropic
alignment. As very little simulation work has been performed on anchoring transitions, it is
hoped that this study can shed some light on their origins. Finally, possible
bistability regions in the anchoring behaviour of the models are examined.


%===============================================================================================
%===============================================================================================



\section{A first surface potential}
\label{s:HNWmodel}

Here, surface induced structural changes are studied using ellipsoidal shaped particles interacting
through the hard Gaussian overlap potential. In this Chapter all
particle-surface interactions are performed using the Hard Needle Wall potential 
(HNW)~\cite{Chrzanowska_Teixera_01} as this provides a simple and intuitive steric interaction which
can be tuned so as to induce either homeotropic or planar arrangement. As a result surface
induced structural changes can be studied using one model along with parameters values
appropriate for both surface arrangements.
Also, this potential allows the study of the transition from one arrangement to the other.


%==============================================================================================
%==============================================================================================
\subsection{The Hard Needle Wall potential}
\label{ss:HNW}


With the Hard Needle Wall (HNW) potential, the particles do not interact directly 
with the surfaces,  rather the surface  interaction is achieved by a needle of length 
$k_S$ placed at the centre of each particle
(Figure~\ref{fig:HGO_wal}). As a result, the HNW potential can be viewed as an extension of the
potential used by Allen~\cite{Allen99} where only the centre of mass interacts with the surface.
The case considered in~\cite{Allen99}, thus, corresponds to the HNW potential with a zero needle
length. The interaction potential between the needle and the surface is defined by 
$\mathcal{V}^\mrm{HNW}$ as~:

\begin{equation}
\mathcal{V}^{HNW} = \left\{ %}
	\begin{array}{ccc}
	0               &\mrm{if}   &|z_i-z_0| \geq \sigma_w  \\
	\infty          &\mrm{if}   &|z_i-z_0| < \sigma_w
	\end{array}
	\right.
\end{equation}

with~:

\begin{equation}
        \sigma_w = \frac{1}{2}\so k_S\cos(\theta)
	\label{eqn:sigma_w_HNW}
\end{equation}

Here $\so$ defines the unit of distance, $k_S$ is the dimensionless needle length and 
$\theta$ the angle 
between the surface normal and the particle's orientation vector, which and also corresponds 
to the Euler zenithal angle.  The behaviour of $\sigma_w(\theta,k_S)$ is shown on 
Figure~\ref{fig:sw_HNW}\\

%----------------------------------------------------
\picW = 7cm
\begin{figure}
	\centering
	\pic{HNWConfig.ps}
	\caption{The HNW configuration}
	\label{fig:HGO_wal}
\end{figure}
%----------------------------------------------------

%------------------------------------------
\picW = 7cm
\begin{figure}
	\centering
	\pic{sw_HNW_2.ps}
	\caption{Evolution of $\sigma_w(k_S, \cos\theta)$ for the HNW surface potential.}
	\label{fig:sw_HNW}
\end{figure}
%------------------------------------------

At high densities, this potential drives the preferred
anchoring behaviour by dictating the volume of absorbed particles that can be subsumed 
into the surface. The only model parameter upon which the anchoring strength and 
angle are dependent is the needle length. According to this, there is a finite volume of 
any `surface particle' that can be absorbed on the surface. The effect of this absorbed 
volume is to reduce the system's free energy due to the increased free volume (and, thus,
entropy) it afford to rest of the system.
The greater the volume  absorbed, the lower the free energy. As a result the most stable
surface arrangement, \ie that which minimizes the free energy, is the one that maximizes 
the absorbed volume.\\

The surface behaviour of the HNW model has already been studied by Chrza\-now\-ska 
\etal\cite{Chrzanowska_Teixera_01} and Cleaver and Teixeira~\cite{Cleaver_Teixeira_01}. For
small $k_S$, the homeotropic arrangement is most stable, whereas the planar arrangement is 
favored for long $k_S$. Some more quantitative insight into this can be obtained by studying the
amount of volume that can be absorbed into the surface as a function of molecular orientation 
and needle length. Approximating the shape of a Gaussian ellipsoid to that of an ellipsoid of
revolution~\cite{Rigby89}, the volume $V_e$ of a single particle of elongation $k$ absorbed into 
the substrate as a function of $k_S$ and $\theta$ can be obtained as described in
Appendix~\ref{chap:A}~:
\begin{equation}
	Ve=\frac{1}{3}\pi
	\lp \frac{1}{2}-\sqrt{\frac{\sigma^\mrm{HNW}_w(k_S,\theta)}{k^2\cos^2\theta+\sin^2\theta}}\rp^2 
	\lp 1 + \sqrt{\frac{\sigma^{HNW}_w(k_S,\cos\theta)}{k^2\cos^2\theta+\sin^2\theta}} \rp
	\label{eqn:Ve_HNW}
\end{equation}

On replacing $\sigma^\mrm{HNW}_w(k_S,\theta)$ with the expression of
Equation~\ref{eqn:sigma_w_HNW}, $V_e$ reads~:
\begin{equation}
	V_e = \frac{1}{8}\pi
	\lp-1+\sqrt{\frac{k_S^2\cos^2\theta}{5+4\cos(2\theta)}}\rp^2
	\lp2+\sqrt{\frac{k_S^2\cos^2\theta}{5+4\cos(2\theta)}}\rp
\end{equation}
A graphical representation of this function is given in Figure~\ref{fig:Ve_HNW}. 

\picW = 8cm
\begin{figure}
	\centering
	\pic{Ve_HNW_fkS.ps}
	\caption{Representation of the absorbed volume $V_e$ of a single particle 
	for the HNW potential as a function of $k_S$ and $\theta$.}
	\label{fig:Ve_HNW}
\end{figure}

This shows that regardless the needle length, the absorbed volume of a single particle is maximal 
for $\theta=\pi/2$,
suggesting that the planar arrangement should be most stable for every case. One exception is the
case $k_S=0$ where both planar and homeotropic arrangement allow the adsorption of an equal
volume (half the ellipsoid volume.)
Clearly this disagrees with the simulation results of~\cite{Chrzanowska_Teixera_01,Cleaver_Teixeira_01} 
where intermediate, short needle lengths
also showed stable homeotropic alignment. This discrepancy can be resolved by also considering
the packing behaviour of the two arrangements. This is described in the next Section.




%==============================================================================================
%==============================================================================================
\subsection{Relative stability of anchoring orientations}
\label{ss:PHstability}

We now consider the behaviour of
a system of numerous hard Gaussian overlap molecules in a confined geometry, interacting with 
the surface through the HNW potential. The aim is to calculate the homeotropic to
planar transition needle length for this system. Each particle close to the surface is 
approximated to be an ellipsoid of revolution with elongation $k=\frac{\sel}{\so}$ and 
with semi axis $a=b=\frac{\so}{2}$ and $c=\frac{\sel}{2}$. The following quantities are also
defined~:

\begin{itemize}
        \item $Veps$~: full volume of an ellipsoid.
	\item $V_P(k_S)$~: volume absorbed by one ellipsoid with $\theta=\frac{\pi}{2}$
	\item $S_P$~: area of the substrate occupied by one ellipsoid with $\theta=\frac{\pi}{2}$
	\item $V_H(k_S)$~: volume absorbed by one ellipsoid with $\theta=0$
        \item $S_H$~: area of the substrate occupied by one ellipsoid with $\theta=0$
\end{itemize}

\picW = 7cm
\begin{figure}[h]
	\centering
	\pic{HPTransTheory_f3.ps}
	\caption{Schematic representation of the geometry considered for the 
	calculation of $V_H(k_S)$.}
	\label{fig:HPTransTheory_f3}
\end{figure}
%----------------------------------------------------------------

The homeotropic to planar transition can be understood by considering the ratio of the volume
absorbed into the surface to the surface occupied by the particle on the substrate (\ie the
projection of the particle onto the substrate) for the two key arrangements. The free energy  
for these will be equal when the two ratios are equal. 
As a result, the problem of finding the anchoring transition needle length reduces to solving~:
%
\begin{equation}
        \frac{V_H(k_S)}{S_H} = \frac{V_P(k_S)}{S_P}
	\label{eqn:P-Htransition}
\end{equation}

%-----------------------------------------------------------------------------------------------
%-----------------------------------------------------------------------------------------------
\subsubsection{expression for $V_P(k_S)$ and $S_P$}
The case of planar alignment is straightforward~:
%
\begin{eqnarray}
        S_P &=& \frac{\pi\so\sel}{4}\\
        V_P &=& \frac{\pi\so^2\sel}{12}
\end{eqnarray}


%-----------------------------------------------------------------------------------------------
%-----------------------------------------------------------------------------------------------
\subsubsection{expression for $S_H$}

In the case of Homeotropic arrangement, $\theta=0 \forall k_S$. As a result, although the
particle will be positioned at a different distances from the surface as a function of $k_S$, the
projection of the particle on to the surface is constant and thus~:
%
\begin{equation}
        S_H = \frac{\pi\so^2}{4}
\end{equation}
%
%-----------------------------------------------------------------------------------------------
%-----------------------------------------------------------------------------------------------
\subsubsection{expression for $V_H(k_S)$ }

The volume that can be absorbed if $\theta=0$, is a function of $k_S$, but is not obtained
trivially. The extreme cases are known~:
\begin{eqnarray*}
        V_H(0) &=& \frac{Veps}{2}\\
	V_H(k) &=& 0
\end{eqnarray*}

The set up shown in Figure~\ref{fig:HPTransTheory_f3} is considered where ultimately 
$V_H(k_S)$ corresponds to the solid volume. First an expression for $V_F$, the volume of 
a portion of an ellipsoid from the tip to a distance $z_0$ is  required as a function 
of $z_0$. From this $V_H(k_S)$ can be identified. Starting from the equation of an ellipsoid~:
\begin{equation}
	\frac{x^2}{a^2} + \frac{y^2}{b^2} + \frac{z^2}{c^2} = 1
	\label{eqn:ellipsoidEqn}
\end{equation}

$V_F$ is given by~;
\begin{equation}
	V_F = \int^{z_0}_{-c} \int^{y_\mrm{max}}_{y_\mrm{min}} \int^{x_\mrm{max}}_{x_\mrm{min}} dx.dy.dz
\end{equation}
%
Using~(\ref{eqn:ellipsoidEqn}), the triple integral transforms to~:
%
\begin{eqnarray}
	V_F &=& \int^{z_0}_{-c}
		\int^{ \frac{b}{c}\sqrt{c^2 - z_0^2} }_{-\frac{b}{c}\sqrt{c^2 - z_0^2}}
		\int^{\frac{a}{b}\sqrt{b^2K - y^2} }_{-\frac{a}{b}\sqrt{b^2K - y^2}}
		dx.dy.dz\\
	\mrm{with\ } K &=& 1 - \frac{z^2}{c^2} \nonumber
\end{eqnarray}
%
%
and thus~:
%
\begin{equation}
	V_F 	= \pi ab \left[  z_0\lp 1 - \frac{z_0^2}{3c^2}\rp + \frac{2c}{3} \right]
	\label{eqn:V_F(z_0)}
\end{equation}

This expression can be checked, considering two known limits.
If $z_0=c$, $V_F = \frac{4}{3} \pi abc$ which is the full volume of an ellipsoid.
If $z_0=-c$, $V_F = 0$, again giving the expected result.\\

$V_H(k_S)$ can then be obtained by identification of the parameters $a,b,c$ and $z_o$ with the
setup shown in Figure~\ref{fig:HPTransTheory_f3} using ~:
\begin{eqnarray*}
	a &=& b = \frac{\so}{2}\\
	c &=& \frac{\sel}{2}	\\
	z_0 &=& \frac{-k_S}{2}
\end{eqnarray*}
%
which leads to~:
%
\begin{equation}
	V_H = \frac{\pi\so^2}{4}
	\left[
	\frac{k_S}{2} \lp \frac{k_S^2}{3\sel^2} -1  \rp + \frac{\sel}{3}
	\right]
	\label{eqn:V_H(Ln)}
\end{equation}



%======================================================================================================
%======================================================================================================
\subsubsection{Transition needle length }

Having obtained expressions for $V_H(k_S)$, $S_H$, $V_P(k_S)$ and $S_P$,
Equation \ref{eqn:P-Htransition} can be solved.
%
\begin{gather}
        \frac{V_H(k_S)}{S_H} = \frac{V_P(k_S)}{S_P}                     \\
        \frac{1}{6k^2\so^2}k_S^3 - \frac{1}{2}k_S + \frac{\so}{3}\lp k-1\rp = 0
\end{gather}

Also, for clarity and generality purposes, the results are best expressed using the reduced 
needle length $k_S^{'} = \frac{k_S}{k}$. This leads to~:
\begin{equation}
	\frac{k}{6\so^2}k_S^{'3} - \frac{k}{2}k_S^{'} +\frac{\so}{3}\lp k-1\rp = 0
	\label{eqn:LnTfinalPoly}
\end{equation}

The transition needle length $k_S^{T\prime}$ is therefore given by the root 
of~\ref{eqn:LnTfinalPoly} satisfying $k_S^{T\prime} \in ]0:1]$
The result is $k-$dependent, the variation of the transition $k_S^{T\prime}$ as a function
of k is shown in Figure~\ref{fig:LnT-k}.

\picW = 10cm
\begin{figure}
        \centering
        \picL{LnT-k.ps}
	\caption{Variation of $k_S^{T\prime}(k)$}
	\label{fig:LnT-k}
\end{figure}


For the two elongations used in this study, that is $k=3$ and $5$, the transition needle length is :
\begin{center}
	\textbf{$k=3$~:} $k_S^{T\prime} \sim 0.4817$	\\
	\textbf{$k=5$~:} $k_S^{T\prime} \sim 0.6084$	\\
\end{center}
%
Thus with $k=3$, the anchoring transition should occur for a reduced needle length of about 50\% 
whereas with $k=5$, the transition should occur for a reduced needle length of about 60\%.






\section{Symmetric anchored systems}
\label{s:simRes}

Here we present simulation results from a study of surface induced structural changes.
Ellipsoidal shaped particles interacting through the hard Gaussian overlap model were 
considered while the HNW model was used for particle-substrate interactions.
Monte Carlo simulations in the canonical ensemble of systems of $N=1000$ particles 
confined in a slab geometry of height $L_z=4\sel$ were performed with
the substrate being located on the top and bottom of the box ($z=-L_z/2$ and $z=L_z/2$); 
the system was periodic in the two other dimensions.
The same anchoring conditions were applied on both substrates (symmetric anchoring).
Using this combination of particle and surface potentials, the aims of this study were 
to model the main influences of confinement in liquid crystalline systems, namely surface 
induced layering and ordering, and to identify the influences of density and needle length 
on the preferred  surface arrangements.
In order to do so, a systematic approach has been undertaken; Figure~\ref{fig:simsDia_HNW}(a) 
and (b) shows the state points at which the simulations have been performed 
as well as the direction of the simulation series. 
For each state points two simulations have been performed, first an equilibration simulation 
of $0.5.10^6$ sweeps has followed by a production run of another $0.5.10^6$ sweeps.

\picW = 10cm
\begin{figure}
	\centering
	\subfigure[$k=3$]{\picL{sims_k3Points.ps}}
	\subfigure[$k=5$]{\picL{sims_k5Points.ps}}
	\caption{Representation of the state points considered for the production of the data
	analysed in this Chapter.}
	\label{fig:simsDia_HNW}
\end{figure}

%==============================================================================================
%==============================================================================================
\subsection{Typical profiles}
\label{ss:typicalProfiles}

The first step is to understand surface induced structural changes for a set of needle
lengths and densities. References~\cite{Chrzanowska_Teixera_01,Cleaver_Teixeira_01} 
provide some information through profiles from systems of particles with $k=5$ and 
$k_S=5$~\cite{Chrzanowska_Teixera_01} and $k_S = 2.75,3.0$~\cite{Cleaver_Teixeira_01}. Here a
more exhaustive set of results are shown.\\

The study of the surface induced structural changes was performed through the computation of 
the $z$-profiles of three different observable.
The density profile $\rho_\ell(z)$ provides information about the layering in the cell 
through the location number and height of its peaks. $Q_{zz}(z)$, provides information 
on the surface induced ordering with respect to the substrate normal. These two profiles can be 
used to characterize the type of surface anchoring adopted. $P_2(z)$ measures the degree 
of orientational order and, hence, the nature of the phase as a function of location in the slab.\\
%
In what follows typical profiles are presented for several needle lengths and system densities. 
These profiles were obtained from several series of simulations each performed with constant density 
and $k_S$ either decreasing and increasing between the limits $[0:k]$.
Typical profiles are shown for densities corresponding to bulk isotropic  and nematic phases. 
Needle lengths are chosen so as to correspond to homeotropic 
($k_S<k^T_S$), competing ($k_S\sim k^T_S$), and planar arrangements ($k_S>k^T_S$).


\subsubsection{Homeotropic anchoring}

An homeotropic arrangement is observed when the orientation of the surface
particles is such that $<\theta>\sim 0$, which for the HNW model occurs for short needle
lengths ($\kSp \leq 0.25$). Typical profiles for this arrangement are showed on
Figures~\ref{fig:typicalProfile_k3_homeo} and~\ref{fig:typicalProfile_k5_homeo} for,
respectively, $k=3$ and $k=5$ using a reduced needle length $\kSp = 0.2$.\\
%
%-----------------------------------------
\picW = 14cm
\begin{figure}
        \centering
        \pic{typeProf_k3_homeo.ps}
	\caption{Typical profiles corresponding to an homeotropic anchoring for $k=3$ 
	and $\kSp = 0.20$.}
	\label{fig:typicalProfile_k3_homeo}
\end{figure}
%-----------------------------------------
%
%-----------------------------------------
\picW = 14cm
\begin{figure}
        \centering
        \pic{typeProf_k5_homeo.ps}
	\caption{Typical profiles corresponding to an homeotropic anchoring for $k=5$ 
	and $\kSp = 0.20$}
	\label{fig:typicalProfile_k5_homeo}
\end{figure}
%-----------------------------------------

From these Figures, the surface induced structural changes associated with the homeotropic 
arrangement can be
observed. $\rho^{*}_\ell(z)$ is an oscillatory function that displays its main peaks at
$|z-z_0|\sim \frac{L_n}{2}$ which corresponds to the first layer of particles. Also, it should
be noted that the periodicity of the peaks is very large (of the order of $\sel$), 
and no more than a total of five to six layers can be observed. 
This is compatible with the height of the simulation slabs of $L_z=4\sel$. The number of peaks in 
$\rho^{*}_\ell(z)$ clearly corresponds to a layered system with end to end alignment.\\

From the $Q_{zz}(z)$ profiles, the homeotropic arrangement can be determined from the regions 
of positive values.
These correspond to regions of high $\rho^{*}_\ell(z)$, that is the interfacial
regions for all densities and also the bulk region of the cell if the global number density
is sufficient to support nematic order. $Q_{zz}(z)$ is also an oscillatory function 
whose maxima match those in
$\rho^{*}_\ell(z)$; the higher the local density, the higher the order and, therefore,
the better the ordering with respect to the wall.\\
%
It should be noted however, that regardless of the number density, $Q_{zz}(z)$ displays negative
values very close to the surface ($|z-z_0|\leq \frac{L_n}{2}$)~; this is  because there are
always a few particles lying parallel to the walls in these regions. However, these planar
particles correspond to regions of very low density, $\rho^{*}_\ell$, and, therefore, their
effect on the overall behaviour of the system is insignificant~; besides the higher the density,
the less noticeable this effect.\\

$P_2(z)$ shows the orientational order as a function of position, a result that is not
easily obtained experimentally. $P_2(z)$ also follows the oscillations of
$\rho^{*}_\ell$, for the same reasons as $Q_{zz}(z)$; regions of high local density 
induce regions of high local in-plane order.\\


\subsubsection{Planar anchoring}

A planar arrangement is observed when the average orientation in the interfacial region is 
parallel to the substrate, that is $<\theta> \sim \frac{\pi}{2}$. In the case of the HNW
model, such an arrangement can be observed using long needle lengths 
($\kSp \geq 0.75$). Typical profiles measuring the surface induced structural changes 
for this arrangement  are shown on Figures~\ref{fig:typicalProfile_k3_planar} 
and~\ref{fig:typicalProfile_k5_planar}, respectively, for $k=3$ and $k=5$.
%
%-----------------------------------------
\picW = 14cm
\begin{figure}
        \centering
        \pic{typeProf_k3_planar.ps}
	\caption{Typical profiles corresponding to a planar arrangement with $k=3$ and $\kSp =
	0.80$.}
	\label{fig:typicalProfile_k3_planar}
\end{figure}
%-----------------------------------------

%-----------------------------------------
\picW = 14cm
\begin{figure}
        \centering
        \pic{typeProf_k5_planar.ps}
	\caption{Typical profiles corresponding to a planar arrangement and $k=5$ and $\kSp =
	0.80$.}
	\label{fig:typicalProfile_k5_planar}
\end{figure}
%-----------------------------------------

The first difference to be noted when comparing these results with those obtained in the case
of an homeotropic arrangement, is the difference in stratification.
%
Because at the surface,
$<\theta>\sim\frac{\pi}{2}$, the particles lie closer to the walls and as 
a result, the main peaks in $\rho^{*}_\ell$ are located at $|z-z_0|\sim 0$. The peak-peak
separation of these
functions are also much smaller than they were in the homeotropic case ($\sim\so$). 
The particles are, thus, now arranged in layers with a side by side alignment between one layer 
and the next.\\

$Q_{zz}(z)$ follows the same behaviour as $\rho^{*}_\ell(z)$ but adopts negative values in regions 
of high $\rho^{*}_\ell$. Therefore  maxima in $\rho^{*}_\ell$, induce minima in $Q_{zz}(z)$
because high planar order is measured by negative values of $Q_{zz}(z)$.
%
Comparison of the absolute value of $Q_{zz}(z)$ in the homeotropic and planar
arrangements, suggests that for a given number density homeotropic ordering with respect 
to the surface is `better' than the corresponding planar ordering. However 
since $Q_{zz} \in[-0.5:1]$, and an isotropic distribution of $\theta$ in a layer induces $Q_{zz}=0$, a
positive value of $Q_{zz}$ should be compared to the double of a negative value. As a result a
planar ordering with $Q_{zz}= -0.5$ is a `good' as an homeotropic ordering with $Q_{zz} = 1$.\\

The form taken by $P_2(z)$ is very similar for homeotropic and planar arrangement as both follow
the behaviour of the corresponding $\rho^{*}_\ell(z)$. The difference between the two
arrangements lies in the peaks separation of $P_2(z)$.


\subsubsection{Competing anchoring}

The term competing anchoring refers to a situation where both planar and homeotropic
arrangements  are of comparable strength. In the case of the HNW model, this corresponds to a 
situation where $\kSp \sim k_S^{T\prime}$. Typical profiles corresponding to competing anchoring are
shown on Figures~\ref{fig:typicalProfile_k3_bist} and~\ref{fig:typicalProfile_k5_bist}
respectively for $k=3$ and $k=5$.\\
These results have been obtained for needle lengths very close but not equal to $k_S^{T\prime}$ as the
values obtained by Equation~\ref{eqn:LnTfinalPoly} correspond to a very idealized case.
%
%
%-----------------------------------------
\picW = 14cm
\begin{figure}[!]
        \centering
        \pic{typeProf_k3_bist.ps}
	\caption{Typical profiles for $k=3$ and $L^{'}_n = 0.48$}
	\label{fig:typicalProfile_k3_bist}
\end{figure}
%-----------------------------------------
%
%-----------------------------------------
\picW = 14cm
\begin{figure}[!]
        \centering
        \pic{typeProf_k5_bist.ps}
	\caption{Typical profiles for $k=5$ and $L^{'}_n = 0.54$}
	\label{fig:typicalProfile_k5_bist}
\end{figure}
%-----------------------------------------

It should be noted here that at high number densities, different profiles suggesting different
surface arrangements were obtained depending on the history of the simulation sequence.
Moreover, as in each case the profiles were obtained from fully equilibrated configurations, the
presence of bistability is suggested here. Both sets of coexisting profiles are shown on the Figures.\\

In the case of \textbf{isotropic densities}, the surface induced layering shows features
reminiscent with
both planar and homeotropic influences. The interfacial region is characterized by two peaks 
of comparable height corresponding to particles with $\theta = 0$ and $\theta = \frac{\pi}{2}$. 
Since the
particles in the interfacial region were equally distributed between those two regions, the
heights of
the two peaks are much smaller than those seen in the cases of strong planar or homeotropic anchoring. 
%
This double behaviour can also be observed on $Q_{zz}(z)$, as, in the interfacial region, 
both negative and positive values corresponding to high local densities can be observed.
%
The $P_2(z)$ profile does not bring much more information as its behaviour follows that of
$\rho^{*}_\ell(z)$.\\


In the case of \textbf{nematic densities}, the behaviour of the system is very different. Here,
the increase in density had induced the particle to align and, therefore, choose one of 
the two possible
surface arrangements. This prevented the system from simultaneously exhibiting both planar 
and homeotropic features. The profiles indicate that in the case of runs with decreasing $k_S$ 
(\ie coming from the planar side), the cell displayed a planar arrangement whereas in the case of
increasing needle lengths an homeotropic arrangement was found. This suggest that in the case of
competing anchoring and high density, the surface arrangement was chosen according to the history
of the system.



%==============================================================================================
%==============================================================================================
\subsection{Influence of density}
\label{ss:rhoInfl}

Observation of the profiles reveals that the density has a strong influence on the
intensity of the surface induced effects; as stated earlier, 
the higher the number density the more intense the surface induced effects. 
In order to further study the influence of density on surface induced structural changes,
Further simulations were carried out
in series with constant $\kSp$ and increasing and decreasing densities. 
From these, the $\rho^{*}_\ell(z,\rho^{*})$, $Q_{zz}(z,\rho^{*})$, and $P_2(z,\rho^{*})$
surfaces have been computed. 
Results for the cases $\kSp = 0.0$ and $\kSp = 1.0$ are shown on Figures~\ref{fig:rhoInfl_k3_L000} 
and~\ref{fig:rhoInfl_k3_L100} for $k=3$. The corresponding results for $k=5$ 
are very similar  and are, therfore, not shown here. Also, because hardly any hysteresis has been found 
between series with increasing and decreasing densities, only the results or series with increasing 
densities are shown.

%-----------------------------------------
\picW = 7cm
\begin{figure}
        \centering
        \subfigure[]{\pic{rho-z-Ln_NVT_Sconf_k3_N1000_fd_L000_LnD.ps}}
	\subfigure[]{\pic{Qzz-z-Ln_NVT_Sconf_k3_N1000_fd_L000_LnD.ps}}
	\subfigure[]{\pic{P2-z-Ln_NVT_Sconf_k3_N1000_fd_L000_LnD.ps}}
	\caption{Influence of $\rho^{*}$ on the z-profiles from series of simulation with
	increasing density of particles with $k=3$ and $\kSp = 0.0$ (homeotropic anchoring).}
	\label{fig:rhoInfl_k3_L000}
\end{figure}
%-----------------------------------------
%-----------------------------------------
\picW = 7cm
\begin{figure}
        \centering
        \subfigure[]{\pic{rho-z-Ln_NVT_Sconf_k3_N1000_fd_L100_LnD.ps}}
	\subfigure[]{\pic{Qzz-z-Ln_NVT_Sconf_k3_N1000_fd_L100_LnD.ps}}
	\subfigure[]{\pic{P2-z-Ln_NVT_Sconf_k3_N1000_fd_L100_LnD.ps}}
	\caption{Influence of $\rho^{*}$ on the z-profiles from series of simulation with
	increasing density of particles with $k=3$ and $\kSp = 0.1$ (planar anchoring).}
	\label{fig:rhoInfl_k3_L100}
\end{figure}
%-----------------------------------------

Those measurements confirm the first observation using a smaller sample of densities 
(Section~\ref{ss:typicalProfiles}). At low densities, the surface induced effects are 
limited to the interfacial regions and the central region of the cell remains unaffected. 
On the $z-$profiles, this is characterized by short ranged surface features.\\
As the density is increased however, the surface influence extends further into the cell up to a
density for which the full slab is uniformally aligned.  The number of peaks on $\rho^{*}_\ell(z)$ 
increases steadily due to layering of the particles. As a result, the absolute values of the ordering 
observables increase as does the distance from the surface at which those functions 
start to decay. This translates to greater orientational ordering that extends further into the cell.

It is interesting to note that in the case of extreme homeotropic anchoring ($\kSp = 0.0$),
regardless the density, the cell never displays uniform alignment since the bulk part of the
cell never orders. This seems to contradict previous observations of systems of particles with
elongation $k=3$ under homeotropic anchoring
(see Figure~\ref{fig:typicalProfile_k3_homeo}). However, due to the reduced needle lengths used
here, the particle volume absorbed by the substrate leads to a lowering of the density in the
bulk regions. This has the effect of shifting the I-N transition to densities out of the range
considered here. This does not, however, question the existence of uniform alignment in the case
$\kSp = 0.0$ as this has been observed with k=5.\\

Another effect that can be observed on Figures~\ref{fig:rhoInfl_k3_L000} and~\ref{fig:rhoInfl_k3_L100}
is that the z gradients in $|Q_{zz}(z)|$ and $\rho^{*}_\ell(z)$ in the interfacial region increase 
when  the bulk part of the cell becomes nematic.  As a result of the bulk orientational order, 
interfacial 
regions come under the ordering influence of both the surfaces, through anchoring effects, and the 
bulk part of the cell, through elastic forces. This improves the quality of the layering and 
ordering in the interfacial regions.


%==============================================================================================
%==============================================================================================
\subsection{Influence of $k^{\prime}_S$}
\label{ss:LnInfluence}

The influence of $\kSp$ on the cell's behaviour is considerable, as this variable
controls the type and strength of the surface anchoring.
%
As the needle lengths is increased
between from zero, the system undergoes a transition from homeotropic 
to planar anchoring. Section~\ref{ss:PHstability} gave the critical $\kSp$ values
for the this transition in the limit of close packing.
%
In the simulations however, the local density at the interfaces is far from being 
that of close packing and is moreover a function of the global density. Therefore some 
shifts in $k_S^{T\prime}$ from the theoretical values are to be expected.\\
%
The purpose of this Section is to study,  in greater detail, the effect of $k^\prime_S$
on the profiles and show the existence and nature of the homeotropic to planar anchoring
transition.
This has been achieved
using simulations performed at constant densities  in series of increasing and decreasing 
needle lengths. 
Two densities known to bulk isotropic and nematic phases have been considered.
Results from the series with $k^\prime_S$ densities and $k=3$ are shown in 
Figure~\ref{fig:LnInfl_k3_d0.28} (isotropic density) and~\ref{fig:LnInfl_k3_d0.34} (nematic
density). The differences between the series with increasing and decreasing $\kSp$ are discussed
in the next Section.\\

%-----------------------------------------
\picW = 7cm
\begin{figure}
        \centering
        \subfigure[]{\pic{rho-z-Ln_NVT_Sconf_k3_N1000_fL_d0.2800_LnD.ps}}
	\subfigure[]{\pic{Qzz-z-Ln_NVT_Sconf_k3_N1000_fL_d0.2800_LnD.ps}}
	\subfigure[]{\pic{P2-z-Ln_NVT_Sconf_k3_N1000_fL_d0.2800_LnD.ps}}
	\caption{influence of $\kSp$ on the z-profiles for $k=3$ and $\rho^{*} = 0.28$.}
	\label{fig:LnInfl_k3_d0.28}
\end{figure}
%-----------------------------------------


%-----------------------------------------
\picW = 7.0cm
\begin{figure}
        \centering
        \subfigure[]{\pic{rho-z-Ln_NVT_Sconf_k3_N1000_fL_d0.3400_LnD.ps}}
	\subfigure[]{\pic{Qzz-z-Ln_NVT_Sconf_k3_N1000_fL_d0.3400_LnD.ps}}
	\subfigure[]{\pic{P2-z-Ln_NVT_Sconf_k3_N1000_fL_d0.3400_LnD.ps}}
	\caption{influence of $\kSp$ on the z-profiles for $k=3$ and $\rho^{*} = 0.34$.}
	\label{fig:LnInfl_k3_d0.34}
\end{figure}
%-----------------------------------------

From Figures~\ref{fig:LnInfl_k3_d0.28} and~\ref{fig:LnInfl_k3_d0.34}, typical features 
discussed previously corresponding to homeotropic and planar anchoring arrangements can be
found by looking at curves corresponding to low and high values of $\kSp$, respectively.
Here, rather,
 focus is brought to bear on the regions corresponding to competing anchoring, as these reveal
the changes that take place during the anchoring transition.\\
%
As the transition region is approached (\eg from the planar side), the height of the peaks in 
$\rho^{*}_\ell$ decrease rapidly, and at the transition, hardly any oscillations can be observed. 
This can be understood easily as at the transition, under equal influence from both type of
anchoring, the particles diffuse almost homogeneously in bimodal surface layers which have
features corresponding to both
anchoring states; as a result, the density profile becomes relatively uniform. As the 
homeotropic anchoring becomes stronger,
the system layers accordingly and a new oscillating pattern corresponding to homeotropic
anchoring can be observed. It is interesting to note that the location 
maxima in $\rho^{*}_\ell(z)$ in the latter case coincide with those of the minima of the former.\\
%
As well as the changes in density, the centre of the slab undergoes an orientational 
disorder-order transition with $\kSp$ as can be
observed on $P_2(z,\kSp)$. The z-dependent oscillation pattern of this function also changes
according to the state of the surface alignment.\\

$Q_{zz}(z,\kSp)$ shows the orientational reorganization that occurs during the transition.
Upon approaching the transition region, this profile changes rapidly between the planar and
homeotropic characteristics. The gradient of this change increases with the density but also
with $k$; the transition is much sharper for $k=5$ than for $k=3$.
Considering the needle length at which the change in the sign of $Q_{zz}$ is observed, the
theoretical predictions made in section~\ref{ss:PHstability} seem to be confirmed, particularly
as the density is increased. A more detailed study of the homeotropic to planar transition
region is presented in Section~\ref{s:surfInfphaseTrans}.






\section{Surface influence on phase transitions.}
\label{s:surfInfphaseTrans}



%===============================================================================================
\subsection{ The $\overline{Q}_{zz}$ and $\overline{P}_2$ observables.}


To determine the location of the planar to homeotropic anchoring transition more precisely 
requires the ability to
characterize quantitatively the nature of the arrangement displayed by a confined system. 
Section~\ref{ss:typicalProfiles} has shown that simple observation of the profiles is not
sufficient to determine the arrangement type as information from both $Q_{zz}(z)$ and 
$\rho^{*}_\ell(z)$ is required. Besides, quantitative information on the anchoring behaviour is
hard to obtain solely by observation of profiles.\\
As the transition point is a function of both the global number density and needle
length, a useful observable for characterising the surface arrangement would be a scalar
able to distinguish 
both the type and strength of the anchoring for a given $(\rho^{*}, \kSp)$ state point.\\
This need is fulfilled by the use of the novel observables $\overline{Q}_{zz}$ and 
$\overline{P}_2$. These are density-profile-weighted averages of,
respectively, $Q_{zz}(z)$ and $P_2(z)$, taken over a given
region of interest. In general $\overline{Q}_{zz}$  and $\overline{P}_2$ are defined as~:
%
\begin{eqnarray}
	\overline{Q}_{zz} &=& \frac{\sum_{z_i} Q^{n}_{zz}(z_i) \rho^{*}_\ell(z_i)}
			{\sum_{z_i} \rho^{*}_\ell(z_i)}		\\
	%
	\nonumber \\ 
	%
	\overline{P}_2 &=& \frac{\sum_{z_i} P_2(z_i) \rho^{*}_\ell(z_i)}
			{\sum_{z_i} \rho^{*}_\ell(z_i)}	
\end{eqnarray}
%
where the $z_i$ considered are restricted on the region of interest.
Here $Q^{n}_{zz}(z)$ is a rescaled version of $Q_{zz}$ such that $Q^{n}_{zz}\in[-1:1]$. Hence the
definition of $Q^{n}_{zz}(z)$~:
%
\begin{equation}
	Q^{n}_{zz} = \left\{	%}
	\begin{array}{ccc}
		Q_{zz}		&\text{if}	&Q_{zz} \geq 0	\\
		2.Q_{zz}	&\text{if}	&Q_{zz} < 0\\
	\end{array}
	\right.
\end{equation}


The computation of the $\overline{Q}_{zz}$ and $\overline{P}_2$ observables has been 
performed on regions of the cell corresponding to the interfacial and bulks domains. 
This has enabled the behaviour of the system to be studied in each region separately. 
The naming convention adopted for the observables
corresponding to each region is described in Table~\ref{tble:SuBuNamingCvention}.\\

\begin{table}[h]
\centering
\begin{tabular}{||c||c||c||}
\hhline{|t:=:t:=:t:=:t|}
key	&Description			&Associated observables		\\
\hhline{|:=::=::=:|}
Sb	&Bottom interfacial region	&$\overline{Q}^{Sb}_{zz}$, $\overline{P}^{Sb}_2$\\
Bu	&Bulk region			&$\overline{Q}^{Bu}_{zz}$, $\overline{P}^{Bu}_2$\\
St	&Top interfacial region		&$\overline{Q}^{St}_{zz}$, $\overline{P}^{St}_2$\\
Su	&Both interfacial region	&$\overline{Q}^{Su}_{zz}$, $\overline{P}^{Su}_2$\\
\hhline{|b:=:b:=:b:=:b|}
\end{tabular}
\caption{Naming convention for the simulation slab regions and associated observable.}
\label{tble:SuBuNamingCvention}
\end{table}

It is now necessary to define an 
appropriate boundary between the interfacial and bulk regions. 
This boundary needs to be located at a point where the surface has no direct influence on
the molecules; as a result the boundary $z_i$ could be chosen such that $|z_i-z_0| = \frac{L_n}{2}$
since at $z=z_i$, the particles can rotate  freely without direct interaction  with the surface. 
This approach fails, however, in the limit of zero needle length, as
it implies $|z_i-z_0| \sim 0$ whereas the $z$-profiles clearly show a non-zero interfacial
regions for all needle lengths.\\
A different approach was taken, therefore, in defining this boundary; the interfacial region
width was made a function of the needle length and density by making the boundary $z_i$ 
dependent on features of the density profiles.
The scheme used is illustrated on Figure~\ref{fig:SuBuDefs}. If the anchoring was found
to be planar (first local maximum of $\rho^{*}_\ell(z)$ at $|z_i-z_0|\sim 0$), 
the interfacial region  was taken to extend from  the surface to the distance corresponding 
to the second maximum in  $\rho^{*}_\ell(z)$. If however, the anchoring was homeotropic 
(first local maximum of $\rho^{*}_\ell(z)$ at $|z_i-z_0|\sim \frac{L_n}{2}$) then the 
interfacial region was taken to extend from
the surface to the first local minimum in $\rho^{*}_\ell(z)$. In those cases with ambiguous 
double peaked density profiles, the first scheme was adopted.

\picW = 7cm
\begin{figure}[h]
	\centering
	\subfigure[planar case]{\picL{planarSuBu.ps}}
	\subfigure[homeotropic case]{\picL{homeoSuBu.ps}}
	\caption{Definition of the slab interfacial and bulk regions.}
	\label{fig:SuBuDefs}
\end{figure}

%===============================================================================================
%===============================================================================================
\subsection{Anchoring transitions.}

Here, the planar to homeotropic anchoring transition is located from measurement of 
$\overline{Q}_{zz}$ as a function of $\rho^{*}$ and $\kSp$ which can be used to construct an 
anchoring phase diagram. Such diagrams have been determined for systems with elongations 
$k=3$ and $k=5$ using 
simulations performed at constant densities and increasing and decreasing needle lengths. 
Results for both series in the interfacial (Su) and bulk (Bu) regions are shown in 
Figures~\ref{fig:QzzPhaseDia_k3} and~\ref{fig:QzzPhaseDia_k5} for $k=3$ and $k=5$ respectively.

%---------------------------------------------------------
\picW = 7cm
\begin{figure}
	\centering
	\subfigure[Series with increasing $\kSp$.]{
	\pic{QzzWaSu_phaseDia_k3_LnU.ps}
	\pic{QzzWaBu_phaseDia_k3_LnU.ps}}
	
	\subfigure[Series with decreasing $\kSp$.]{
	\pic{QzzWaSu_phaseDia_k3_LnD.ps}
	\pic{QzzWaBu_phaseDia_k3_LnD.ps}}
	
	\subfigure[Bistability phase diagram (\ie difference between (a) and (b)).]{
	\pic{QzzWaSu_bistPhaseDia_k3.ps}
	\pic{QzzWaBu_bistPhaseDia_k3.ps}}
	\caption{Anchoring phase diagrams of $\overline{Q}_{zz}$ for $k=3$ for the surface 
	(left) and bulk (right) regions of the cell. }
	\label{fig:QzzPhaseDia_k3}
\end{figure}

\picW = 7cm
\begin{figure}
	\centering
	\subfigure[Series with increasing $\kSp$.]{
	\pic{QzzWaSu_phaseDia_k5_LnU.ps}
	\pic{QzzWaBu_phaseDia_k5_LnU.ps}}
	
	\subfigure[Series with decreasing $\kSp$.]{
	\pic{QzzWaSu_phaseDia_k5_LnD.ps}
	\pic{QzzWaBu_phaseDia_k5_LnD.ps}}
	
	\subfigure[Bistability phase diagram (\ie difference between (a) and (b)).]{
	\pic{QzzWaSu_bistPhaseDia_k5.ps}
	\pic{QzzWaBu_bistPhaseDia_k5.ps}}
	\caption{Anchoring phase diagrams of $\overline{Q}_{zz}$ for $k=5$ for the surface 
	(left) and bulk (right) regions of the cell. }
	\label{fig:QzzPhaseDia_k5}
\end{figure}
%---------------------------------------------------------

The results obtained for the two elongations are very similar. In the \textbf{interfacial
region}, the anchoring transitions occur at $k_S/k$ values close to those predicted in 
section~\ref{ss:PHstability}, as can be observed from the lines of constant $\overline{Q}^{Su}_{zz} =
0$. For higher densities, the agreement between  the simulation and theoretical result can be
seen to improve. 
Also the region around $\kSp$ becomes sharper with increasing density indicating a 
possible discontinuous transition between planar and homeotropic anchoring states.\\

In the \textbf{bulk region}, little surface influence can be observed at low density, as
the values of $\overline{Q}^{Bu}_{zz}$ remain close to zero due to the systems orientational
isotropy. As the number density is increased,  the local density in the bulk 
regions reaches values corresponding to bulk nematic densities. The surface influence then extends 
further into the cell and sharp anchoring transitions become apparent at needle lengths similar to
those suggested by the interfacial region anchoring diagram.
This, however, occurs for global number densities significantly greater than the isotropic to nematic
transition densities of the equivalent bulk system (see Section~\ref{HGO_cal_res}).
This indicates that the I-N transitions in the bulk regions
were shifted to higher number densities due to the presence of the surfaces. this is further
discussed in Section~\ref{ss:surfInflINTans}.\\

The anchoring phase diagrams are also found to be asymmetric in that bulk
planar ordering develops at lower densities than its homeotropic counterpart. 
As stated earlier, this is due, in part, to the increased absorbed volume in the case of
homeotropic anchoring which is sufficient to prevent the observation of uniformly 
ordered slabs of homeotropic arrangement in the limit $\kSp=0$ for the range of density 
considered here. That said, interpolation of the result determined here to higher densities
suggests that homeotropically ordered phases should exist for $\kSp < k^{T\prime}_{S}$.

%==============================================================================================
\subsection{Anchoring bistability}

Another interesting feature of Figures~\ref{fig:QzzPhaseDia_k3} and~\ref{fig:QzzPhaseDia_k5}
comes from the comparison of the diagrams for increasing and decreasing
needle lengths (\ie diagrams (a) and (b)). As the density is increased, the hysteresis between 
the two set of results also 
increases. This confirms an earlier observation that in conditions corresponding to 
competing  anchoring according the $z$-profiles observed can be dependent on 
sample's history (recall 
Figures~\ref{fig:typicalProfile_k3_bist} and~\ref{fig:typicalProfile_k5_bist}).\\
%
Since all of the data used to construct those diagrams were obtained from equilibrated systems,
these discrepancies suggest possible bistable behaviour for state points close to the
anchoring transition. This bistability has been measured more precisely by computing the
absolute value of the difference between results obtained with series of increasing and
decreasing needle lengths. This difference equals $0$ if the two diagram agree and $2$ for full
bistability. The results for both elongations in the interfacial and bulk regions of the slab,
shown on Figures~\ref{fig:QzzPhaseDia_k3}(c) and~\ref{fig:QzzPhaseDia_k5}(c)
indicate for both $k=3$ and $k=5$ distinct bistable regions at high densities.\\

In order to demonstrate the existence of this bistability, an attempt to
switch the cell from planar to homeotropic and back has been carried out. For this, a previously
equilibrated system of $N=1000$ particles with $k=3$, $\rho^{*}=0.34$ and $\kSp = 0.5$ was
taken. This configuration was extracted from a series performed with decreasing densities which 
showed planar anchoring at this state point. The switching was performed through the series of 
simulations $R_1$ to $R_5$ listed in Table~\ref{tble:bistFieldConditions}, \ie by 
applying and removing an electric field $\vect{E} = E\vecth{z}$ and
taking the dielectric constant $\chi_e$ to be, alternately, positive and negative.
The effect of the field is to align the  particles parallel or perpendicular with $\vect{E}$
respectively for positive and negative values of $\chi_e$, respectively.\\
While this setup is admittedly somewhat unrealistic, it can nevertheless be related to an
experimental system in which the mesogens employed can display different dielectric constant 
according to the frequency of an applied AC field.\\

\begin{table}
\centering
\begin{tabular}{||c||c||c||c||c||}
	\hhline{|t:=:t:=:t:=:t:=:t:=:t|}
	Run		&	$\vecth{E}$	&$E$	& $\chi_e$	&run length\\
	\hhline{|:=::=::=::=::=:|}
	$R_1$	&	(0,0,0)		&0.0	&0.0	&$0.25.10^6$\\
	$R_2$	&	(0,0,1)		&6.0	&0.5	&$0.25.10^6$\\
	$R_3$	&	(0,0,0)		&0.0	&0.0	&$1.00.10^6$\\
	$R_4$	&	(0,0,1)		&6.0	&-0.5	&$0.25.10^6$\\
	$R_5$	&	(0,0,0)		&0.0	&0.0	&$0.50.10^6$\\
	\hhline{|b:=:b:=:b:=:b:=:b:=:b|}
\end{tabular}
\caption{Electric parameterisation in the switching between the planar and
homeotropic states of the bistable system.}
\label{tble:bistFieldConditions}
\end{table}

If the considered state points correspond to a bistable region, both planar and homeotropic phases
should be obtained in the field-off runs provided the sample is prealigned suitably for each
arrangement. The purpose of applying the field with each value of $\chi_e$ is, thus, to perform 
this pre-alignment operation.


\picW = 6cm
\begin{figure}
	\centering
	\subfigure[start $R_1$]{\pic{HGO_box_init.ps}}
	\subfigure[end $R_1$]{\pic{HGO_box_Sconf_switch_01.ps}}
	\subfigure[end $R_2$]{\pic{HGO_box_Sconf_switch_02.ps}}
	\subfigure[end $R_3$]{\pic{HGO_box_Sconf_switch_03.ps}}
	\subfigure[end $R_4$]{\pic{HGO_box_Sconf_switch_04.ps}}
	\subfigure[end $R_5$]{\pic{HGO_box_Sconf_switch_05.ps}}
	\caption{Configuration snapshots corresponding to the initial (start) and final
	configurations of runs $R_1$ to $R_5$.}
	\label{fig:HGOk3BistSnaps}
\end{figure}


\picW = 12cm
\begin{figure}
	\centering
	\subfigure[$\overline{Q}_{zz}^{Su}(n)$]{\picL{QzzWaSu_k3_Bist.ps}}
	\subfigure[$\overline{Q}_{zz}^{Bu}(n)$]{\picL{QzzWaBu_k3_Bist.ps}}
	\caption{Evolution of $\overline{Q}_{zz}$ in the interfacial(a) and bulk(b) regions as a
	function of $n$, the number of sweeps.}
	\label{fig:QzzWak3Bist}
\end{figure}

The configuration snapshots corresponding to the initial and final states from each run are shown on 
Figure~\ref{fig:HGOk3BistSnaps}(a) to (f). The corresponding behaviour of
$\overline{Q}^{Su}_{zz}$ and $\overline{Q}^{Bu}_{zz}$, as a function of Monte Carlo sweeps, are
shown in Figures~\ref{fig:QzzWak3Bist}(a) and~\ref{fig:QzzWak3Bist}(b). Also, for comparison,
the values of $\overline{Q}_{zz}$ at this state point and corresponding to the two different surface
arrangements as obtained from the runs with increasing and decreasing $\kSp$ are shown as
horizontal lines.\\
%
The results confirm the existence of the bistable region. 
Run $R_1$ shows that the system remains stable in its initial planar arrangement; after 
reorientation of the particles along $\vecth{z}$ by the applied field (run $R_2$), 
the  system equilibrates naturally to a  homeotropic arrangement (run $R_3$) although the
final value of $\overline{Q}_{zz}$ is higher than that obtained from the previous runs performed
in the computation of the anchoring phase diagrams. This discrepancy might be induced by the
application of the strong value of the field which forced all interfacial particles 
to take an almost perfect homeotropic alignment. Upon removal of the field, 
the system equilibrated towards the stable homeotropic state, but due to packing constraints 
fewer particles were allowed to take a planar orientation close to the surface as in the case 
of previous simulations. This should not however question the equilibrium of the state obtained
here.
%
Upon changing the molecular dielectric susceptibility to $\chi_e<0$, reapplication of the field 
(run $R_4$) recreates a planar arrangement which relaxes to the original stable state 
upon field removal (run $R_5$). In this case, the system equilibrated to the same value of
$\overline{Q}_{zz}$ as that obtained previously as in the case of planar anchoring, there is no
instance of homeotropic alignment in the interfacial region.\\
%
It should be noted also, that the `response times' of the systems were different in the bulk and
interfacial region; however the Monte Carlo technique was used and this does not follow the time
evolution of the systems. An appropriate study of the dynamic behaviour of the system would have
required the use of the Molecular Dynamics techniques, but this was not of prime interest here.
The purpose of these simulations was to prove the existence of the bistability behaviour of the
model, and within the simulation run lengths available here, this has been fulfilled.


%==============================================================================================
\subsection{The I-N transition.}
\label{ss:surfInflINTans}

Here, the influence of confinement on the liquid crystalline phase behaviour of the model is
studied, specifically, the influence of confinement upon the location of the I-N transition is of
interest. To some extend, this issue has already been addressed for similar systems 
studied very recently by Zhou~\etal\cite{ZhouChen03}. This work was based on simulations of 
hard Gaussian overlap particles of elongation $k\in [2:3]$ confined between plane structureless
walls represented by a surface potential describing the interaction between an ellipsoid and 
a plane.
The authors found that the effect of confinement was to shift the location of the I-N transition
towards lower number densities. Another effect was enhancement of orientational order in that particles 
whose shape anisotropy was not sufficient for the formation of liquid crystalline phases in the
bulk (\ie 3d) displayed ordered phases with an order parameter consistent with a nematic 
phase in confined systems.\\

The effect of confinement upon the phase behaviour of the model studied here was assessed by
computing the variation of $\overline{P}_2$ and the average local density
$\overline{\rho^{*}_\ell}$ in bulk and surface regions as a function of the overall number density 
$\rho^{*}$ and reduced needle length $k_S/k$.
Here the  approach taken for the study of the anchoring transition was applied,
using $P_2(z)$ as the main observable.\\
The order phase diagrams of the system as a function of $\rho^{*}$ and $\kSp$ have been
computed from simulations with constant $\rho^{*}$ performed in series of increasing and decreasing
$\kSp$. The difference between this and the study of the anchoring transition is that now lines of
constant $\kSp$ on the diagram are of interest;
the results shown here have been computed using series of simulations with constant density 
rather than constant needle lengths as more data were available for the former. 
However comparison of data obtained from series with constant density and varying needle length 
show that ultimately both series agree. For the sake of completeness, 
Figure~\ref{fig:P2Wa-rho_fd} shows a sample of the results obtained for series of simulation 
with constant needle length and increasing densities.
The order phase diagrams (obtained from the series with constant number density) 
for $k=3$ in the interfacial and bulk regions are shown on 
Figure~\ref{fig:P2PhaseDia_k3}, and the evolution of $\overline{\rho^{*}_\ell}$ with 
$\rho^{*}$ and $k_S/k$ is shown on Figure~\ref{fig:rhoLPhaseDia_k3}. 
From these, the effects of confinement on the I-N transition can be assessed.\\

%---------------------------------------------------------
\picW = 10cm
\begin{figure}
	\centering
	\subfigure[interfacial region]{\picL{P2WaSu-rho_k3.ps}}
	\subfigure[bulk region]{\picL{P2WaBu-rho_k3.ps}}
	\caption{Evolution of $\overline{P}_2$ in the interfacial(a) and bulk(b) regions as
	obtained from simulations at constant $k_s/k$ and increasing densities.}
	\label{fig:P2Wa-rho_fd}
\end{figure}

%---------------------------------------------------------
\picW = 7cm
\begin{figure}
	\centering
	\subfigure[Series with increasing needle length]
	{\pic{P2WaSu_phaseDia_k3_LnU.ps}\pic{P2WaBu_phaseDia_k3_LnU.ps}}
	\subfigure[Series with decreasing needle length]
	{\pic{P2WaSu_phaseDia_k3_LnD.ps}\pic{P2WaBu_phaseDia_k3_LnD.ps}}
	\caption{Order phase diagrams for $k=3$ in the interfacial (left) and bulk (right) regions of the
	slab obtained from series with increasing and decreasing density.}
	\label{fig:P2PhaseDia_k3}
\end{figure}

%---------------------------------------------------------
\picW = 7cm
\begin{figure}
	\centering
	\subfigure[Series with increasing needle length]
	{\pic{rhoLSu_k3_S4.ps}\pic{rhoLBu_k3_S4.ps}}
	\subfigure[Series with decreasing needle length]
	{\pic{rhoLSu_k3_S2.ps}\pic{rhoLBu_k3_S2.ps}}
	\caption{average local density $\overline{\rho^{*}_\ell}$ for $k=3$ 
	in the interfacial (left) and bulk (right) regions of the
	slab obtained from series with increasing and decreasing density.}
	\label{fig:rhoLPhaseDia_k3}
\end{figure}


%---------------------------------------------------------
%\picW = 7cm
%\begin{figure}
%	\centering
%	\subfigure[Series with increasing needle lengths]
%	{\pic{P2WaSu_phaseDia_k5_LnU.ps}\pic{P2WaBu_phaseDia_k5_LnU.ps}}
%	
%	\subfigure[Series with decreasing needle lengths]
%	{\pic{P2WaSu_phaseDia_k5_LnD.ps}\pic{P2WaBu_phaseDia_k5_LnD.ps}}
%	\caption{Order phase diagrams for $k=5$ in the interfacial (left) and bulk (right) 
%	regions of the slab obtained from series with increasing and decreasing density.}
%	\label{fig:P2PhaseDia_k5}
%\end{figure}


%---------------------------------------------------------

Observation of these results reveals the main effects of confinement mentioned in
Chapter~\ref{chap:two} and at the beginning of this Chapter.
In the interfacial region, Figure~\ref{fig:rhoLPhaseDia_k3} shows an enhanced density
which, in turn, results in higher order as shown on Figure~\ref{fig:P2PhaseDia_k3}. As a result
of this, the average local density in the bulk region is reduced, and so is the degree of
ordering. Generally, therefore, the effect of confinement on these systems is to shift the 
isotropic-nematic transition to lower number densities in the interfacial region and to higher 
number densities in the bulk region.\\ 

Close observation of Figures~\ref{fig:P2PhaseDia_k3} and~\ref{fig:rhoLPhaseDia_k3}
shows, however, that the anchoring conditions have a strong influence on the surface induced 
shifts in the local
density and order parameter values. 
In particular and in the region corresponding to competing anchoring \ie $k_S \sim 0.5$,
$\overline{\rho^{*}_\ell}$ shows a sudden decrease in value which is accompanied by a region
of low orientational order. However, this effect seems to be much stronger on 
$\overline{P}_2$ than is suggested by the behaviour of $\overline{\rho^{*}_\ell}$. This is
because,
in addition to the reduced average local density, the double peaked nature of the $z$-profiles
indicates the particles to have competing preferred orientations, which reduces
the value of the order parameter. This leads to a higher shift in the number density of the 
I-N transition of competing anchoring cases than would be expected purely from local density
effects.\\

The results observed here are consistent with those obtained by Zhou~\etal\cite{ZhouChen03}. The
shifts in $\overline{\rho^{*}_\ell}$ and $\overline{P}_2$ with number density confirms 
that for non-competing anchoring conditions, 
the principal  effect of confinement is to enhance order in the systems and shift the 
location of the I-N transition towards lower number densities. 
Calculating the observables independently in the interfacial and bulk regions
shows that both regions exhibit a qualitatively but not quantitatively similar behaviour; this
was not, however, observed in~\cite{ZhouChen03}, where all observables were averaged
over the full samples, so that different shifts in the bulk and surface I-N 
transition densities were not monitored. 
Finally, we have shown that the  behaviour of the film is dependent on the type 
of anchoring applied; this issue was not addressed in~\cite{ZhouChen03} where only the 
case of strong planar anchoring was considered.






%===============================================================================================
%===============================================================================================
\conclusion

In this Chapter, the study of surface induced structural changes on a confined system of hard
Gaussian overlap particles has been addressed. The choice of the hard needle wall potential for
surface interaction allowed the observation of two stable surface arrangements, namely planar
and homeotropic according to $k_S$, the length of the needle embedded in the particles. The
mechanism responsible for the change in surface arrangement as a function of $k_S$ is the
varying amount of molecular volume that can be absorbed into the surface.\\
A systematic study of the behaviour of the system as a function of number density and needle 
length has been performed. From
this, an anchoring transition between the two arrangements has been identified and located
through the computation of anchoring phase diagrams. Also, differences between the diagrams
obtained from series of simulations performed with, respectively increasing and decreasing 
needle lengths have been used to identify bistability regions, where both homeotropic 
and planar arrangement remain stable on timescale of a simulation run length.\\ 
It has been shown that generally, the effect of confinement is to shift the isotropic-nematic 
transition to lower number densities close to the surfaces and to higher number densities 
in the bulk region.
In addition, it has been shown that the amount by which this transition is shifted varies with
the nature of anchoring conditions adopted.\\


%===============================================================================================
%===============================================================================================
