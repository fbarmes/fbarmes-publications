\section{Symmetric anchored systems}
\label{s:simRes}

Here we present simulation results from a study of surface induced structural changes.
Ellipsoidal shaped particles interacting through the hard Gaussian overlap model were 
considered while the HNW model was used for particle-substrate interactions.
Monte Carlo simulations in the canonical ensemble of systems of $N=1000$ particles 
confined in a slab geometry of height $L_z=4\sel$ were performed with
the substrate being located on the top and bottom of the box ($z=-L_z/2$ and $z=L_z/2$); 
the system was periodic in the two other dimensions.
The same anchoring conditions were applied on both substrates (symmetric anchoring).
Using this combination of particle and surface potentials, the aims of this study were 
to model the main influences of confinement in liquid crystalline systems, namely surface 
induced layering and ordering, and to identify the influences of density and needle length 
on the preferred  surface arrangements.
In order to do so, a systematic approach has been undertaken; Figure~\ref{fig:simsDia_HNW}(a) 
and (b) shows the state points at which the simulations have been performed 
as well as the direction of the simulation series. 
For each state points two simulations have been performed, first an equilibration simulation 
of $0.5.10^6$ sweeps has followed by a production run of another $0.5.10^6$ sweeps.

\picW = 10cm
\begin{figure}
	\centering
	\subfigure[$k=3$]{\picL{sims_k3Points.ps}}
	\subfigure[$k=5$]{\picL{sims_k5Points.ps}}
	\caption{Representation of the state points considered for the production of the data
	analysed in this Chapter.}
	\label{fig:simsDia_HNW}
\end{figure}

%==============================================================================================
%==============================================================================================
\subsection{Typical profiles}
\label{ss:typicalProfiles}

The first step is to understand surface induced structural changes for a set of needle
lengths and densities. References~\cite{Chrzanowska_Teixera_01,Cleaver_Teixeira_01} 
provide some information through profiles from systems of particles with $k=5$ and 
$k_S=5$~\cite{Chrzanowska_Teixera_01} and $k_S = 2.75,3.0$~\cite{Cleaver_Teixeira_01}. Here a
more exhaustive set of results are shown.\\

The study of the surface induced structural changes was performed through the computation of 
the $z$-profiles of three different observable.
The density profile $\rho_\ell(z)$ provides information about the layering in the cell 
through the location number and height of its peaks. $Q_{zz}(z)$, provides information 
on the surface induced ordering with respect to the substrate normal. These two profiles can be 
used to characterize the type of surface anchoring adopted. $P_2(z)$ measures the degree 
of orientational order and, hence, the nature of the phase as a function of location in the slab.\\
%
In what follows typical profiles are presented for several needle lengths and system densities. 
These profiles were obtained from several series of simulations each performed with constant density 
and $k_S$ either decreasing and increasing between the limits $[0:k]$.
Typical profiles are shown for densities corresponding to bulk isotropic  and nematic phases. 
Needle lengths are chosen so as to correspond to homeotropic 
($k_S<k^T_S$), competing ($k_S\sim k^T_S$), and planar arrangements ($k_S>k^T_S$).


\subsubsection{Homeotropic anchoring}

An homeotropic arrangement is observed when the orientation of the surface
particles is such that $<\theta>\sim 0$, which for the HNW model occurs for short needle
lengths ($\kSp \leq 0.25$). Typical profiles for this arrangement are showed on
Figures~\ref{fig:typicalProfile_k3_homeo} and~\ref{fig:typicalProfile_k5_homeo} for,
respectively, $k=3$ and $k=5$ using a reduced needle length $\kSp = 0.2$.\\
%
%-----------------------------------------
\picW = 14cm
\begin{figure}
        \centering
        \pic{typeProf_k3_homeo.ps}
	\caption{Typical profiles corresponding to an homeotropic anchoring for $k=3$ 
	and $\kSp = 0.20$.}
	\label{fig:typicalProfile_k3_homeo}
\end{figure}
%-----------------------------------------
%
%-----------------------------------------
\picW = 14cm
\begin{figure}
        \centering
        \pic{typeProf_k5_homeo.ps}
	\caption{Typical profiles corresponding to an homeotropic anchoring for $k=5$ 
	and $\kSp = 0.20$}
	\label{fig:typicalProfile_k5_homeo}
\end{figure}
%-----------------------------------------

From these Figures, the surface induced structural changes associated with the homeotropic 
arrangement can be
observed. $\rho^{*}_\ell(z)$ is an oscillatory function that displays its main peaks at
$|z-z_0|\sim \frac{L_n}{2}$ which corresponds to the first layer of particles. Also, it should
be noted that the periodicity of the peaks is very large (of the order of $\sel$), 
and no more than a total of five to six layers can be observed. 
This is compatible with the height of the simulation slabs of $L_z=4\sel$. The number of peaks in 
$\rho^{*}_\ell(z)$ clearly corresponds to a layered system with end to end alignment.\\

From the $Q_{zz}(z)$ profiles, the homeotropic arrangement can be determined from the regions 
of positive values.
These correspond to regions of high $\rho^{*}_\ell(z)$, that is the interfacial
regions for all densities and also the bulk region of the cell if the global number density
is sufficient to support nematic order. $Q_{zz}(z)$ is also an oscillatory function 
whose maxima match those in
$\rho^{*}_\ell(z)$; the higher the local density, the higher the order and, therefore,
the better the ordering with respect to the wall.\\
%
It should be noted however, that regardless of the number density, $Q_{zz}(z)$ displays negative
values very close to the surface ($|z-z_0|\leq \frac{L_n}{2}$)~; this is  because there are
always a few particles lying parallel to the walls in these regions. However, these planar
particles correspond to regions of very low density, $\rho^{*}_\ell$, and, therefore, their
effect on the overall behaviour of the system is insignificant~; besides the higher the density,
the less noticeable this effect.\\

$P_2(z)$ shows the orientational order as a function of position, a result that is not
easily obtained experimentally. $P_2(z)$ also follows the oscillations of
$\rho^{*}_\ell$, for the same reasons as $Q_{zz}(z)$; regions of high local density 
induce regions of high local in-plane order.\\


\subsubsection{Planar anchoring}

A planar arrangement is observed when the average orientation in the interfacial region is 
parallel to the substrate, that is $<\theta> \sim \frac{\pi}{2}$. In the case of the HNW
model, such an arrangement can be observed using long needle lengths 
($\kSp \geq 0.75$). Typical profiles measuring the surface induced structural changes 
for this arrangement  are shown on Figures~\ref{fig:typicalProfile_k3_planar} 
and~\ref{fig:typicalProfile_k5_planar}, respectively, for $k=3$ and $k=5$.
%
%-----------------------------------------
\picW = 14cm
\begin{figure}
        \centering
        \pic{typeProf_k3_planar.ps}
	\caption{Typical profiles corresponding to a planar arrangement with $k=3$ and $\kSp =
	0.80$.}
	\label{fig:typicalProfile_k3_planar}
\end{figure}
%-----------------------------------------

%-----------------------------------------
\picW = 14cm
\begin{figure}
        \centering
        \pic{typeProf_k5_planar.ps}
	\caption{Typical profiles corresponding to a planar arrangement and $k=5$ and $\kSp =
	0.80$.}
	\label{fig:typicalProfile_k5_planar}
\end{figure}
%-----------------------------------------

The first difference to be noted when comparing these results with those obtained in the case
of an homeotropic arrangement, is the difference in stratification.
%
Because at the surface,
$<\theta>\sim\frac{\pi}{2}$, the particles lie closer to the walls and as 
a result, the main peaks in $\rho^{*}_\ell$ are located at $|z-z_0|\sim 0$. The peak-peak
separation of these
functions are also much smaller than they were in the homeotropic case ($\sim\so$). 
The particles are, thus, now arranged in layers with a side by side alignment between one layer 
and the next.\\

$Q_{zz}(z)$ follows the same behaviour as $\rho^{*}_\ell(z)$ but adopts negative values in regions 
of high $\rho^{*}_\ell$. Therefore  maxima in $\rho^{*}_\ell$, induce minima in $Q_{zz}(z)$
because high planar order is measured by negative values of $Q_{zz}(z)$.
%
Comparison of the absolute value of $Q_{zz}(z)$ in the homeotropic and planar
arrangements, suggests that for a given number density homeotropic ordering with respect 
to the surface is `better' than the corresponding planar ordering. However 
since $Q_{zz} \in[-0.5:1]$, and an isotropic distribution of $\theta$ in a layer induces $Q_{zz}=0$, a
positive value of $Q_{zz}$ should be compared to the double of a negative value. As a result a
planar ordering with $Q_{zz}= -0.5$ is a `good' as an homeotropic ordering with $Q_{zz} = 1$.\\

The form taken by $P_2(z)$ is very similar for homeotropic and planar arrangement as both follow
the behaviour of the corresponding $\rho^{*}_\ell(z)$. The difference between the two
arrangements lies in the peaks separation of $P_2(z)$.


\subsubsection{Competing anchoring}

The term competing anchoring refers to a situation where both planar and homeotropic
arrangements  are of comparable strength. In the case of the HNW model, this corresponds to a 
situation where $\kSp \sim k_S^{T\prime}$. Typical profiles corresponding to competing anchoring are
shown on Figures~\ref{fig:typicalProfile_k3_bist} and~\ref{fig:typicalProfile_k5_bist}
respectively for $k=3$ and $k=5$.\\
These results have been obtained for needle lengths very close but not equal to $k_S^{T\prime}$ as the
values obtained by Equation~\ref{eqn:LnTfinalPoly} correspond to a very idealized case.
%
%
%-----------------------------------------
\picW = 14cm
\begin{figure}[!]
        \centering
        \pic{typeProf_k3_bist.ps}
	\caption{Typical profiles for $k=3$ and $L^{'}_n = 0.48$}
	\label{fig:typicalProfile_k3_bist}
\end{figure}
%-----------------------------------------
%
%-----------------------------------------
\picW = 14cm
\begin{figure}[!]
        \centering
        \pic{typeProf_k5_bist.ps}
	\caption{Typical profiles for $k=5$ and $L^{'}_n = 0.54$}
	\label{fig:typicalProfile_k5_bist}
\end{figure}
%-----------------------------------------

It should be noted here that at high number densities, different profiles suggesting different
surface arrangements were obtained depending on the history of the simulation sequence.
Moreover, as in each case the profiles were obtained from fully equilibrated configurations, the
presence of bistability is suggested here. Both sets of coexisting profiles are shown on the Figures.\\

In the case of \textbf{isotropic densities}, the surface induced layering shows features
reminiscent with
both planar and homeotropic influences. The interfacial region is characterized by two peaks 
of comparable height corresponding to particles with $\theta = 0$ and $\theta = \frac{\pi}{2}$. 
Since the
particles in the interfacial region were equally distributed between those two regions, the
heights of
the two peaks are much smaller than those seen in the cases of strong planar or homeotropic anchoring. 
%
This double behaviour can also be observed on $Q_{zz}(z)$, as, in the interfacial region, 
both negative and positive values corresponding to high local densities can be observed.
%
The $P_2(z)$ profile does not bring much more information as its behaviour follows that of
$\rho^{*}_\ell(z)$.\\


In the case of \textbf{nematic densities}, the behaviour of the system is very different. Here,
the increase in density had induced the particle to align and, therefore, choose one of 
the two possible
surface arrangements. This prevented the system from simultaneously exhibiting both planar 
and homeotropic features. The profiles indicate that in the case of runs with decreasing $k_S$ 
(\ie coming from the planar side), the cell displayed a planar arrangement whereas in the case of
increasing needle lengths an homeotropic arrangement was found. This suggest that in the case of
competing anchoring and high density, the surface arrangement was chosen according to the history
of the system.



%==============================================================================================
%==============================================================================================
\subsection{Influence of density}
\label{ss:rhoInfl}

Observation of the profiles reveals that the density has a strong influence on the
intensity of the surface induced effects; as stated earlier, 
the higher the number density the more intense the surface induced effects. 
In order to further study the influence of density on surface induced structural changes,
Further simulations were carried out
in series with constant $\kSp$ and increasing and decreasing densities. 
From these, the $\rho^{*}_\ell(z,\rho^{*})$, $Q_{zz}(z,\rho^{*})$, and $P_2(z,\rho^{*})$
surfaces have been computed. 
Results for the cases $\kSp = 0.0$ and $\kSp = 1.0$ are shown on Figures~\ref{fig:rhoInfl_k3_L000} 
and~\ref{fig:rhoInfl_k3_L100} for $k=3$. The corresponding results for $k=5$ 
are very similar  and are, therfore, not shown here. Also, because hardly any hysteresis has been found 
between series with increasing and decreasing densities, only the results or series with increasing 
densities are shown.

%-----------------------------------------
\picW = 7cm
\begin{figure}
        \centering
        \subfigure[]{\pic{rho-z-Ln_NVT_Sconf_k3_N1000_fd_L000_LnD.ps}}
	\subfigure[]{\pic{Qzz-z-Ln_NVT_Sconf_k3_N1000_fd_L000_LnD.ps}}
	\subfigure[]{\pic{P2-z-Ln_NVT_Sconf_k3_N1000_fd_L000_LnD.ps}}
	\caption{Influence of $\rho^{*}$ on the z-profiles from series of simulation with
	increasing density of particles with $k=3$ and $\kSp = 0.0$ (homeotropic anchoring).}
	\label{fig:rhoInfl_k3_L000}
\end{figure}
%-----------------------------------------
%-----------------------------------------
\picW = 7cm
\begin{figure}
        \centering
        \subfigure[]{\pic{rho-z-Ln_NVT_Sconf_k3_N1000_fd_L100_LnD.ps}}
	\subfigure[]{\pic{Qzz-z-Ln_NVT_Sconf_k3_N1000_fd_L100_LnD.ps}}
	\subfigure[]{\pic{P2-z-Ln_NVT_Sconf_k3_N1000_fd_L100_LnD.ps}}
	\caption{Influence of $\rho^{*}$ on the z-profiles from series of simulation with
	increasing density of particles with $k=3$ and $\kSp = 0.1$ (planar anchoring).}
	\label{fig:rhoInfl_k3_L100}
\end{figure}
%-----------------------------------------

Those measurements confirm the first observation using a smaller sample of densities 
(Section~\ref{ss:typicalProfiles}). At low densities, the surface induced effects are 
limited to the interfacial regions and the central region of the cell remains unaffected. 
On the $z-$profiles, this is characterized by short ranged surface features.\\
As the density is increased however, the surface influence extends further into the cell up to a
density for which the full slab is uniformally aligned.  The number of peaks on $\rho^{*}_\ell(z)$ 
increases steadily due to layering of the particles. As a result, the absolute values of the ordering 
observables increase as does the distance from the surface at which those functions 
start to decay. This translates to greater orientational ordering that extends further into the cell.

It is interesting to note that in the case of extreme homeotropic anchoring ($\kSp = 0.0$),
regardless the density, the cell never displays uniform alignment since the bulk part of the
cell never orders. This seems to contradict previous observations of systems of particles with
elongation $k=3$ under homeotropic anchoring
(see Figure~\ref{fig:typicalProfile_k3_homeo}). However, due to the reduced needle lengths used
here, the particle volume absorbed by the substrate leads to a lowering of the density in the
bulk regions. This has the effect of shifting the I-N transition to densities out of the range
considered here. This does not, however, question the existence of uniform alignment in the case
$\kSp = 0.0$ as this has been observed with k=5.\\

Another effect that can be observed on Figures~\ref{fig:rhoInfl_k3_L000} and~\ref{fig:rhoInfl_k3_L100}
is that the z gradients in $|Q_{zz}(z)|$ and $\rho^{*}_\ell(z)$ in the interfacial region increase 
when  the bulk part of the cell becomes nematic.  As a result of the bulk orientational order, 
interfacial 
regions come under the ordering influence of both the surfaces, through anchoring effects, and the 
bulk part of the cell, through elastic forces. This improves the quality of the layering and 
ordering in the interfacial regions.


%==============================================================================================
%==============================================================================================
\subsection{Influence of $k^{\prime}_S$}
\label{ss:LnInfluence}

The influence of $\kSp$ on the cell's behaviour is considerable, as this variable
controls the type and strength of the surface anchoring.
%
As the needle lengths is increased
between from zero, the system undergoes a transition from homeotropic 
to planar anchoring. Section~\ref{ss:PHstability} gave the critical $\kSp$ values
for the this transition in the limit of close packing.
%
In the simulations however, the local density at the interfaces is far from being 
that of close packing and is moreover a function of the global density. Therefore some 
shifts in $k_S^{T\prime}$ from the theoretical values are to be expected.\\
%
The purpose of this Section is to study,  in greater detail, the effect of $k^\prime_S$
on the profiles and show the existence and nature of the homeotropic to planar anchoring
transition.
This has been achieved
using simulations performed at constant densities  in series of increasing and decreasing 
needle lengths. 
Two densities known to bulk isotropic and nematic phases have been considered.
Results from the series with $k^\prime_S$ densities and $k=3$ are shown in 
Figure~\ref{fig:LnInfl_k3_d0.28} (isotropic density) and~\ref{fig:LnInfl_k3_d0.34} (nematic
density). The differences between the series with increasing and decreasing $\kSp$ are discussed
in the next Section.\\

%-----------------------------------------
\picW = 7cm
\begin{figure}
        \centering
        \subfigure[]{\pic{rho-z-Ln_NVT_Sconf_k3_N1000_fL_d0.2800_LnD.ps}}
	\subfigure[]{\pic{Qzz-z-Ln_NVT_Sconf_k3_N1000_fL_d0.2800_LnD.ps}}
	\subfigure[]{\pic{P2-z-Ln_NVT_Sconf_k3_N1000_fL_d0.2800_LnD.ps}}
	\caption{influence of $\kSp$ on the z-profiles for $k=3$ and $\rho^{*} = 0.28$.}
	\label{fig:LnInfl_k3_d0.28}
\end{figure}
%-----------------------------------------


%-----------------------------------------
\picW = 7.0cm
\begin{figure}
        \centering
        \subfigure[]{\pic{rho-z-Ln_NVT_Sconf_k3_N1000_fL_d0.3400_LnD.ps}}
	\subfigure[]{\pic{Qzz-z-Ln_NVT_Sconf_k3_N1000_fL_d0.3400_LnD.ps}}
	\subfigure[]{\pic{P2-z-Ln_NVT_Sconf_k3_N1000_fL_d0.3400_LnD.ps}}
	\caption{influence of $\kSp$ on the z-profiles for $k=3$ and $\rho^{*} = 0.34$.}
	\label{fig:LnInfl_k3_d0.34}
\end{figure}
%-----------------------------------------

From Figures~\ref{fig:LnInfl_k3_d0.28} and~\ref{fig:LnInfl_k3_d0.34}, typical features 
discussed previously corresponding to homeotropic and planar anchoring arrangements can be
found by looking at curves corresponding to low and high values of $\kSp$, respectively.
Here, rather,
 focus is brought to bear on the regions corresponding to competing anchoring, as these reveal
the changes that take place during the anchoring transition.\\
%
As the transition region is approached (\eg from the planar side), the height of the peaks in 
$\rho^{*}_\ell$ decrease rapidly, and at the transition, hardly any oscillations can be observed. 
This can be understood easily as at the transition, under equal influence from both type of
anchoring, the particles diffuse almost homogeneously in bimodal surface layers which have
features corresponding to both
anchoring states; as a result, the density profile becomes relatively uniform. As the 
homeotropic anchoring becomes stronger,
the system layers accordingly and a new oscillating pattern corresponding to homeotropic
anchoring can be observed. It is interesting to note that the location 
maxima in $\rho^{*}_\ell(z)$ in the latter case coincide with those of the minima of the former.\\
%
As well as the changes in density, the centre of the slab undergoes an orientational 
disorder-order transition with $\kSp$ as can be
observed on $P_2(z,\kSp)$. The z-dependent oscillation pattern of this function also changes
according to the state of the surface alignment.\\

$Q_{zz}(z,\kSp)$ shows the orientational reorganization that occurs during the transition.
Upon approaching the transition region, this profile changes rapidly between the planar and
homeotropic characteristics. The gradient of this change increases with the density but also
with $k$; the transition is much sharper for $k=5$ than for $k=3$.
Considering the needle length at which the change in the sign of $Q_{zz}$ is observed, the
theoretical predictions made in section~\ref{ss:PHstability} seem to be confirmed, particularly
as the density is increased. A more detailed study of the homeotropic to planar transition
region is presented in Section~\ref{s:surfInfphaseTrans}.



