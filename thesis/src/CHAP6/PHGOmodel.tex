

\section{The PHGO model.}

Here a novel model for non-centrosymmetric particles is described. The route adopted
is to extend the generalized Gay-Berne (GGBP) potential~\cite{CleaverCare96} to the description of
non-centrosymmetric particles thus leading to the so called Parametric Hard Gaussian Overlap (PHGO)
model.

%================================================================================================
%================================================================================================
\subsection{Obtaining the contact distance.}

The generalized Gay-Berne model describes the interaction between two ellipsoidal particles 
$i$, $j$ with arbitrary lengths $\ell_i$, $\ell_j$ and breadth $d_i$, $d_j$ and orientations
$\ui$ and $\uj$ separated by an intermolecular vector $\vect{r}_{ij} = r_{ij}\rij$. The contact 
distance for this model is given by~:
%
\begin{multline}
\sijr = \\
\sigma_0 \left[ 1 - \chi\left\{
    \frac{\alpha^2\dotProdP{\rij}{\ui}^2 + \alpha^{-2}\dotProdP{\rij}{\uj}
    -2\chi\dotProdP{\rij}{\ui}\dotProdP{\rij}{\uj}\dotProdP{\ui}{\uj}  }{1-\chi^2\dotProdP{\ui}{\uj}^2}
    \right\}\right]^{-\frac{1}{2}}
    \label{eqn:GGB_CD_original}
\end{multline}

with~:
\begin{eqnarray*}
    \sigma_0 &=& \sqrt{\frac{d^2_i + d^2_j}{2}}               \\
    %
    \alpha^2 &=& \left[ \frac{ (\ell^2_i-d^2_i)(\ell^2_j+d^2_i)}
    {(\ell^2_j-d^2_j)(\ell^2_i+d^2_j)}\right]^{\half}         \\
    %
    \chi &=& \left[ \frac{(\ell^2_i-d^2_i)(\ell^2_j-d^2_j)}
    {(\ell^2_j+d^2_i)(\ell^2_i+d^2_j)}\right]^{\half}.         \\
    %
\end{eqnarray*}
If, alternatively, the brackets containing the length and breadth values are grouped as~:
%
\begin{equation*}
	\begin{array}{cc}	
	A = (\ell^2_i-d^2_i) &B = (\ell^2_j-d^2_j) \\
	C = (\ell^2_j+d^2_i) &D = (\ell^2_i+d^2_j),
    	\end{array}
\end{equation*}
%
the shape parameter can be rewritten as~:
\begin{multline}
    \sijr = \\
    \sigma_0 \left[ 1 - \frac{ AC\dotProdP{\rij}{\ui}^2 + BD\dotProdP{\rij}{\uj}^2
    - 2AB\dotProdP{\rij}{\ui}\dotProdP{\rij}{\uj}\dotProdP{\ui}{\uj} }
    {CD - AB\dotProdP{\ui}{\uj}^2}
    \right]^{-\half}
    \label{eqn:GGB_CD_modified}
\end{multline}

Although the two forms of Equations~\ref{eqn:GGB_CD_original} and~\ref{eqn:GGB_CD_modified} 
are mathematically equivalent, the second is to be preferred for
implementation in a simulation code as it is free of possible division by zero or complex
numbers in the limit $\ell=0$ and $d = 0$.\\

By design, Equation~\ref{eqn:GGB_CD_modified} is restricted to the description of particles with 
fixed $\ell$ and $d$. The extension from the $GGB$ to the $PHGO$ contact distance is to consider
particles for which $\ell$ and $d$ vary parametrically with $\dotProdP{\rij}{\ui}$ and 
$\dotProdP{\rij}{\uj}$. As a result the shape of $i$ as `seen' by $j$ 
can be tuned to be a function of the relative position and orientation of the particles. 
This makes it possible to model particles with arbitrary non-centrosymmetric shapes although the
approximation made by the model can lead to anomalies if non-convex particle shapes are used.

This method can therefore be applied to the description of particles having the B\'ezier shape used 
by the Stone expansion of the HP model as these do satisfy the convexity requirements of the model.
When two such particles interact with their sharp ends, $\ell/d$ needs to be 
relatively large, whereas the  blunt end interaction requires an $\ell/d$ ratio rather nearer to unity. 
In order to avoid discontinuities between these two limiting cases, a multitude of 
parametric forms is possible; here the form of $\ell_i$ and $d_i$ is limited to simple 
polynomials of the polar angle $\dotProdP{\rij}{\ui}$, that is~:
\begin{eqnarray}
        d_i(\dotProd{\rij}{\ui}) &=& a_{d,0} +\ldots + a_{d,n}(\dotProd{\rij}{\ui})^n\\     
	\label{eqn:drui}
        \ell_i(\dotProd{\rij}{\ui}) &=& a_{\ell,0} +\ldots + a_{\ell,m}(\dotProd{\rij}{\ui})^m.
\end{eqnarray}

The main limitation of this model is that the description of concave
particles (i.e. dumbells) is not avaliable.
The advantages are, however, numerous as the analytical form makes it possible to use the model
in both MC and MD codes. Also, this monosite model does not introduce any discontinuities in
the particle-particle contact surfaces.
Although the use of polynomials to describe the dependency of $\ell$ and $d$ over 
$\dotProdP{\rij}{\ui}$ might seem simplistic, this allows a very
straightforward implementation in the coefficients $A,B,C$ and $D$ by simply adding higher order
terms. Reversibly, the contact distance of the HGO model can be recovered in taking the limit $m=n=0$.\\
Finally simulations of multi-component fluids is an easy extension of this model, as a set of 
$a_{\ell,i}$, $a_{d,i}$ can be assigned to every type of particle. Systems of particles whose
shapes vary within a simulation run can also be achieved using this approach.

%================================================================================================
%================================================================================================
\subsection{Parameterizing B\'ezier pears.}

In order to apply the PHGO approach to pear-shaped particles, the coefficients $a_{\ell,i}$ and
$a_{d,i}$ need to be computed. This has been performed by numerically fitting the particle-point
potential to a geometrical shape. For this, a numerical simplex  method~\cite{NumericalRecipes}
has been chosen.\\
The shape that is considered here is a simple update of the B\'ezier shape described in
Section~\ref{s:HP_model}, the difference being that the control points $q_2$ and $q_3$ are set
to be always coincident. This makes the B\'ezier shape easily scalable with $k$. 
This new B\'ezier shape is shown on Figure~\ref{fig:GBP_BzPear} and the associated control
points coordinates as a function of $k$ are given on Table~\ref{tble:GBP_CtrlCoords}.

%=============================
\picW = 12cm
\begin{figure}
	\centering
	\hspace*{2cm}\picL{GBP_BzPear.ps}
	\caption{B\'ezier shape for parameterized with the PHGO model.}
	\label{fig:GBP_BzPear}
\end{figure}
%=============================

\begin{table}
	\centering
	\begin{tabular}{||c||c||c||}
	\hhline{|t:=:t:=:t:=:t|}
	\hspace{5mm}$q_i$\hspace{5mm}	&\hspace{5mm}x\hspace{5mm}	&\hspace{5mm}y\hspace{5mm}	\\
	\hhline{|:=::=::=:|}
	$q_1$				&$-\frac{1}{2}\sigma_0$		&$0$			\\
	$q_2$				&$0$				&$\frac{2}{3}k\sigma_0$	\\
	$q_3$				&$0$				&$\frac{2}{3}k\sigma_0$	\\
	$q_4$				&$\frac{1}{2}\sigma_0$		&$0$			\\
	$q_5$				&$\sigma_0$			&$-\frac{2}{3}k\sigma_0$\\
	$q_6$				&$-\sigma_0$			&$-\frac{2}{3}k\sigma_0$\\
	\hhline{|b:=:b:=:b:=:b|}
	\end{tabular}
	\caption{B\'ezier control points coordinates for the B\'ezier pear used with the PHGO model.}
	\label{tble:GBP_CtrlCoords}
\end{table}

%=============================

Using the geometrical properties of the B\'ezier curves, it is possible 
to extract the coordinates of each point of the pear surface and, hence, the numerical data
corresponding to the particle-point contact distance as a function of $\dotProdP{\ui}{\rij}$. 
These data are then fitted to the expression for the PHGO contact distance with
$\ell_j=d_i=0$ which in this case reduces to~:
\begin{equation}
	\sir = \frac{d_i(\dotProd{\rij}{\ui})l_i(\dotProd{\rij}{\ui})}
	{\left\{ l_i^2(\dotProd{\rij}{\ui}) +
	\dotProdP{\ui}{\rij}^2\left[ d_i^2(\dotProd{\rij}{\ui})- l_i^2(\dotProd{\rij}{\ui})
	\right] \right\}^{\frac{1}{2}}}
	\label{eqn:GBP_sir}
\end{equation}

The values of the $a_{d,\alpha}$ and $a_{\ell,\alpha}$ coefficients obtained from the fitting
procedure are shown on Table~\ref{tble:GBP_dili}. $n=10$ and
$m=1$ have been chosen as this was found to be the best compromise between simulation speed 
and fit accuracy.
%
%=======================================================================================================
%=======================================================================================================
\tabul = 1.0cm
\begin{table}
    \centering
    \begin{tabular}{||c||c||c||c||}
    \hhline{|:t=:t:=:t:=:t:=:t|}
    \TAB\TAB    &   \TAB$k=3$\TAB   &   \TAB$k=4$\TAB   &   \TAB$k=5$\TAB   \\
    \hhline{|:=::=::=::=:|}
    $a_{d,0} $       &   0.501852454     &    0.501377232    &    0.497721868    \\
    $a_{d,1} $       &   -0.141145314    &   -0.129608758    &   -0.123155821    \\
    $a_{d,2} $       &   -0.060542359    &   -0.074219217    &    0.024405876    \\
    $a_{d,3} $       &   0.225813650     &   0.484166441     &    0.723627215    \\
    $a_{d,4} $       &   0.832274021     &   0.923492941     &    0.389831429    \\
    $a_{d,5} $       &   -1.015039575    &   -1.987232902    &   -3.018638148    \\
    $a_{d,6} $       &   -2.504045172    &   -2.943008017    &   -1.951629076    \\
    $a_{d,7} $       &   1.375313426     &   2.808075172     &    4.413215403    \\
    $a_{d,8} $       &   3.196830129     &   3.815344782     &    2.998417509    \\
    $a_{d,9} $       &   -0.699241457    &   -1.426641750    &   -2.241573216    \\
    $a_{d,10}$       &   -1.430400139    &   -1.682476460    &   -1.416614353    \\
    \hhline{|:=::=::=::=:|}
    $a_{\ell,0} $       &   1.498259615     &   1.995906501     &    2.493069403    \\
    $a_{\ell,1} $       &   -0.002027616    &   -0.004518187    &   -0.008067236    \\
    \hhline{|:b=:b:=:b:=:b:=:b|}
    \end{tabular}
    \caption{Values of the $a_{d,\alpha}$ and $a_{\ell,\alpha}$ for the PHGO pears with $k=3$, 4 and 5}
    \label{tble:GBP_dili}
\end{table}
%=======================================================================================================
%
The accuracy of the fits can be checked by comparing the contact distances
calculated using the coefficients of Table~\ref{tble:GBP_dili} and the actual B\'ezier shape
(Figure~\ref{fig:GBP_2dFit}).
The comparison reveals a very good correspondence between the two sets. The critical region 
being for $\dotProdP{\rij}{\ui} \sim 0$. The level of correspondence between the numerical and
analytical result to be achieved in this region dictates the number of coefficients to be used.
While using $n=m=1$ allows a good fit for the head and tail regions of the particles
up to $n=10$ coefficients in $a_{d,\alpha}$ were required to achieve the quality of fits shown on
Figure~\ref{fig:GBP_2dFit}.
%
%===========================
\picW=8.0cm
\begin{figure}
	\centering
	\subfigure[$k=3$]{\picL{GBP_profile_k3.ps}}
	\subfigure[$k=4$]{\picL{GBP_profile_k4.ps}}
	\subfigure[$k=5$]{\picL{GBP_profile_k5.ps}}
	\caption{Comparison between the numerical and geometrical contact distance for the PHGO
	particle-point potential.}
	\label{fig:GBP_2dFit}
\end{figure}
%===========================
%

The quality of these fits does not, however, guarantee an appropriate particle-particle
interaction and the validity of the latter is assessed by computing the contact surfaces
equivalent to those considered for
the HP model (\eg Figures~\ref{fig:HP_k3_contactSurf} and~\ref{fig:HP_k5_contactSurf} ). 
Again three contact surfaces were computed corresponding
to parallel, anti-parallel and T-geometry interactions. The results for $k=3$ and $k=5$ are
shown on Figures~\ref{fig:GBP_contactSurfaces_k3} and~\ref{fig:GBP_contactSurfaces_k5}.

%===========================
\picW = 7.0cm
\begin{figure}
	\centering
	\subfigure[parallel]{\pic{GBP_k3_ppShape_0PI.ps}}
	\subfigure[anti-parallel]{\pic{GBP_k3_ppShape_PI.ps}}
	\subfigure[T-geometry, x-z view]{\pic{GBP_k3_ppShape_0.5PI.ps}}
	\subfigure[T-geometry, y-z view]{\pic{GBP_k3_ppShape_0.5PI_2.ps}}
	\caption{Contact surfaces for the PHGO model and $k=3$.}
	\label{fig:GBP_contactSurfaces_k3}
\end{figure}
%===========================

%===========================
\picW = 7.0cm
\begin{figure}
	\centering
	\subfigure[parallel]{\pic{GBP_k5_ppShape_0PI.ps}}
	\subfigure[anti-parallel]{\pic{GBP_k5_ppShape_PI.ps}}
	\subfigure[T-geometry, x-z view]{\pic{GBP_k5_ppShape_0.5PI.ps}}
	\subfigure[T-geometry, y-z view]{\pic{GBP_k5_ppShape_0.5PI_2.ps}}
	\caption{Contact surfaces for the PHGO model and $k=5$.}
	\label{fig:GBP_contactSurfaces_k5}
\end{figure}
%===========================


These surfaces show the required shapes, that is approximately ellipsoidal shapes
for the parallel interactions and pear shapes for the anti-parallel
interactions. Although there is not really a required shape for the T-geometry, the latter do
not show any discontinuities that would indicate potential problems.
They are also reasonably consistent with those obtained for the corresponding HP contact surfaces.\\

The difference between the contact surfaces for this model and the HP model are readily
understandable. For the PHGO model, ridges in the contact surfaces have disappeared save for small
oscillations which are a direct consequence of the degree of truncation of the fits shown on
Figure~\ref{fig:GBP_2dFit}. However the low amplitude of these oscillation suggests that the PHGO
model should be free of the interlocking phenomena shown by the HP model.




