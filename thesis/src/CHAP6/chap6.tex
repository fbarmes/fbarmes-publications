

\chapter{Bulk simulations of pear shaped particles}
\label{chap:six}

%===================================================================================================
%===================================================================================================
\introduction

The last two Chapters have dealt with surface induced effects in systems of hard Gaussian 
ellipsoids subject to a range of different surface potentials. For some of these, bistable regions 
corresponding to different surface arrangements have been found. Here, a new line of work is
considered, the focus of which the behaviour of non-centrosymmetric, pear-shaped particles.
In the next Chapter, these two threads will be brought together in a study of these pear-shaped
particles in confined geometries.\\
It is recalled that the ultimate aim of this thesis is to model surface induced switching between
two arrangements of an hybrid anchored cell following the theoretical
treatment of Davidson and Mottram~\cite{DavidsonMottram02}. The system properties required
to achieve this 
are a bistable surface and a stable nematic phase of flexoelectric particles. The former problem
has already been addressed; this Chapter deals with the latter. For this, the development of a
model for pear-shaped particles is studied as particles of that shape are thought to display 
flexoelectric behaviour~\cite{Meyer69}.\\
In this Chapter, two models are studied, the first one hereafter referred to as the HP model is a
hard version of the potential  used in~\cite{BerardiRicci01}. Despite having a rich phase
behaviour, 
we show that this model does not have a stable nematic phase. As a result a second model,
which we term the Parametric Hard Gaussian Overlap (PHGO) model, has been developed which does
yield the required phase behaviour. The work presented 
in this Chapter has been submitted for publication to Physical Review E~\cite{BarmesRicci03}.

%===================================================================================================
%===================================================================================================


\section{The HP model}
\label{s:HP_model}

In 2001, Berardi \etal developed the first soft, single site model for non-centro\-sym\-metric
anisotropic particles
hereafter referred to as the soft pear (SP) model. This model uses a contact distance describing a
pear shape embedded within a Gay-Berne like potential. This model was taken as a base for the 
HP model described here. The HP model defines a steric potential $\mathcal{V}^\mrm{HP}$ between 
two pear-shaped objects whose contact distance is that of the SP model~\cite{BerardiRicci01}
as~:
%
\begin{equation}
	\mathcal{V}^{HP} = \left\{ 	%}
	\begin{array}{ccc}
		0	&\mathrm{if}	&r_{ij} \geq \sigma^{HP}(\ijr) \\
		\infty	&\mathrm{if}	&r_{ij} < \sigma^{HP}(\ijr) 
	\end{array}
	\right.
\end{equation}

where $\sigma^{HP}(\ijr)$ is the contact distance between two particles $i$ and $j$
with orientations $\ui$ and $\uj$ and and $\rij =\frac{\vect{r}_{ij}}{r_{ij}}$ 
where $\vect{r}_{ij}$ is the intermolecular separation. 
With this model, the contact distance is obtained using a numerical method 
following the approach of Zewdie~\cite{Zewdie98a,Zewdie98b}. The prototype shape of the particles is
defined using a set of two B\'ezier curves (Fig~\ref{fig:HP_bezier_k3}). The coordinates of the
control points of these, $q_{1..6}$, are given in Table~\ref{tble:HP_ctrlPts_k3}.

\picW = 12cm
\begin{figure}
	\centering
	\hspace*{2.5cm}\picL{HP_BzPear.ps}
	\caption{The B\'ezier curve used for the geometrical definition of the pear shaped of
	the HP model.}
	\label{fig:HP_bezier_k3}
\end{figure}

\begin{table}
	\centering	
	\begin{tabular}{||c||c||c||}
	\hhline{|t:=:t:=:t:=:t|}
	\hspace*{2mm} $q_i$ \hspace{2mm} &\hspace{2mm} x \hspace{2mm}	&\hspace{2mm} y \hspace{2mm}\\
	\hhline{|:=::=::=:|}
	$q_1$	&$-0.5$			&$0$		\\
	$q_2$	&$-0.5+ h\tan\alpha$	&$h$		\\
	$q_3$	&$0.5 - h\tan\alpha$	&$h$		\\
	$q_4$	&$0.5$			&$0$		\\
	$q_5$	&$-0.5 - h\tan\alpha$	&$-h$		\\
	$q_6$	&$0.5 + h\tan\alpha$	&$-h$		\\
	\hhline{|b:=:b:=:b:=:b|}
	\end{tabular}
	\caption{Coordinates of the B\'ezier control points for the HP model and $k=3$.}
	\label{tble:HP_ctrlPts_k3}
\end{table}


Following this a numerical
contact distance $\mathcal{L}(\ui,\uj,\rij )$ is computed for a given
set of $\ui, \uj$. This numerical distance is then fitted to a truncated Stone expansion as~:
\begin{eqnarray}
	%
	\sijr   &\simeq&  \mathcal{L}(\ui,\uj,\rij) \nonumber \\
	        &=& \sum_{L_1,L_2,L_3}\sigma_{L_1, L_2, L_3}S^{*L_1, L_2, L_3}(\ui,\uj,\rij)
	\label{eqn:Stone_sigma1}
\end{eqnarray}

where $S^{L_1, L_2, L_3}$ is a Stone function~\cite{Stone}, and the expansion coefficients 
$\sigma_{L_1, L_2,L_3}$ are given by~:
\begin{equation}
	%
	\sigma_{L_1, L_2, L_3} = \frac
	{
	\int \mathcal{L}(\ui,\uj,\rij)S^{L_1, L_2, L_3}
	(\ui,\uj,\rij)d\ui d\uj d\rij
	}
	{
	\int S^{*L_1, L_2, L_3}(\ui,\uj,\rij)S^{L_1, L_2, L_3}
	(\ui,\uj,\rij)d\ui d\uj d\rij
	}
	\label{eqn:Stone_sigma2}
\end{equation}

The non-zero coefficients $\sigma_{L_1, L_2, L_3}$ for particles with elongation $k=3$ and \\
$\{L_1, L_2, L_3\} \leq 6$ are given
in~\cite{BerardiRicci01}~; the corresponding coefficients for $k=5$ are
given in Table~\ref{tble:HP_sigma_k5}.\\

\begin{table}
    \centering
    \begin{tabular}{||cc||cc||cc||}
    \hhline{|t:==:t:==:t:==:t|}
      $[000]$ & $    1.90456 $ & $[011]$ & $    0.51113 $ & $[101]$ & $    0.51113 $  \\
      $[022]$ & $    2.01467 $ & $[202]$ & $    2.01467 $ & $[033]$ & $   -0.11376 $  \\
      $[303]$ & $   -0.11376 $ & $[044]$ & $    0.91479 $ & $[404]$ & $    0.91479 $  \\
      $[055]$ & $   -0.29937 $ & $[505]$ & $   -0.29937 $ & $[066]$ & $    0.41523 $  \\
      $[606]$ & $    0.41523 $ & $[110]$ & $   -0.03942 $ & $[121]$ & $   -0.45400 $  \\
      $[211]$ & $   -0.45400 $ & $[123]$ & $    0.59579 $ & $[213]$ & $    0.59579 $  \\
      $[132]$ & $    0.17137 $ & $[312]$ & $    0.17137 $ & $[143]$ & $   -0.27083 $  \\
      $[413]$ & $   -0.27083 $ & $[220]$ & $   -0.56137 $ & $[222]$ & $   -2.78379 $ \\
      $[224]$ & $    2.41676 $ & $[231]$ & $    0.31104 $ & $[321]$ & $    0.31104 $  \\
      $[233]$ & $    0.45382 $ & $[323]$ & $    0.45382 $ & $[242]$ & $    0.38115 $  \\
      $[422]$ & $    0.38115 $ & $[244]$ & $   -1.69388 $ & $[424]$ & $   -1.69388 $  \\
      $[246]$ & $    1.40664 $ & $[426]$ & $    1.40664 $ & $[330]$ & $   -0.07836 $  \\
      $[440]$ & $   -0.17713 $ & $[442]$ & $   -0.52246 $ &    &   \\
    \hhline{|b:==:b:==:b:==:b|}
    \end{tabular}
    \caption{The non zero $\sigma_{L_1, L_2, L_3}$ coefficients for $k=5$.}
    \label{tble:HP_sigma_k5}
\end{table}


The original study of the SP model revealed stable nematic and smectic phases and, using an
appropriate energy parameterisation, the same phases with net polar order were also obtained.
A reasonable expectation, therefore, is that
the steric version of this model should exhibit at least a stable nematic phase.
Such a correlation is found virtually in all soft LC models and their hard-particle equivalent.
The best example of this is seen on comparing the Gay-Berne~\cite{deMiguelRull91,deMiguelRull91a} 
and hard Gaussian overlap phase behaviours~\cite{DeMiguelDelRio01}.
As the former is reduced to its steric equivalent, the smectic phases disappear whereas the 
nematic-isotropic phase transition not only persists, but remains at virtually the same density.\\

As the HP model can not be reduced to the interaction between a particle and a point, the actual
shape of the particle and, hence, the accuracy with which this model represents the geometrical
B\'ezier curve is not available. However, for the two elongation considered in this study, 
some insight into the potential possible behaviour can be obtained by computation 
of the contact surfaces between two particles. 
Such surfaces show the location of the contact point between a particle $i$ located at the origin
with fixed orientation $\ui$ and a particle $j$ with fixed orientation $\uj$ whose position
is uniformly distributed on the unit sphere. Example surfaces, for
parallel particles ($\ui=\uj=\vecth{z}$), anti-parallel particles ($\ui=-\uj=\vecth{z}$) 
and the T-geometry ($\ui=\vecth{z}$, $\uj=\vecth{x}$), are shown on 
Figure~\ref{fig:HP_k3_contactSurf}.

\picW = 7cm
\begin{figure}
	\centering
	\subfigure[parallel particles]{\pic{HP_3DProf_k3_0.ps}}
	\subfigure[anti-parallel particles]{\pic{HP_3DProf_k3_PI.ps}}
	\subfigure[T-geometry, x-z view]{\pic{HP_3DProf_k3_0.5PI.ps}}
	\subfigure[T-geometry, y-z view]{\pic{HP_3DProf_k3_0.5PI_2.ps}}
	\caption{Contact surfaces for HP particles with $k=3$.}
	\label{fig:HP_k3_contactSurf}
\end{figure}

\picW = 7cm
\begin{figure}
	\centering
	\subfigure[parallel particles]{\pic{HP_3DProf_k5_0.ps}}
	\subfigure[anti-parallel particles]{\pic{HP_3DProf_k5_PI.ps}}
	\subfigure[T-geometry, x-z view]{\pic{HP_3DProf_k5_0.5PI.ps}}
	\subfigure[T-geometry, y-z view]{\pic{HP_3DProf_k5_0.5PI_2.ps}}
	\caption{Contact surfaces for HP particles with $k=5$.}
	\label{fig:HP_k5_contactSurf}
\end{figure}

Generally, these show the expected interaction, but it is evident that several non-convex
fractures are present; this raises the prospect of possible
enhanced stability for some
configurations, specifically for those of interlocking side by side anti-parallel particles. 
The consequence of  this would be to unduly stabilize 
these configurations and as a result prevent particles from `sliding' along one another.
Comparison of these contact surfaces for different elongations reveals different shape and
convexity behaviours. Thus rather different phase behaviours are to be expected for systems of 
particles with different elongations.\\
%
Problems related to non convexity of contact surfaces have previously been found by
Williamson and Jackson~\cite{WilliamsonJackson98} in their study of systems of linear hard 
sphere chains (LHSC). For that model, the authors noticed that the lack of convexity in the  
the particle-particle interaction surfaces lead to the formation of glassy phases. For the LHSC
model, this problem
could be resolved by the inclusion of reptation moves in the Monte Carlo sequence. 
This, however, is not possible with the single site HP model; the consequence of this is
addressed in Section~\ref{s:HPresults}.\\
%
Also the lack of convexity of the contact surfaces for the HP model would suggest that the HP
model might display a rather different phase behaviour than that obtained using
the SP model. This is because the energy minima of the soft tapered model systematically 
correspond to molecular separations greater than the contact distance and, therefore, the effects
of non convexity might not have been accessible to the models considered by the Italian group.





\section{Simulation results using the HP model.}
\label{s:HPresults}

\subsection{Particles with $k=3$.}

The phase behaviour of the HP model was computed using Monte Carlo simulations of bulk systems 
of $N=1250$ particles in the isothermal-isobaric ensemble. For better sampling of phase space 
and to ease the formation of possible polar phases, `flip moves' were included using a probabilistic
scheme. These moves involved the inversion of the molecular orientation vector $\ui$ and 
accounted for 20\% of each particle's attempted moves. Compression moves which changed every
box length independently were also carried out using a  probabilistic scheme, on average once 
every two MC sweeps (\ie moves per particle). 
Run lengths were of the order of $1.10^6$ sweeps for equilibration and production, but for 
some of the highest densities up to $5.10^6$ sweeps were necessary to achieve
equilibration.\\

The first series of runs used a compression sequence; the pressure range considered was chosen so that 
it would induce densities of the same order of magnitude as those found for the
isotropic-nematic transition of the hard Gaussian overlap model (see Chapter~\ref{chap:three}.)
The pressure and order parameter results obtained are shown in 
Figure~\ref{fig:HP_k3_phaseDia} under the `compression' label.

%=================================
\picW = 7cm
\begin{figure}
	\subfigure[$P(\rho^{*})$]{\picL{HP_P-rho_k3.ps}}
	\subfigure[$\langle P_2\rangle(\rho^{*})$]{\picL{HP_P2-rho_k3.ps}}
	\caption{Results from constant NPT Monte Carlo simulations in compression and melting
	sequences of the HP model with $k=3$.}
	\label{fig:HP_k3_phaseDia}
\end{figure}
%=================================

These reveal that during the compression sequence, no spontaneous ordering was observed.
Thus, in order to test for the stability of the nematic phase, another series of simulations was
performed in a melting sequence of decreasing pressures. The starting configuration for this series 
was a high density configuration obtained from the compression series, the particles being field
aligned along $\vecth{z}$ so as to obtain a nematic phase with an order parameter 
$P_2\sim 0.8$. The results for this sequence are shown on  Figure~\ref{fig:HP_k3_phaseDia} 
under the `melting' label. Surprisingly, these results suggest a
stable nematic phase for densities $\rho^{*} > 0.3$ and $P^{*}>8.0$. This situation seems to
be very similar to that found by Williamson and Jackson~\cite{WilliamsonJackson98} 
where a stable nematic phase was present although the model failed to order spontaneously.
A further test of the systems behaviour was provided through monitoring the mobility of the molecules
(Figure~\ref{fig:HP_k3_dc}). This was measured using~:
\begin{equation}
	\langle\delta r^2(n)\rangle  = \langle({\bf r}_n - {\bf r}_0)^2\rangle 
\end{equation}
%
%===========================
\picW = 7cm
\begin{figure}
	\centering
	\subfigure[compression sequence]{\picL{HP_k3_cpress_dc.ps}}
	\subfigure[melting sequence]{\picL{HP_k3_melt_dc.ps}}
	\caption[Evolution of $\langle\delta r^2(n)\rangle$ as a function of $n$ the number of
	Monte Carlo sweeps for the HP model and k=3.]
	{Evolution of $\langle\delta r^2(n)\rangle$ as a function of $n$ the number of
	Monte Carlo sweeps for the HP model and k=3. The data
	on the left are for a constant $NVT$ compression sequence with $N=1000$ and the data on
	the right for a constant NPT melting sequence with $N=1250$.}
	\label{fig:HP_k3_dc}
\end{figure}
%===========================
%
where ${\bf r}_n - {\bf r}_0$ is the displacement vector moved by a given particle in $n$ 
consecutive MC
sweeps and the angled brackets indicate an average over all particles and the run length. In MC 
simulations with fixed maximum particle displacement (as was the case here), Brownian diffusion 
dictates that  $\langle\delta r^2(n)\rangle$ should increase linearly with $n$ in a fluid phase.
The greater the gradient of $\langle\delta r^2(n)\rangle$, the more fluid the studied phase.\\

These measurements show that for both the compression and melting series, the
diffusion of the particles decreased monotonically with increased density or pressure. If the
results are fitted to equations of the form $y=a +bx$, $b$ decreases of about two orders of magnitude
between the lowest and highest densities shown. The conclusion from this is that the highest density 
configurations did not correspond to fluid, but to glassy phases which inhibited the ordering of the
system. In the melting series, in contrast, the lack of fluidity at the highest densities 
may have prevented the corresponding systems from disordering into the equilibrium structure. 
This argument is supported by the order of magnitude of the diffusion which is
about half the value obtained from the compression series, thus illustrating the frustrated
nature of the ordered phases obtained. The reduction of move acceptance rate from 40\% down to 10\%
between simulations at the lowest and highest density  further illustrates this glassiness.\\


%============================
\picW = 5.5cm
\begin{figure}
	\centering
	\subfigure[$\rho^{*}=0.28$]{\pic{HP_box_NVT_k3_b_N1000_d0.2800_0.50M.ps}}
	\hspace*{2cm}
	\subfigure[$\rho^{*}=0.32$]{\pic{HP_box_NVT_k3_b_N1000_d0.3200_0.50M.ps}}
	\caption{Configuration snapshots of the HP model and $k=3$ at $\rho^{*}=0.28$ and
	$\rho^{*}=0.32$.}
	\label{HP_k3_snaps}
\end{figure}
%============================

%============================
\picW = 7cm
\begin{figure}
	\centering
	\subfigure[$g_\parallel(r_\parallel)$]{\picL{HP_k3_grParallel.iso.ps}}
	\subfigure[$g_\perp(r_\perp)$]{\picL{HP_k3_grPerpLayer.iso.ps}}
	
	\subfigure[$g^\mrm{mol}_\parallel(r_\parallel)$]{\picL{HP_k3_grParallel.fun.ps}}
	\subfigure[$g^\mrm{mol}_\perp(r_\perp)$]{\picL{HP_k3_grPerpLayer.fun.ps}}
	\caption{Pair correlation functions for the HP model with $k=3$ computed with respect to
	$\vecth{n}$(top) and $\ui$ (bottom).}
	\label{HP_k3_grs}
\end{figure}
%============================


More insight into the phases produced by those two series of simulations can be obtained by
studying simulation snapshots (\eg Figure~\ref{HP_k3_snaps}) and computation of the pair
correlation functions.
Both the standard $g_\parallel(r_\parallel)$, $g_\perp(r_\perp)$ and 
$g^\mrm{mol}_\parallel(r_\parallel)$, $g^\mrm{mol}_\perp(r_\perp)$ were computed (see
Chapter~\ref{chap:three} for a definition). These functions are shown on 
Figure~\ref{HP_k3_grs}.\\
Little difference can be observed between the two sets of functions; 
$g_\parallel(r_\parallel)$ and $g^\mrm{mol}_\parallel(r_\parallel)$  indicate that as 
the density is increased, some 
short range features appear, corresponding to distances less than the molecular elongation.
Comparing this with the configuration snapshots suggests that those short range features
reflect the tendency of particles to align anti-parallel with their closest neighbors.
$g_\perp(r_\perp)$ and $g^\mrm{mol}_\perp(r_\perp)$ also fail to show signs of ordering, 
and little variation can be
observed throughout the density range considered. However, some signs of the preferred pairwise 
anti-parallel ordering are visible for the highest density in the form of a peak at $r\sim 0.9$.\\ 

The main conclusion to be drawn from these series of simulations is that the HP model with
$k=3$ shows rather surprising phase behaviour. With increased density, the particles exhibit short
ranged anti-parallel ordering. The rapid decrease in particle mobility, however leads to the 
formation of glassy phases and prevents the formation of nematic phases. Instead, at high
density the particles form a glassy phase made of small domains with local anti-parallel 
alignment.


%=================================================================================================
%=================================================================================================
\subsection{Particles with $k=5$.}

Due to the lack of ordering shown by particles with $k=3$ and following the idea that
increasing the molecular elongation stabilizes the nematic phase, a second study of the HP model
was made using particles with $k=5$. 
Systems of $N=1000$ particles were simulated using the Monte Carlo method 
in both the canonical and isothermal-isobaric ensembles and with inclusion of the `flip' moves.
Similar volume change algorithms and run lengths were used as for runs with $k=3$.
However, only a compression series was performed. The results obtained from this
(shown on Figure~\ref{fig:HP_k5_phaseDia}) indicate a somewhat
different phase behaviour for this
longer elongation.  
$P(\rho^{*})$ shows an inflexion point at $\rho^{*}\sim 0.12$, suggesting a transition to a more
ordered phase. This corresponds to a rapid increase in $\langle P_2\rangle$, having the common `S' shape of a
first order transition, which is consistent with ordering to a nematic phase. However the values of 
$\langle P_2\rangle$ are still short of that usually associated with nematic order. For this system, 
the computation of 
$\langle\delta r^2(n)\rangle$ (Figure~\ref{fig:HP_k5_dc}) shows that all 
configurations remained fluid throughout the pressure/density range considered.\\

%=====================
\picW = 7.0cm
\begin{figure}
	\centering
	\subfigure{\picL{HP_P-rho_k5.ps}}
	\subfigure{\picL{HP_P2-rho_k5.ps}}
	\caption{Results from the bulk compression simulation series with the HP model and
	$k=5$.}
	\label{fig:HP_k5_phaseDia}
\end{figure}
%=====================
%
%=====================
\picW = 12cm
\begin{figure}
	\centering
	\picL{HP_k5_cpress_dc.ps}
	\caption{Results for the diffusion of the HP model with $k=5$ for simulation in the
	isothermal-isobaric ensemble in a compression series. }
	\label{fig:HP_k5_dc}
\end{figure}
%=====================

The nature of the phases formed by this model was investigated by observing configuration
snapshots (\eg Figure~\ref{fig:HP_k5_d0.14_snaps}) and computing  pair correlation
functions (Figure~\ref{HP_k5_grs}). Again both the standard $g_\parallel(r_\parallel)$, 
$g_\perp(r_\perp)$ and  molecule based $g^\mrm{mol}_\parallel(r_\parallel)$, 
$g^\mrm{mol}_\perp(r_\perp)$ were computed.\\
%
As the density increased, the particles were found to order but also form layers with different 
orientations.  As a result, although the intralayer ordering was very high, this was not 
reflected by the value  of the overall nematic order parameter as the directions of the layers 
were very different, almost perpendicular in places. This is illustrated by the configuration shown on  
Figure~\ref{fig:HP_k5_d0.14_snaps}(b), obtained from a simulation performed in the canonical 
ensemble at
$\rho^{*}=0.14$. A side view of the same configuration (Figure~\ref{fig:HP_k5_d0.14_snaps}(c)), 
would however, suggest phase with a unique alignment direction and smectic-like ordering,
with the exception of those particles with different orientations which are sandwiched 
between the `layers'. 
This discrepancy shows the full extent of the non homogenous character of the high density
structures obtained here and explains the moderate values of $\langle P_2 \rangle$.\\
The likelihood of this type of configuration transforming into a smectic phase at higher
densities appears low as the particle mobilities at these high densities are rather low. 
This question has, however, been investigated by performing simulations in the 
isothermal-isobaric ensemble  where the three box lengths were allowed to vary independently.
This approach is known to facilitate the 
possible formation of smectic phases as the box accommodates the system's natural layer spacing. 
These simulations however lead to the same structures as were seen 
in the canonical ensemble.  Therefore, this multi-domain layered configuration seems to be the 
most stable state for this model at high pressures/densities.

%=====================
\picW = 5.5cm
\begin{figure}
	\centering
	\subfigure[$\rho^{*}=0.10$]{\pic{HP_box_NVT_k5_b_N1000_d0.1000_0.50M.ps}}
	
	\subfigure[$\rho^{*}=0.14$, bottom view]{\pic{HP_box_NVT_k5_b_N1000_d0.1400_0.50M_01.ps}}
	\hspace{2cm}
	\subfigure[$\rho^{*}=0.14$, side view]{\pic{HP_box_NVT_k5_b_N1000_d0.1400_0.50M_02.ps}}
	\caption{Configuration snapshots for the HP model with $k=5$ at $\rho^{*}=0.14$. Two
	views are presented the bottom of the simulation box (left) and a side view looking down
	along $\vecth{y}$ and $\vecth{z}$ pointing upward.}
	\label{fig:HP_k5_d0.14_snaps}
\end{figure}
%=====================


%============================
\picW = 7cm
\begin{figure}
	\centering
	\subfigure[$g_\parallel(r_\parallel)$]{\picL{HP_k5_grParallel.iso.ps}}
	\subfigure[$g_\perp(r_\perp)$]{\picL{HP_k5_grPerpLayer.iso.ps}}
	
	\subfigure[$g_\parallel(r_\parallel)$]{\picL{HP_k5_grParallel.fun.ps}}
	\subfigure[$g_\perp(r_\perp)$]{\picL{HP_k5_grPerpLayer.fun.ps}}
	\caption{Pair correlation functions for the HP model with $k=5$ computed with respect to
	$\vecth{n}$(top) and $\ui$ (bottom).}
	\label{HP_k5_grs}
\end{figure}
%============================


A quantitative insight into those phases is provided by the pair correlation functions as shown on
Figure~\ref{HP_k5_grs}. Again the functions computed with respect to $\vecth{n}$ and $\ui$ show
little qualitative differences. As the density is increased, $g_\parallel(r_\parallel)$ and 
$g^\mrm{mol}_\parallel(r_\parallel)$ display
a series of peaks accounting for the domain ordering. The distance between two successive peaks of about
$6\sigma_0$ corresponds to the distance between two layers separated by a small number of
interstitial particles with perpendicular orientation.
The lack of periodicity in these functions reveals 
that this phase is not smectic, however. The intra-layer, in contrast, is rather 
close to that of a smectic phase as $g^\mrm{mol}_\perp(r_\perp)$ shows up to four peaks with an average 
separation $\sim\so$  accounting for the successive parallel and antiparallel neighbours.\\

Thus, for the elongation of $k=5$, the HP model shows a rather surprising phase behaviour which is
subtly different from that observed for $k=3$. For the longer elongation, the model proved to
remain fluid over the pressure range considered. With increased pressure, a transition to an
ordered phase was observed. At high pressure, the particles formed layered domains with different
orientations and high, smectic-like, intra-layer order.
The HP model does not, however, fulfill the requirement of having the stable nematic phase 
needed to simulate the
confined flexoelectric particle systems mentioned at the beginning of this
Chapter. The modeling of pear-shaped particles which form nematic phases can, however, be achieved 
using a totally different route to the expression for the contact distance.
This new approach is described in the following Sections.




\section{The PHGO model.}

Here a novel model for non-centrosymmetric particles is described. The route adopted
is to extend the generalized Gay-Berne (GGBP) potential~\cite{CleaverCare96} to the description of
non-centrosymmetric particles thus leading to the so called Parametric Hard Gaussian Overlap (PHGO)
model.

%================================================================================================
%================================================================================================
\subsection{Obtaining the contact distance.}

The generalized Gay-Berne model describes the interaction between two ellipsoidal particles 
$i$, $j$ with arbitrary lengths $\ell_i$, $\ell_j$ and breadth $d_i$, $d_j$ and orientations
$\ui$ and $\uj$ separated by an intermolecular vector $\vect{r}_{ij} = r_{ij}\rij$. The contact 
distance for this model is given by~:
%
\begin{multline}
\sijr = \\
\sigma_0 \left[ 1 - \chi\left\{
    \frac{\alpha^2\dotProdP{\rij}{\ui}^2 + \alpha^{-2}\dotProdP{\rij}{\uj}
    -2\chi\dotProdP{\rij}{\ui}\dotProdP{\rij}{\uj}\dotProdP{\ui}{\uj}  }{1-\chi^2\dotProdP{\ui}{\uj}^2}
    \right\}\right]^{-\frac{1}{2}}
    \label{eqn:GGB_CD_original}
\end{multline}

with~:
\begin{eqnarray*}
    \sigma_0 &=& \sqrt{\frac{d^2_i + d^2_j}{2}}               \\
    %
    \alpha^2 &=& \left[ \frac{ (\ell^2_i-d^2_i)(\ell^2_j+d^2_i)}
    {(\ell^2_j-d^2_j)(\ell^2_i+d^2_j)}\right]^{\half}         \\
    %
    \chi &=& \left[ \frac{(\ell^2_i-d^2_i)(\ell^2_j-d^2_j)}
    {(\ell^2_j+d^2_i)(\ell^2_i+d^2_j)}\right]^{\half}.         \\
    %
\end{eqnarray*}
If, alternatively, the brackets containing the length and breadth values are grouped as~:
%
\begin{equation*}
	\begin{array}{cc}	
	A = (\ell^2_i-d^2_i) &B = (\ell^2_j-d^2_j) \\
	C = (\ell^2_j+d^2_i) &D = (\ell^2_i+d^2_j),
    	\end{array}
\end{equation*}
%
the shape parameter can be rewritten as~:
\begin{multline}
    \sijr = \\
    \sigma_0 \left[ 1 - \frac{ AC\dotProdP{\rij}{\ui}^2 + BD\dotProdP{\rij}{\uj}^2
    - 2AB\dotProdP{\rij}{\ui}\dotProdP{\rij}{\uj}\dotProdP{\ui}{\uj} }
    {CD - AB\dotProdP{\ui}{\uj}^2}
    \right]^{-\half}
    \label{eqn:GGB_CD_modified}
\end{multline}

Although the two forms of Equations~\ref{eqn:GGB_CD_original} and~\ref{eqn:GGB_CD_modified} 
are mathematically equivalent, the second is to be preferred for
implementation in a simulation code as it is free of possible division by zero or complex
numbers in the limit $\ell=0$ and $d = 0$.\\

By design, Equation~\ref{eqn:GGB_CD_modified} is restricted to the description of particles with 
fixed $\ell$ and $d$. The extension from the $GGB$ to the $PHGO$ contact distance is to consider
particles for which $\ell$ and $d$ vary parametrically with $\dotProdP{\rij}{\ui}$ and 
$\dotProdP{\rij}{\uj}$. As a result the shape of $i$ as `seen' by $j$ 
can be tuned to be a function of the relative position and orientation of the particles. 
This makes it possible to model particles with arbitrary non-centrosymmetric shapes although the
approximation made by the model can lead to anomalies if non-convex particle shapes are used.

This method can therefore be applied to the description of particles having the B\'ezier shape used 
by the Stone expansion of the HP model as these do satisfy the convexity requirements of the model.
When two such particles interact with their sharp ends, $\ell/d$ needs to be 
relatively large, whereas the  blunt end interaction requires an $\ell/d$ ratio rather nearer to unity. 
In order to avoid discontinuities between these two limiting cases, a multitude of 
parametric forms is possible; here the form of $\ell_i$ and $d_i$ is limited to simple 
polynomials of the polar angle $\dotProdP{\rij}{\ui}$, that is~:
\begin{eqnarray}
        d_i(\dotProd{\rij}{\ui}) &=& a_{d,0} +\ldots + a_{d,n}(\dotProd{\rij}{\ui})^n\\     
	\label{eqn:drui}
        \ell_i(\dotProd{\rij}{\ui}) &=& a_{\ell,0} +\ldots + a_{\ell,m}(\dotProd{\rij}{\ui})^m.
\end{eqnarray}

The main limitation of this model is that the description of concave
particles (i.e. dumbells) is not avaliable.
The advantages are, however, numerous as the analytical form makes it possible to use the model
in both MC and MD codes. Also, this monosite model does not introduce any discontinuities in
the particle-particle contact surfaces.
Although the use of polynomials to describe the dependency of $\ell$ and $d$ over 
$\dotProdP{\rij}{\ui}$ might seem simplistic, this allows a very
straightforward implementation in the coefficients $A,B,C$ and $D$ by simply adding higher order
terms. Reversibly, the contact distance of the HGO model can be recovered in taking the limit $m=n=0$.\\
Finally simulations of multi-component fluids is an easy extension of this model, as a set of 
$a_{\ell,i}$, $a_{d,i}$ can be assigned to every type of particle. Systems of particles whose
shapes vary within a simulation run can also be achieved using this approach.

%================================================================================================
%================================================================================================
\subsection{Parameterizing B\'ezier pears.}

In order to apply the PHGO approach to pear-shaped particles, the coefficients $a_{\ell,i}$ and
$a_{d,i}$ need to be computed. This has been performed by numerically fitting the particle-point
potential to a geometrical shape. For this, a numerical simplex  method~\cite{NumericalRecipes}
has been chosen.\\
The shape that is considered here is a simple update of the B\'ezier shape described in
Section~\ref{s:HP_model}, the difference being that the control points $q_2$ and $q_3$ are set
to be always coincident. This makes the B\'ezier shape easily scalable with $k$. 
This new B\'ezier shape is shown on Figure~\ref{fig:GBP_BzPear} and the associated control
points coordinates as a function of $k$ are given on Table~\ref{tble:GBP_CtrlCoords}.

%=============================
\picW = 12cm
\begin{figure}
	\centering
	\hspace*{2cm}\picL{GBP_BzPear.ps}
	\caption{B\'ezier shape for parameterized with the PHGO model.}
	\label{fig:GBP_BzPear}
\end{figure}
%=============================

\begin{table}
	\centering
	\begin{tabular}{||c||c||c||}
	\hhline{|t:=:t:=:t:=:t|}
	\hspace{5mm}$q_i$\hspace{5mm}	&\hspace{5mm}x\hspace{5mm}	&\hspace{5mm}y\hspace{5mm}	\\
	\hhline{|:=::=::=:|}
	$q_1$				&$-\frac{1}{2}\sigma_0$		&$0$			\\
	$q_2$				&$0$				&$\frac{2}{3}k\sigma_0$	\\
	$q_3$				&$0$				&$\frac{2}{3}k\sigma_0$	\\
	$q_4$				&$\frac{1}{2}\sigma_0$		&$0$			\\
	$q_5$				&$\sigma_0$			&$-\frac{2}{3}k\sigma_0$\\
	$q_6$				&$-\sigma_0$			&$-\frac{2}{3}k\sigma_0$\\
	\hhline{|b:=:b:=:b:=:b|}
	\end{tabular}
	\caption{B\'ezier control points coordinates for the B\'ezier pear used with the PHGO model.}
	\label{tble:GBP_CtrlCoords}
\end{table}

%=============================

Using the geometrical properties of the B\'ezier curves, it is possible 
to extract the coordinates of each point of the pear surface and, hence, the numerical data
corresponding to the particle-point contact distance as a function of $\dotProdP{\ui}{\rij}$. 
These data are then fitted to the expression for the PHGO contact distance with
$\ell_j=d_i=0$ which in this case reduces to~:
\begin{equation}
	\sir = \frac{d_i(\dotProd{\rij}{\ui})l_i(\dotProd{\rij}{\ui})}
	{\left\{ l_i^2(\dotProd{\rij}{\ui}) +
	\dotProdP{\ui}{\rij}^2\left[ d_i^2(\dotProd{\rij}{\ui})- l_i^2(\dotProd{\rij}{\ui})
	\right] \right\}^{\frac{1}{2}}}
	\label{eqn:GBP_sir}
\end{equation}

The values of the $a_{d,\alpha}$ and $a_{\ell,\alpha}$ coefficients obtained from the fitting
procedure are shown on Table~\ref{tble:GBP_dili}. $n=10$ and
$m=1$ have been chosen as this was found to be the best compromise between simulation speed 
and fit accuracy.
%
%=======================================================================================================
%=======================================================================================================
\tabul = 1.0cm
\begin{table}
    \centering
    \begin{tabular}{||c||c||c||c||}
    \hhline{|:t=:t:=:t:=:t:=:t|}
    \TAB\TAB    &   \TAB$k=3$\TAB   &   \TAB$k=4$\TAB   &   \TAB$k=5$\TAB   \\
    \hhline{|:=::=::=::=:|}
    $a_{d,0} $       &   0.501852454     &    0.501377232    &    0.497721868    \\
    $a_{d,1} $       &   -0.141145314    &   -0.129608758    &   -0.123155821    \\
    $a_{d,2} $       &   -0.060542359    &   -0.074219217    &    0.024405876    \\
    $a_{d,3} $       &   0.225813650     &   0.484166441     &    0.723627215    \\
    $a_{d,4} $       &   0.832274021     &   0.923492941     &    0.389831429    \\
    $a_{d,5} $       &   -1.015039575    &   -1.987232902    &   -3.018638148    \\
    $a_{d,6} $       &   -2.504045172    &   -2.943008017    &   -1.951629076    \\
    $a_{d,7} $       &   1.375313426     &   2.808075172     &    4.413215403    \\
    $a_{d,8} $       &   3.196830129     &   3.815344782     &    2.998417509    \\
    $a_{d,9} $       &   -0.699241457    &   -1.426641750    &   -2.241573216    \\
    $a_{d,10}$       &   -1.430400139    &   -1.682476460    &   -1.416614353    \\
    \hhline{|:=::=::=::=:|}
    $a_{\ell,0} $       &   1.498259615     &   1.995906501     &    2.493069403    \\
    $a_{\ell,1} $       &   -0.002027616    &   -0.004518187    &   -0.008067236    \\
    \hhline{|:b=:b:=:b:=:b:=:b|}
    \end{tabular}
    \caption{Values of the $a_{d,\alpha}$ and $a_{\ell,\alpha}$ for the PHGO pears with $k=3$, 4 and 5}
    \label{tble:GBP_dili}
\end{table}
%=======================================================================================================
%
The accuracy of the fits can be checked by comparing the contact distances
calculated using the coefficients of Table~\ref{tble:GBP_dili} and the actual B\'ezier shape
(Figure~\ref{fig:GBP_2dFit}).
The comparison reveals a very good correspondence between the two sets. The critical region 
being for $\dotProdP{\rij}{\ui} \sim 0$. The level of correspondence between the numerical and
analytical result to be achieved in this region dictates the number of coefficients to be used.
While using $n=m=1$ allows a good fit for the head and tail regions of the particles
up to $n=10$ coefficients in $a_{d,\alpha}$ were required to achieve the quality of fits shown on
Figure~\ref{fig:GBP_2dFit}.
%
%===========================
\picW=8.0cm
\begin{figure}
	\centering
	\subfigure[$k=3$]{\picL{GBP_profile_k3.ps}}
	\subfigure[$k=4$]{\picL{GBP_profile_k4.ps}}
	\subfigure[$k=5$]{\picL{GBP_profile_k5.ps}}
	\caption{Comparison between the numerical and geometrical contact distance for the PHGO
	particle-point potential.}
	\label{fig:GBP_2dFit}
\end{figure}
%===========================
%

The quality of these fits does not, however, guarantee an appropriate particle-particle
interaction and the validity of the latter is assessed by computing the contact surfaces
equivalent to those considered for
the HP model (\eg Figures~\ref{fig:HP_k3_contactSurf} and~\ref{fig:HP_k5_contactSurf} ). 
Again three contact surfaces were computed corresponding
to parallel, anti-parallel and T-geometry interactions. The results for $k=3$ and $k=5$ are
shown on Figures~\ref{fig:GBP_contactSurfaces_k3} and~\ref{fig:GBP_contactSurfaces_k5}.

%===========================
\picW = 7.0cm
\begin{figure}
	\centering
	\subfigure[parallel]{\pic{GBP_k3_ppShape_0PI.ps}}
	\subfigure[anti-parallel]{\pic{GBP_k3_ppShape_PI.ps}}
	\subfigure[T-geometry, x-z view]{\pic{GBP_k3_ppShape_0.5PI.ps}}
	\subfigure[T-geometry, y-z view]{\pic{GBP_k3_ppShape_0.5PI_2.ps}}
	\caption{Contact surfaces for the PHGO model and $k=3$.}
	\label{fig:GBP_contactSurfaces_k3}
\end{figure}
%===========================

%===========================
\picW = 7.0cm
\begin{figure}
	\centering
	\subfigure[parallel]{\pic{GBP_k5_ppShape_0PI.ps}}
	\subfigure[anti-parallel]{\pic{GBP_k5_ppShape_PI.ps}}
	\subfigure[T-geometry, x-z view]{\pic{GBP_k5_ppShape_0.5PI.ps}}
	\subfigure[T-geometry, y-z view]{\pic{GBP_k5_ppShape_0.5PI_2.ps}}
	\caption{Contact surfaces for the PHGO model and $k=5$.}
	\label{fig:GBP_contactSurfaces_k5}
\end{figure}
%===========================


These surfaces show the required shapes, that is approximately ellipsoidal shapes
for the parallel interactions and pear shapes for the anti-parallel
interactions. Although there is not really a required shape for the T-geometry, the latter do
not show any discontinuities that would indicate potential problems.
They are also reasonably consistent with those obtained for the corresponding HP contact surfaces.\\

The difference between the contact surfaces for this model and the HP model are readily
understandable. For the PHGO model, ridges in the contact surfaces have disappeared save for small
oscillations which are a direct consequence of the degree of truncation of the fits shown on
Figure~\ref{fig:GBP_2dFit}. However the low amplitude of these oscillation suggests that the PHGO
model should be free of the interlocking phenomena shown by the HP model.








\section{Phase behaviour of the PHGO model.}


In order to test the PHGO model applied to the B\'ezier pear shape, bulk constant $NPT$ MC simulations
were performed on systems of $N=1000$ particles with elongations $k=3,4$ and $5$. 
As with the HP model simulations, orientation inversion moves were also included in the MC scheme. 
These accounted for $20\%$ of the total number of attempted moves. The volume change scheme was 
performed on average, once two sweeps and allowed each box side to change its length independently. 
Typical runs comprised of $0.5.10^6$ sweeps for equilibration and production. Close to phase
transitions, 
additional runs were performed on a case by case basis  to ensure that equilibration was
achieved.  All simulations were performed in a compression sequence.


%===============================================================================================
\subsection{Particles with $k=3$ and 4.}


%==================================================
\picW = 7cm
\begin{figure}
	\subfigure[$k=3$]{\picL{GBP_P-rho_k3.ps}\picL{GBP_P2-rho_k3.ps}}
	\subfigure[$k=4$]{\picL{GBP_P-rho_k4.ps}\picL{GBP_P2-rho_k4.ps}}
	\caption{Phase diagram for the PHGO pears with $k=3$ and 4.}
	\label{fig:PHGO_phaseDia_k3k4}
\end{figure}
%==================================================

The equation of state and order parameter behaviour obtained from simulations with $k=3$ 
and $4$ are shown in 
Figure~\ref{fig:PHGO_phaseDia_k3k4}. The phase behaviour for these two elongations is fairly
similar; both $P(\rho^{*})$ curves show an inflexion point, respectively at $\rho^{*}\sim 0.30$ and
$\sim 0.20$, which corresponds to an increase of $P_2(\rho^{*})$ to values of the order
of $0.15$ immediately followed by a rather steep decrease.\\ 
The `plateau' in $P(\rho^{*})$ indicates some sort of phase transition to a more ordered phase
but, as this is only hardly reported on $P_2$, this phase change does not corresponds to 
an isotropic-nematic phase transition.
Some more insight into the phase behaviour can be obtained by observation of configuration
snapshots (\eg Figure~\ref{fig:GBP_snaps_k3k4}).

%==================================================
\picW = 5.5cm
\begin{figure}
	\subfigure[$k=3$ isotropic]{\pic{GBP_box_NPT_k3-2_N1000_P3.0000_0.50M_S1.1.ps}}
	\hspace{2cm}
	\subfigure[$k=3$ domain ordered]{\pic{GBP_box_NPT_k3-2_N1000_P7.0000_0.50M_S1.1.ps}}
	
	\subfigure[$k=4$ isotropic]{\pic{GBP_box_NPT_k4_N1000_P1.8000_0.50M.ps}}
	\hspace{2cm}
	\subfigure[$k=4$ domain ordered]{\pic{GBP_box_NPT_k4_N1000_P5.0000_0.50M.ps}}
	\caption{Configuration snapshots for the pear PHGO model and $k=3$ and 4.}
	\label{fig:GBP_snaps_k3k4}
\end{figure}
%==================================================

These show that upon increasing the pressure, both systems underwent a phase transition from
isotropic to the what we term as a domain ordered BILAYER phase where, the particles form 
domains in which 
the local order is very high but where the orientation changes from one domain to the other. 
Unlike with the HP model, this seems to have been a transition between two genuinely
liquid states as the systems maintained high mobility throughout the density ranges considered here
(see Figure~\ref{fig:PHGO_dc_k3k4}). Also, the configuration snapshots suggest continuous
orientation changes in moving from one domain to another (\eg Figure~\ref{fig:GBP_snaps_k3k4}b
and d,) unlike the very sharp domain
boundaries seen in the HP systems (\eg Figure~\ref{fig:HP_k5_d0.14_snaps}b and c.)\\

\picW = 7cm
\begin{figure}
	\centering
	\subfigure[$k=3$]{\picL{GBP_k3-2_cpress_dc.ps}}
	\subfigure[$k=4$]{\picL{GBP_k4_cpress_dc.ps}}
	\caption{Evolution of $\langle\delta r^2(n)\rangle$ for the pear PHGO model and $k=3$
	and 4.}
	\label{fig:PHGO_dc_k3k4}
\end{figure}

More quantitative insight into those domain ordered phases have been obtained through
computation of the pair correlation functions. Both $g_\parallel(r_\parallel)$,
$g^\mrm{mol}_\parallel(r_\parallel)$ (Figure~\ref{fig:gr_PHGO_k4}(a)) and $g_\perp(r_\perp)$,
$g^\mrm{mol}_\perp(r_\perp)$ (Figure~\ref{fig:gr_PHGO_k4}(b)) have been computed  for $k=4$ as
the domains are best observed for this elongation.\\
%
$g_\parallel(r_\parallel)$ and $g_\perp(r_\perp)$ show some short ranged structure with 
increasing density but do not give much information about the structures observed 
on the snapshots because the choice of reference ($\vecth{n}$) for the computation of these 
curves does not allow one to `follow' the orientation of the particles in the domains.\\
A better reference is the molecular orientation as used when calculating 
$g^\mrm{mol}_\parallel(r_\parallel)$
and $g^\mrm{mol}_\perp(r_\perp)$ (Figure~\ref{fig:gr_PHGO_k4}(c) and (d)).
In this case, the appearance of almost periodic  behaviour in $g^\mrm{mol}_\parallel(r_\parallel)$ 
shows some smectic-like ordering with increased density. $g^\mrm{mol}_\perp(r_\perp)$, however,
shows that those domain are fairly short ranged as only three peaks can be observed, the first one 
accounting for the direct antiparallel neighbors and the subsequent maxima correspond to further
shells of parallel and anti-parallel particles.\\


\picW = 7cm
\begin{figure}
	\centering
	\subfigure[$g_\parallel(r_\parallel)$]{\picL{GBP_k4_grParallel.iso.ps}}
	\subfigure[$g_\perp(r_\perp)$]{\picL{GBP_k4_grPerpLayer.iso.ps}}

	\subfigure[$g_\parallel(r_\parallel)$]{\picL{GBP_k4_grParallel.fun.ps}}
	\subfigure[$g_\perp(r_\perp)$]{\picL{GBP_k4_grPerpLayer.fun.ps}}
	\caption{Pair correlation functions for the pear PHGO model of $k=4$ computed with respect to 
	$\vecth{n}$ (top) and to $\ui$ (bottom).}
	\label{fig:gr_PHGO_k4}
\end{figure}

Thus the PHGO model applied to the B\'ezier pears with elongation $k=3$ and $4$ has shown some very 
interesting phase behaviour. Despite the improved contact surfaces, that prevented the formation 
of glassy phases, still no nematic phases could be observed. Rather, upon increasing the
density, the systems underwent a transition from isotropic to a domain ordered phase.

%===============================================================================================
\subsection{Particles with $k=5$.}

The phase behaviour for particles with $k=5$ proved to be qualitatively different from the shorter
elongations and resembled that of other common LC models, such as the HGO fluid.

$P^{*}(\rho^{*})$ shows a plateau at $P^{*}\sim 1.2$ which corresponds to an `S' shaped increase in 
$P_2(\rho^{*})$ to values typical of a nematic
phase. This phase lacked polar order throughout the density range
considered as shown by the low values  of $P_1(\rho^{*})$. Configuration snapshots illustrating 
these isotropic and nematic phases are shown on Figure~\ref{fig:GBP_k5_snaps}(a) and (b).
%
%
%==================================================
\picW = 7cm
\begin{figure}
	\picL{GBP_P-rho_k5.ps}
	\picL{GBP_P2-rho_k5.ps}
	\caption{Phase diagram for the PHGO pears with $k=5$.}
	\label{fig:PHGO_phaseDia_k5}
\end{figure}
%==================================================
%
%
Surprisingly a second feature can be observed on the phase diagram in the form of a second inflexion
point in $P^{*}(\rho^{*})$ at higher pressure. This corresponds to a second, smaller `S' shaped
increased in $P_2(\rho^{*})$ marking a second phase transition, this time to a smectic phase.
As the PHGO model is simply a generalization of the Hard Gaussian Overlap model to 
non-centrosymmetric shapes, this second phase transition was not expected. Observation of the 
configuration snapshots at the highest densities confirmed the existence of the smectic phase 
(\eg Figure~\ref{fig:GBP_k5_snaps}(c) .)
All the phases found can be shown to be fluid as the gradient in $\left< \delta r^2(n) \right>$
stayed rather high throughout the density range considered here (Figure~\ref{fig:GBP_k5_dc}).\\

%==================================================
\picW = 12cm
\begin{figure}
	\centering
	\picL{GBP_k5_cpress_dc.ps}
	\caption{Evolution of $\left< \delta r^2(n) \right>$ as a function of pressure for the
	pear PHGO model with $k=5$.}
	\label{fig:GBP_k5_dc}
\end{figure}
%==================================================


%==================================================

\begin{figure}
	\centering
	\picW = 5cm
	\subfigure[isotropic]{\pic{GBP_box_k5_P1.0000.ps}}
	\hspace{2cm}
	\picW = 5.5cm
	\subfigure[nematic]{\pic{GBP_box_k5_P1.5000.ps}}
	\picW = 5.5cm
	\subfigure[smectic]{\pic{GBP_box_k5_P2.8000.ps}}
	\caption{Configuration snapshots for the PHGO pears with $k=5$.}
	\label{fig:GBP_k5_snaps}
\end{figure}
%==================================================


%==================================================
\picW = 7cm
\begin{figure}
	\centering
	\subfigure[$g_\parallel(r_\parallel)$]{\picL{GBP_k5_grParallel.ps}}
	\subfigure[$g_\perp(r_\perp)$]{\picL{GBP_k5_grPerpLayer.ps}}
	\caption{Pair correlation functions for the pear PHGO model of $k=5$ computed with respect to 
	$\vecth{n}$.}
	\label{fig:GBP_k5_gr}
\end{figure}
%==================================================


In order to gain more insight into the phases found here but also to characterize the
smectic phase more precisely, the pair correlation functions have been computed parallel and 
perpendicular with respect to the nematic director $\vecth{n}$. Only this reference frame has
been taken here as, in the absence of domain ordered phases, 
computation of $g^\mrm{mol}_\parallel(r_\parallel)$ and $g^\mrm{mol}_\perp(r_\perp)$ was not
required.\\
%
As shown in Figure~\ref{fig:GBP_k5_gr}, increase in pressure leads to the development of
periodic oscillations in $g_\parallel(r_\parallel)$; the amplitudes of these fluctuations 
were found to grow with increased pressure. In the smectic phase, $g_\parallel(r_\parallel)$ 
became fully periodic, the repeating pattern being composed of one main 
peak between two smaller ones.\\
At $P^{*}=2.8$, the distance between the two main peaks was about  
$7.389\sigma_0 \approx 1.5k\sigma_0$. This corresponds to  the separation of layers with 
the same polar orientation. The two smaller peaks account for the two spacing of antiparallel 
layer arrangement.  The presence of two peaks (rather than one) can be explained by 
the difference between the preferred tail-tail and  head-head contact distances. The area under 
these peaks is about half that under the main peaks as each accounts for one type of 
relative alignment (head-head \textbf{or} tail-tail) whereas the main peak account for two 
types of interaction (head-tail pointing up \textbf{and} head-tail pointing down). At the same
pressure of $P^{*} = 2.8$, the peak separation of about $\so$ shows the smectic to be a \smA
phase. Moreover, the interdigitation between the layers as revealed by the configuration
snapshots (\eg Figure~\ref{fig:GBP_k5_snaps}(c)) further identifies this smectic phase as a
bilayered smectic $\mrm{A_2}$.\\

Comparison of $g_\parallel(r_\parallel)$ for different values of $P^{*}$ in the smectic phase
shows an interesting compressibility behaviour and also allows to identify the different 
peaks with their associated particles interaction geometry. Upon increasing the pressure, 
the system density rises  and the intra-layer separation decreases whereas the bilayer 
separation increases  (Figure~\ref{fig:GBP_k5_gr}(a) and (b)). 
From the measured $g_\parallel(r_\parallel)$ data in the pressure range $[2.4:3.8]$,
it is found that the distance between the 
main peaks, which corresponds to the separation of the bilayers, increases from $7.38$ 
to $7.66$. The distance from the main peak to the first minor peak, which corresponds to 
the strongly interdigitating `tail-tail' configuration increases from $2.49$ to $2.76$, 
whereas that to the second minor peak, corresponding to the weakly interdigitating `head-head' 
alignment remains effectively constant at $4.85$. Thus, the in-plane compression induced by 
this increase in pressure leads to a 10\% increase in the separation within the interdigitated 
bilayers that comprise the smectic ${\rm A_2}$ phase.\\
Figure~\ref{fig:GBP_k5_grDetails} shows the period of $g_\parallel(r_\parallel)$ at
$P^{*}=2.8$~; the distance $C-H_1$ corresponds to the bilayer separation, that is the separation
between particles with the same polar orientation ($\lhd\lhd$ and $\rhd\rhd$). The $C-L_1$
distance corresponds to the separation between the strongly interdigitated particles in a 
tail-tail ($\rhd\lhd$) configuration and the $C-L_2$ distance corresponds to the separation
between particles in a head-head ($\lhd\rhd$) configuration.\\

\picW = 10cm
\begin{figure}
	\centering
	\picL{GBP_k5_grParallelDetails.ps}
	\caption{Details of $g_\parallel(r_\parallel)$ for a system of PHGO particles with $k=5$
	and $P^{*}=2.8$, which corresponds to a smectic phase.}
	\label{fig:GBP_k5_grDetails}
\end{figure}


The results from the simulation of the pear PHGO model with $k=5$ thus shows an interesting
phase behaviour that includes stable nematic and interdigitated \smA phases. As a result 
the prerequisite 
condition for the creation of the model for surface induced switching is fulfilled and pear 
PHGO particles can be used in confined geometries in an attempt to achieve the main goal of this
thesis. Results  of the work performed to this end are presented in Chapter~\ref{chap:seven}.







%===================================================================================================
%===================================================================================================
\conclusion


In this Chapter the simulation of non-centrosymmetric, pear-shaped particles has been addressed
using two models. The first model used a steric version of the model developed by 
Berardi~\etal\cite{BerardiRicci01}.
The contact distance of this is obtained fitting a numerical contact distance
to a truncated Stone expansion. However, this introduced non-convex contact surfaces and, 
as a result, the model with
elongations $k=3$ and $5$ did not display stable nematic phases as they were systematically 
preceeded by glassy phases. A totally new approach was then taken in the form of the  so called
parametric hard Gaussian overlap model. For the shorter elongations $k=3$ and $4$, the model
did not show any nematic phases; rather at high densities, the molecules order into
interdigitated domains. 
Upon increasing the elongation to $k=5$, the model was found to display both
nematic and interdigitated smectic $\mrm{A_2}$ phases. The latter showed some interesting
anisotropic compressibility behaviour where compression had the effect of decreasing the
intra-layer particle separation while increasing the distances between bilayers. As a result, 
the PHGO model with an elongation $k=5$ can be used for the modeling of surface induced 
switching in hybrid anchored systems of flexoelectric particles; the implementation of this 
is discussed in the next Chapter.






