

\section{Future work}


The work performed in this Thesis has lead to the development of a model for a novel bistable
LCD cell where two way switching is achieved by the application of a directional
electric field pulse. Much of the work that will directly follow this Thesis concerns the study and
improvement of this model. The first task is certainly the implementation of a method
for the calculation of the slay flexoelectric coefficient as this would allow a good quantitative
measurement of the effects of subsequent alterations to the model. Also, the use of the
molecular dynamics method would provide interesting results by giving access to the true 
dynamics of relaxation properties of the model systems.\\
%
The cell model itself should also be improved, for instance by the use of mixtures of ellipsoidal
shaped (HGO) and pear shaped particles (PHGO) which would allow independent tuning and/or
enhancement of the bulk
flexoelectric properties of the cell and the field-direction-dependence of its surface 
behaviour. This would also make the model a better description of a real liquid crystal cell,
since mixtures of several different components are commonly used.\\
Finally, incorporation of attractive forces into the molecular models should be
considered so as to render them more realistic and give access to better control mechanisms for
tuning phase and anchoring behaviour. This could include addition of quadrupolar contributions 
in the particle-particle interaction as it has been shown that quadrupole interactions are 
predominant in the origins of flexoelectricity in real mesogens.\\
All of these refinements should lead to the development of, more realistic modeling of both the
LCD cell considered in Chapter~\ref{chap:seven} and the more fundamental behaviour needed to
make such cell. The work presented in this Thesis therefore represent a step along the path
towards providing more efficient display cells as well as an academic study into the development
of models for confined and flexoelectric liquid crystal behaviour.

