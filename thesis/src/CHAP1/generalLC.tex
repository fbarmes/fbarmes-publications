
\section{Characterization of liquid crystals.}



%==================================================================================================
%==================================================================================================
\subsection{The liquid crystalline phases}

The liquid crystalline phases refer to states of matter that exist between the isotropic liquid and
crystalline solid. A liquid crystal phase is formed by mesogenic particles, hence the term
mesogen used to refer to a molecule that forms a meso\-phase or liquid crystalline phase. A mesophase 
shares properties with both the liquid phase (flow, zero resistance to shear) and the
crystalline phase (long range positional and/or orientational order, anisotropic optical properties).
The term liquid crystal actually encompasses several different phases, the most common of which are
smectic and nematic~\cite{LiquidCrystals, Nature_phase, Intro_LC}.\\
A necessary but not sufficient requirement for a molecule to form a liquid crystalline 
phase (or mesophase) is a strong anisotropy in shape; mesogens are either
calamitic (rod shaped) or discotic (disc like). Their phase transitions can be temperature
driven (thermotropic mesogens) or density driven (lyotropic mesogens).

%=====================================================================================================
%=====================================================================================================
\subsubsection{Calamitic mesogens}

Typically a calamitic mesogen contains an aromatic rigid core, formed from 1,4-phenyl groups, to
which one or more flexible alkyl chains are attached~\cite{introSoftMatter}. Short alkyl chains are 
typical of nematogens (mesogens that form nematic phases) while longer alkyl chains are related to
smectogens (mesogens that form smectic phases).\\
Liquid crystalline phases can be enhanced by increasing the length and polarisability of the
molecule as well as the addition of a terminal cyano group which induces polar interactions between
pairs of molecules. Lateral substituents (usually attached at the side of molecules in aromatic
cores) can influence molecular packing. For instance, adding a fluoro group enhances
polarisability but disrupts molecular packing leading to a shift in the
isotropic-nematic transition. Creating a lateral dipole can promote formation of a tilted \smC
and, in the case of chiral phases, gives rise to ferroelectricity. Further details regarding the 
effects of specific functions on liquid crystalline phase behaviour can be found in~\cite{greenBook}.

\picW=7cm
\begin{figure}
	\centering
	\subfigure[5CB]{\pic{5cb.ps}}
	\subfigure[8CB]{\pic{8cb.ps}}
	\caption{4-pentyl-4'-cyanobiphenyl (5CB) and 4-octyl-4'-cyanobiphenyl (8CB) molecules}
	\label{fig:nCB}
\end{figure}

The classic example of mesogenic substances are the nCB family shown on Figure~\ref{fig:nCB}. 
Here the aromatic core is made of a meta biphenyl; on one end of this core is the flexible tail, 
an alkyl chain of $n$ carbons ($C_nH_{2n+1}$,) and on the other end the head, composed by a 
cyano group. The influence of the alkyl chain length is readily observable by comparing the 
phase sequences of the 5CB and 8CB.

\begin{center}
\begin{tabular}{cc}
\textbf{5CB}~:  &Crystal $^{23^\circ\mathrm{C}}\rightarrow$ 
		Nematic $^{35^\circ\mathrm{C}}\rightarrow$  Isotropic\\


\textbf{8CB}~:	&Crystal $^{21^\circ\mathrm{C}}\rightarrow$ 
		\SmA $^{32.5^\circ\mathrm{C}}\rightarrow$ 
		Nematic $^{40^\circ\mathrm{C}}\rightarrow$  
		isotropic
\end{tabular}
\end{center}

\picW=6cm
\begin{figure}
	\centering
	\subfigure[]{\fbox{\begin{minipage}{\picW}\pic{nematic.ps}\end{minipage}}}
	\hspace*{10mm}
	\subfigure[]{\fbox{\begin{minipage}{\picW}\pic{smectic.ps}\end{minipage}}}
	\caption{The nematic (a) and smectic (b) phases.}
	\label{fig:N_Sm}
\end{figure}

The two main liquid crystalline phases available to calamitic mesogens are nematic and smectic.\\
The nematic phase is the simplest liquid crystalline phase and can be formed by calamitic and
discotic mesogens alike. This phase is characterized by~:
\begin{itemize}
	\item no long range translational order
	\item long range orientational order.
\end{itemize}
%
In the nematic phase (Figure~\ref{fig:N_Sm}(a)) the molecular positions are randomly distributed across 
the sample but their long axes all point, on average, towards the same direction, the director 
$\vecth{n}$. Also in the case of a nematic phase with a zero polar moment, the symmetry 
properties of the phase remain unchanged upon inversion of the director's direction.
If chiral molecules are used, a cholesteric or chiral nematic phase can be obtained. 
The difference between this and  the standard nematic phase is that in the former, the director 
twists as a function of position.\\


The smectic phase is characterized by~:
\begin{itemize}
	\item long range translational order (\ie one or two dimensions)
	\item long range orientational order.
\end{itemize}
%
In smectic phases (Figure~\ref{fig:N_Sm}(b),) as well as pointing along a common direction, the molecules are organized in
layers. According to the angle between the director and the layer normal as well
as any in plane positional ordering, several different smectic phases can be identified as shown
on Figure~\ref{fig:SmPhases}.

\picW = 3cm
\begin{figure}
	\centering
	\subfigure[\SmA]
	{\fbox{\begin{minipage}{6cm}\centering\pic{SmA.ps}\pic{SmA_top.ps}\end{minipage}}}
	%
	\subfigure[\SmB]
	{\fbox{\begin{minipage}{6cm}\centering\pic{SmA.ps}\pic{SmB_top.ps}\end{minipage}}}
	%
	\subfigure[\SmC]
	{\fbox{\begin{minipage}{6cm}\centering\pic{SmC.ps}\pic{SmC_top.ps}\end{minipage}}}
	%
	\subfigure[\SmI]
	{\fbox{\begin{minipage}{6cm}\centering\pic{SmC.ps}\pic{SmI_top.ps}\end{minipage}}}
	%
	\caption{Different types of smectic phases. Each Subfigure shows a front view on the
	left and a top view on the right. All these phases lack long range positional order.}
	\label{fig:SmPhases}
\end{figure}


%=====================================================================================================
%=====================================================================================================
\subsubsection{Discotic mesogens}

\picW = 7cm
\begin{figure}
	\centering
	\pic{HHTT.ps}
	\caption{Molecular representation of the HHTT molecule
	(2,3,6,7,11-hexa\-hexyl\-thio\-triphenylene).}
	\label{fig:typDisc}
\end{figure}


\picW = 4.5cm
\begin{figure}
	\centering
	\subfigure[Disordered (d)]{\pic{Col_d_h.ps}}
	\subfigure[Ordered (o)]{\pic{Col_o_h.ps}}
	\subfigure[Tilted (t)]{\pic{Col_t_h.ps}}
	\caption{The different discotic columnar phases.}
	\label{fig:discColumnPhases}
\end{figure}

\picW = 4.5cm
\begin{figure}
	\centering
	\subfigure[Hexagonal (h)]{\pic{topCol_h.ps}}
	\subfigure[Rectangular (r)]{\pic{topCol_r.ps}}
	\subfigure[Oblique (ob)]{\pic{topCol_ob.ps}}
	\caption{The different discotic columnar phases.}
	\label{fig:discColumnStack}
\end{figure}


Discotic molecules typically have a core composed of aromatic rings connected in an
approximately circular arrangement from which alkyl chains extend radially 
(see Figure~\ref{fig:typDisc}).  Discotic mesogens form discotic nematic and columnar phases. Several
types of the latter exist (see Figure~\ref{fig:discColumnPhases},) namely disordered (d), ordered (o) 
and tilted (t)  and for each of these there can be three column arrangements namely 
hexagonal (h), rectangular (r)  and oblique (ob). For instance a hexagonal disordered columnar 
phase is referred to as $\mathrm{Col}_{\mathrm{hd}}$ and an oblique ordered columnar phase as
$\mathrm{Col}_{\mathrm{ob,o}}$. 


