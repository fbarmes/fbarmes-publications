
\chapter{The liquid crystalline phases}
\label{chap:one}

\introduction

Up to the end of the $19^{\mathrm{th}}$ century, the three known states of matter were gas,
liquid and solid. According to dictionary definitions~\cite{physicsDictionary, structMatter} 
a substance is gaseous if it expands to fill its container homogeneously, regardless of its volume, 
displays a high compressibility and shows no positional correlations on length scales greater than
the molecular size. At the other extreme, a solid retains its shape, can support shear and its
atoms are restricted in space, oscillating about a fixed position. Two type of solids
can be distinguished; in crystalline solids, the atoms are regularly spaced on a 3 dimensional
lattice, whereas in amorphous or glassy solids, the atoms are disordered on a large scale
but ordered in the range of a few molecular lengths.\\
Liquid is a state of matter which exists between the two preceeding phases. A liquid substance 
only fills part of its container and its localisation is largely controlled by the gravitational 
force. Actually a liquid is only one component of a two phase system as every liquid is always 
searching to be in equilibrium with its own saturated vapor. A liquid has no rigidity, gives 
no resistance to shear under static conditions and has a small volume compressibility. On a 
microscopic scale, the positions of the constituent molecules are randomly distributed and no 
long range order can be found. Molecules in a liquid are subject to Brownian diffusion.\\

However, this picture changed somewhat in the late eighteen eighties. Slightly before then a number 
of scientist noticed some uncommon crystallization behaviour in certain substances which were
found to  transform from isotropic
liquid to a non-crystalline form before undergoing full crystallization. At the time, this
was attributed to the presence of impurities in the samples. The actual discovery of liquid
crystals is attributed to Friedrich Reinitzer~\cite{Reinitzer88} who, in 1888, was studying 
a compound related to cholesterol (cholesteryl benzoate). Reinitzer observed what he referred 
to as `two melting points' and identified the new phase; this was later termed liquid crystal 
by his friend and colleague Otto Lehmann~\cite{Lehmann89} who performed the first polarised 
optical microscopic measurements on liquid crystals.\\
Although not much studied at first because of the lack of the direct applications, 
interest in liquid crystals increased dramatically during the $20^{\mathrm{th}}$ century, mostly 
because of Liquid Crystal Display (LCD) applications; a wide understanding of these phases and 
the molecules that form them has now been built up.



%===================================================================================================
%===================================================================================================

\section{Characterization of liquid crystals.}



%==================================================================================================
%==================================================================================================
\subsection{The liquid crystalline phases}

The liquid crystalline phases refer to states of matter that exist between the isotropic liquid and
crystalline solid. A liquid crystal phase is formed by mesogenic particles, hence the term
mesogen used to refer to a molecule that forms a meso\-phase or liquid crystalline phase. A mesophase 
shares properties with both the liquid phase (flow, zero resistance to shear) and the
crystalline phase (long range positional and/or orientational order, anisotropic optical properties).
The term liquid crystal actually encompasses several different phases, the most common of which are
smectic and nematic~\cite{LiquidCrystals, Nature_phase, Intro_LC}.\\
A necessary but not sufficient requirement for a molecule to form a liquid crystalline 
phase (or mesophase) is a strong anisotropy in shape; mesogens are either
calamitic (rod shaped) or discotic (disc like). Their phase transitions can be temperature
driven (thermotropic mesogens) or density driven (lyotropic mesogens).

%=====================================================================================================
%=====================================================================================================
\subsubsection{Calamitic mesogens}

Typically a calamitic mesogen contains an aromatic rigid core, formed from 1,4-phenyl groups, to
which one or more flexible alkyl chains are attached~\cite{introSoftMatter}. Short alkyl chains are 
typical of nematogens (mesogens that form nematic phases) while longer alkyl chains are related to
smectogens (mesogens that form smectic phases).\\
Liquid crystalline phases can be enhanced by increasing the length and polarisability of the
molecule as well as the addition of a terminal cyano group which induces polar interactions between
pairs of molecules. Lateral substituents (usually attached at the side of molecules in aromatic
cores) can influence molecular packing. For instance, adding a fluoro group enhances
polarisability but disrupts molecular packing leading to a shift in the
isotropic-nematic transition. Creating a lateral dipole can promote formation of a tilted \smC
and, in the case of chiral phases, gives rise to ferroelectricity. Further details regarding the 
effects of specific functions on liquid crystalline phase behaviour can be found in~\cite{greenBook}.

\picW=7cm
\begin{figure}
	\centering
	\subfigure[5CB]{\pic{5cb.ps}}
	\subfigure[8CB]{\pic{8cb.ps}}
	\caption{4-pentyl-4'-cyanobiphenyl (5CB) and 4-octyl-4'-cyanobiphenyl (8CB) molecules}
	\label{fig:nCB}
\end{figure}

The classic example of mesogenic substances are the nCB family shown on Figure~\ref{fig:nCB}. 
Here the aromatic core is made of a meta biphenyl; on one end of this core is the flexible tail, 
an alkyl chain of $n$ carbons ($C_nH_{2n+1}$,) and on the other end the head, composed by a 
cyano group. The influence of the alkyl chain length is readily observable by comparing the 
phase sequences of the 5CB and 8CB.

\begin{center}
\begin{tabular}{cc}
\textbf{5CB}~:  &Crystal $^{23^\circ\mathrm{C}}\rightarrow$ 
		Nematic $^{35^\circ\mathrm{C}}\rightarrow$  Isotropic\\


\textbf{8CB}~:	&Crystal $^{21^\circ\mathrm{C}}\rightarrow$ 
		\SmA $^{32.5^\circ\mathrm{C}}\rightarrow$ 
		Nematic $^{40^\circ\mathrm{C}}\rightarrow$  
		isotropic
\end{tabular}
\end{center}

\picW=6cm
\begin{figure}
	\centering
	\subfigure[]{\fbox{\begin{minipage}{\picW}\pic{nematic.ps}\end{minipage}}}
	\hspace*{10mm}
	\subfigure[]{\fbox{\begin{minipage}{\picW}\pic{smectic.ps}\end{minipage}}}
	\caption{The nematic (a) and smectic (b) phases.}
	\label{fig:N_Sm}
\end{figure}

The two main liquid crystalline phases available to calamitic mesogens are nematic and smectic.\\
The nematic phase is the simplest liquid crystalline phase and can be formed by calamitic and
discotic mesogens alike. This phase is characterized by~:
\begin{itemize}
	\item no long range translational order
	\item long range orientational order.
\end{itemize}
%
In the nematic phase (Figure~\ref{fig:N_Sm}(a)) the molecular positions are randomly distributed across 
the sample but their long axes all point, on average, towards the same direction, the director 
$\vecth{n}$. Also in the case of a nematic phase with a zero polar moment, the symmetry 
properties of the phase remain unchanged upon inversion of the director's direction.
If chiral molecules are used, a cholesteric or chiral nematic phase can be obtained. 
The difference between this and  the standard nematic phase is that in the former, the director 
twists as a function of position.\\


The smectic phase is characterized by~:
\begin{itemize}
	\item long range translational order (\ie one or two dimensions)
	\item long range orientational order.
\end{itemize}
%
In smectic phases (Figure~\ref{fig:N_Sm}(b),) as well as pointing along a common direction, the molecules are organized in
layers. According to the angle between the director and the layer normal as well
as any in plane positional ordering, several different smectic phases can be identified as shown
on Figure~\ref{fig:SmPhases}.

\picW = 3cm
\begin{figure}
	\centering
	\subfigure[\SmA]
	{\fbox{\begin{minipage}{6cm}\centering\pic{SmA.ps}\pic{SmA_top.ps}\end{minipage}}}
	%
	\subfigure[\SmB]
	{\fbox{\begin{minipage}{6cm}\centering\pic{SmA.ps}\pic{SmB_top.ps}\end{minipage}}}
	%
	\subfigure[\SmC]
	{\fbox{\begin{minipage}{6cm}\centering\pic{SmC.ps}\pic{SmC_top.ps}\end{minipage}}}
	%
	\subfigure[\SmI]
	{\fbox{\begin{minipage}{6cm}\centering\pic{SmC.ps}\pic{SmI_top.ps}\end{minipage}}}
	%
	\caption{Different types of smectic phases. Each Subfigure shows a front view on the
	left and a top view on the right. All these phases lack long range positional order.}
	\label{fig:SmPhases}
\end{figure}


%=====================================================================================================
%=====================================================================================================
\subsubsection{Discotic mesogens}

\picW = 7cm
\begin{figure}
	\centering
	\pic{HHTT.ps}
	\caption{Molecular representation of the HHTT molecule
	(2,3,6,7,11-hexa\-hexyl\-thio\-triphenylene).}
	\label{fig:typDisc}
\end{figure}


\picW = 4.5cm
\begin{figure}
	\centering
	\subfigure[Disordered (d)]{\pic{Col_d_h.ps}}
	\subfigure[Ordered (o)]{\pic{Col_o_h.ps}}
	\subfigure[Tilted (t)]{\pic{Col_t_h.ps}}
	\caption{The different discotic columnar phases.}
	\label{fig:discColumnPhases}
\end{figure}

\picW = 4.5cm
\begin{figure}
	\centering
	\subfigure[Hexagonal (h)]{\pic{topCol_h.ps}}
	\subfigure[Rectangular (r)]{\pic{topCol_r.ps}}
	\subfigure[Oblique (ob)]{\pic{topCol_ob.ps}}
	\caption{The different discotic columnar phases.}
	\label{fig:discColumnStack}
\end{figure}


Discotic molecules typically have a core composed of aromatic rings connected in an
approximately circular arrangement from which alkyl chains extend radially 
(see Figure~\ref{fig:typDisc}).  Discotic mesogens form discotic nematic and columnar phases. Several
types of the latter exist (see Figure~\ref{fig:discColumnPhases},) namely disordered (d), ordered (o) 
and tilted (t)  and for each of these there can be three column arrangements namely 
hexagonal (h), rectangular (r)  and oblique (ob). For instance a hexagonal disordered columnar 
phase is referred to as $\mathrm{Col}_{\mathrm{hd}}$ and an oblique ordered columnar phase as
$\mathrm{Col}_{\mathrm{ob,o}}$. 





\section{Flexoelectric liquid crystals.}


%======================================================================================================
%======================================================================================================
\subsection{The flexoelectric effect.}


When a liquid crystal is subjected to a splay or bend strain, a net electrical polarization can
be created. This effect was first explained in 1969 by R.B. Meyer~\cite{Meyer69}. Reciprocally,
flexoelectric mesogens will splay or bend upon application of an electric field according to the
reverse flexoelectric effect.\\
Originally the flexoelectric effect was explained to arise for particles with a strong shape
anisotropy (\ie pear or banana shaped particles) and a strong permanent dipole~\cite{Meyer69}. 
According to Meyer, a splay distortion 
would arise for wedge  or pear shaped particles while a bend distortion is associated with banana shaped
particles. The relation between the electrical polarization $\vect{P}$ and the flexoelectric splay 
and bend coefficients $e_{1z}$ and $e_{3x}$ is given by~\cite{deGennes}:
\begin{equation}
	\vect{P} = e_{1z}\lp \vecth{n}(\nabla\centerdot\vecth{n}) \rp 
	+ e_{3x}\lp\nabla\times\vecth{n}\rp\times\vecth{n}
\end{equation}

\picW = 5cm
\begin{figure}
	\centering
	\subfigure[no applied field]{\fbox{\pic{pearNematic.ps}}}
	\hspace*{10mm}
	\subfigure[field applied]{\fbox{\pic{pearSplay.ps}}}
	
	\subfigure[no applied field]{\fbox{\pic{bananaNematic.ps}}}
	\hspace*{10mm}
	\subfigure[field applied]{\fbox{\pic{bananaBend.ps}}}
	\caption[Representation of the polarisation induced distortions in flexoelectric systems.]
	{Representation of the polarisation induced distortions in flexoelectric systems. 
	Pear shaped particles (top) have a splay distortion while banana shaped particles
	(bottom) display a bend distortion. The distorted phases correspond to a situation with
	an electric field $\vect{E}$ applied vertically and pointing towards to the top of the page.}
\end{figure}

Means for the calculation of the flexoelectric coefficients were later devised by Derzhanski and
Petrov~\cite{DerzhanskiPetrov71a,DerzhanskiPetrov71a} and Helfrich~\cite{Helfrich71}. The latter
author also extended Meyer's theory to polarizable molecules~\cite{Helfrich74}\\
%
Eight years after Meyer's original theory, another mechanism for flexoelectricity in liquid
crystals was proposed by Prost and Marcerou~\cite{ProstMarcerou77}. In this work, 
the electrostatics of uniaxial phases were examined to recognize the link between polarization and 
strain. A two term expression for the  flexoelectric tensor was derived (see Equation 2.9
of~\cite{ProstMarcerou77}) where 
the first term corresponds to a dipolar effect equivalent to that proposed by Meyer, while the second 
relates to a non zero quadrupole moment that pertains even with non pear or banana shaped molecule 
(\ie ellipsoids).  The authors also showed that both mechanisms contributed to the total flexoelectric 
coefficient  $f = e_{1z} + e_{3x}$ with similar orders of magnitude.\\
%
The implication of this result is that flexoelectricity is an intrinsic property of liquid
crystals as most mesogens have non zero quadrupole moments. Also, this effect should be
observable not only in the nematic phase as proposed by Meyer, but also in the isotropic
phase (see~\cite{ProstMarcerou78b} for the first experimental verification of this) as well 
as the smectic phase~\cite{ProstPershan76}.

%======================================================================================================
%======================================================================================================
\subsection{The flexoelectric coefficients.}

Due to the comparability of the contributions from the dipolar and quadrupolar mechanisms to the
flexoelectric coefficients, identification of which contribution is the largest
is not a trivial exercise. Resolving this question requires a means of separating the two effects.
One such method is the study of the thermal  dependence of the flexoelectric coefficient $f$.  
From the study of Prost and  Marcerou~\cite{ProstMarcerou77, ProstMarcerou78a} 
the expression for $f$ is~:
%
\begin{eqnarray}
	f &=& e_{1z} + e_{3x} \hspace*{1cm} \mrm{(in\ Meyer's\ notation)}	\\
	f &=& f^M + f^Q	\\
	f &=& 	K_{11}\frac{\epsilon^0_\parallel - \epsilon^1_\parallel}{4\pi\mu_\parallel}
		+ K_{33}\frac{\epsilon^0_\perp - \epsilon^1_\perp}{4\pi\mu_\perp}
		-\frac{2}{3}N\theta_a S
		\label{eqn:flexoCoeff}
\end{eqnarray}

In the first two terms representing $f^M$, $K_{11}$ and $K_{33}$ are respectively the splay 
and bend Frank Oseen elastic constant~\cite{deGennes}, $\epsilon^0_\parallel$ and 
$\epsilon^0_\perp$ are the dielectric constant taken respectively parallel and perpendicular to 
the director $\vecth{n}$ and $\mu^{-1}$ measures the degree of asymmetry of the compound under 
interest. In the last term representing $f^Q$, $N$ is the number of particles per unit volume,
$\theta_a$ is the quadrupole tensor as defined in~\cite{ProstMarcerou77} and $S$ is the order
parameter as defined in~\cite{deGennes}.\\

From Equation~\ref{eqn:flexoCoeff}, it is readily observable that $f$ depends linearly 
on $K_{11}$ and $K_{33}$ that is
linearly with $S^2$ if Meyer's contribution is of more significance. If the quadrupolar
contribution is the larger, then $f$ is a linear function of $S$.\\

Although several methods are available for the measurement of the flexoelectric
coefficients~\cite{deGennes,LiquidCrystals,MazzullaGiuchi00,JewellSambles02}, the best seems to be 
the ``interdigital electrodes technique" (see~\cite{ProstPershan76} and~\cite{MarcerouProst80} 
for a detailed description) as used by Marcerou and Prost. This setup involves the use of
a spatially periodic electric field (created by the interdigitated electrodes) which induces a 
periodic distortion of the liquid crystal.  This distortion corresponds to a grating which scatters 
light.  Measurements of the flexoelectric coefficients are thus performed through the measurement 
of the scattered light intensity.\\

Measurements of the thermal dependence of $f$ has been performed by Marcerou and 
Prost~\cite{MarcerouProst80} for four different mesogens ranging from symmetric to strongly dipolar. 
Although the expectation from this study was that the quadrupolar contribution would prove stronger for 
the most symmetrical mesogens, with Meyer's contribution being stronger for the dipolar
particles, the results 
showed that the quadrupole contribution was actually stronger for virtually all compounds. The only case
where  Meyer's contribution was more significant was that of a molecule with strong steric constraints as
already suggested a previous study~\cite{ProstMarcerou78a}. This last  finding was later confirmed 
in~\cite{DozovMartinotLagarde83} in a study of cyanobiphenyl components using a different
measurement technique. In parallel with this, using the Onsager-like theory of Straley~\cite{Straley76}, 
Osipov~\cite{Osipov83} showed that the dipole flexoelectricity is significant only for molecules 
with large transverse dipoles; this condition is met by the mesogens mentioned above which showed 
Meyer's flexoelectricity.\\
Some subsequent theoretical work from the same author~\cite{Osipov84}, using a Landau-de Gennes 
formalism contradicted the  general results from this series of experiments, as it predicted a 
$S^2$-like  variation.  This discrepancy is likely  explained by the lack of conformational 
freedom in the theoretical treatment, however.\\
Using density functional theory, Singh and Singh~\cite{SinghSingh89} showed that, given increased 
knowledge of the molecular parameters of a system, the flexoelectric coefficients 
(taking into account both mechanisms) can be  accurately calculated. The main drawback of this
treatment is its restriction to rigid particles. This restriction was subsequently lifted by 
Ferrarini~\cite{Ferarrini01} who applied mean-field treatment to MBBA which took into account 
the structure of the molecule's transformers.

Model for flexoelectric pear-shaped molecules have been designed for use in computer 
simulations~\cite{StelzerBerardi99,BilleterPelcovits00}. These have found non zero splay and near to
zero bend flexoelectric coefficients in accordance with Meyer's theory. These models were
subsequently refined so as to become monosite as opposed to multisite~\cite{BerardiRicci01},
leading to phases with net polar order. More details on these studies 
are given in Chapter~\ref{chap:two}.\\

The interest in flexoelectric particles is not only of academic interest as the phenomenon finds
some very important applications in bistable liquid crystal display
devices~\cite{DavidsonMottram02,DennistonYeomans} which could lead to the development of the
next generation of displays.


\section{Theoretical approach to liquid crystals.}


As interest in liquid crystal grows, the number of theories used to describe their
complicated phase behaviour increases similarly. Here the focus is brought on to bear on to molecular
theories that, using statistical mechanics, take the intermolecular potential as a starting point 
from which to deduce the macroscopic phase behaviour.\\
%
Theoretical studies of simple atomic fluids~\cite{theorySimpleLiquids} show that the liquid 
phase can be described effectively using intermolecular potentials of the Lennard-Jones form,
\ie containing both long range attractive and short range repulsive component. Mesogens 
can be represented with a similar class of potentials, though account has to be taken of the
inherent in their elongated shape. One significant question considered by these theories is
which of the repulsive and attractive components of the interaction are of greater importance in
mesophase formation. As a result, theories have been developed which consider both of these
contributions to the anisotropic potentials and thus quantify their respective influences. The two main
and complementary approaches are described in the following Sections, namely Onsager and Maier-Saupe
theories.


%===============================================================================================
%===============================================================================================
\subsection{Density functional theory}

The phase structure of simple atomistic fluids can be described successfully using hard spheres as
models, that is considering only the repulsive part of their pair
potential~\cite{BarkerHenderson76}. The extension of this principle to the mesoscale was
first achieved by Lars Onsager in 1949~\cite{Onsager49} in a study of colloidal particles
(tobacco virus). The main idea underlying this seminal work is that the mechanism for spontaneous 
ordering in a system of hard molecules is based on competition between the orientational entropy that
destroys nematic order and the positional entropy that favors it~\cite{deGennes}.\\

Onsager theory is derived from the cluster approach~\cite{greenBook} and is a density functional theory 
in which the free energy $\mathcal{F}$ is expressed as a density virial expansion~\cite{PadillaVelasco97}.
\begin{eqnarray}
	\mathcal{F} &=& \mathcal{F}^{id} + \mathcal{F}^{ex}	\\
	\mathcal{F}^{id} &=& N(\log\rho - 1) + N \int f(\vecth{u})\log\lp 4\pi f(\vecth{u}) \rp d\vecth{u} \\
	\mathcal{F}^{ex} &=& \rho\mathcal{B}_2 \lp f(\vecth{u}) \rp
			+ \frac{1}{2}\rho^{2}\mathcal{B}_3 \lp f(\vecth{u}) \rp
			+ \frac{1}{3}\rho^{3}\mathcal{B}_4 \lp f(\vecth{u}) \rp
			+ \ldots	\label{exess_F}
\end{eqnarray}
%
Where $\mathcal{F}^{id} $ and $\mathcal{F}^{ex} $ are the ideal and excess parts of $\mathcal{F}$ and
$f(\vecth{u}) $ is the orientational distribution function that depends on the particle orientation
vector $\vecth{u}$. Each $\mathcal{B}_i$ represents the excluded volume in a cluster of $i$ 
particles.

Onsager made the following assumptions~:
\begin{itemize}
	\item The molecules (spherocylinders of length $L$ and diameter $D$) interact only through steric
	repulsion (no interpenetration).
	%
	\item The volume fraction is much smaller than $1$.
	%
	\item The rods are very long ($L \gg D$).
\end{itemize}

Within these assumptions, Onsager showed that the virial expansion can be truncated after the second
coefficient; he showed that the third one vanishes and assumed the same for the the following
coefficients. This makes his approach the simplest form of density functional theory.

The implication of his results is that, in the limits considered, the N-I transition 
can be explained exclusively using short range repulsive forces. However, the approximations made
in this theory worsen
considerably as the particle elongation is reduced because the density of the transition is not
vanishing. Therefore this theory can not be applied to mesogens with a standard elongation of 3 to
5. Early computer simulations of hard prolate particles by Vieillard-Baron~\cite{VieillardBaron72} and 
later Frenkel \etal~\cite{FrenkelMulder81,FrenkelMulder85} showed qualitative but not 
quantitative agreement with Onsager's theory\\
This does not imply that density functional theory can not be applied to liquid crystalline
behaviours, only that, for the accurate description of mesogens, more virial coefficients are
needed. The most obvious approach, then is to calculate higher order coefficients, as
been done in~\cite{Vega97,VelascoPadila94}, but the difficulty of the task increases
significantly with the order of the coefficients. A better approach is the use of resummation
techniques such as the y-expansion~\cite{MulderFrenkel85,TjiptoMargoEvans90} that allow the indirect 
inclusion of high order coefficients. Some other resummation methods have been
used successfully on single component~\cite{Parsons79,Williamson95} and 
mixture systems~\cite{StroobantsLekkerkerker84,CampAllen96} leading, recently, to considerable
improvements in the description of anisotropic fluids~\cite{CinacchiSchmid02}.



%===============================================================================================
%===============================================================================================
\subsection{Maier Saupe theory.}


In the early $20^\mathrm{th}$ Century, Born showed that the anisotropic component of the pair potential
was responsible for the order-disorder transitions in nematic phases~\cite{greenBook}.
This was later expanded by Maier and Saupe to give rise to the so called Maier Saupe (MS)
theory~\cite{MaierSaupe58,MaierSaupe59,MaierSaupe60}.\\
The configurational partition function of a fluid is expressed as~:
\begin{equation}
	Q_N = \frac{1}{N!}\int e^{-\beta U(\vect{X}^N) }d\vect{X}^N
\end{equation}
where $\vect{X}^N$ represents the full set of positional and orientational coordinates for the $N$
particles of the fluid. In the case of a perfect gas, as every particle's behaviour  can be
taken to be independent of all others~\cite{structMatter}, the partition function can be 
transformed into the
product of $N$ single particle partition functions, each of which is easily solvable. However,
in the case
of a fluid phase, and even more so a mesophase, the relatively high density implies that each 
particle's energy is dependent on several other particles' coordinates so that the previous 
simplification is no longer valid.\\
MS theory aims to resolve this difficulty using the so called molecular field
approximation. This approximation removes the need for the consideration of each individual 
pair potential; rather each particle is taken to reside in a field which mimics the presence of
all the other particles. Effectively each particle is assumed to be was moving in an 
energy continuum. As a result $U(\vect{X}^N)$ can be written as the sum of $N$ energies, each of 
which is a function of the coordinates of a single particle~:
\begin{equation}
	U(\vect{X}^N) = \sum_{i=1}^N U(\vect{X}_i)
\end{equation}

The problem, then, is to find an expression for the mean-potential experienced by the particles. 
Several approaches have been used for this. The most intuitive approach is to average the anisotropic 
pair potential over the coordinates of one  particle~\cite{HumphriesJames71} while one of the most
simple and rigorous is to start from the singlet distribution function and solve the
Bogoliubov-Born-Green-Kirkwood-Yvon hierarchy of equations as described in~\cite{greenBook}. 
In a translationally invariant situation, the general form of the final mean potential depends only
on $\beta$, the angle between the particle under consideration and the director $\vecth{n}$, so:
\begin{equation}
	U(\beta) = -\epsilon \overline{P_2} P_2(cos^2\beta)
\end{equation}
where $\overline{P_2}$ is the nematic order parameter, $P_2(x)$ is the second order Legendre
polynomial of $x$ and $\epsilon$ is a scaling constant which is given by~:
\begin{equation}
	\epsilon = -V\lp\frac{\partial \epsilon}{\partial V}\rp_T
\end{equation}
Another form for $\epsilon$ is given in~\cite{Intro_LC} by the identification~:
\begin{equation}
	\epsilon = \frac{A}{V^2}
\end{equation}
where the value of $A$ is determined by the interaction properties of the molecules.
In any case the general approach for the implementation of Maier Saupe theory is as follows~:
\begin{itemize}
	\item Choose the anisotropic term of the pair potential
	\item Implement the mean field approximation
	\item Determine the mean potential
	\item Deduce the singlet distribution function
	\item Use this function to calculate the entropy, Helmholtz free energy and order parameters.
\end{itemize}
The two major elements of the theory that influence its accuracy are the mean field 
approximation and the form of the mean potential.\\

Maier Saupe theory can be tested using both computer simulation and real experiment, though
 the former has an advantage since it can test the validity of the mean field approximation 
as the mean-potential can be specified in the `computer experiment'. The theory is reasonably
successful in describing, qualitatively, the behaviour of mesogens, showing a first order IN
transition. The temperature dependence of the order 
parameters is also well described qualitatively. The limits of the Maier Saupe theory become 
apparent in the quantitative predictions, errors being attributed, in part, to the use of 
the mean field approximation (although some improvement can be made by improving the form of the 
potential.) A more fundamental weakness of the mean field approximation is that it neglects
spacial and orientational correlations between molecules.\\

The main conclusion that can be drawn from this is that while MS theory is most effective at 
long range it can successfully describe liquid crystalline behaviour. This implies that short-range 
repulsive forces have little role to play, whereas short range potentials have been shown to
be responsible for both the ordering of nematics and the structure of normal liquids. This apparent
discrepancy can be resolved by appreciating that the long range attractive 
part of the potential used in MS theory  can be regarded  as describing the interactions 
between clusters of highly ordered particles. 








\section{Experimental study of liquid crystals.}

%===============================================================================================
%===============================================================================================
\subsection{Optical polarising microscopy.}

Optical polarising microscopy is the main technique used for liquid crystalline phase
characterization~\cite{Intro_LC}. Historically it was also the first technique used by Lehmann
when he studied the liquid crystalline samples provided by Reinitzer.\\
The technique consists of observing, under a microscope, a sample sandwiched between crossed polarizers. 
An isotropically liquid phase does not affect the light and, therefore, no
light can cross the analyzer. In the case of a liquid crystalline phase, however, the birefringent 
property of the material induces refraction of the light according to the director orientation. 
Since only the component of the refracted light parallel to the analyzer polarization direction is
transmitted, the intensity of the transmitted light varies from white if
$\dotProdP{\vecth{n}}{\vecth{p}}= 0$ to black if $\dotProdP{\vecth{n}}{\vecth{p}}= 1$ where 
$\vecth{p}$ is the polarisation direction direction of the analyser. Moreover, because of defects 
in the structure and disclination lines, the orientation of $\vecth{n}$ typically varies with 
position and, therefore, so does the intensity of transmitted light.
Distinct patterns can be observed for different mesophases. Examples of these patterns can be
found in~\cite{Nature_phase}.\\
Nematic phases induce the so called Schlieren textures where black threads marking the
disclination lines can be observed. These threads lead to the name `nematic' being used for this
phase. \SmA liquid crystals have a  `fan-like' pattern when viewed through crossed polarizers and 
\smC a combination of both. The \smB phase induces an altogether different pattern of `mosaic' or 
broken fan textures. However not every phase can be distinguished clearly. For example, \smI 
and \smF phases induce patterns very similar to that of the \smC phase which makes the task 
of identification very difficult if this is the only technique to be used.



%===============================================================================================
%===============================================================================================
\subsection{Differential scanning calorimetry.}

Differential scanning calorimetry (DSC) is a technique used to complement optical 
polarising microscopy in the phase characterization of liquid crystals~\cite{Intro_LC}. 
This technique measures
enthalpy changes ($\Delta H$) at phase transitions. The phase type of the sample is not 
examined using this technique, but the value of the enthalpy gives some information about the degree 
of molecular order in a mesophase.\\
With this technique, two independently heated furnaces are used. One is empty or contains a 
reference sample (usually gold) and  the second contains the sample under study. Both furnaces 
are linked to control loops which
insure that they are kept at the same temperature as each other. Upon cooling or heating, the heat 
absorbed by or released from the sample in order to keep the furnaces at the same temperature is 
measured. Differences between the heat measurement for the two furnaces indicate phase transitions. 
With this technique, temperatures  ranging from $-180^\circ C$ to $600^\circ C$ can be accessed.\\

Thermodynamics state that there are two types of phase transition discontinuous ($1^\mathrm{st}$ order)
and continuous ($2^\mathrm{nd}$ order) corresponding to discontinuities in, respectively, the first and
second derivatives of the Gibbs free energy $G$~:
%
\begin{gather}
	G = H - TS\\
	\lp\frac{\partial G}{\partial T}\rp = -S	\\
	\lp\frac{\partial^2 G}{\partial T^2}\rp = -\frac{C_p}{T}
\end{gather}
%
Thus, a first order transition induces a discontinuity in the entropy and a peak of the DSC baseline
can be observed, whereas a second order transition is indicated by  inflexion of the baseline.\\
Thus a DSC trace can reveal phase transitions that would be missed by optical 
polarising microscopy because of the smallness of changes in the optical properties. Conversely phase
transitions with small enthalpy changes but rather different optical properties can be missed
with the DSC but are easily detected with optical polarising microscopy.


%===============================================================================================
%===============================================================================================
\subsection{X-ray and neutron diffraction.}

X-ray diffraction is one of the most effective techniques for liquid crystalline phase
characterization~\cite{introSoftMatter, greenBook}. Here the mesophase is characterised by analysis of
the diffraction pattern of an X-ray beam incident upon a sample in which the molecules are
aligned with the beam. According to  Bragg's law, diffraction
is obtained at an angle $\theta$ when $\lambda = 2d\sin\theta$ where $\lambda$ is the light
wavelength and $d$ the intermolecular spacing.\\
%
Liquid crystal diffraction patterns are characterized by two lateral vertical clear areas that account 
for the vertical alignment of the molecules. In the case of a nematic, horizontal clear areas,
corresponding to diffuse low intensity peaks, can be observed above and below the centre of the
pattern. In the case of a smectic, the latter are replaced by points corresponding to sharp high
intensity peaks induced by the smectic layering. In the case of a \smA, those points
are located on the vertical axis whereas they lie at an angle in the case of a tilted
smectic such as a monodomain \smC.


\section{Applications}



%===============================================================================================
%===============================================================================================
\subsection{Liquid Crystal Displays}

Since the discovery of electro-optical effects in liquid crystals in about 1968, their main
applications have been in the display technology~\cite{Shanks82}. The first
device that could be used industrially is the Twisted Nematic 
device (see Figure~\ref{fig:TNcell}).\\
%
This device uses a liquid crystal sandwiched between two electrically conductive glass plates 
rubbed so as to induce a planar surface arrangement. This cell is placed between crossed polarizers. 
The direction of rubbing on the glass plates is made parallel to the local polarizing direction, thus 
inducing a twist of $90^\circ$  from the top to the bottom of the cell. Due to the optical 
anisotropy of the liquid crystal, the direction of polarization of the light in the cell is twisted 
(following the director,) as a result of which the light is transmitted through the analyzer. 
This correspond to a light cell.\\
Upon application of an electric field $\vect{E}$ between the glass plates, the 
molecules reorient to be approximately parallel to $\vect{E}$; in this state, the polarization 
of the light is left unchanged by the liquid crystal molecules and therefore no light can be 
transmitted. This corresponds to the dark state.

\picW = 7cm
\begin{figure}
	\centering
	\pic{TNCell.ps}
	\caption{Schematic representation of the Twisted Nematic display cell. The `on' state
	without applied field is shown on the left and the `off' state with applied electric
	field represented by the black arrow is shown on the right.}
	\label{fig:TNcell}
\end{figure}

The TN cell is mostly used in the wrist-watch type of display~\cite{Geelhaar98} and can employ
different types of compound~\cite{Eidenschink85,Eidenschink89}, the most simple of which is the 
5CB. In order to meet the requirements for more sophisticated displays, such as those used in 
cellular phones or  laptop computers, more advanced display cell have been designed. The first 
such improvement is the Supertwisted Nematic Cell (STN). This is a direct refinement of the TN 
cell but with more advanced mesogens allowing a twist of $270^\circ$ instead of $90^\circ$. This 
results in a sharper and faster transition between the light and dark states.\\
Further refinements have been achieved by improving the addressing of the display's individual
pixels. This lead to the development of the active matrix TFT display which are, to date, the most
used screens for laptop computers.\\
The current trend is towards the development of bistable  
displays~\cite{KreuzerTschudi92,GiocondoLelidis99,DavidsonMottram02}
where, by having two stable arrangements corresponding to the light and dark states of the cell, the
electric field no longer needs to be applied to maintain the dark state. Rather an
electric pulse is used to switch the cell. The main advantages of such display are their much
reduced power consumption, which is of crucial importance in portable devices. Additionally, such
displays can be used as optical storage devices. Further developments in this area include 
tristable nematics displays, which, have the potential to yield extremely versatile
functionality~\cite{KimYoneya02}.

%===============================================================================================
%===============================================================================================
\subsection{Other applications.}

There are numerous applications for liquid crystals other than display. In laser optics, for
instance, a laser can produce grid patterns in a nematic phase which, in turn, can be used as
switches for another laser. Polymer dispersed liquid crystal (PDLC) sheet can be formed into
big panels which, upon application of an electric field, can be made opaque or transparent. A
rather versatile bathroom window can be produced that way, as shown in~\cite{Intro_LC}.\\
Liquid crystals have also been proposed for applications in 
engineering~\cite{Eidenschink88,Eidenschink89a,EidenschinkKonrath99} that exploit the variation of
the viscosity coefficients of mesogens in different phases. This may lead to the development of
very efficient lubricants or bearings which can act as breaks if their temperature exceeds some
threshold.\\
Finally, one possibly surprising area for the application of liquid crystal science is the human body
itself, but many living cells and viruses display and utilise liquid crystalline 
prospectives~\cite{Goodby98}.


%===================================================================================================
%===================================================================================================




