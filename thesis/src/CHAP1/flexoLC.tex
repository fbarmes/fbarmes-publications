

\section{Flexoelectric liquid crystals.}


%======================================================================================================
%======================================================================================================
\subsection{The flexoelectric effect.}


When a liquid crystal is subjected to a splay or bend strain, a net electrical polarization can
be created. This effect was first explained in 1969 by R.B. Meyer~\cite{Meyer69}. Reciprocally,
flexoelectric mesogens will splay or bend upon application of an electric field according to the
reverse flexoelectric effect.\\
Originally the flexoelectric effect was explained to arise for particles with a strong shape
anisotropy (\ie pear or banana shaped particles) and a strong permanent dipole~\cite{Meyer69}. 
According to Meyer, a splay distortion 
would arise for wedge  or pear shaped particles while a bend distortion is associated with banana shaped
particles. The relation between the electrical polarization $\vect{P}$ and the flexoelectric splay 
and bend coefficients $e_{1z}$ and $e_{3x}$ is given by~\cite{deGennes}:
\begin{equation}
	\vect{P} = e_{1z}\lp \vecth{n}(\nabla\centerdot\vecth{n}) \rp 
	+ e_{3x}\lp\nabla\times\vecth{n}\rp\times\vecth{n}
\end{equation}

\picW = 5cm
\begin{figure}
	\centering
	\subfigure[no applied field]{\fbox{\pic{pearNematic.ps}}}
	\hspace*{10mm}
	\subfigure[field applied]{\fbox{\pic{pearSplay.ps}}}
	
	\subfigure[no applied field]{\fbox{\pic{bananaNematic.ps}}}
	\hspace*{10mm}
	\subfigure[field applied]{\fbox{\pic{bananaBend.ps}}}
	\caption[Representation of the polarisation induced distortions in flexoelectric systems.]
	{Representation of the polarisation induced distortions in flexoelectric systems. 
	Pear shaped particles (top) have a splay distortion while banana shaped particles
	(bottom) display a bend distortion. The distorted phases correspond to a situation with
	an electric field $\vect{E}$ applied vertically and pointing towards to the top of the page.}
\end{figure}

Means for the calculation of the flexoelectric coefficients were later devised by Derzhanski and
Petrov~\cite{DerzhanskiPetrov71a,DerzhanskiPetrov71a} and Helfrich~\cite{Helfrich71}. The latter
author also extended Meyer's theory to polarizable molecules~\cite{Helfrich74}\\
%
Eight years after Meyer's original theory, another mechanism for flexoelectricity in liquid
crystals was proposed by Prost and Marcerou~\cite{ProstMarcerou77}. In this work, 
the electrostatics of uniaxial phases were examined to recognize the link between polarization and 
strain. A two term expression for the  flexoelectric tensor was derived (see Equation 2.9
of~\cite{ProstMarcerou77}) where 
the first term corresponds to a dipolar effect equivalent to that proposed by Meyer, while the second 
relates to a non zero quadrupole moment that pertains even with non pear or banana shaped molecule 
(\ie ellipsoids).  The authors also showed that both mechanisms contributed to the total flexoelectric 
coefficient  $f = e_{1z} + e_{3x}$ with similar orders of magnitude.\\
%
The implication of this result is that flexoelectricity is an intrinsic property of liquid
crystals as most mesogens have non zero quadrupole moments. Also, this effect should be
observable not only in the nematic phase as proposed by Meyer, but also in the isotropic
phase (see~\cite{ProstMarcerou78b} for the first experimental verification of this) as well 
as the smectic phase~\cite{ProstPershan76}.

%======================================================================================================
%======================================================================================================
\subsection{The flexoelectric coefficients.}

Due to the comparability of the contributions from the dipolar and quadrupolar mechanisms to the
flexoelectric coefficients, identification of which contribution is the largest
is not a trivial exercise. Resolving this question requires a means of separating the two effects.
One such method is the study of the thermal  dependence of the flexoelectric coefficient $f$.  
From the study of Prost and  Marcerou~\cite{ProstMarcerou77, ProstMarcerou78a} 
the expression for $f$ is~:
%
\begin{eqnarray}
	f &=& e_{1z} + e_{3x} \hspace*{1cm} \mrm{(in\ Meyer's\ notation)}	\\
	f &=& f^M + f^Q	\\
	f &=& 	K_{11}\frac{\epsilon^0_\parallel - \epsilon^1_\parallel}{4\pi\mu_\parallel}
		+ K_{33}\frac{\epsilon^0_\perp - \epsilon^1_\perp}{4\pi\mu_\perp}
		-\frac{2}{3}N\theta_a S
		\label{eqn:flexoCoeff}
\end{eqnarray}

In the first two terms representing $f^M$, $K_{11}$ and $K_{33}$ are respectively the splay 
and bend Frank Oseen elastic constant~\cite{deGennes}, $\epsilon^0_\parallel$ and 
$\epsilon^0_\perp$ are the dielectric constant taken respectively parallel and perpendicular to 
the director $\vecth{n}$ and $\mu^{-1}$ measures the degree of asymmetry of the compound under 
interest. In the last term representing $f^Q$, $N$ is the number of particles per unit volume,
$\theta_a$ is the quadrupole tensor as defined in~\cite{ProstMarcerou77} and $S$ is the order
parameter as defined in~\cite{deGennes}.\\

From Equation~\ref{eqn:flexoCoeff}, it is readily observable that $f$ depends linearly 
on $K_{11}$ and $K_{33}$ that is
linearly with $S^2$ if Meyer's contribution is of more significance. If the quadrupolar
contribution is the larger, then $f$ is a linear function of $S$.\\

Although several methods are available for the measurement of the flexoelectric
coefficients~\cite{deGennes,LiquidCrystals,MazzullaGiuchi00,JewellSambles02}, the best seems to be 
the ``interdigital electrodes technique" (see~\cite{ProstPershan76} and~\cite{MarcerouProst80} 
for a detailed description) as used by Marcerou and Prost. This setup involves the use of
a spatially periodic electric field (created by the interdigitated electrodes) which induces a 
periodic distortion of the liquid crystal.  This distortion corresponds to a grating which scatters 
light.  Measurements of the flexoelectric coefficients are thus performed through the measurement 
of the scattered light intensity.\\

Measurements of the thermal dependence of $f$ has been performed by Marcerou and 
Prost~\cite{MarcerouProst80} for four different mesogens ranging from symmetric to strongly dipolar. 
Although the expectation from this study was that the quadrupolar contribution would prove stronger for 
the most symmetrical mesogens, with Meyer's contribution being stronger for the dipolar
particles, the results 
showed that the quadrupole contribution was actually stronger for virtually all compounds. The only case
where  Meyer's contribution was more significant was that of a molecule with strong steric constraints as
already suggested a previous study~\cite{ProstMarcerou78a}. This last  finding was later confirmed 
in~\cite{DozovMartinotLagarde83} in a study of cyanobiphenyl components using a different
measurement technique. In parallel with this, using the Onsager-like theory of Straley~\cite{Straley76}, 
Osipov~\cite{Osipov83} showed that the dipole flexoelectricity is significant only for molecules 
with large transverse dipoles; this condition is met by the mesogens mentioned above which showed 
Meyer's flexoelectricity.\\
Some subsequent theoretical work from the same author~\cite{Osipov84}, using a Landau-de Gennes 
formalism contradicted the  general results from this series of experiments, as it predicted a 
$S^2$-like  variation.  This discrepancy is likely  explained by the lack of conformational 
freedom in the theoretical treatment, however.\\
Using density functional theory, Singh and Singh~\cite{SinghSingh89} showed that, given increased 
knowledge of the molecular parameters of a system, the flexoelectric coefficients 
(taking into account both mechanisms) can be  accurately calculated. The main drawback of this
treatment is its restriction to rigid particles. This restriction was subsequently lifted by 
Ferrarini~\cite{Ferarrini01} who applied mean-field treatment to MBBA which took into account 
the structure of the molecule's transformers.

Model for flexoelectric pear-shaped molecules have been designed for use in computer 
simulations~\cite{StelzerBerardi99,BilleterPelcovits00}. These have found non zero splay and near to
zero bend flexoelectric coefficients in accordance with Meyer's theory. These models were
subsequently refined so as to become monosite as opposed to multisite~\cite{BerardiRicci01},
leading to phases with net polar order. More details on these studies 
are given in Chapter~\ref{chap:two}.\\

The interest in flexoelectric particles is not only of academic interest as the phenomenon finds
some very important applications in bistable liquid crystal display
devices~\cite{DavidsonMottram02,DennistonYeomans} which could lead to the development of the
next generation of displays.

