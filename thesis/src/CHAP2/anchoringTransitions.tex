
\section{Anchoring Transitions}

The study of confined liquid crystalline systems has shown that different
particle-substrate interactions can lead to various types of surface anchoring. 
A number of experimental 
studies have reported that, upon changing some of the experimental conditions (\ie temperature,
structure of the aligning agent), transitions between two  surface arrangements can be observed
which leads to a change in the bulk alignment. 
Transitions which meet these requirements
are called \emph{anchoring transitions}. However, even in studies where the surface induced 
structural changes have been thoroughly studied, the mechanisms underlying these anchoring 
transitions remain relatively unexplained.
From the many experimental studies reporting anchoring transitions, it can be deduced that
 a concensus about the origins of these transitions is still lacking since a number of 
different mechanisms have been proposed to explain the origins of anchoring
transitions.\\ 

The first observed anchoring transitions were temperature  
driven~\cite{PatelYokoyama93,JagemalmKomitov1997}~; other authors have shown that anchoring
transitions can be obtained by a change in the conformation of the surface aligning
agent~\cite{ZhuLu94,ZhuWei94}. Light can also induce anchoring transitions by changing the
structure of the substrate molecules (\eg light induced cis-trans isomerization) and, therefore, the 
particle-substrate interaction~\cite{BarberoPopaNita00}. This is a feature that can be exploited
in optical storage devices. The absorption behaviour of a liquid crystal on the 
substrate~\cite{AlkhairallaAllison99,AlkhairallaBoden02} or of volatile molecule on the 
liquid crystal substrate interface~\cite{Jerome93} can also lead to anchoring transitions if the 
amount of absorbed particles  or the nature of the  absorption is changed.
One last mechanism is the memory effect whereby a multistable
anchoring system can preferentially adopt one of its possible anchoring states due to the history 
of the sample~\cite{Jerome91,StoenescuMartinotLagarde99}.\\

Although a number of mechanisms underlying anchoring transitions are in principle known
(see~\cite{Sluckin95} for a review) very few theoretical analyses have been performed. 
Teixeira and Sluckin~\cite{TeixeiraSluckin92,TeixeiraSluckin92a} used a Landau-de Gennes free
energy functional to study the planar to homeotropic anchoring transition in systems of liquid
crystals confined by different substrates. They found a rich anchoring behaviour, despite 
a number of simplifications that had to be made, which helped in the identification of the
possible mechanisms responsible for anchoring transition; in this case, the compositions
of binary mixtures of liquid crystals and the amount of adsorption at the surface. Subsequently 
Teixeira~\etal\cite{TeixeiraSluckin93} used a Landau-de Gennes formalism to observe
a temperature driven anchoring transition at the interface between a liquid crystal and smooth 
solid surface, thus confirming the experimental findings.\\
The effect of non-uniform substrates (\ie microtextured) has been studied by Qian and
Sheng~\cite{ZhengQianSheng96,ZhengQianSheng97} using a Landau-de Gennes formalism. They show
that the effect of the substrate is to induce temperature dependent tilt angles separated by
phase transitions.\\
%
The literature on computer simulations of anchoring transition is extremely scarce. 
Cleaver and Teixeira~\cite{Cleaver_Teixeira_01} have studied systems of hard Gaussian
overlap particles confined in a slab geometry and interacting with smooth substrates via the
hard needle wall potential. There, adsorption phenomena induced an homeotropic to planar
anchoring transition as the surface potential parameterisation was changed. More details into
similar systems can be found in Chapter~\ref{chap:four}.
Another simulation study of an anchoring transition was that of
Lange and Schmid~\cite{LangeSchmid02,LangeSchmid02a,LangeSchmid02c} who 
observed an anchoring transition between tilted and homeotropic arrangements
in a system of ellipsoidal Gay-Berne confined by grafted polymer chains (made of
Gay-Berne ellipsoids) as the grafting density was changed.




