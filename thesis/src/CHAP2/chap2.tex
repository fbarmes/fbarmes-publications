
\chapter{Computer simulations of liquid crystals}
\label{chap:two}

\introduction

Computer simulations act as a link between theory and experiment. Experimental analysis is
limited in that, usually,  only bulk averaged properties are measurable meanwhile, the complicated
nature of mesogens and their interaction potentials prevents the possibility of a full theoretical 
treatment capable of  predicting the phase behaviour of real mesogens. These
limitations can be resolved using computer simulations, where, by using a model interaction
in an appropriately designed `computer experiment', full molecular insight into the system can be
gained. Using the results of statistical mechanics, macroscopic properties can be computed 
which, in turn, can be used to test the theories from which the model originated.
The validity of the model can, thus, be checked against experimental results. Computer simulations can
also act as predictive tools in the design of novel compounds or for the study of systems under
conditions that can only be attained in the laboratory with great difficulty and cost.\\
The first part of this Chapter contains a description of the two main methods used in the simulations 
of liquid crystals at a
molecular scale, namely the Monte Carlo method and Molecular dynamics. Following this, previous
work on the computer simulation of liquid crystals is reviewed. Here attention is concentrated on
generic hard particles and Gay-Berne models, and their extension to treat flexoelectric
behaviour.
The Chapter then concludes with a review of progress on the simulation of confined liquid 
crystalline systems and their anchoring properties.


%==============================================================================
%==============================================================================


\section{Molecular modeling techniques.}


%===============================================================================================
%===============================================================================================
\subsection{The Monte Carlo Method}

The Monte Carlo (MC) method is used to find the solution of mathematical problems 
using a probabilistic approach and is currently applied to a wide variety of different problems.
According to Hammersley and Handscomb, `` Monte Carlo methods comprise of that branch of
experimental mathematics which is concerned with experiments on random numbers''~\cite{MCMethods}.
In the molecular physics of liquid crystals, the term Monte Carlo usually refers to the 
specific sampling method as proposed by Metropolis \etal\cite{MR2T2}.

\subsubsection{The Metropolis solution.}

Considering a system of $N$ particles in the canonical (constant \NVT) ensemble and assuming pairwise
interactions between the particles~; the total potential $\mathcal{V}$ is given
by~\cite{greenBook}~:
\begin{equation}
	\mathcal{V} = \sum_{i\neq j}^{N} U(\vect{X}(i), \vect{X}(j) ) \hspace*{5mm} i,j\in[1\ldots N]
\end{equation}
where $\vect{X}$ is the complete set of positions, orientations and momenta of the system.

In this canonical ensemble, a time independent configurational property can be 
obtained from~:
\begin{equation}
	\langle\mathcal{A}\rangle_{\mrm{real}} = \int \rho_{\mrm{NVT}}(\vect{X})\mathcal{A}(\vect{X})
		d\vect{X}^N
\end{equation}
with~:
\begin{eqnarray}
	\rho_{\mrm{NVT}} &=& \frac{e^{-\beta \mathcal{V}}}{Q_\mrm{NVT}}
\end{eqnarray}
%
$Q_\mrm{NVT}$, being the partition function for the canonical ensemble.
If the system is ergotic, $\langle\mathcal{A}\rangle_\mrm{real}$ can be obtained by averaging its
instantaneous values over a sufficient number of uncorrelated state points, 
$\vect{\Gamma}_i$, provided that they appear with a probability proportional to the probability density 
of the considered ensemble. Therefore~:
\begin{equation}
	<\mathcal{A}>_{\mrm{real}} = \frac{1}{M}\sum_{i=1}^{M} \mathcal{A}_i 
\end{equation}

Metropolis \etal designed a stochastic process for creating such a sequence of state points in
the canonical ensemble where each configuration $\vect{\Gamma}_i$ appears with a probability
$e^{-\beta\mathcal{V}}$. This sequence corresponds to a discrete Markov chain, that is a stochastic
sequence of states within each step of which memory extends only to the preceeding state~\cite{AandT}.\\

The theory of discrete Markov chains~\cite{theorySimpleLiquids} shows that the probability,
$\rho_b$, that the system evolves from state $a$ to state $b$ is given by~:
\begin{equation}
	\rho_b = \rho_a \pi_{ab},
\end{equation}
where $\pi_{ab}$ is the transition matrix. Intuitively the properties of $\pi_{ab}$ read~:
\begin{eqnarray}
	\pi_{ab} &\geq& 0\\
	\sum_{b}\pi_{ab} &=& 1.
	\label{eqn:MCcdtion1}
\end{eqnarray}
%
Also the condition of microscopic reversibility requires that the probability of going from state
$b$ to $a$ is equal to that of the reverse transition and, therefore~:
\begin{equation}
	\rho_a\pi_{ab} = \rho_b\pi_{ba}.
	\label{eqn:MCcdtion2}
\end{equation}

The transition matrix for the system under consideration is not directly available, however the 
limiting distribution $\rho_\infty$ is known to be the probability density of the canonical ensemble
($\rho_{\mrm{NVT}}$), that is~:
\begin{eqnarray}
	\rho_{\infty} &=& \rho_{\mrm{NVT}}(\vect{\Gamma}_\infty)\\
		 &=& \frac{1}{Q_{\mrm{NVT}}}e^{-\beta\mathcal{V}(\vect{\Gamma})}.
\end{eqnarray}

The scheme introduced by Metropolis \etal allows the construction of an appropriate phase space
trajectory which obeys Equations~\ref{eqn:MCcdtion1} and~\ref{eqn:MCcdtion2}.\\

In its most basic form, the Metropolis method
considers a two dimensional system of $N$ atoms (with
straightforward extension to three dimensions.) Phase space is 
sampled by choosing one particle $i$ at random and assigning it a new random position within a
square of arbitrary size centered on the particle's old position. The move is accepted if it is
downhill in energy $(\delta\mathcal{V}_{ab} = \mathcal{V}_b - \mathcal{V}_a \leq 0)$. If,
however, the move is uphill in energy $(\delta\mathcal{V}_{ab} > 0)$, then the move is accepted 
with a probability $e^{-\beta\delta\mathcal{V}_{ab}}$. This is performed by generating a
random number $\xi \in [0:1]$. The move is accepted if $\xi \leq 
e^{-\beta\delta\mathcal{V}_{ab}}$ and rejected otherwise. This sequence of particle
choosing-moving is then repeated until a sufficient number of uncorrelated moves are achieved.\\

The method can be extended to three dimensional systems of non spherical and even non rigid
molecules where new states are created by changing the positions and the orientations of the
molecules~\cite{AandT}. A summary of the Monte Carlo algorithm in the canonical ensemble is
given on Figure~\ref{fig:MCalgo}.

\begin{figure}
\centering
\Ovalbox{
\begin{minipage}{10cm}
	\vspace*{3mm}
	\begin{enumerate}
	\item Choose a particle $i$ at random
	\item Assign a new random position and orientation 
	\item \textbf{if} ($\delta\mathcal{V}_{ab} \leq 0$) \textbf{OR} ($\xi \leq e^{-\beta\delta\mathcal{V}_{ab}}$)\\
		\hspace*{20mm} Accept move\\
	\textbf{else}\\
		\hspace*{20mm} Reject move
	\item Store instantaneous observable.
	\item Return to step 1 until $n_{\mrm{step}}$ performed.
	\item Compute observable average .
	\end{enumerate}
	\vspace*{3mm}
\end{minipage}}
\caption{The Monte Carlo algorithm in the canonical ensemble}
\label{fig:MCalgo}
\end{figure}

According to the nature of the system studied, other moves have been
used such as reptation moves~\cite{WilliamsonJackson98} or flip moves~\cite{BerardiRicci01}.
Similarly Monte Carlo simulation of flexible molecules can be achieved by moving sub-molecular
segments independently. As a result for more advanced systems, one Monte Carlo move is composed
of several different type of molecular moves (change in orientation, position, flip, reptation
\etc). If at any Monte Carlo step, different combination of molecular moves are performed, the
use of a given move type should be probabilistic so as to keep the Markov chain 
stochastic~\cite{AandT}.\\


\subsubsection{Extension to the isothermal-isobaric ensemble.}

The Metropolis solution was first applied in the canonical ensemble, but it can
readily be extended to other ensembles such as the isothermal-isobaric as proposed by
Wood~\cite{Wood68}. The extension of the MC method into another ensemble requires knowledge of
its probability density $\rho_{\mrm{ens}}$. For the isothermal-isobaric 
ensemble, this is~:
\begin{eqnarray}
	\rho_{\mrm{NPT}} = \frac{1}{Q_{\mrm{NPT}}}e^{-\beta(\mathcal{V}(\Gamma) +PV )}
\end{eqnarray}

where $P$ represents the pressure and $V$ the volume.
The implementation of the MC method in this alternative ensemble requires the generation of a Markov
chain with a state probability proportional to $e^{-\beta(\mathcal{V}(\Gamma) +PV )}$.
This is achieved using a similar algorithm to that for the canonical
ensemble, the difference being that volume changes are performed in order to keep the pressure
constant. Because of the computational
overhead associated with volume changes, they are typically attempted with a frequency of 
once every $n$ sweeps (\ie $n$ attempted move per particle) where $n$ is typically 
$\in [1:10]$. Volume changes are assessed by testing the variation in enthalpy 
$\delta H$~\cite{AandT} given by~:
\begin{equation}
	\delta H_{ab} = \delta\mathcal{V}_{ab} + P(V_b - V_a)-\frac{N}{\beta}\ln\lp\frac{V_b}{V_a}\rp.
\end{equation}
%
A given volume change move is accepted if $\delta H_{ab} \leq 0$ or 
$\xi \leq e^{-\beta \delta H_{ab}}$ and rejected otherwise.\\

Several methods can be used to generate the volume changes. One of these involves generating a random
change in volume ($\delta V$,) computing the corresponding changes in box lengths and
rescaling the particle coordinates accordingly. However this imposes the constraint that
the simulation box remain cubic and involves changing, simultaneously, the lengths of all three
box sides. Another scheme is to change every box length independently
by choosing a box dimension randomly and assigning it a new length using a random variation. 
This method allows the box shape to change and, if necessary, adapt to the nature of the phase of 
the system under study. 


%===============================================================================================
%===============================================================================================
\subsection{Generating random orientations}


The generation of new random orientations is not a trivial exercise as can be the generation of
new random positions. Here two methods for the generation of random orientations are described,
namely the Barker-Watts method and the so called Local Frame method.

\subsubsection{The problem}

The aim is to generate a new orientation $\un$ given an initial orientation
$\uo$  so that the distribution of possible trial orientations is uniform in a 
portion of the unit sphere delimited by a chosen boundary.\\
The orientation vectors $\vecth{u}$ are defined according to $\theta$ and $\phi$,
respectively, the zenithal and azimuthal Euler angles~:

\begin{equation}
	\vecth{u} =
	\lp
	\begin{array}{c}
		u_x\\	u_y\\	u_z
	\end{array}
	\rp
	=
	\lp
	\begin{array}{c}
		\cos\phi\sin\theta	\\
        	\sin\phi\sin\theta	\\
		\cos\theta
	\end{array}
	\rp
\end{equation}
%
with~:
%
\begin{eqnarray*}
	\theta 	&\in& 	[0:\pi]	\\
        \phi 	&\in&	[-\pi:\pi]
\end{eqnarray*}

The generation of $\un$ is performed so that~:
\begin{eqnarray}
	\theta_{n} 	&=& \theta_o + \delta\theta	\\
        \phi_{n}	&=& \phi_o + \delta\phi
\end{eqnarray}
Where $\delta\theta$ and $\delta\phi$ are random angular displacement defined by the limiting 
conditions~:
\begin{eqnarray*}
	\delta\theta 	&\in& 	[0:\theta_\mrm{max}]	\\
        \delta\phi 		&\in&	[-\pi:\pi]
\end{eqnarray*}

It can be shown that the direct generation of $\delta\theta$ as~:
\begin{equation*}
	\theta_n = \theta_o + (2\xi_\theta -1)\delta\theta_\mrm{max}	
\end{equation*}
leads to non-uniform distribution of $\delta\theta$, in conflict with the
MC move acceptance criterion~\cite{AandT}. Ra\-ther,
random $\cos\theta$ should be generated as~:
\begin{equation*}
	\cos\theta_n = \cos\theta_o + (2\xi_\theta -1)\delta(\cos\theta_\mrm{max})
\end{equation*}

%==================================================================================================
%==================================================================================================
\subsubsection{The Barker Watts method}

The so called Barker Watts method~\cite{BarkerWatts69} has been proposed as a fast method for
generating random orientation. In Monte Carlo codes, orientation vectors are best represented
by unit vectors rather than by explicitely stating the Euler angles. The Barker-Watts method 
allows the generation of
random orientations without the computational overhead associated with the use of 
trigonometric functions. With this method a new orientation is generated as~:
\begin{equation}
	\un = \vect{A}_{\alpha}\uo
\end{equation}
where $\vect{A}_{\alpha}$ is one of the rotation matrices $\vect{A}_x,\vect{A}_y,\vect{A}_z$,
chosen at random~:
\begin{eqnarray}
	\vect{A}_x &=& \lp
	\begin{array}{ccc}
	1		&0		&0		\\
	0		&\cos\theta^R	&\sin\theta^R	\\
	0		&-\sin\theta^R	&\cos\theta^R
	\end{array}
	\rp
	\\
%
	\vect{A}_y &=& \lp
	\begin{array}{ccc}
	\cos\theta^R	&0		&-\sin\theta^R	\\
	0		&1		&0		\\
	\sin\theta^R	&0		&\cos\theta^R
	\end{array}
	\rp	
	\\
%
	\vect{A}_z &=& \lp
	\begin{array}{ccc}
	\cos\theta^R	&\sin\theta^R	&0	\\
	-\sin\theta^R	&\cos\theta^R	&0	\\
	0		&0		&1
	\end{array}
	\rp
\end{eqnarray}
and $\theta^R$ is a random angle so that $\theta^R \in [0:\theta_\mrm{max}]$.

This method has the advantage of being very fast, but also presents some possible
drawbacks. If $\delta\theta_\mrm{max} = \pi$, the generation of $2.10^6$ $\un$ with
$\uo=\vecth{z}$ shows that only a small portion of the unit sphere is available (see
Figure~\ref{fig:BWdistro}(a).) This
does not usually prevent good phase space sampling in the Monte Carlo sequence as the particles
follow Brownian motion and, thus, $\uo$ is not constant. 
Therefore if $\un$ at step $t+1$ is created using $\un$ from step $t$, the full unit 
sphere is available (see Figure~\ref{fig:BWdistro}(b).) This behaviour can, however, raise some 
problems in simulations with  very low acceptance rates or where the director is aligned to a
fixed direction by, \eg, a surface interaction or applied field.

\picW = 7cm
\begin{figure}
	\centering
	\subfigure[$\uo^{(t)} = (0,0,1)$]{\picL{distroBWNoCarry.ps}}
	\subfigure[$\uo^{(t)} = \un^{(t-1)}$]{\picL{distroBWCarry.ps}}
	\caption{Distribution of the Euler angles for generated random configuration using the
	Barker-Watts method. The Figure on the left corresponds to a generation with constant 
	$\uo$ and the Figure on the right to a distribution using changing input orientations.}
	\label{fig:BWdistro}
\end{figure}

%==================================================================================================
%==================================================================================================
\subsubsection{The Local Frame method}

The Local Frame method has been designed in order to provide a method which samples the full
unit sphere at every step if $\delta\theta_{max} =\pi$. The principle here is to generate a random
orientation $\unfp$ in the molecular frame $f^{\prime}$ which is transformed into $\unf$ in the
laboratory frame $f$ using an appropriate rotation matrix.\\
$\unfp$ is given by~:
\begin{equation}
	\unfp = \lp
	\begin{array}{ccc}
		\cos\phi_R\sin\theta_R	\\
		\sin\phi_R\sin\theta_R	\\
		\cos\theta_R
	\end{array}\rp
\end{equation}
with~:
\begin{eqnarray}
	\cos\theta_R 	&=& 1-\xi_\theta( 1-(\cos\theta_\mrm{max} ))	\\
	\phi_R		&=& (2\xi_\phi -1)\pi
\end{eqnarray}
%
and where $\xi_\theta$ and $\xi_\phi$ are random numbers in $[0:1]$. The transformation of 
$\unfp$ into $\unf$ is given by~:
\begin{equation}
	\unf = \vect{R}^{-1}_T\unfp
\end{equation}
%
where $\vect{R}_T$ is the rotation matrix that transforms a vector $\vecth{u}$ from
$\vect{u}=(\cos\phi\sin\theta,\- \sin\phi\sin\theta,\- \cos\theta)$ in $f$
to $\vecth{u} = (0, 0, 1)$ in $f^\prime$. In a right handed coordinate system,
this transformation is a 
rotation of $\phi$ about $\vecth{z}$ followed by a rotation of $\theta$ about $\vecth{y}$ thus~:
%
\begin{gather}
	\vect{R}_T = \vect{R_{\theta,y}}\vect{R_{\phi,z}}	\\
	\vect{R}_T = \lp
		\begin{array}{ccc}
		\cos\theta	&0	&-\sin\theta	\\
                0		&1	&0		\\
                \sin\theta	&0	&\cos\theta
		\end{array}\rp
	.\lp
		\begin{array}{ccc}
		\cos\phi	&\sin\phi	&0	\\
                -\sin\phi	&\cos\phi	&0	\\
                0		&0		&1
                \end{array}\rp	\\
        %
        \vect{R}_T = \lp
		\begin{array}{ccc}
			\cos\theta\cos\phi	&\cos\theta\sin\phi	&-\sin\theta	\\
                        -\sin\phi		&\cos\phi		&0		\\
                        \sin\theta\cos\phi	&\sin\theta\sin\phi	&\cos\theta
                \end{array}
        \rp
\end{gather}

and therefore~:
\begin{equation}
\vect{R}^{-1}_T = \lp
\begin{array}{ccc}
\cos\phi\cos\theta	&-\sin\phi	&\cos\phi\sin\theta	\\
\sin\phi\cos\theta	&\cos\phi	&\sin\phi\sin\theta		\\
-\sin\theta		&0		&\cos\theta
\end{array}
\rp
\end{equation}

\begin{figure}
	\centering
	\subfigure[$\uo^{(t)} = (0,0,1)$]{\picL{distroLFNoCarry.ps}}
	\subfigure[$\uo^{(t)} = \un^{(t-1)}$]{\picL{distroLFCarry.ps}}
	\caption{Distribution of the Euler angles for generated random configuration using the
	Local-Frame method. The Figure on the left corresponds to a generation with constant 
	$\uo$ and the Figure on the right to a distribution using changing input orientations.}
	\label{fig:LFdistro}
\end{figure}


If $\theta_\mrm{max} =\pi$, this method allows one to sample from the full unit sphere at every random
generation (see Figure~\ref{fig:LFdistro})~; however it presents the drawback of being slower
than the Barker-Watts method. As a
result the latter is still preferentially used for Monte Carlo simulations with s sufficiently
high acceptance rate. The Local Frame method, on the other hand, shows its strength 
in the generation of random initial configurations or for simulations where acceptable phase 
space sampling requires the use of simulation parameters inducing a low acceptance rate.


%===============================================================================================
%===============================================================================================
\subsection{Molecular Dynamics}

The Molecular Dynamics (MD) method was introduced by Alder and Wainwright in
1959~\cite{AlderWainwright59} in an attempt to describe the time evolution of fluids at a
molecular level~; at the time only the Monte Carlo method was
available for the task. The technique involves solving the simultaneous Newtonian
(spherical molecules) or the combined Newtonian-Euler (non-spherical molecules) equations of
motion for all particles in the system over a finite (and usually short) time. The method is
based on the statistical mechanics result that an ensemble average of a given property
$\mathcal{A}_\mathrm{real}$ of an ergotic system can be obtained from the time average of 
its instantaneous values as~:
%
\begin{eqnarray}
        \mathcal{A}_\mathrm{real} &=& \left<  \mathcal{A}\lp\vect{X}(t)\rp \right>_\mrm{time} \nonumber \\
                                &=& \frac{1}{t_{obs}}\int^{t_{obs}}_{0}
\mathcal{A}\lp\vect{X}(t)\rp dt \end{eqnarray}

where $\vect{X}(t)$ describes the set of positional and orientational coordinates of the $N$
particles system at a time $t$. The general algorithm of the Molecular Dynamics method is 
described on Figure~\ref{fig:MDalgo}.\\
%

\begin{figure}
\centering
\Ovalbox{
\begin{minipage}{10cm}
	\vspace*{3mm}
	\begin{enumerate}
	\item Create an initial configuration.
	\item Calculate the forces on each particle.
	\item Update the particles positions and velocities to the current time step.
	\item Calculate the instantaneous properties.
	\item Return to step 2 until $n$ time steps performed.
	\item Compute the time average properties.
	\end{enumerate}
	\vspace*{3mm}
\end{minipage}}
\caption{The Molecular Dynamics algorithm.}
\label{fig:MDalgo}
\end{figure}

With this process, the method used to calculate the force field is of critical importance as it
governs the equilibrium behaviour of the model. Solving the equations of motion is also the
most time consuming part of the whole algorithm. Several algorithms have been developed for the
optimization of this task, the most common of which are the original Verlet 
algorithm~\cite{verletForce}, the so called `leap-frog' algorithm~\cite{leapFrog} and the 
velocity-Verlet algorithm~\cite{velocityVerlet}. The success of these lies in their time 
efficiency and ease of implementation.\\
The advantage of the MD technique is that it allows the study of transport properties of particles
and, therefore, renders possible the study of relaxation phenomena, which cannot be addressed
using the Monte Carlo method.\\
One difficulty that arises with Molecular Dynamics is that when faced with systems having both short 
and long time scale oscillations, the time step employed must be small enough 
to capture the high frequency behaviour~; nevertheless, the total run length is required to be
sufficient to allow the system to exhibit long time
scales phenomena. This problem is very common in the study of polymers, and biological systems.\\
Also it should be noted that Molecular Dynamics requires the use of a differentiable model, as
the forces and torques are derived from gradient of the intermolecular
potential~\cite{AandT}. This said, discontinuous potentials, such as 
the square well potential~\cite{AlderWainwright59,AlderWainwright60} have been used in MD
simulations. In these case, the
whole algorithm needs to be re-cast so as to consider binary collisions rather than fixed
time steps. With more complicated non-differentiable models, however, the MD method cannot 
be used.




\section{Molecular models of Liquid Crystals}

At a molecular scale, a first approximation of the shape of liquid crystals molecules 
would be a rod or a disc for, respectively, calamitic and discotic mesogens. 
Using such objects, interacting through appropriate intermolecular
potentials, provides an excellent testbed for theories on the origins of liquid crystals
phases. Two types of particles are commonly used. Hard particles models have their basis in 
Onsager theory, which shows that short range repulsive interactions alone can lead to the formation
of liquid crystal phases such as the nematic phase. A refinement of these models is to
incorporate the attractive forces which are responsible for the more sophisticated mesophases.
Section~\ref{ss:HP} discuss the first type of models while the effects of attractive forces
for the most popular soft model, namely the Gay-Berne model, are presented in Section~\ref{ss:GB}.

%==================================================================================================
%==================================================================================================
\subsection{Hard particle models.}
\label{ss:HP}

%=============================
%=============================
\subsubsection{Hard ellipsoids of revolution}

The structure of simple atomic fluids can be described efficiently and surprisingly accurately 
using hard sphere models. 
The most obvious extension of this result that incorporate the anisotropy of mesogens is to
represent the particles by hard ellipsoids of revolution (HER) with semi axes $a=b\neq c$. For
this model, the length to width ratio $k$ is defined by $k=\frac{a}{c}$. Also both discotic
and calamitic particles can be modeled according to the value of $k$.
Although spherocylinders were used in Onsager's treatment, the HER can be expected to follow the
Onsager solution as it can be shown that for both shapes, the expression for the excluded volume 
of a pair of particles as a function of their relative orientation $\alpha$ takes the form
$V_\mrm{excl}\sim\sin\alpha$ $\alpha$~\cite{FrenkelMulder85}.\\
%
The main drawback of the HER model is that the contact distance between two ellipsoids can not be
expressed in a closed form. 
The first Monte Carlo simulation of this model was performed on a two dimensional system by 
Vieillard-Baron in the early 1970's~\cite{VieillardBaron72}. This work gave the first algorithm for the
contact distance between two ellipses and showed that excluded volume effects
play an important role in mesophase formation. However, limitations in computer power
prevented Vieillard-Baron from establishing the behaviour of three dimensional systems. The contact 
distance computation algorithm was later refined by Perram~\etal\cite{PerramWertheim84,PerramWertheim85} 
and a tentative phase diagram for three
dimensional systems was proposed by Frenkel~\etal\cite{FrenkelMulder81}. This showed
the existence of four different phases namely isotropic, nematic, plastic crystal and ordered
crystal. A more thorough investigation by Frenkel and Mulder~\cite{FrenkelMulder85} established the
range of stability of those phases; more specifically nematic phases were found for $k \geq
2.75$, the transition density reducing with increased molecular elongation.
Comparison of those results with the y-expansion
density functional theory where the free energy expansion was cut at the third 
term~\cite{MulderFrenkel85} showed good agreement for $\frac{1}{3} \leq k \leq 3$; a worsening 
of the theoretical predictions was found with more extreme $k$ values, however.\\
Despite the argument from Zarragoicoecha~\etal\cite{Zarragoicoecha92} that the stability of 
the nematic phase for $k=3$ is an artifact of the limited system size employed by Frenkel \etal
simulations, subsequent work
by Allen and Mason~\cite{AllenMason86} for several system sizes confirmed the validity
of the phase diagram proposed by Frenkel~\etal
An extension of this phase diagram was later proposed by Camp~\etal\cite{CampMason96} who
computed of the exact location of the isotropic-nematic phase transition using Gibbs-Duhem
integration techniques.\\

Studies by Allen~\etal~\cite{Allen90,CampAllen97} of a biaxial version of the hard ellipsoid
model ($a \neq b \neq c$) for $\frac{c}{a} = 10$ and $1 \leq \frac{b}{a} \leq 10$ showed the
existence of isotropic, nematic, discotic nematic and biaxial phases. Again, the exact location of the
isotropic-nematic and isotropic-discotic nematic phase transitions were computed using
Gibbs-Duhem integration methods.

The extensive studies performed on this model have established a comprehensive and very precise 
phase diagram as a function of molecular elongation. The range of stability of the different
phases can be summarized as~:
%
\begin{itemize}
	\item plastic solid~: $k \in [\frac{1}{1.25}:1.25]$
	\item ordered solid~: $k \in [\frac{1}{2}:2]$
	\item discotic nematic~: $k \leq \frac{1}{2.75}$
	\item nematic~: $k \geq \frac{1}{2.75}$
\end{itemize}

The major conclusion that can be drawn from these results is that Onsager's solution applied
at the intermediate values $k\sim 3-5$ corresponding to the elongation of common mesogens. 

%=============================
%=============================
\subsubsection{Hard spherocylinders.}

Another very popular model for calamitic liquid crystals is the hard spherocylinder (HSC). This is the
model used by Onsager, that is a cylinder of length $L$ and diameter $D$ terminated by two 
hemispherical caps of diameter $D$. The elongation $k$ of such an object is therefore
$k=\frac{L+D}{D}$. The reasons for the use of this model are threefold. First it is the shape
used in Onsager's theory and therefore the most obvious choice to make when testing this theory 
using computer simulations. Also for computational purposes, the model presents the advantage of 
having a tractable expression 
for the contact distance. Finally the spherocylinder shape resembles closely that of the 
liquid-crystal-phase-forming tobacco virus~\cite{ZazadzinskiMeyer86,DogicFraden97}. There is,
however, no obvious extension of this model to discotic particles~; equivalence could be
considered using cut spheres~\cite{VeermanFrenkel92} or sort cylindrical 
segments~\cite{BatesFrenkel98}\\

The first computer simulation on hard spherocylinders was performed by 
Vieillard-Baron~\cite{VieillardBaron74} using elongations $k=2$ and $3$, but this study did not find
any stable nematic phases. Limitations in computer power prevented the author from fully
investigating a system with $k=6$. In 1987, Frenkel~\cite{Frenkel87,Frenkel87a} performed the
first simulations of three dimensional systems of hard spherocylinders with
elongation $k \in [0:5]$ and free translation and rotation; the authors found the model to 
exhibit three phases, namely~: isotropic liquid, nematic and
\smA. With the shorter elongation $k=3$, Veerman and Frenkel~\cite{VeermanFrenkel96} 
found the nematic and smectic phases to become respectively unstable and metastable. A more
complete phase diagram was later proposed by Mc Grother~\etal\cite{McGrotherWilliamson96} for
different values of $k$ in the range $[3:5]$ and confirmed the results from Frenkel~\etal 
The authors also showed that as $k$ was increased, the \smA~ phase was stable for
$\frac{L}{D}\geq 3.2$ whereas the nematic phase required $\frac{L}{D}\sim 4$. 
Bolhuis and Frenkel~\cite{BolhuisFrenkel97}
later refined and completed this phase diagram into the Onsager limit.\\
From these studies the phase diagram of the HSC can be summarized as~:
\begin{itemize}
	\item Nematic~: $\frac{L}{D} \geq 3.7$.
	\item \SmA~: $\frac{L}{D} \geq 3.1$.
\end{itemize}

%=============================
%=============================
%\subsubsection{Other models.}

%==================================================================================================
%==================================================================================================
\subsection{The Gay-Berne model.}
\label{ss:GB}


Despite their simplicity and their success in modeling liquid crystalline phases, the steric
models discussed above have limitations in that they can not be used to model
every liquid crystalline phase (\eg \smB) and the effect of attractive interactions can
not be studied.\\
The first mathematically tractable model for soft particles was developed in 1972 by Berne and 
Pechukas~\cite{BernePechukas72}. This model describes the interaction between soft ellipsoidal
particles through a potential $\mathcal{V}^{BP}$ which is alternately repulsive and
attractive at short and long ranges. This potential however presented  unrealistic
features such as having equal well depths for end to end and side by side parallel molecular
arrangements.
These deficiencies were resolved by Gay and Berne~\cite{GayBerne81} who modified the functional
form of the Berne-Pechukas potential so that it could give a reasonable fit to a 
linear arrangement of four 
Lennard-Jones  sites. This resulted in the now widely used Gay-Berne model 
$\mathcal{V}^{GB}$ expressed as~:
\begin{eqnarray}
	\mathcal{V}^{GB} &=& 4\epsilon(\ui,\uj,\rij)\left\{ R^{12} - R^{6} \right\} \\
	\mrm{with\ }R &=&  \frac{\so}{r - \sijr +\so} \nonumber.
\end{eqnarray}

Where $\sijr$ is the shape function for two Gaussian ellipsoids, as determined
originally by Berne and Pechukas~\cite{BernePechukas72},
%
\begin{equation}
	\sijr = \so\left\{
	1 - \frac{1}{2}\chi\left[ 
	\frac{ \lp\dotProd{\rij}{\ui} + \dotProd{\rij}{\uj}\rp^2 }{1 + \chi(\dotProd{\ui}{\uj})}
      + \frac{ \lp\dotProd{\rij}{\ui} - \dotProd{\rij}{\uj}\rp^2 }{1 - \chi(\dotProd{\ui}{\uj})}
	\right] \right\}^{-\frac{1}{2}}
\end{equation}

$\so$ defines the unit of distance and $\chi$ is the shape anisotropy parameter defined using $k$ the
length to breadth ratio as~:
\begin{equation}
	\chi = \frac{k^2-1}{k^2+1}.
\end{equation}

The energy strength parameter is defined as~:
\begin{equation}
	\eijr = \eo\eOne^{\nu}(\ui,\uj)\eTwo^{\mu}(\uj,\uj,\rij)
\end{equation}

with~:
\begin{equation}
	\eOne(\ui,\uj) = \left[1 - \chi^2(\dotProd{\ui}{\uj})^2\right]^{-\frac{1}{2}}
\end{equation}

and
\begin{equation}
	\eTwo(\ui,\uj,\rij) = 
	1 - \frac{1}{2}\chi^\prime \left[ 
	\frac{ \lp\dotProd{\rij}{\ui} + \dotProd{\rij}{\uj}\rp^2 }{1 + \chi^\prime(\dotProd{\ui}{\uj})}
      + \frac{ \lp\dotProd{\rij}{\ui} - \dotProd{\rij}{\uj}\rp^2 }{1 - \chi^\prime(\dotProd{\ui}{\uj})}
	\right].
\end{equation}

Here $\chi^\prime$ is the energy anisotropy parameter defined using $k^\prime$ the ratio of end to
end and side by side well depth ($\epsilon_{ee}$ and $\epsilon_{ss}$ respectively)~:
\begin{eqnarray}
	\chi^{\prime} &=& \frac{k^{\prime \mu^{-1}} - 1}{k^{\prime \mu^{-1}} + 1}	\\
	k^\prime &=& \frac{\epsilon_{ee}}{\epsilon_{ss}}	\nonumber
\end{eqnarray}

The Gay-Berne model behaviour can be easily tuned through modification of the four parameters
$k,k^\prime,\mu$ and $\nu$.\\
Preliminary simulation results by Adam~\etal\cite{AdamLuckhurst87}, using the parameterisation 
$GB(k,k^\prime,\nu,\mu) = GB(3,5,1,2)$, showed the Gay-Berne model
to be suitable for liquid crystal modeling as both isotropic and nematic phases were observed.
Subsequent work by Luckhurst and Stephens~\cite{LuckhurstStephens90} using the slightly
different parameterisation $GB(3,5,2,1)$ found a much richer phase diagram containing isotropic,
nematic, \smA, \smB and crystal phases. Thanks to a very thorough study by the 
Seville group~\cite{ChalamGubbins91,deMiguelRull91,deMiguelRull91a},
the full liquid crystalline phase diagram of the model was then determined for
$GB(3,5,1,2)$.\\
%
The parameterisation $GB(3,5,3,1)$ was used by other groups~\cite{BerardiEmerson93,
AllenWarren96}; while this gives the same isotropic, nematic and smectic phases as the
previous parameterisation, the increased value of $\mu$ allows for a wider nematic region.\\
%
The substantially different parameterisation $GB(4.4, 39.6^{-1}, 0.74, 0.8)$ was used by Luckhurst
and Simmonds~\cite{LuckhurstSimmonds93} in an attempt to use a model better fitted to
representing
real mesogens. This parameterisation was obtained from fitting the Gay-Berne potential to a uniaxial
version of a realistic potential of the p-terphenyl molecule. The authors found a phase behaviour
compatible with their aim; the model displayed isotropic, nematic and smectic phases.
Subsequently, Bates and Luckhurst~\cite{BatesLuckhurst99} performed a thorough study using 
the parameterisation  $GB(4.4,20^{-1},1,1)$ and found the model to exhibit isotropic, nematic, \smA
and \smB phases in good agreement with the behaviour of the real mesogens this parameterisation was
modeling.
%
An investigation into the generic effects of the attractive part of the 
potential~\cite{deMiguelDelRio96} showed that
smectic order is favoured as $k^\prime$ is increased, thus showing the importance of attractive
forces for the formation of smectic phases by ellipsoidal particles. A similar study into the
effects of molecular elongation on the Gay-Berne phase diagram~\cite{BrownAllen98} showed significant
changes notably in the location of the isotropic-nematic phase transition.\\

More recently, various extensions of the Gay-Berne model have been performed. 
Cleaver \etal\cite{CleaverCare96} generalised the potential to give an interaction for particles 
with different elongations thus opening up the possibility of modeling LC mixtures.
Zewdie~\cite{Zewdie98a,Zewdie98b} developed a Corner-like potential where the range and
energy strength parameters are expanded in terms of a complete orthogonal basis set, namely
Stone~\cite{Stone} functions. The advantage of this method is that each term in the energy
strength parameter can be associated with a given type of interaction thus allowing fine tuning
of the model. Application of this approach to the modeling of discotic particles~\cite{Zewdie98a} 
lead to the
formation of isotropic, discotic-nematic, columnar and crystal phases while the modeling of
calamitic particles (using an equivalent of $GB(3,5,1,2)$) showed isotropic, nematic, \smA, \smB and
crystal phases in agreement with~\cite{AdamLuckhurst87}.\\
%
Biaxial particles have been studied by the Bologna 
group~\cite{BerardiZannoni00,Zannoni02} using a generalization of the Gay-Berne
potential to the interaction between two arbitrary ellipsoidal particles presented
in~\cite{BerardiFava98}. These studies have indicated that the answer to the much argued about 
question
of the existence of biaxial phases is that, such phases can exist since both biaxial nematic and
smectic phases have been observed.\\
%
The inclusion of dipole electrostatic moments into Gay-Berne particles has been studied by
Houssa~\etal~\cite{HoussaRull98, HoussaMcGrother99} and revealed that although the location of
the phase transitions of the model are insensitive to the strength of the dipoles, the
electrostatic forces were found to have a considerable effect on the nature of the observed
phases.
multipole electrostatic moments has also been considered. Specifically, 
Berardi \etal~\cite{BerardiOrlandi03} included two outboard permanent dipoles at various angles
from the axis and observed tilted smectic phases whose intra-layer arrangement was compatible
with \smJ and tetragonal \smT. The inclusion of quadrupole moments into a $GB(4,5,2,1)$
system~\cite{Withers} proved to have strong effects upon the smectic regions of the phase
diagrams such as the replacement of a \smB by a \smJ phase.
%
These developments, along with the
extended possibility of parameterisation, makes, the Gay-Berne model a very versatile one which can
be applied to a wide number of different types of interactions. 
%A number of different applications is reviewed in~\cite{Zannoni02}.






\section{Modeling of flexoelectric particles}


Flexoelectricity is an important property to be considered in the design of materials for use in 
liquid crystal devices.  It has been shown, theoretically, that flexoelectricity can be used
as the driver for the switching in new generation LCDs~\cite{DavidsonMottram02}.
Although thoroughly studied experimentally and theoretically (see Chapter~\ref{chap:one}), 
computer simulation studies of flexoelectric particles are relatively scarce, mainly
due to the difficulty of modeling the shape anisotropy if Meyer's principle is to be considered.
Models showing ferroelectric behaviour, have, however been well 
studied~\cite{WeiPatey92,WeiPatey92a}.\\

One attempt at modeling flexoelectric particles was performed by Neal
\etal\cite{NealParker97} in their study of molecules formed from rigid assemblies of 
three Gay-Berne sites. One
of these models was a triangular arrangement of parallel particles whose overall shape
resembled that of a pear. This system exhibited an isotropic to smectic ordering transition in
which the particles adopted anti-parallel orientations in adjacent layers. A subsequent
attempt at modeling pear shaped particles was performed by Stelzer \etal\cite{StelzerBerardi99}
using Gay-Berne sites to one end of each was connected a Lennard-Jones sphere.
Isotropic, nematic and smectic phases were found for this model. The computation of the
flexoelectric coefficients gave a non-zero splay coefficient and, within error estimates, a zero
bend  coefficient in accordance with Meyer's theory. Subsequent simulations by Billeter and
Pelcovits~\cite{BilleterPelcovits00}, using a slightly different energy parameterisation and a
different method for the computation of the flexoelectric coefficient showed results in
agreement with~\cite{StelzerBerardi99}.\\
Berardi~\etal\cite{BerardiRicci01} subsequently developed 
a single-site potential for pear shaped particles using Zewdie's Stone expansion
approach~\cite{Zewdie98a,Zewdie98b}. This study was rather successful as this computationally
efficient model showed
isotropic, nematic and smectic phases, the latter two of which, upon application of an appropriate
energy parameterisation, exhibited net polar order. Further details regarding this potential are
provided in Chapter~\ref{chap:six}





\section{Liquid crystals in confined geometries}


The behaviour of nematic liquid crystals close to a confining substrate is of fundamental 
importance for liquid crystal display design~\cite{Shanks82}. This has lead to many
experimentalist studies into the interfacial properties of confined liquid 
crystals~\cite{Jerome93,ZhuangMarucci94}. These types of systems have also been well studied
theoretically using a wealth of different techniques such as van der Waals 
theory~\cite{DelRioTeloDaGamma95}, density functional theory~\cite{Teixeira97,OsipovHess93} and 
mean field approximation~\cite{TjiptoMargoSullivan88}. Comparatively, the number 
of molecular computer simulations of confined liquid crystal systems is rather small.\\

From the wealth of experimental data available, it is clear that the confinement of liquid
crystals breaks the symmetry of the confined fluid by inducing two main effects~\cite{Jerome91} 
namely positional layering and orientational ordering through a mechanism called \emph{anchoring}.\\
The first effect is a universal consequence of confinement and has been observed by
Schoen~\cite{Schoen96,Schoen96a} through computer simulations of atomic (\ie Lennard-Jones)
fluids. On the other, hand surface-induced orientational order is specific to molecules with 
shape anisotropy such as mesogens. For these particles, several surface arrangements can be
observed according to the particles' orientations with respect to the substrate as shown on
Figure~\ref{fig:SurfArrangements}. These arrangement
are homeotropic, planar and tilted  with respectively $\theta = 0$, $\frac{\pi}{2}$ and $\in
]0:\frac{\pi}{2}[$, where $\theta$ is the angle between the surface normal and the mean particle
orientation. Further details regarding variations on these three basic types of surface ordering 
(\ie monostable, multistable and degenerate) are given in~\cite{Jerome91}.
The surface energy, or anchoring energy $f_S$ is commonly taken to be related to $\theta_S$, the
angle between the surface director and the natural anchoring angle, by
$f_s=\frac{1}{2}W_0\sin^2(\theta_S)$~\cite{Jerome91,Allen99}
where $W_0$ is the anchoring strength and measures the ease with which the director
can deviate from the imposed anchoring direction.\\


\picW=4.5cm
\begin{figure}
	\centering
	\subfigure[Planar]{\pic{typePlanar.ps}}
	\subfigure[Homeotropic]{\pic{typeHomeo.ps}}
	\subfigure[Tilted]{\pic{typeTilt.ps}}
	\caption[The main surface arrangements]{Illustration of the three main surface arrangements.
	The substrates are represented by the gray semi-transparent thick lines.}
	\label{fig:SurfArrangements}
\end{figure}

Early computer simulation studies of confined liquid crystalline systems used the
Lebwohl-Lasher lattice based models~\cite{TeloDaGammaTarazona90,CleaverAllen93} and were successful in
describing the enhanced order found in interfacial regions and shifts in the location of the 
isotropic-nematic transition due to the confinement. However the very nature of such a lattice
model introduces limitations, particularly its neglect of surface induced layering.\\

Subsequently, 
Chalam \etal\cite{ChalamGubbins91} used the Monte Carlo technique on a system of confined 
Gay-Berne particles using a separable form for the particle-surface potential. This
study confirmed the behaviour seen in lattice simulations, showing a shift of the 
isotropic-nematic transition to
higher temperatures due to the stabilization of the liquid crystalline phases by the
substrate. However the use of a surface potential which was separable into spacial and 
angular parts meant that the shape of the particles was not properly taken into account. This
was remedied by Zhang \etal\cite{ZhangChakrabarti96} who used the same molecular model with 
a separable substrate potential and observed tilted layers at the surface. Wall and
Cleaver~\cite{WallCleaver97}, using a modified surface potential, extended this study so as to
include the effect of changing temperature and phase. 
Another modification of this surface potential, involving azimuthal coupling, was used by Latham and 
Cleaver~\cite{LathamCleaver00} in systems of confined
mixtures of Gay-Berne particles. In these studies the alterations to the surface potential were
restricted to the well-depth anisotropy term, whereas the shape function of the potential was
always that for the interaction between a Gaussian ellipsoid and a sphere.\\

Although it has been shown that, in a confined system, the bulk orientation is often determined by the
orientational distribution of the surface particles~\cite{ZhangChakrabarti96}, some
authors have found rather different behaviour. Gruhn and
Schoen~\cite{GruhnSchoen97,GruhnSchoen98,GruhnSchoen98a} studied very thin films of Gay-Berne 
particles in a slab geometry confined between surfaces of rigidly fixed atoms; 
upon changing the thickness of the film, the orientation of bulk 
region particles was found to change regularly from planar to homeotropic. Another example 
of this kind of behaviour is 
provided by the work of Palermo \etal\cite{PalermoBiscarini98} who observed abrupt changes 
in particle orientation when moving from the surface to the bulk regions in systems of Gay-Berne 
particles absorbed at graphite surfaces.\\

Simulations of confined hard particles have also been performed. Allen~\cite{Allen99} used a system of 
hard ellipsoid with elongation $k=15$ confined so that their centres of mass interacted 
sterically with smooth substrates. A homeotropic substrate arrangement was observed.
Subsequently van Roij~\etal\cite{VanRoijDijkstra00, VanRoijDijkstra00a, DijkstraVanRoij01} used hard
spherocylinders confined between a hard smooth wall and an isotropic liquid crystal and observed
surface induced wetting and planar ordering. These studies also showed that the planar
arrangement is the natural state of hard-rod nematic phase in contact with a flat surface and 
when surface absorption is not made possible.\\
Chrzanowska~\etal and Cleaver~\etal\cite{Chrzanowska_Teixera_01, Cleaver_Teixeira_01} used the
Hard Gaussian Overlap (HGO) model (\ie a hard version of the Gay-Berne model) in symmetric and hybrid
anchored films using the Hard Needle Wall (HNW) potential as a surface model. 
Simulations of symmetrically anchored systems showed that appropriate tuning of the 
HNW potential lead the preferred surface arrangement to switch between planar and homeotropic. 
Hybrid anchored systems exhibited  a discontinuous transition from bent-director (or HAN) to
uniform director arrangements as the
anchoring coefficients of the surfaces were made sufficiently different. This result was
consistent with 
experimental observations~\cite{VandenbrouckValignat99} of 5CB molecules spun cast onto 
silicon wafers which were subject to
planar anchoring at the solid substrate and homeotropic anchoring at the free surface. Upon
increasing the film thickness from a few molecular lengths to more than 20 nm, the molecules
which first formed small islands proved able to form a stable film at increased thicknesses 
since they were then able to adopt a bent-director. 
A similar result was shown theoretically by \v{S}arlah and \v{Z}umer~\cite{SarlahZummer99} 
that is, very thin films with hybrid anchoring do not show a bent-director structure.\\

Realistic simulations of confined molecules have also been performed. Cleaver and
Tildesley~\cite{CleaverTildesley94} have performed energy minimisations on systems of 8CB
molecules, represented by an assembly of 22 spherical sites, absorbed on either a smooth or graphite 
planar substrates. They found that strips of $50$ molecules formed structures almost fully
compatible with scanning tunneling microscopy (STM) observations. Later, Yoneya and 
Iwakabe~\cite{YoneyaIwakabe95} performed molecular dynamics simulations on systems of 8
molecules of 8CB anchored on graphite and initially arranged in the structures shown by the 
STM experiment. However, due to the small system size used and the lack of periodic boundary 
conditions, these arrangements proved to be unstable. These limitations were later 
removed by Cleaver~\etal\cite{CleaverCallaway95} who studied periodic systems of monolayers of
8CB and 10CB molecules anchored on graphite using energy minimisation and molecular dynamics
techniques. The findings of this study showed structures fully consistent with the STM
observations.\\
A more systematic series of simulations was conducted by Binger and
Hanna~\cite{BingerHanna99,BingerHanna00,BingerHanna01} who performed realistic molecular
dynamics and molecular mechanics simulations of various liquid crystals molecules (\eg 5CB, 8CB,
MBF). Systems ranging from single molecules up to two monolayers anchored on different polymeric 
substrates (\eg PE, PVA, Nylon 6) were investigated. The authors found that for most substrates, 
the molecules adopt planar arrangements with specific favoured conformations. 
In the most recent of these studies~\cite{BingerHanna01}
the atoms of the substrate were replaced by a pseudo potential which had the combined
effects of saving computer power and making the substrate model more generic. The results from
this last set of simulations proved to be very encouraging as they were fully consistent 
with previous results despite the increased simulation speed.







\section{Anchoring Transitions}

The study of confined liquid crystalline systems has shown that different
particle-substrate interactions can lead to various types of surface anchoring. 
A number of experimental 
studies have reported that, upon changing some of the experimental conditions (\ie temperature,
structure of the aligning agent), transitions between two  surface arrangements can be observed
which leads to a change in the bulk alignment. 
Transitions which meet these requirements
are called \emph{anchoring transitions}. However, even in studies where the surface induced 
structural changes have been thoroughly studied, the mechanisms underlying these anchoring 
transitions remain relatively unexplained.
From the many experimental studies reporting anchoring transitions, it can be deduced that
 a concensus about the origins of these transitions is still lacking since a number of 
different mechanisms have been proposed to explain the origins of anchoring
transitions.\\ 

The first observed anchoring transitions were temperature  
driven~\cite{PatelYokoyama93,JagemalmKomitov1997}~; other authors have shown that anchoring
transitions can be obtained by a change in the conformation of the surface aligning
agent~\cite{ZhuLu94,ZhuWei94}. Light can also induce anchoring transitions by changing the
structure of the substrate molecules (\eg light induced cis-trans isomerization) and, therefore, the 
particle-substrate interaction~\cite{BarberoPopaNita00}. This is a feature that can be exploited
in optical storage devices. The absorption behaviour of a liquid crystal on the 
substrate~\cite{AlkhairallaAllison99,AlkhairallaBoden02} or of volatile molecule on the 
liquid crystal substrate interface~\cite{Jerome93} can also lead to anchoring transitions if the 
amount of absorbed particles  or the nature of the  absorption is changed.
One last mechanism is the memory effect whereby a multistable
anchoring system can preferentially adopt one of its possible anchoring states due to the history 
of the sample~\cite{Jerome91,StoenescuMartinotLagarde99}.\\

Although a number of mechanisms underlying anchoring transitions are in principle known
(see~\cite{Sluckin95} for a review) very few theoretical analyses have been performed. 
Teixeira and Sluckin~\cite{TeixeiraSluckin92,TeixeiraSluckin92a} used a Landau-de Gennes free
energy functional to study the planar to homeotropic anchoring transition in systems of liquid
crystals confined by different substrates. They found a rich anchoring behaviour, despite 
a number of simplifications that had to be made, which helped in the identification of the
possible mechanisms responsible for anchoring transition; in this case, the compositions
of binary mixtures of liquid crystals and the amount of adsorption at the surface. Subsequently 
Teixeira~\etal\cite{TeixeiraSluckin93} used a Landau-de Gennes formalism to observe
a temperature driven anchoring transition at the interface between a liquid crystal and smooth 
solid surface, thus confirming the experimental findings.\\
The effect of non-uniform substrates (\ie microtextured) has been studied by Qian and
Sheng~\cite{ZhengQianSheng96,ZhengQianSheng97} using a Landau-de Gennes formalism. They show
that the effect of the substrate is to induce temperature dependent tilt angles separated by
phase transitions.\\
%
The literature on computer simulations of anchoring transition is extremely scarce. 
Cleaver and Teixeira~\cite{Cleaver_Teixeira_01} have studied systems of hard Gaussian
overlap particles confined in a slab geometry and interacting with smooth substrates via the
hard needle wall potential. There, adsorption phenomena induced an homeotropic to planar
anchoring transition as the surface potential parameterisation was changed. More details into
similar systems can be found in Chapter~\ref{chap:four}.
Another simulation study of an anchoring transition was that of
Lange and Schmid~\cite{LangeSchmid02,LangeSchmid02a,LangeSchmid02c} who 
observed an anchoring transition between tilted and homeotropic arrangements
in a system of ellipsoidal Gay-Berne confined by grafted polymer chains (made of
Gay-Berne ellipsoids) as the grafting density was changed.





%==============================================================================
%==============================================================================
