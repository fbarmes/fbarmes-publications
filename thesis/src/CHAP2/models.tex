\section{Molecular models of Liquid Crystals}

At a molecular scale, a first approximation of the shape of liquid crystals molecules 
would be a rod or a disc for, respectively, calamitic and discotic mesogens. 
Using such objects, interacting through appropriate intermolecular
potentials, provides an excellent testbed for theories on the origins of liquid crystals
phases. Two types of particles are commonly used. Hard particles models have their basis in 
Onsager theory, which shows that short range repulsive interactions alone can lead to the formation
of liquid crystal phases such as the nematic phase. A refinement of these models is to
incorporate the attractive forces which are responsible for the more sophisticated mesophases.
Section~\ref{ss:HP} discuss the first type of models while the effects of attractive forces
for the most popular soft model, namely the Gay-Berne model, are presented in Section~\ref{ss:GB}.

%==================================================================================================
%==================================================================================================
\subsection{Hard particle models.}
\label{ss:HP}

%=============================
%=============================
\subsubsection{Hard ellipsoids of revolution}

The structure of simple atomic fluids can be described efficiently and surprisingly accurately 
using hard sphere models. 
The most obvious extension of this result that incorporate the anisotropy of mesogens is to
represent the particles by hard ellipsoids of revolution (HER) with semi axes $a=b\neq c$. For
this model, the length to width ratio $k$ is defined by $k=\frac{a}{c}$. Also both discotic
and calamitic particles can be modeled according to the value of $k$.
Although spherocylinders were used in Onsager's treatment, the HER can be expected to follow the
Onsager solution as it can be shown that for both shapes, the expression for the excluded volume 
of a pair of particles as a function of their relative orientation $\alpha$ takes the form
$V_\mrm{excl}\sim\sin\alpha$ $\alpha$~\cite{FrenkelMulder85}.\\
%
The main drawback of the HER model is that the contact distance between two ellipsoids can not be
expressed in a closed form. 
The first Monte Carlo simulation of this model was performed on a two dimensional system by 
Vieillard-Baron in the early 1970's~\cite{VieillardBaron72}. This work gave the first algorithm for the
contact distance between two ellipses and showed that excluded volume effects
play an important role in mesophase formation. However, limitations in computer power
prevented Vieillard-Baron from establishing the behaviour of three dimensional systems. The contact 
distance computation algorithm was later refined by Perram~\etal\cite{PerramWertheim84,PerramWertheim85} 
and a tentative phase diagram for three
dimensional systems was proposed by Frenkel~\etal\cite{FrenkelMulder81}. This showed
the existence of four different phases namely isotropic, nematic, plastic crystal and ordered
crystal. A more thorough investigation by Frenkel and Mulder~\cite{FrenkelMulder85} established the
range of stability of those phases; more specifically nematic phases were found for $k \geq
2.75$, the transition density reducing with increased molecular elongation.
Comparison of those results with the y-expansion
density functional theory where the free energy expansion was cut at the third 
term~\cite{MulderFrenkel85} showed good agreement for $\frac{1}{3} \leq k \leq 3$; a worsening 
of the theoretical predictions was found with more extreme $k$ values, however.\\
Despite the argument from Zarragoicoecha~\etal\cite{Zarragoicoecha92} that the stability of 
the nematic phase for $k=3$ is an artifact of the limited system size employed by Frenkel \etal
simulations, subsequent work
by Allen and Mason~\cite{AllenMason86} for several system sizes confirmed the validity
of the phase diagram proposed by Frenkel~\etal
An extension of this phase diagram was later proposed by Camp~\etal\cite{CampMason96} who
computed of the exact location of the isotropic-nematic phase transition using Gibbs-Duhem
integration techniques.\\

Studies by Allen~\etal~\cite{Allen90,CampAllen97} of a biaxial version of the hard ellipsoid
model ($a \neq b \neq c$) for $\frac{c}{a} = 10$ and $1 \leq \frac{b}{a} \leq 10$ showed the
existence of isotropic, nematic, discotic nematic and biaxial phases. Again, the exact location of the
isotropic-nematic and isotropic-discotic nematic phase transitions were computed using
Gibbs-Duhem integration methods.

The extensive studies performed on this model have established a comprehensive and very precise 
phase diagram as a function of molecular elongation. The range of stability of the different
phases can be summarized as~:
%
\begin{itemize}
	\item plastic solid~: $k \in [\frac{1}{1.25}:1.25]$
	\item ordered solid~: $k \in [\frac{1}{2}:2]$
	\item discotic nematic~: $k \leq \frac{1}{2.75}$
	\item nematic~: $k \geq \frac{1}{2.75}$
\end{itemize}

The major conclusion that can be drawn from these results is that Onsager's solution applied
at the intermediate values $k\sim 3-5$ corresponding to the elongation of common mesogens. 

%=============================
%=============================
\subsubsection{Hard spherocylinders.}

Another very popular model for calamitic liquid crystals is the hard spherocylinder (HSC). This is the
model used by Onsager, that is a cylinder of length $L$ and diameter $D$ terminated by two 
hemispherical caps of diameter $D$. The elongation $k$ of such an object is therefore
$k=\frac{L+D}{D}$. The reasons for the use of this model are threefold. First it is the shape
used in Onsager's theory and therefore the most obvious choice to make when testing this theory 
using computer simulations. Also for computational purposes, the model presents the advantage of 
having a tractable expression 
for the contact distance. Finally the spherocylinder shape resembles closely that of the 
liquid-crystal-phase-forming tobacco virus~\cite{ZazadzinskiMeyer86,DogicFraden97}. There is,
however, no obvious extension of this model to discotic particles~; equivalence could be
considered using cut spheres~\cite{VeermanFrenkel92} or sort cylindrical 
segments~\cite{BatesFrenkel98}\\

The first computer simulation on hard spherocylinders was performed by 
Vieillard-Baron~\cite{VieillardBaron74} using elongations $k=2$ and $3$, but this study did not find
any stable nematic phases. Limitations in computer power prevented the author from fully
investigating a system with $k=6$. In 1987, Frenkel~\cite{Frenkel87,Frenkel87a} performed the
first simulations of three dimensional systems of hard spherocylinders with
elongation $k \in [0:5]$ and free translation and rotation; the authors found the model to 
exhibit three phases, namely~: isotropic liquid, nematic and
\smA. With the shorter elongation $k=3$, Veerman and Frenkel~\cite{VeermanFrenkel96} 
found the nematic and smectic phases to become respectively unstable and metastable. A more
complete phase diagram was later proposed by Mc Grother~\etal\cite{McGrotherWilliamson96} for
different values of $k$ in the range $[3:5]$ and confirmed the results from Frenkel~\etal 
The authors also showed that as $k$ was increased, the \smA~ phase was stable for
$\frac{L}{D}\geq 3.2$ whereas the nematic phase required $\frac{L}{D}\sim 4$. 
Bolhuis and Frenkel~\cite{BolhuisFrenkel97}
later refined and completed this phase diagram into the Onsager limit.\\
From these studies the phase diagram of the HSC can be summarized as~:
\begin{itemize}
	\item Nematic~: $\frac{L}{D} \geq 3.7$.
	\item \SmA~: $\frac{L}{D} \geq 3.1$.
\end{itemize}

%=============================
%=============================
%\subsubsection{Other models.}

%==================================================================================================
%==================================================================================================
\subsection{The Gay-Berne model.}
\label{ss:GB}


Despite their simplicity and their success in modeling liquid crystalline phases, the steric
models discussed above have limitations in that they can not be used to model
every liquid crystalline phase (\eg \smB) and the effect of attractive interactions can
not be studied.\\
The first mathematically tractable model for soft particles was developed in 1972 by Berne and 
Pechukas~\cite{BernePechukas72}. This model describes the interaction between soft ellipsoidal
particles through a potential $\mathcal{V}^{BP}$ which is alternately repulsive and
attractive at short and long ranges. This potential however presented  unrealistic
features such as having equal well depths for end to end and side by side parallel molecular
arrangements.
These deficiencies were resolved by Gay and Berne~\cite{GayBerne81} who modified the functional
form of the Berne-Pechukas potential so that it could give a reasonable fit to a 
linear arrangement of four 
Lennard-Jones  sites. This resulted in the now widely used Gay-Berne model 
$\mathcal{V}^{GB}$ expressed as~:
\begin{eqnarray}
	\mathcal{V}^{GB} &=& 4\epsilon(\ui,\uj,\rij)\left\{ R^{12} - R^{6} \right\} \\
	\mrm{with\ }R &=&  \frac{\so}{r - \sijr +\so} \nonumber.
\end{eqnarray}

Where $\sijr$ is the shape function for two Gaussian ellipsoids, as determined
originally by Berne and Pechukas~\cite{BernePechukas72},
%
\begin{equation}
	\sijr = \so\left\{
	1 - \frac{1}{2}\chi\left[ 
	\frac{ \lp\dotProd{\rij}{\ui} + \dotProd{\rij}{\uj}\rp^2 }{1 + \chi(\dotProd{\ui}{\uj})}
      + \frac{ \lp\dotProd{\rij}{\ui} - \dotProd{\rij}{\uj}\rp^2 }{1 - \chi(\dotProd{\ui}{\uj})}
	\right] \right\}^{-\frac{1}{2}}
\end{equation}

$\so$ defines the unit of distance and $\chi$ is the shape anisotropy parameter defined using $k$ the
length to breadth ratio as~:
\begin{equation}
	\chi = \frac{k^2-1}{k^2+1}.
\end{equation}

The energy strength parameter is defined as~:
\begin{equation}
	\eijr = \eo\eOne^{\nu}(\ui,\uj)\eTwo^{\mu}(\uj,\uj,\rij)
\end{equation}

with~:
\begin{equation}
	\eOne(\ui,\uj) = \left[1 - \chi^2(\dotProd{\ui}{\uj})^2\right]^{-\frac{1}{2}}
\end{equation}

and
\begin{equation}
	\eTwo(\ui,\uj,\rij) = 
	1 - \frac{1}{2}\chi^\prime \left[ 
	\frac{ \lp\dotProd{\rij}{\ui} + \dotProd{\rij}{\uj}\rp^2 }{1 + \chi^\prime(\dotProd{\ui}{\uj})}
      + \frac{ \lp\dotProd{\rij}{\ui} - \dotProd{\rij}{\uj}\rp^2 }{1 - \chi^\prime(\dotProd{\ui}{\uj})}
	\right].
\end{equation}

Here $\chi^\prime$ is the energy anisotropy parameter defined using $k^\prime$ the ratio of end to
end and side by side well depth ($\epsilon_{ee}$ and $\epsilon_{ss}$ respectively)~:
\begin{eqnarray}
	\chi^{\prime} &=& \frac{k^{\prime \mu^{-1}} - 1}{k^{\prime \mu^{-1}} + 1}	\\
	k^\prime &=& \frac{\epsilon_{ee}}{\epsilon_{ss}}	\nonumber
\end{eqnarray}

The Gay-Berne model behaviour can be easily tuned through modification of the four parameters
$k,k^\prime,\mu$ and $\nu$.\\
Preliminary simulation results by Adam~\etal\cite{AdamLuckhurst87}, using the parameterisation 
$GB(k,k^\prime,\nu,\mu) = GB(3,5,1,2)$, showed the Gay-Berne model
to be suitable for liquid crystal modeling as both isotropic and nematic phases were observed.
Subsequent work by Luckhurst and Stephens~\cite{LuckhurstStephens90} using the slightly
different parameterisation $GB(3,5,2,1)$ found a much richer phase diagram containing isotropic,
nematic, \smA, \smB and crystal phases. Thanks to a very thorough study by the 
Seville group~\cite{ChalamGubbins91,deMiguelRull91,deMiguelRull91a},
the full liquid crystalline phase diagram of the model was then determined for
$GB(3,5,1,2)$.\\
%
The parameterisation $GB(3,5,3,1)$ was used by other groups~\cite{BerardiEmerson93,
AllenWarren96}; while this gives the same isotropic, nematic and smectic phases as the
previous parameterisation, the increased value of $\mu$ allows for a wider nematic region.\\
%
The substantially different parameterisation $GB(4.4, 39.6^{-1}, 0.74, 0.8)$ was used by Luckhurst
and Simmonds~\cite{LuckhurstSimmonds93} in an attempt to use a model better fitted to
representing
real mesogens. This parameterisation was obtained from fitting the Gay-Berne potential to a uniaxial
version of a realistic potential of the p-terphenyl molecule. The authors found a phase behaviour
compatible with their aim; the model displayed isotropic, nematic and smectic phases.
Subsequently, Bates and Luckhurst~\cite{BatesLuckhurst99} performed a thorough study using 
the parameterisation  $GB(4.4,20^{-1},1,1)$ and found the model to exhibit isotropic, nematic, \smA
and \smB phases in good agreement with the behaviour of the real mesogens this parameterisation was
modeling.
%
An investigation into the generic effects of the attractive part of the 
potential~\cite{deMiguelDelRio96} showed that
smectic order is favoured as $k^\prime$ is increased, thus showing the importance of attractive
forces for the formation of smectic phases by ellipsoidal particles. A similar study into the
effects of molecular elongation on the Gay-Berne phase diagram~\cite{BrownAllen98} showed significant
changes notably in the location of the isotropic-nematic phase transition.\\

More recently, various extensions of the Gay-Berne model have been performed. 
Cleaver \etal\cite{CleaverCare96} generalised the potential to give an interaction for particles 
with different elongations thus opening up the possibility of modeling LC mixtures.
Zewdie~\cite{Zewdie98a,Zewdie98b} developed a Corner-like potential where the range and
energy strength parameters are expanded in terms of a complete orthogonal basis set, namely
Stone~\cite{Stone} functions. The advantage of this method is that each term in the energy
strength parameter can be associated with a given type of interaction thus allowing fine tuning
of the model. Application of this approach to the modeling of discotic particles~\cite{Zewdie98a} 
lead to the
formation of isotropic, discotic-nematic, columnar and crystal phases while the modeling of
calamitic particles (using an equivalent of $GB(3,5,1,2)$) showed isotropic, nematic, \smA, \smB and
crystal phases in agreement with~\cite{AdamLuckhurst87}.\\
%
Biaxial particles have been studied by the Bologna 
group~\cite{BerardiZannoni00,Zannoni02} using a generalization of the Gay-Berne
potential to the interaction between two arbitrary ellipsoidal particles presented
in~\cite{BerardiFava98}. These studies have indicated that the answer to the much argued about 
question
of the existence of biaxial phases is that, such phases can exist since both biaxial nematic and
smectic phases have been observed.\\
%
The inclusion of dipole electrostatic moments into Gay-Berne particles has been studied by
Houssa~\etal~\cite{HoussaRull98, HoussaMcGrother99} and revealed that although the location of
the phase transitions of the model are insensitive to the strength of the dipoles, the
electrostatic forces were found to have a considerable effect on the nature of the observed
phases.
multipole electrostatic moments has also been considered. Specifically, 
Berardi \etal~\cite{BerardiOrlandi03} included two outboard permanent dipoles at various angles
from the axis and observed tilted smectic phases whose intra-layer arrangement was compatible
with \smJ and tetragonal \smT. The inclusion of quadrupole moments into a $GB(4,5,2,1)$
system~\cite{Withers} proved to have strong effects upon the smectic regions of the phase
diagrams such as the replacement of a \smB by a \smJ phase.
%
These developments, along with the
extended possibility of parameterisation, makes, the Gay-Berne model a very versatile one which can
be applied to a wide number of different types of interactions. 
%A number of different applications is reviewed in~\cite{Zannoni02}.



