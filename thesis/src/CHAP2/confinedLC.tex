
\section{Liquid crystals in confined geometries}


The behaviour of nematic liquid crystals close to a confining substrate is of fundamental 
importance for liquid crystal display design~\cite{Shanks82}. This has lead to many
experimentalist studies into the interfacial properties of confined liquid 
crystals~\cite{Jerome93,ZhuangMarucci94}. These types of systems have also been well studied
theoretically using a wealth of different techniques such as van der Waals 
theory~\cite{DelRioTeloDaGamma95}, density functional theory~\cite{Teixeira97,OsipovHess93} and 
mean field approximation~\cite{TjiptoMargoSullivan88}. Comparatively, the number 
of molecular computer simulations of confined liquid crystal systems is rather small.\\

From the wealth of experimental data available, it is clear that the confinement of liquid
crystals breaks the symmetry of the confined fluid by inducing two main effects~\cite{Jerome91} 
namely positional layering and orientational ordering through a mechanism called \emph{anchoring}.\\
The first effect is a universal consequence of confinement and has been observed by
Schoen~\cite{Schoen96,Schoen96a} through computer simulations of atomic (\ie Lennard-Jones)
fluids. On the other, hand surface-induced orientational order is specific to molecules with 
shape anisotropy such as mesogens. For these particles, several surface arrangements can be
observed according to the particles' orientations with respect to the substrate as shown on
Figure~\ref{fig:SurfArrangements}. These arrangement
are homeotropic, planar and tilted  with respectively $\theta = 0$, $\frac{\pi}{2}$ and $\in
]0:\frac{\pi}{2}[$, where $\theta$ is the angle between the surface normal and the mean particle
orientation. Further details regarding variations on these three basic types of surface ordering 
(\ie monostable, multistable and degenerate) are given in~\cite{Jerome91}.
The surface energy, or anchoring energy $f_S$ is commonly taken to be related to $\theta_S$, the
angle between the surface director and the natural anchoring angle, by
$f_s=\frac{1}{2}W_0\sin^2(\theta_S)$~\cite{Jerome91,Allen99}
where $W_0$ is the anchoring strength and measures the ease with which the director
can deviate from the imposed anchoring direction.\\


\picW=4.5cm
\begin{figure}
	\centering
	\subfigure[Planar]{\pic{typePlanar.ps}}
	\subfigure[Homeotropic]{\pic{typeHomeo.ps}}
	\subfigure[Tilted]{\pic{typeTilt.ps}}
	\caption[The main surface arrangements]{Illustration of the three main surface arrangements.
	The substrates are represented by the gray semi-transparent thick lines.}
	\label{fig:SurfArrangements}
\end{figure}

Early computer simulation studies of confined liquid crystalline systems used the
Lebwohl-Lasher lattice based models~\cite{TeloDaGammaTarazona90,CleaverAllen93} and were successful in
describing the enhanced order found in interfacial regions and shifts in the location of the 
isotropic-nematic transition due to the confinement. However the very nature of such a lattice
model introduces limitations, particularly its neglect of surface induced layering.\\

Subsequently, 
Chalam \etal\cite{ChalamGubbins91} used the Monte Carlo technique on a system of confined 
Gay-Berne particles using a separable form for the particle-surface potential. This
study confirmed the behaviour seen in lattice simulations, showing a shift of the 
isotropic-nematic transition to
higher temperatures due to the stabilization of the liquid crystalline phases by the
substrate. However the use of a surface potential which was separable into spacial and 
angular parts meant that the shape of the particles was not properly taken into account. This
was remedied by Zhang \etal\cite{ZhangChakrabarti96} who used the same molecular model with 
a separable substrate potential and observed tilted layers at the surface. Wall and
Cleaver~\cite{WallCleaver97}, using a modified surface potential, extended this study so as to
include the effect of changing temperature and phase. 
Another modification of this surface potential, involving azimuthal coupling, was used by Latham and 
Cleaver~\cite{LathamCleaver00} in systems of confined
mixtures of Gay-Berne particles. In these studies the alterations to the surface potential were
restricted to the well-depth anisotropy term, whereas the shape function of the potential was
always that for the interaction between a Gaussian ellipsoid and a sphere.\\

Although it has been shown that, in a confined system, the bulk orientation is often determined by the
orientational distribution of the surface particles~\cite{ZhangChakrabarti96}, some
authors have found rather different behaviour. Gruhn and
Schoen~\cite{GruhnSchoen97,GruhnSchoen98,GruhnSchoen98a} studied very thin films of Gay-Berne 
particles in a slab geometry confined between surfaces of rigidly fixed atoms; 
upon changing the thickness of the film, the orientation of bulk 
region particles was found to change regularly from planar to homeotropic. Another example 
of this kind of behaviour is 
provided by the work of Palermo \etal\cite{PalermoBiscarini98} who observed abrupt changes 
in particle orientation when moving from the surface to the bulk regions in systems of Gay-Berne 
particles absorbed at graphite surfaces.\\

Simulations of confined hard particles have also been performed. Allen~\cite{Allen99} used a system of 
hard ellipsoid with elongation $k=15$ confined so that their centres of mass interacted 
sterically with smooth substrates. A homeotropic substrate arrangement was observed.
Subsequently van Roij~\etal\cite{VanRoijDijkstra00, VanRoijDijkstra00a, DijkstraVanRoij01} used hard
spherocylinders confined between a hard smooth wall and an isotropic liquid crystal and observed
surface induced wetting and planar ordering. These studies also showed that the planar
arrangement is the natural state of hard-rod nematic phase in contact with a flat surface and 
when surface absorption is not made possible.\\
Chrzanowska~\etal and Cleaver~\etal\cite{Chrzanowska_Teixera_01, Cleaver_Teixeira_01} used the
Hard Gaussian Overlap (HGO) model (\ie a hard version of the Gay-Berne model) in symmetric and hybrid
anchored films using the Hard Needle Wall (HNW) potential as a surface model. 
Simulations of symmetrically anchored systems showed that appropriate tuning of the 
HNW potential lead the preferred surface arrangement to switch between planar and homeotropic. 
Hybrid anchored systems exhibited  a discontinuous transition from bent-director (or HAN) to
uniform director arrangements as the
anchoring coefficients of the surfaces were made sufficiently different. This result was
consistent with 
experimental observations~\cite{VandenbrouckValignat99} of 5CB molecules spun cast onto 
silicon wafers which were subject to
planar anchoring at the solid substrate and homeotropic anchoring at the free surface. Upon
increasing the film thickness from a few molecular lengths to more than 20 nm, the molecules
which first formed small islands proved able to form a stable film at increased thicknesses 
since they were then able to adopt a bent-director. 
A similar result was shown theoretically by \v{S}arlah and \v{Z}umer~\cite{SarlahZummer99} 
that is, very thin films with hybrid anchoring do not show a bent-director structure.\\

Realistic simulations of confined molecules have also been performed. Cleaver and
Tildesley~\cite{CleaverTildesley94} have performed energy minimisations on systems of 8CB
molecules, represented by an assembly of 22 spherical sites, absorbed on either a smooth or graphite 
planar substrates. They found that strips of $50$ molecules formed structures almost fully
compatible with scanning tunneling microscopy (STM) observations. Later, Yoneya and 
Iwakabe~\cite{YoneyaIwakabe95} performed molecular dynamics simulations on systems of 8
molecules of 8CB anchored on graphite and initially arranged in the structures shown by the 
STM experiment. However, due to the small system size used and the lack of periodic boundary 
conditions, these arrangements proved to be unstable. These limitations were later 
removed by Cleaver~\etal\cite{CleaverCallaway95} who studied periodic systems of monolayers of
8CB and 10CB molecules anchored on graphite using energy minimisation and molecular dynamics
techniques. The findings of this study showed structures fully consistent with the STM
observations.\\
A more systematic series of simulations was conducted by Binger and
Hanna~\cite{BingerHanna99,BingerHanna00,BingerHanna01} who performed realistic molecular
dynamics and molecular mechanics simulations of various liquid crystals molecules (\eg 5CB, 8CB,
MBF). Systems ranging from single molecules up to two monolayers anchored on different polymeric 
substrates (\eg PE, PVA, Nylon 6) were investigated. The authors found that for most substrates, 
the molecules adopt planar arrangements with specific favoured conformations. 
In the most recent of these studies~\cite{BingerHanna01}
the atoms of the substrate were replaced by a pseudo potential which had the combined
effects of saving computer power and making the substrate model more generic. The results from
this last set of simulations proved to be very encouraging as they were fully consistent 
with previous results despite the increased simulation speed.





