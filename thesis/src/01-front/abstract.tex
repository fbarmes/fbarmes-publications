


\hrule
\begin{center}
	\textbf{\large Abstract}
\end{center}
%\vspace*{5mm}
\hrule
\vspace*{10mm}

%\begin{abstract}
In this Thesis, systems of confined and flexoelectric liquid crystal systems have been studied
using molecular computer simulations. The aim of this work was to provide a molecular model 
of a bistable display cell in which switching is induced through the application of directional
electric field pulses.\\

%========================================
%	CONFINED SYSTEMS - SYMMETRIC
%========================================
In the first part of this Thesis, the study of confined systems of liquid crystalline particles has
been addressed. Computation of the anchoring phase diagrams for three different surface interaction
models showed that the hard needle wall and rod-surface potentials induce both planar and
homeotropic alignment separated by a bistability region, this being stronger and wider
for the rod-surface varant. The results obtained using the rod-sphere surface model, in
contrast, showed that tilted surface arrangements can be induced by surface absorption mechanisms.
%
Equivalent studies of hybrid anchored systems showed that a bend director structure
can be obtained in a slab with monostable homeotropic anchoring at the top surface and bistable
anchoring at the bottom, provided that the slab height is sufficiently large and the top homeotropic
anchoring is not too strong.\\

%========================================
%	PEAR SHAPED PARTICLES
%========================================
In the second part of the Thesis, the development of models for tapered (pear-shaped) mesogens
has been addressed. The first model considered, the truncated Stone expansion model, proved to be
unsuccessful in that it did not display liquid crystalline phases.
This drawback was then overcome using the alternative parametric hard Gaussian overlap model 
which was found to display
a much richer phase behaviour. With a molecular elongation $k=5$, both nematic and
interdigitated smectic ${\rm A_2}$ phases were obtained.\\

%========================================
%	CONFINED PEARS - LCD MODELLING
%========================================
In the final part of this Thesis, the knowledge acquired from the two previous studies was
united in an attempt to model a bistable display cell. 
Switching between the hybrid aligned nematic and vertical states of the cell was
successfully performed using pear shaped particles with both dielectric and dipolar 
couplings to an applied field. Also, a parameter window was identified, for 
values of the electric field, dielectric susceptibility and dipole moment, for which directional
switching is achievable between the bistable states.

%\end{abstract}



