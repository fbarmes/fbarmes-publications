

\section{Aims}


The work presented in this Thesis addresses the study of confined and flexoelectric liquid
crystalline systems by means of molecular simulations. The final aim is the development of
a model for a display cell represented by an hybrid anchored slab with homeotropic anchoring
at the top surface and bistable homeotropic-planar anchoring on the bottom surface. In such
a cell, switching between the two stable states, the so called HAN and vertical states, is
thought to be induced by the flexoelectric characteristic of the molecules and application 
of an appropriate electric pulse~\cite{DavidsonMottram02}. In order to successfully model 
such a system requires three key aspects to be investigated.\\

%------------------------------------------------
%	Confined Systems
%------------------------------------------------
First the study of confined liquid crystalline systems is addressed. The aim here is to perform a
thorough investigation of surface-induced structural changes on a system of confined
ellipsoidal-shaped hard particles in a slab geometry using a range of
different surface potentials. In this study both symmetric and hybrid anchoring conditions are
used. The main goals here are the localization and characterization of the planar-homeotropic
anchoring transition and the identification of possible regions of bistability between the two
surface arrangements. Hybrid anchored systems are also investigated so as to determine the
parameterisation necessary to maintain continuous director profiles in simulations of HAN 
arrangement.\\
%
%In order to achieve this, a simple surface potential, namely the hard needle wall potential, is
%first used so as to identify the main surface induced effects and study the homeotropic to
%planar anchoring transition. This surface model is then replaced by alternative, more realistic
%ones and the effect of changing the surface potential are studied. Finally the consequences of
%using hybrid confined systems with different anchoring conditions are investigated, specifically
%the effect of a top surface homeotropic anchoring upon the bistability behaviour of the bottom
%surface is studied as well as the structural transition between the two surface induced
%arrangements. 
Results corresponding to this first part of the Thesis are presented in
Chapter~\ref{chap:four} and~\ref{chap:five}.\\


%------------------------------------------------
%	Pear shaped particles
%------------------------------------------------
The second aim of this Thesis is the development of molecular
model of hard pear-shaped particles which are thought to exhibit flexoelectric
behaviour. 
%This is achieved through the use of two different models, namely the Stone expansion
%and parametric hard Gaussian overlap models. For both models, it is intended  to compute the bulk phase
%diagrams as well as the structural observables of the produced phases. 
A key target from this study is a model for pear shaped particles which displays liquid 
crystalline phases,
most specifically a stable nematic phase which forms spontaneously upon compression.
Results corresponding to this second part of the Thesis are presented in Chapter~\ref{chap:six}.\\

%------------------------------------------------
%	Switching hybrid anchored pears
%------------------------------------------------
In the last part of this Thesis, the results obtained from the two previous studies are brought
together towards the final aim of this Thesis, and the simulation of confined systems of pear
shaped particles is addressed. Here, modeling is performed of a display cell having hybrid 
anchoring conditions with homeotropic
arrangement on the top surface and homeotropic-planar bistable
anchoring on the bottom surface. Switching between the two stable states of this cell, the so
called HAN and vertical states, is
attempted through application of an alternatively positive or
negative electric pulse. The aim here is to determine some of the molecular mechanism relevant to
this recent development in LCD technology. Results corresponding to this last part of the Thesis
are presented in Chapter~\ref{chap:seven}.

