

\section{Organisation of the Thesis}

%----------------------------------------------------------
%	CHAPTER	1 : the LC phases
%----------------------------------------------------------
Chapter~\ref{chap:one} provides some background information about liquid crystals. Specifically the
characterization of the different bulk liquid crystalline phases is discussed and the class of
flexoelectric particles and their properties is considered. This Chapter then follows on
the description of the two main theories of liquid crystals as well as the experimental
techniques most commonly used. As a conclusion the different applications of liquid crystals are
presented.\\

%----------------------------------------------------------
%	CHAPTER	2 : Simulation of LC
%----------------------------------------------------------
Chapter~\ref{chap:two} reviews the techniques and models relevant to the molecular simulation of
liquid crystals as well as the properties of confined liquid crystalline systems and their
anchoring transitions. The simulation techniques and models relevant to liquid crystals are
reviewed first, followed by a survey of the modeling of flexoelectric particles. The last part
of this Chapter concentrates on the simulation of confined systems and of their anchoring
transitions.\\

%----------------------------------------------------------
%	CHAPTER	3 : HGO model
%----------------------------------------------------------
Chapter~\ref{chap:three} combines review and results and focuses specifically on the
simulation of hard Gaussian overlap (HGO) particles. The literature concerning this
specific molecular model is reviewed and the techniques used for the computation
of observables are presented. Some preliminary results for the bulk behaviour of the HGO
models are then given, considering both calamitic and discotic particles.\\


%----------------------------------------------------------
%	CHAPTER	4 : Effects of surfaces
%----------------------------------------------------------
Chapter~\ref{chap:four} presents the results corresponding to
the first part of the study of confined systems. Here the surface induced
effects on systems of HGO particles confined in a slab geometry are studied.
Using a simple surface potential, namely the hard needle wall potential (HNW), these effects are
characterized and their regions of stability compared with analytical results. From this, the
anchoring transition between the two stable surface arrangements (planar and homeotropic) is
located as a function of density and anchoring conditions. Here, regions of bistability between the
two surface arrangements are identified.\\

%----------------------------------------------------------
%	CHAPTER	5 : advanced confined systems
%----------------------------------------------------------
In Chapter~\ref{chap:five}, more advanced confined configurations are studied. In the first part,
the surface induced effects obtained using two more realistic surface potentials, namely the rod-sphere
(RSP) and the rod-surface (RSUP) potentials, are investigated and the results compared
with an analytical treatment.
In the second part of this Chapter, the case of hybrid systems is investigated using the HNW
potential. Systems of HGO particles are considered confined in a slab geometry with different 
anchoring conditions on each surface. The effects of hybrid anchoring on the bistability regions
are investigated and the structural transition between the homeotropic and planar surface
arrangements is investigated.  Finally the possibility of obtaining a continuous transition
between these two is considered.\\

%----------------------------------------------------------
%	CHAPTER	6 : bulk pear shaped particles
%----------------------------------------------------------
Chapter~\ref{chap:six} related to a different line of work. Here models for the description of hard
pear-shaped particles are developed. For this, two models are considered. The first model is
the so called Stone expansion model,  a steric version of a potential used previously by
the Bologna group~\cite{BerardiRicci01} while the second model is the parametric hard Gaussian 
overlap (PHGO) which was developed within
this project to resolve some difficulties experienced with the former model. The bulk phase diagrams
and structural observables of the phases obtained using these models are presented and their
applicability for the modeling of pear shaped liquid crystal molecules is discussed.\\

%----------------------------------------------------------
%	CHAPTER	7 : flexo switching
%----------------------------------------------------------
Chapter~\ref{chap:seven} is the last substantial Chapter of this thesis. Here, the knowledge 
acquired from the preceding studies of
confined ellipsoidal particles and the bulk behaviour of pear shaped models is brought together
in a study of confined systems of flexoelectric particles. The aim here is to achieve
directional field induced switching between the two stable states of a hybrid anchored display 
cell model.
In order to achieve this, a modified version of the RSUP model is implemented, 
and the resulting surface induced structural changes studied.
Specifically a region of bistability between the planar and homeotropic anchoring states 
is sought. This region of bistability is then used to investigate the mechanisms of easy and 
hard switching between the HAN and vertical states of the cell; the relevance of the model 
to the operation of bistable cell is then discussed.\\

Finally, the main results and conclusions of the Thesis are brought together, and suitable areas
for future work are listed.


