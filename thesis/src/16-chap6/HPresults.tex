
\section{Simulation results using the HP model.}
\label{s:HPresults}

\subsection{Particles with $k=3$.}

The phase behaviour of the HP model was computed using Monte Carlo simulations of bulk systems 
of $N=1250$ particles in the isothermal-isobaric ensemble. For better sampling of phase space 
and to ease the formation of possible polar phases, `flip moves' were included using a probabilistic
scheme. These moves involved the inversion of the molecular orientation vector $\ui$ and 
accounted for 20\% of each particle's attempted moves. Compression moves which changed every
box length independently were also carried out using a  probabilistic scheme, on average once 
every two MC sweeps (\ie moves per particle). 
Run lengths were of the order of $1.10^6$ sweeps for equilibration and production, but for 
some of the highest densities up to $5.10^6$ sweeps were necessary to achieve
equilibration.\\

The first series of runs used a compression sequence; the pressure range considered was chosen so that 
it would induce densities of the same order of magnitude as those found for the
isotropic-nematic transition of the hard Gaussian overlap model (see Chapter~\ref{chap:three}.)
The pressure and order parameter results obtained are shown in 
Figure~\ref{fig:HP_k3_phaseDia} under the `compression' label.

%=================================
\picW = 7cm
\begin{figure}
	\subfigure[$P(\rho^{*})$]{\picL{HP_P-rho_k3.ps}}
	\subfigure[$\langle P_2\rangle(\rho^{*})$]{\picL{HP_P2-rho_k3.ps}}
	\caption{Results from constant NPT Monte Carlo simulations in compression and melting
	sequences of the HP model with $k=3$.}
	\label{fig:HP_k3_phaseDia}
\end{figure}
%=================================

These reveal that during the compression sequence, no spontaneous ordering was observed.
Thus, in order to test for the stability of the nematic phase, another series of simulations was
performed in a melting sequence of decreasing pressures. The starting configuration for this series 
was a high density configuration obtained from the compression series, the particles being field
aligned along $\vecth{z}$ so as to obtain a nematic phase with an order parameter 
$P_2\sim 0.8$. The results for this sequence are shown on  Figure~\ref{fig:HP_k3_phaseDia} 
under the `melting' label. Surprisingly, these results suggest a
stable nematic phase for densities $\rho^{*} > 0.3$ and $P^{*}>8.0$. This situation seems to
be very similar to that found by Williamson and Jackson~\cite{WilliamsonJackson98} 
where a stable nematic phase was present although the model failed to order spontaneously.
A further test of the systems behaviour was provided through monitoring the mobility of the molecules
(Figure~\ref{fig:HP_k3_dc}). This was measured using~:
\begin{equation}
	\langle\delta r^2(n)\rangle  = \langle({\bf r}_n - {\bf r}_0)^2\rangle 
\end{equation}
%
%===========================
\picW = 7cm
\begin{figure}
	\centering
	\subfigure[compression sequence]{\picL{HP_k3_cpress_dc.ps}}
	\subfigure[melting sequence]{\picL{HP_k3_melt_dc.ps}}
	\caption[Evolution of $\langle\delta r^2(n)\rangle$ as a function of $n$ the number of
	Monte Carlo sweeps for the HP model and k=3.]
	{Evolution of $\langle\delta r^2(n)\rangle$ as a function of $n$ the number of
	Monte Carlo sweeps for the HP model and k=3. The data
	on the left are for a constant $NVT$ compression sequence with $N=1000$ and the data on
	the right for a constant NPT melting sequence with $N=1250$.}
	\label{fig:HP_k3_dc}
\end{figure}
%===========================
%
where ${\bf r}_n - {\bf r}_0$ is the displacement vector moved by a given particle in $n$ 
consecutive MC
sweeps and the angled brackets indicate an average over all particles and the run length. In MC 
simulations with fixed maximum particle displacement (as was the case here), Brownian diffusion 
dictates that  $\langle\delta r^2(n)\rangle$ should increase linearly with $n$ in a fluid phase.
The greater the gradient of $\langle\delta r^2(n)\rangle$, the more fluid the studied phase.\\

These measurements show that for both the compression and melting series, the
diffusion of the particles decreased monotonically with increased density or pressure. If the
results are fitted to equations of the form $y=a +bx$, $b$ decreases of about two orders of magnitude
between the lowest and highest densities shown. The conclusion from this is that the highest density 
configurations did not correspond to fluid, but to glassy phases which inhibited the ordering of the
system. In the melting series, in contrast, the lack of fluidity at the highest densities 
may have prevented the corresponding systems from disordering into the equilibrium structure. 
This argument is supported by the order of magnitude of the diffusion which is
about half the value obtained from the compression series, thus illustrating the frustrated
nature of the ordered phases obtained. The reduction of move acceptance rate from 40\% down to 10\%
between simulations at the lowest and highest density  further illustrates this glassiness.\\


%============================
\picW = 5.5cm
\begin{figure}
	\centering
	\subfigure[$\rho^{*}=0.28$]{\pic{HP_box_NVT_k3_b_N1000_d0.2800_0.50M.ps}}
	\hspace*{2cm}
	\subfigure[$\rho^{*}=0.32$]{\pic{HP_box_NVT_k3_b_N1000_d0.3200_0.50M.ps}}
	\caption{Configuration snapshots of the HP model and $k=3$ at $\rho^{*}=0.28$ and
	$\rho^{*}=0.32$.}
	\label{HP_k3_snaps}
\end{figure}
%============================

%============================
\picW = 7cm
\begin{figure}
	\centering
	\subfigure[$g_\parallel(r_\parallel)$]{\picL{HP_k3_grParallel.iso.ps}}
	\subfigure[$g_\perp(r_\perp)$]{\picL{HP_k3_grPerpLayer.iso.ps}}
	
	\subfigure[$g^\mrm{mol}_\parallel(r_\parallel)$]{\picL{HP_k3_grParallel.fun.ps}}
	\subfigure[$g^\mrm{mol}_\perp(r_\perp)$]{\picL{HP_k3_grPerpLayer.fun.ps}}
	\caption{Pair correlation functions for the HP model with $k=3$ computed with respect to
	$\vecth{n}$(top) and $\ui$ (bottom).}
	\label{HP_k3_grs}
\end{figure}
%============================


More insight into the phases produced by those two series of simulations can be obtained by
studying simulation snapshots (\eg Figure~\ref{HP_k3_snaps}) and computation of the pair
correlation functions.
Both the standard $g_\parallel(r_\parallel)$, $g_\perp(r_\perp)$ and 
$g^\mrm{mol}_\parallel(r_\parallel)$, $g^\mrm{mol}_\perp(r_\perp)$ were computed (see
Chapter~\ref{chap:three} for a definition). These functions are shown on 
Figure~\ref{HP_k3_grs}.\\
Little difference can be observed between the two sets of functions; 
$g_\parallel(r_\parallel)$ and $g^\mrm{mol}_\parallel(r_\parallel)$  indicate that as 
the density is increased, some 
short range features appear, corresponding to distances less than the molecular elongation.
Comparing this with the configuration snapshots suggests that those short range features
reflect the tendency of particles to align anti-parallel with their closest neighbors.
$g_\perp(r_\perp)$ and $g^\mrm{mol}_\perp(r_\perp)$ also fail to show signs of ordering, 
and little variation can be
observed throughout the density range considered. However, some signs of the preferred pairwise 
anti-parallel ordering are visible for the highest density in the form of a peak at $r\sim 0.9$.\\ 

The main conclusion to be drawn from these series of simulations is that the HP model with
$k=3$ shows rather surprising phase behaviour. With increased density, the particles exhibit short
ranged anti-parallel ordering. The rapid decrease in particle mobility, however leads to the 
formation of glassy phases and prevents the formation of nematic phases. Instead, at high
density the particles form a glassy phase made of small domains with local anti-parallel 
alignment.


%=================================================================================================
%=================================================================================================
\subsection{Particles with $k=5$.}

Due to the lack of ordering shown by particles with $k=3$ and following the idea that
increasing the molecular elongation stabilizes the nematic phase, a second study of the HP model
was made using particles with $k=5$. 
Systems of $N=1000$ particles were simulated using the Monte Carlo method 
in both the canonical and isothermal-isobaric ensembles and with inclusion of the `flip' moves.
Similar volume change algorithms and run lengths were used as for runs with $k=3$.
However, only a compression series was performed. The results obtained from this
(shown on Figure~\ref{fig:HP_k5_phaseDia}) indicate a somewhat
different phase behaviour for this
longer elongation.  
$P(\rho^{*})$ shows an inflexion point at $\rho^{*}\sim 0.12$, suggesting a transition to a more
ordered phase. This corresponds to a rapid increase in $\langle P_2\rangle$, having the common `S' shape of a
first order transition, which is consistent with ordering to a nematic phase. However the values of 
$\langle P_2\rangle$ are still short of that usually associated with nematic order. For this system, 
the computation of 
$\langle\delta r^2(n)\rangle$ (Figure~\ref{fig:HP_k5_dc}) shows that all 
configurations remained fluid throughout the pressure/density range considered.\\

%=====================
\picW = 7.0cm
\begin{figure}
	\centering
	\subfigure{\picL{HP_P-rho_k5.ps}}
	\subfigure{\picL{HP_P2-rho_k5.ps}}
	\caption{Results from the bulk compression simulation series with the HP model and
	$k=5$.}
	\label{fig:HP_k5_phaseDia}
\end{figure}
%=====================
%
%=====================
\picW = 12cm
\begin{figure}
	\centering
	\picL{HP_k5_cpress_dc.ps}
	\caption{Results for the diffusion of the HP model with $k=5$ for simulation in the
	isothermal-isobaric ensemble in a compression series. }
	\label{fig:HP_k5_dc}
\end{figure}
%=====================

The nature of the phases formed by this model was investigated by observing configuration
snapshots (\eg Figure~\ref{fig:HP_k5_d0.14_snaps}) and computing  pair correlation
functions (Figure~\ref{HP_k5_grs}). Again both the standard $g_\parallel(r_\parallel)$, 
$g_\perp(r_\perp)$ and  molecule based $g^\mrm{mol}_\parallel(r_\parallel)$, 
$g^\mrm{mol}_\perp(r_\perp)$ were computed.\\
%
As the density increased, the particles were found to order but also form layers with different 
orientations.  As a result, although the intralayer ordering was very high, this was not 
reflected by the value  of the overall nematic order parameter as the directions of the layers 
were very different, almost perpendicular in places. This is illustrated by the configuration shown on  
Figure~\ref{fig:HP_k5_d0.14_snaps}(b), obtained from a simulation performed in the canonical 
ensemble at
$\rho^{*}=0.14$. A side view of the same configuration (Figure~\ref{fig:HP_k5_d0.14_snaps}(c)), 
would however, suggest phase with a unique alignment direction and smectic-like ordering,
with the exception of those particles with different orientations which are sandwiched 
between the `layers'. 
This discrepancy shows the full extent of the non homogenous character of the high density
structures obtained here and explains the moderate values of $\langle P_2 \rangle$.\\
The likelihood of this type of configuration transforming into a smectic phase at higher
densities appears low as the particle mobilities at these high densities are rather low. 
This question has, however, been investigated by performing simulations in the 
isothermal-isobaric ensemble  where the three box lengths were allowed to vary independently.
This approach is known to facilitate the 
possible formation of smectic phases as the box accommodates the system's natural layer spacing. 
These simulations however lead to the same structures as were seen 
in the canonical ensemble.  Therefore, this multi-domain layered configuration seems to be the 
most stable state for this model at high pressures/densities.

%=====================
\picW = 5.5cm
\begin{figure}
	\centering
	\subfigure[$\rho^{*}=0.10$]{\pic{HP_box_NVT_k5_b_N1000_d0.1000_0.50M.ps}}
	
	\subfigure[$\rho^{*}=0.14$, bottom view]{\pic{HP_box_NVT_k5_b_N1000_d0.1400_0.50M_01.ps}}
	\hspace{2cm}
	\subfigure[$\rho^{*}=0.14$, side view]{\pic{HP_box_NVT_k5_b_N1000_d0.1400_0.50M_02.ps}}
	\caption{Configuration snapshots for the HP model with $k=5$ at $\rho^{*}=0.14$. Two
	views are presented the bottom of the simulation box (left) and a side view looking down
	along $\vecth{y}$ and $\vecth{z}$ pointing upward.}
	\label{fig:HP_k5_d0.14_snaps}
\end{figure}
%=====================


%============================
\picW = 7cm
\begin{figure}
	\centering
	\subfigure[$g_\parallel(r_\parallel)$]{\picL{HP_k5_grParallel.iso.ps}}
	\subfigure[$g_\perp(r_\perp)$]{\picL{HP_k5_grPerpLayer.iso.ps}}
	
	\subfigure[$g_\parallel(r_\parallel)$]{\picL{HP_k5_grParallel.fun.ps}}
	\subfigure[$g_\perp(r_\perp)$]{\picL{HP_k5_grPerpLayer.fun.ps}}
	\caption{Pair correlation functions for the HP model with $k=5$ computed with respect to
	$\vecth{n}$(top) and $\ui$ (bottom).}
	\label{HP_k5_grs}
\end{figure}
%============================


A quantitative insight into those phases is provided by the pair correlation functions as shown on
Figure~\ref{HP_k5_grs}. Again the functions computed with respect to $\vecth{n}$ and $\ui$ show
little qualitative differences. As the density is increased, $g_\parallel(r_\parallel)$ and 
$g^\mrm{mol}_\parallel(r_\parallel)$ display
a series of peaks accounting for the domain ordering. The distance between two successive peaks of about
$6\sigma_0$ corresponds to the distance between two layers separated by a small number of
interstitial particles with perpendicular orientation.
The lack of periodicity in these functions reveals 
that this phase is not smectic, however. The intra-layer, in contrast, is rather 
close to that of a smectic phase as $g^\mrm{mol}_\perp(r_\perp)$ shows up to four peaks with an average 
separation $\sim\so$  accounting for the successive parallel and antiparallel neighbours.\\

Thus, for the elongation of $k=5$, the HP model shows a rather surprising phase behaviour which is
subtly different from that observed for $k=3$. For the longer elongation, the model proved to
remain fluid over the pressure range considered. With increased pressure, a transition to an
ordered phase was observed. At high pressure, the particles formed layered domains with different
orientations and high, smectic-like, intra-layer order.
The HP model does not, however, fulfill the requirement of having the stable nematic phase 
needed to simulate the
confined flexoelectric particle systems mentioned at the beginning of this
Chapter. The modeling of pear-shaped particles which form nematic phases can, however, be achieved 
using a totally different route to the expression for the contact distance.
This new approach is described in the following Sections.

