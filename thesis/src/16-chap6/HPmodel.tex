

\section{The HP model}
\label{s:HP_model}

In 2001, Berardi \etal developed the first soft, single site model for non-centro\-sym\-metric
anisotropic particles
hereafter referred to as the soft pear (SP) model. This model uses a contact distance describing a
pear shape embedded within a Gay-Berne like potential. This model was taken as a base for the 
HP model described here. The HP model defines a steric potential $\mathcal{V}^\mrm{HP}$ between 
two pear-shaped objects whose contact distance is that of the SP model~\cite{BerardiRicci01}
as~:
%
\begin{equation}
	\mathcal{V}^{HP} = \left\{ 	%}
	\begin{array}{ccc}
		0	&\mathrm{if}	&r_{ij} \geq \sigma^{HP}(\ijr) \\
		\infty	&\mathrm{if}	&r_{ij} < \sigma^{HP}(\ijr) 
	\end{array}
	\right.
\end{equation}

where $\sigma^{HP}(\ijr)$ is the contact distance between two particles $i$ and $j$
with orientations $\ui$ and $\uj$ and and $\rij =\frac{\vect{r}_{ij}}{r_{ij}}$ 
where $\vect{r}_{ij}$ is the intermolecular separation. 
With this model, the contact distance is obtained using a numerical method 
following the approach of Zewdie~\cite{Zewdie98a,Zewdie98b}. The prototype shape of the particles is
defined using a set of two B\'ezier curves (Fig~\ref{fig:HP_bezier_k3}). The coordinates of the
control points of these, $q_{1..6}$, are given in Table~\ref{tble:HP_ctrlPts_k3}.

\picW = 12cm
\begin{figure}
	\centering
	\hspace*{2.5cm}\picL{HP_BzPear.ps}
	\caption{The B\'ezier curve used for the geometrical definition of the pear shaped of
	the HP model.}
	\label{fig:HP_bezier_k3}
\end{figure}

\begin{table}
	\centering	
	\begin{tabular}{||c||c||c||}
	\hhline{|t:=:t:=:t:=:t|}
	\hspace*{2mm} $q_i$ \hspace{2mm} &\hspace{2mm} x \hspace{2mm}	&\hspace{2mm} y \hspace{2mm}\\
	\hhline{|:=::=::=:|}
	$q_1$	&$-0.5$			&$0$		\\
	$q_2$	&$-0.5+ h\tan\alpha$	&$h$		\\
	$q_3$	&$0.5 - h\tan\alpha$	&$h$		\\
	$q_4$	&$0.5$			&$0$		\\
	$q_5$	&$-0.5 - h\tan\alpha$	&$-h$		\\
	$q_6$	&$0.5 + h\tan\alpha$	&$-h$		\\
	\hhline{|b:=:b:=:b:=:b|}
	\end{tabular}
	\caption{Coordinates of the B\'ezier control points for the HP model and $k=3$.}
	\label{tble:HP_ctrlPts_k3}
\end{table}


Following this a numerical
contact distance $\mathcal{L}(\ui,\uj,\rij )$ is computed for a given
set of $\ui, \uj$. This numerical distance is then fitted to a truncated Stone expansion as~:
\begin{eqnarray}
	%
	\sijr   &\simeq&  \mathcal{L}(\ui,\uj,\rij) \nonumber \\
	        &=& \sum_{L_1,L_2,L_3}\sigma_{L_1, L_2, L_3}S^{*L_1, L_2, L_3}(\ui,\uj,\rij)
	\label{eqn:Stone_sigma1}
\end{eqnarray}

where $S^{L_1, L_2, L_3}$ is a Stone function~\cite{Stone}, and the expansion coefficients 
$\sigma_{L_1, L_2,L_3}$ are given by~:
\begin{equation}
	%
	\sigma_{L_1, L_2, L_3} = \frac
	{
	\int \mathcal{L}(\ui,\uj,\rij)S^{L_1, L_2, L_3}
	(\ui,\uj,\rij)d\ui d\uj d\rij
	}
	{
	\int S^{*L_1, L_2, L_3}(\ui,\uj,\rij)S^{L_1, L_2, L_3}
	(\ui,\uj,\rij)d\ui d\uj d\rij
	}
	\label{eqn:Stone_sigma2}
\end{equation}

The non-zero coefficients $\sigma_{L_1, L_2, L_3}$ for particles with elongation $k=3$ and \\
$\{L_1, L_2, L_3\} \leq 6$ are given
in~\cite{BerardiRicci01}~; the corresponding coefficients for $k=5$ are
given in Table~\ref{tble:HP_sigma_k5}.\\

\begin{table}
    \centering
    \begin{tabular}{||cc||cc||cc||}
    \hhline{|t:==:t:==:t:==:t|}
      $[000]$ & $    1.90456 $ & $[011]$ & $    0.51113 $ & $[101]$ & $    0.51113 $  \\
      $[022]$ & $    2.01467 $ & $[202]$ & $    2.01467 $ & $[033]$ & $   -0.11376 $  \\
      $[303]$ & $   -0.11376 $ & $[044]$ & $    0.91479 $ & $[404]$ & $    0.91479 $  \\
      $[055]$ & $   -0.29937 $ & $[505]$ & $   -0.29937 $ & $[066]$ & $    0.41523 $  \\
      $[606]$ & $    0.41523 $ & $[110]$ & $   -0.03942 $ & $[121]$ & $   -0.45400 $  \\
      $[211]$ & $   -0.45400 $ & $[123]$ & $    0.59579 $ & $[213]$ & $    0.59579 $  \\
      $[132]$ & $    0.17137 $ & $[312]$ & $    0.17137 $ & $[143]$ & $   -0.27083 $  \\
      $[413]$ & $   -0.27083 $ & $[220]$ & $   -0.56137 $ & $[222]$ & $   -2.78379 $ \\
      $[224]$ & $    2.41676 $ & $[231]$ & $    0.31104 $ & $[321]$ & $    0.31104 $  \\
      $[233]$ & $    0.45382 $ & $[323]$ & $    0.45382 $ & $[242]$ & $    0.38115 $  \\
      $[422]$ & $    0.38115 $ & $[244]$ & $   -1.69388 $ & $[424]$ & $   -1.69388 $  \\
      $[246]$ & $    1.40664 $ & $[426]$ & $    1.40664 $ & $[330]$ & $   -0.07836 $  \\
      $[440]$ & $   -0.17713 $ & $[442]$ & $   -0.52246 $ &    &   \\
    \hhline{|b:==:b:==:b:==:b|}
    \end{tabular}
    \caption{The non zero $\sigma_{L_1, L_2, L_3}$ coefficients for $k=5$.}
    \label{tble:HP_sigma_k5}
\end{table}


The original study of the SP model revealed stable nematic and smectic phases and, using an
appropriate energy parameterisation, the same phases with net polar order were also obtained.
A reasonable expectation, therefore, is that
the steric version of this model should exhibit at least a stable nematic phase.
Such a correlation is found virtually in all soft LC models and their hard-particle equivalent.
The best example of this is seen on comparing the Gay-Berne~\cite{deMiguelRull91,deMiguelRull91a} 
and hard Gaussian overlap phase behaviours~\cite{DeMiguelDelRio01}.
As the former is reduced to its steric equivalent, the smectic phases disappear whereas the 
nematic-isotropic phase transition not only persists, but remains at virtually the same density.\\

As the HP model can not be reduced to the interaction between a particle and a point, the actual
shape of the particle and, hence, the accuracy with which this model represents the geometrical
B\'ezier curve is not available. However, for the two elongation considered in this study, 
some insight into the potential possible behaviour can be obtained by computation 
of the contact surfaces between two particles. 
Such surfaces show the location of the contact point between a particle $i$ located at the origin
with fixed orientation $\ui$ and a particle $j$ with fixed orientation $\uj$ whose position
is uniformly distributed on the unit sphere. Example surfaces, for
parallel particles ($\ui=\uj=\vecth{z}$), anti-parallel particles ($\ui=-\uj=\vecth{z}$) 
and the T-geometry ($\ui=\vecth{z}$, $\uj=\vecth{x}$), are shown on 
Figure~\ref{fig:HP_k3_contactSurf}.

\picW = 7cm
\begin{figure}
	\centering
	\subfigure[parallel particles]{\pic{HP_3DProf_k3_0.ps}}
	\subfigure[anti-parallel particles]{\pic{HP_3DProf_k3_PI.ps}}
	\subfigure[T-geometry, x-z view]{\pic{HP_3DProf_k3_0.5PI.ps}}
	\subfigure[T-geometry, y-z view]{\pic{HP_3DProf_k3_0.5PI_2.ps}}
	\caption{Contact surfaces for HP particles with $k=3$.}
	\label{fig:HP_k3_contactSurf}
\end{figure}

\picW = 7cm
\begin{figure}
	\centering
	\subfigure[parallel particles]{\pic{HP_3DProf_k5_0.ps}}
	\subfigure[anti-parallel particles]{\pic{HP_3DProf_k5_PI.ps}}
	\subfigure[T-geometry, x-z view]{\pic{HP_3DProf_k5_0.5PI.ps}}
	\subfigure[T-geometry, y-z view]{\pic{HP_3DProf_k5_0.5PI_2.ps}}
	\caption{Contact surfaces for HP particles with $k=5$.}
	\label{fig:HP_k5_contactSurf}
\end{figure}

Generally, these show the expected interaction, but it is evident that several non-convex
fractures are present; this raises the prospect of possible
enhanced stability for some
configurations, specifically for those of interlocking side by side anti-parallel particles. 
The consequence of  this would be to unduly stabilize 
these configurations and as a result prevent particles from `sliding' along one another.
Comparison of these contact surfaces for different elongations reveals different shape and
convexity behaviours. Thus rather different phase behaviours are to be expected for systems of 
particles with different elongations.\\
%
Problems related to non convexity of contact surfaces have previously been found by
Williamson and Jackson~\cite{WilliamsonJackson98} in their study of systems of linear hard 
sphere chains (LHSC). For that model, the authors noticed that the lack of convexity in the  
the particle-particle interaction surfaces lead to the formation of glassy phases. For the LHSC
model, this problem
could be resolved by the inclusion of reptation moves in the Monte Carlo sequence. 
This, however, is not possible with the single site HP model; the consequence of this is
addressed in Section~\ref{s:HPresults}.\\
%
Also the lack of convexity of the contact surfaces for the HP model would suggest that the HP
model might display a rather different phase behaviour than that obtained using
the SP model. This is because the energy minima of the soft tapered model systematically 
correspond to molecular separations greater than the contact distance and, therefore, the effects
of non convexity might not have been accessible to the models considered by the Italian group.



