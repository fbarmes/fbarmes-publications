


\section{Phase behaviour of the PHGO model.}


In order to test the PHGO model applied to the B\'ezier pear shape, bulk constant $NPT$ MC simulations
were performed on systems of $N=1000$ particles with elongations $k=3,4$ and $5$. 
As with the HP model simulations, orientation inversion moves were also included in the MC scheme. 
These accounted for $20\%$ of the total number of attempted moves. The volume change scheme was 
performed on average, once two sweeps and allowed each box side to change its length independently. 
Typical runs comprised of $0.5.10^6$ sweeps for equilibration and production. Close to phase
transitions, 
additional runs were performed on a case by case basis  to ensure that equilibration was
achieved.  All simulations were performed in a compression sequence.


%===============================================================================================
\subsection{Particles with $k=3$ and 4.}


%==================================================
\picW = 7cm
\begin{figure}
	\subfigure[$k=3$]{\picL{GBP_P-rho_k3.ps}\picL{GBP_P2-rho_k3.ps}}
	\subfigure[$k=4$]{\picL{GBP_P-rho_k4.ps}\picL{GBP_P2-rho_k4.ps}}
	\caption{Phase diagram for the PHGO pears with $k=3$ and 4.}
	\label{fig:PHGO_phaseDia_k3k4}
\end{figure}
%==================================================

The equation of state and order parameter behaviour obtained from simulations with $k=3$ 
and $4$ are shown in 
Figure~\ref{fig:PHGO_phaseDia_k3k4}. The phase behaviour for these two elongations is fairly
similar; both $P(\rho^{*})$ curves show an inflexion point, respectively at $\rho^{*}\sim 0.30$ and
$\sim 0.20$, which corresponds to an increase of $P_2(\rho^{*})$ to values of the order
of $0.15$ immediately followed by a rather steep decrease.\\ 
The `plateau' in $P(\rho^{*})$ indicates some sort of phase transition to a more ordered phase
but, as this is only hardly reported on $P_2$, this phase change does not corresponds to 
an isotropic-nematic phase transition.
Some more insight into the phase behaviour can be obtained by observation of configuration
snapshots (\eg Figure~\ref{fig:GBP_snaps_k3k4}).

%==================================================
\picW = 5.5cm
\begin{figure}
	\subfigure[$k=3$ isotropic]{\pic{GBP_box_NPT_k3-2_N1000_P3.0000_0.50M_S1.1.ps}}
	\hspace{2cm}
	\subfigure[$k=3$ domain ordered]{\pic{GBP_box_NPT_k3-2_N1000_P7.0000_0.50M_S1.1.ps}}
	
	\subfigure[$k=4$ isotropic]{\pic{GBP_box_NPT_k4_N1000_P1.8000_0.50M.ps}}
	\hspace{2cm}
	\subfigure[$k=4$ domain ordered]{\pic{GBP_box_NPT_k4_N1000_P5.0000_0.50M.ps}}
	\caption{Configuration snapshots for the pear PHGO model and $k=3$ and 4.}
	\label{fig:GBP_snaps_k3k4}
\end{figure}
%==================================================

These show that upon increasing the pressure, both systems underwent a phase transition from
isotropic to the what we term as a domain ordered BILAYER phase where, the particles form 
domains in which 
the local order is very high but where the orientation changes from one domain to the other. 
Unlike with the HP model, this seems to have been a transition between two genuinely
liquid states as the systems maintained high mobility throughout the density ranges considered here
(see Figure~\ref{fig:PHGO_dc_k3k4}). Also, the configuration snapshots suggest continuous
orientation changes in moving from one domain to another (\eg Figure~\ref{fig:GBP_snaps_k3k4}b
and d,) unlike the very sharp domain
boundaries seen in the HP systems (\eg Figure~\ref{fig:HP_k5_d0.14_snaps}b and c.)\\

\picW = 7cm
\begin{figure}
	\centering
	\subfigure[$k=3$]{\picL{GBP_k3-2_cpress_dc.ps}}
	\subfigure[$k=4$]{\picL{GBP_k4_cpress_dc.ps}}
	\caption{Evolution of $\langle\delta r^2(n)\rangle$ for the pear PHGO model and $k=3$
	and 4.}
	\label{fig:PHGO_dc_k3k4}
\end{figure}

More quantitative insight into those domain ordered phases have been obtained through
computation of the pair correlation functions. Both $g_\parallel(r_\parallel)$,
$g^\mrm{mol}_\parallel(r_\parallel)$ (Figure~\ref{fig:gr_PHGO_k4}(a)) and $g_\perp(r_\perp)$,
$g^\mrm{mol}_\perp(r_\perp)$ (Figure~\ref{fig:gr_PHGO_k4}(b)) have been computed  for $k=4$ as
the domains are best observed for this elongation.\\
%
$g_\parallel(r_\parallel)$ and $g_\perp(r_\perp)$ show some short ranged structure with 
increasing density but do not give much information about the structures observed 
on the snapshots because the choice of reference ($\vecth{n}$) for the computation of these 
curves does not allow one to `follow' the orientation of the particles in the domains.\\
A better reference is the molecular orientation as used when calculating 
$g^\mrm{mol}_\parallel(r_\parallel)$
and $g^\mrm{mol}_\perp(r_\perp)$ (Figure~\ref{fig:gr_PHGO_k4}(c) and (d)).
In this case, the appearance of almost periodic  behaviour in $g^\mrm{mol}_\parallel(r_\parallel)$ 
shows some smectic-like ordering with increased density. $g^\mrm{mol}_\perp(r_\perp)$, however,
shows that those domain are fairly short ranged as only three peaks can be observed, the first one 
accounting for the direct antiparallel neighbors and the subsequent maxima correspond to further
shells of parallel and anti-parallel particles.\\


\picW = 7cm
\begin{figure}
	\centering
	\subfigure[$g_\parallel(r_\parallel)$]{\picL{GBP_k4_grParallel.iso.ps}}
	\subfigure[$g_\perp(r_\perp)$]{\picL{GBP_k4_grPerpLayer.iso.ps}}

	\subfigure[$g_\parallel(r_\parallel)$]{\picL{GBP_k4_grParallel.fun.ps}}
	\subfigure[$g_\perp(r_\perp)$]{\picL{GBP_k4_grPerpLayer.fun.ps}}
	\caption{Pair correlation functions for the pear PHGO model of $k=4$ computed with respect to 
	$\vecth{n}$ (top) and to $\ui$ (bottom).}
	\label{fig:gr_PHGO_k4}
\end{figure}

Thus the PHGO model applied to the B\'ezier pears with elongation $k=3$ and $4$ has shown some very 
interesting phase behaviour. Despite the improved contact surfaces, that prevented the formation 
of glassy phases, still no nematic phases could be observed. Rather, upon increasing the
density, the systems underwent a transition from isotropic to a domain ordered phase.

%===============================================================================================
\subsection{Particles with $k=5$.}

The phase behaviour for particles with $k=5$ proved to be qualitatively different from the shorter
elongations and resembled that of other common LC models, such as the HGO fluid.

$P^{*}(\rho^{*})$ shows a plateau at $P^{*}\sim 1.2$ which corresponds to an `S' shaped increase in 
$P_2(\rho^{*})$ to values typical of a nematic
phase. This phase lacked polar order throughout the density range
considered as shown by the low values  of $P_1(\rho^{*})$. Configuration snapshots illustrating 
these isotropic and nematic phases are shown on Figure~\ref{fig:GBP_k5_snaps}(a) and (b).
%
%
%==================================================
\picW = 7cm
\begin{figure}
	\picL{GBP_P-rho_k5.ps}
	\picL{GBP_P2-rho_k5.ps}
	\caption{Phase diagram for the PHGO pears with $k=5$.}
	\label{fig:PHGO_phaseDia_k5}
\end{figure}
%==================================================
%
%
Surprisingly a second feature can be observed on the phase diagram in the form of a second inflexion
point in $P^{*}(\rho^{*})$ at higher pressure. This corresponds to a second, smaller `S' shaped
increased in $P_2(\rho^{*})$ marking a second phase transition, this time to a smectic phase.
As the PHGO model is simply a generalization of the Hard Gaussian Overlap model to 
non-centrosymmetric shapes, this second phase transition was not expected. Observation of the 
configuration snapshots at the highest densities confirmed the existence of the smectic phase 
(\eg Figure~\ref{fig:GBP_k5_snaps}(c) .)
All the phases found can be shown to be fluid as the gradient in $\left< \delta r^2(n) \right>$
stayed rather high throughout the density range considered here (Figure~\ref{fig:GBP_k5_dc}).\\

%==================================================
\picW = 12cm
\begin{figure}
	\centering
	\picL{GBP_k5_cpress_dc.ps}
	\caption{Evolution of $\left< \delta r^2(n) \right>$ as a function of pressure for the
	pear PHGO model with $k=5$.}
	\label{fig:GBP_k5_dc}
\end{figure}
%==================================================


%==================================================

\begin{figure}
	\centering
	\picW = 5cm
	\subfigure[isotropic]{\pic{GBP_box_k5_P1.0000.ps}}
	\hspace{2cm}
	\picW = 5.5cm
	\subfigure[nematic]{\pic{GBP_box_k5_P1.5000.ps}}
	\picW = 5.5cm
	\subfigure[smectic]{\pic{GBP_box_k5_P2.8000.ps}}
	\caption{Configuration snapshots for the PHGO pears with $k=5$.}
	\label{fig:GBP_k5_snaps}
\end{figure}
%==================================================


%==================================================
\picW = 7cm
\begin{figure}
	\centering
	\subfigure[$g_\parallel(r_\parallel)$]{\picL{GBP_k5_grParallel.ps}}
	\subfigure[$g_\perp(r_\perp)$]{\picL{GBP_k5_grPerpLayer.ps}}
	\caption{Pair correlation functions for the pear PHGO model of $k=5$ computed with respect to 
	$\vecth{n}$.}
	\label{fig:GBP_k5_gr}
\end{figure}
%==================================================


In order to gain more insight into the phases found here but also to characterize the
smectic phase more precisely, the pair correlation functions have been computed parallel and 
perpendicular with respect to the nematic director $\vecth{n}$. Only this reference frame has
been taken here as, in the absence of domain ordered phases, 
computation of $g^\mrm{mol}_\parallel(r_\parallel)$ and $g^\mrm{mol}_\perp(r_\perp)$ was not
required.\\
%
As shown in Figure~\ref{fig:GBP_k5_gr}, increase in pressure leads to the development of
periodic oscillations in $g_\parallel(r_\parallel)$; the amplitudes of these fluctuations 
were found to grow with increased pressure. In the smectic phase, $g_\parallel(r_\parallel)$ 
became fully periodic, the repeating pattern being composed of one main 
peak between two smaller ones.\\
At $P^{*}=2.8$, the distance between the two main peaks was about  
$7.389\sigma_0 \approx 1.5k\sigma_0$. This corresponds to  the separation of layers with 
the same polar orientation. The two smaller peaks account for the two spacing of antiparallel 
layer arrangement.  The presence of two peaks (rather than one) can be explained by 
the difference between the preferred tail-tail and  head-head contact distances. The area under 
these peaks is about half that under the main peaks as each accounts for one type of 
relative alignment (head-head \textbf{or} tail-tail) whereas the main peak account for two 
types of interaction (head-tail pointing up \textbf{and} head-tail pointing down). At the same
pressure of $P^{*} = 2.8$, the peak separation of about $\so$ shows the smectic to be a \smA
phase. Moreover, the interdigitation between the layers as revealed by the configuration
snapshots (\eg Figure~\ref{fig:GBP_k5_snaps}(c)) further identifies this smectic phase as a
bilayered smectic $\mrm{A_2}$.\\

Comparison of $g_\parallel(r_\parallel)$ for different values of $P^{*}$ in the smectic phase
shows an interesting compressibility behaviour and also allows to identify the different 
peaks with their associated particles interaction geometry. Upon increasing the pressure, 
the system density rises  and the intra-layer separation decreases whereas the bilayer 
separation increases  (Figure~\ref{fig:GBP_k5_gr}(a) and (b)). 
From the measured $g_\parallel(r_\parallel)$ data in the pressure range $[2.4:3.8]$,
it is found that the distance between the 
main peaks, which corresponds to the separation of the bilayers, increases from $7.38$ 
to $7.66$. The distance from the main peak to the first minor peak, which corresponds to 
the strongly interdigitating `tail-tail' configuration increases from $2.49$ to $2.76$, 
whereas that to the second minor peak, corresponding to the weakly interdigitating `head-head' 
alignment remains effectively constant at $4.85$. Thus, the in-plane compression induced by 
this increase in pressure leads to a 10\% increase in the separation within the interdigitated 
bilayers that comprise the smectic ${\rm A_2}$ phase.\\
Figure~\ref{fig:GBP_k5_grDetails} shows the period of $g_\parallel(r_\parallel)$ at
$P^{*}=2.8$~; the distance $C-H_1$ corresponds to the bilayer separation, that is the separation
between particles with the same polar orientation ($\lhd\lhd$ and $\rhd\rhd$). The $C-L_1$
distance corresponds to the separation between the strongly interdigitated particles in a 
tail-tail ($\rhd\lhd$) configuration and the $C-L_2$ distance corresponds to the separation
between particles in a head-head ($\lhd\rhd$) configuration.\\

\picW = 10cm
\begin{figure}
	\centering
	\picL{GBP_k5_grParallelDetails.ps}
	\caption{Details of $g_\parallel(r_\parallel)$ for a system of PHGO particles with $k=5$
	and $P^{*}=2.8$, which corresponds to a smectic phase.}
	\label{fig:GBP_k5_grDetails}
\end{figure}


The results from the simulation of the pear PHGO model with $k=5$ thus shows an interesting
phase behaviour that includes stable nematic and interdigitated \smA phases. As a result 
the prerequisite 
condition for the creation of the model for surface induced switching is fulfilled and pear 
PHGO particles can be used in confined geometries in an attempt to achieve the main goal of this
thesis. Results  of the work performed to this end are presented in Chapter~\ref{chap:seven}.




