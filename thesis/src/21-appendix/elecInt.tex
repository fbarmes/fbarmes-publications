

\chapter{Particle-field electrostatic interactions}
\label{chap:B}

Here, an expression is derived for the interaction between a particle and uniform electric field.
Two types of interaction are considered here, namely the dielectric and dipolar interactions.
The former describes the interaction between the polarisability of a liquid crystal molecule 
and an applied electric
field, while the latter is the interaction between a permanent dipolar moment and the
applied field.\\

The setup considered is that of a particle $i$ with orientation 
$\ui = (\cos\phi\sin\theta, \- \sin\phi\sin\theta, \cos\theta)$ subject to a constant
electric field $\vect{E}$ as shown in Figure~\ref{fig:partFieldSetup}. Although in principle, the
particle should be considered to be subject to a constant electric displacement $\vect{D}$ 
rather than a constant field, the approximation of constant $\vect{E}$ is made so as to avoid
the complication of including the Maxwell relations.


\picW = 6cm
\begin{figure}
	\centering
	\pic{partFieldSetup.ps}
	\caption{Schematic representation of the geometry considered to calculate $U_{pf}$}
	\label{fig:partFieldSetup}
\end{figure}



%================================================================================================
%================================================================================================
\section{Dielectric interaction}

The field induced polarisation on the dielectric is given by~:
\begin{equation}
	\vect{P} = \epsilon_0.\gvect{\chi_e}.\vect{E}
\end{equation}
%
where $\vect{P}$ is the polarisation induced by the electric field $\vect{E}$,
$\epsilon_0$ is the dielectric permittivity and $\gvect{\chi_e}$ is the 
dielectric susceptibility tensor~:
%
\begin{equation}
	\gvect{\chi_e} = \left(
	\begin{array}{ccc}
		\chi_{e\perp}	&0			&0	\\
		0			&\chi_{e\perp}	&0	\\
		0			&0			&\chi_{e\parallel}
	\end{array}
	\right).
\end{equation}

The dielectric energy of the particle is then given by~:
\begin{equation}
	U_{die} = -\frac{1}{2}\dotProd{\vect{D}}{\vect{E}}
\end{equation}
%
with $\vect{D}$ the electric displacement
\begin{equation}
	\vect{D} = \epsilon_0\vect{E} + \vect{P}.
\end{equation}

Therefore, the electric energy can be expressed as~:
\begin{equation}
	U_{die} = -\frac{1}{2} \dotProd{\lp \epsilon_0\vect{E} + \vect{P} \rp}{\vect{E}}
\end{equation}

and, since $\mrm{d}\vect{E} = 0$, $\mrm{d}U_{die}$ can be expressed as~:
\begin{equation}
	\mrm{d}U_{die} = -\frac{1}{2}\vect{E}\mrm{d}\vect{P}
	\label{eqn:dUe}
\end{equation}

The expression for $\mrm{d}\vect{P}$ can be obtained by using
$\vect{E}$ and $\vect{P}$ expressed in the molecular frame in term of $\vect{E_\parallel}$,
$\vect{P_\parallel}$ and $\vect{E_\perp}$, $\vect{P_\perp}$ the components of $\vect{E}$ 
and $\vect{P}$ respectively parallel and perpendicular to $\ui$. Thus~:
%
\begin{eqnarray}
	\vect{E} &=& \vect{E_\parallel} + \vect{E_\perp}	\\
	\vect{E_\parallel} &=& \dotProdP{\vect{E}}{\ui}\ui	\\
	\vect{E_\perp} &=& \vect{E} - \dotProdP{\vect{E}}{\ui}\ui
\end{eqnarray}

similarly~:
%
\begin{eqnarray}
	\vect{P} &=& \vect{P_\parallel} + \vect{P_\perp}	\\
	\vect{P_\parallel} &=& \epsilon_0 \chi_{e\parallel}\vect{E_\parallel}	\\
	\vect{P_\perp} &=& \epsilon_0 \chi_{e\perp}\vect{E_\perp}
\end{eqnarray}

which leads to~:
\begin{eqnarray}
	\vect{P} &=& \epsilon_0
		\lp \chi_{e\parallel} \vect{E_\parallel}
		 + \chi_{e\perp}\vect{E_\perp} \rp \\
	%
	\vect{P} &=& \epsilon_0 \left[
		\chi_{e\parallel}  \dotProdP{\vect{E}}{\ui}\ui
		+ \chi_{e\perp} \lp \vect{E} - \dotProdP{\vect{E}}{\ui}\ui \rp
		\right]	 
\end{eqnarray}

and with~:
\begin{equation}
	\delta\epsilon =
		\chi_{e\parallel} - \chi_{e\perp} 
\end{equation}

\begin{equation}
	\vect{P} = \epsilon_0\lp \chi_{e\perp}\vect{E} +
	\delta\epsilon\dotProdP{\vect{E}}{\ui}\ui  \rp	
\end{equation}

and therefore~:
\begin{equation}
	\mrm{d}\vect{P} = \epsilon_0\delta\epsilon\left[
	\dotProdP{\vect{E}}{\mrm{d}\ui}\ui + \dotProdP{\vect{E}}{\ui}\mrm{d}\ui
	\right]
\end{equation}

Equation~\ref{eqn:dUe} then becomes~:
\begin{eqnarray}
	\mrm{d}U_{die} &=& -\frac{1}{2}\epsilon_0\delta\epsilon \vect{E}\cdot
		\left[
		\dotProdP{\vect{E}}{\mrm{d}\ui}\ui + \dotProdP{\vect{E}}{\ui}\mrm{d}\ui
		\right]	 \\
	%
	\mrm{d}U_{die} &=& -\epsilon_0\delta\epsilon
	\dotProdP{\vect{E}}{\mrm{d}\ui}\dotProdP{\vect{E}}{\ui}
\end{eqnarray}

Taking the electric field to be $\vect{E} = E\vecth{E}=E\vecth{z}$, an expression for 
$\mrm{d}U_e$ in terms of $\theta$ can be obtained as~:
%
\begin{eqnarray}
	\mrm{d}U_{die} &=& -\epsilon_0\delta\epsilon\lp -E \sin\theta\mrm{d}\theta\rp
		\lp E \cos\theta\rp	\\
	\mrm{d}U_{die} &=& \epsilon_0\delta\epsilon E^2 \sin\theta\cos\theta\mrm{d}\theta
\end{eqnarray}

Thus for every $\theta_0$ the electric energy corresponding to the dielectric interaction is
given by~:
\begin{gather}
	U_{die} = \epsilon_0\delta\epsilon E^2 \int^{\theta_0}_{0} 
		\sin\theta\cos\theta\mrm{d}\theta  \\
	%
	U_{die} = \epsilon_0\delta\epsilon E^2 \lp -\frac{1}{2}\cos^2\theta\rp \\
	\fbox{$U_{die} = -\frac{1}{2}\epsilon_0\delta\epsilon\dotProdP{\vect{E}}{\ui}^2$}
\end{gather}


%================================================================================================
%================================================================================================
\section{Dipolar interaction}

The case of the dipolar interaction is much simpler than the previous one. The field interacts
with a dipolar moment with energy:

\begin{equation}
	U_{dip} = -\dotProdP{\gvect{\mu}}{\vect{E}}
\end{equation}

For $\gvect{\mu}$, the dipolar moment, given by~:
\begin{equation}
	\gvect{\mu} = \mu\ui,
\end{equation}

the energy corresponding to the dipolar interaction is given by~:
\begin{equation}
	\fbox{$U_{dip} = -\mu\dotProdP{E}{\ui}$}
\end{equation}

%================================================================================================
%================================================================================================
\section{Particle-field interaction}


The energy $U_{pf,i}$ of the particle-field interaction on one particle $i$ is the sum of 
the dielectric and dipolar contribution and hence~:
\begin{eqnarray}
	U_{pf,i} &=& U_{die} + U_{dip}	 \\
	U_{pf,i} &=& -\frac{1}{2}\epsilon_0\delta\epsilon\dotProdP{\vect{E}}{\ui}^2
		-\mu\dotProdP{E}{\ui}.
\end{eqnarray}

As a result, the total energy per particle corresponding to the particle-field interaction 
is simply~:
\begin{equation}
	\fbox{$U_{pf} = \frac{1}{N}\sum^{N}_{i=1} -\frac{1}{2}\epsilon_0\delta\epsilon\dotProdP{\vect{E}}{\ui}^2
		-\mu\dotProdP{E}{\ui}$}
\end{equation}




