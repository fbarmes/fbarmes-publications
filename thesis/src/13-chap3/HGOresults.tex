

\section{Computer simulations}

In this Section, results from Monte Carlo computer simulations of bulk systems of hard Gaussian
overlap particles are presented. Although most of
these simulations do not lead to new results (as the phase diagram of the model is already
known,) they do provide a good test-bed for the simulation code used to produce the 
novel results given in Chapter~\ref{chap:four} to~\ref{chap:six}.\\
Two sets of results are presented here. First the bulk simulation of calamitic particles is
considered using two elongations $k=3$ and $5$. The phase diagrams from these simulations are
compared with those extracted from the literature~\cite{PadillaVelasco97,DeMiguelDelRio01}
in order to validate the simulation code. Results from the simulation of discotic particles are
then presented using the elongations $k=1/3$ and $1/5$.



%=================================================================================================
%=================================================================================================
\subsection{Calamitic particles}
\label{HGO_cal_res}
Calamitic mesogens have been simulated using the hard Gaussian overlap model in the canonical
and isothermal-isobaric ensembles and in compression sequences. 
Systems of $N=1000$ particles with elongation $k=3$ and $5$  were used. Typical runs consisted 
of $5.10^5$ to $1.10^6$ sweeps (\ie attempted moves per particle) for both equilibration and
production. The phase diagrams of the model were generated by computing $P^{*}(\rho^{*})$ 
from the constant $NPT$ runs and $\Ptwo(\rho^{*})$ from both sets of runs. Those are shown on
Figure~\ref{fig:HGO_phaseDia_k3_5} respectively for $k=3$ and $5$.\\


\picW = 7cm
\begin{figure}
	\centering
	\subfigure[]{\picL{HGO_P-rho_k3.ps}\picL{HGO_P2-rho_k3.ps}}
	\subfigure[]{\picL{HGO_P-rho_k5.ps}\picL{HGO_P2-rho_k5.ps}}
	\caption{Phase diagrams obtained from Monte Carlo simulations in the constant $NPT$ and
	constant $NVT$ ensembles of systems of $N=1000$ hard Gaussian overlap particles with
	$k=3$(a) and $k=5$(b)}
	\label{fig:HGO_phaseDia_k3_5}
\end{figure}




\picW = 5cm
\begin{figure}
	\centering
	\subfigure[$k=3$, $P^{*}=3.5$]{\pic{HGO_box_k3_iso.ps}}
	\hspace*{1cm}
	\subfigure[$k=3$, $P^{*}=8.0$]{\pic{HGO_box_k3_nem.ps}}
	
	\subfigure[$k=5$, $P^{*}=0.8$]{\pic{HGO_box_k5_iso.ps}}
	\hspace*{1cm}
	\subfigure[$k=5$, $P^{*}=1.5$]{\pic{HGO_box_k5_nem.ps}}
	\caption{Typical configuration snapshots obtained from constant pressure Monte Carlo
	simulation of systems of $N=1000$ HGO particles with elongation $k=3$(a,b) and $5$(c,d).}
	\label{fig:HGOSnaps_k3_5}
\end{figure}


For both elongations the $P^{*}(\rho^{*})$ curves show a `plateau' characteristic of a first order
transition. These correspond on $\Ptwo(\rho^{*})$ to a sharp, `S'-shaped increase in the nematic
order parameter from values corresponding to an isotropic phase to those consistent with a
nematic phase. Observation of the configuration snapshots (\eg Figure~\ref{fig:HGOSnaps_k3_5}) 
confirms this. From these data $P^{*}_{in}$, the pressure at the isotropic-nematic coexistence and
$\rho^{*}_{i}$ and $\rho^{*}_{n}$ respectively the density at coexistence of the isotropic and
nematic phases can be estimated. Since the isotropic to nematic transition was not of specific
interest in this thesis, those have not been determined with the great accuracy that
techniques such as thermodynamic integration allow~\cite{CampMason96,CampAllen97}. Rather, 
$\rho^{*}_{i}$ and $\rho^{*}_{n}$ are taken to be the densities corresponding respectively to the
beginning and end of the `plateau' in $P^{*}(\rho^{*})$. $P^{*}_{in}$ is taken to be the
average of the pressures corresponding to $\rho^{*}_{i}$ and $\rho^{*}_{n}$.
The coexistence data are shown in Table~\ref{tble:HGOTransitionData_k3_5} 
along with the results from Padilla and Velasco~\cite{PadillaVelasco97} and de Miguel 
and Mart\'{\i}n del R\'{\i}o~\cite{DeMiguelDelRio01} for comparison.

\begin{table}
\centering
\begin{tabular}{||c||c||c||c||c||c||c||}
\hhline{|t:=:t:=:t:=:t:=:t:=:t:=:t:=:t|}
Source		&$k$	&System size	&$P^{*}_{in}$	&$\rho^{*}_i$	&$\rho^{*}_n$	&$\frac{\rho^{*}_n -\rho^{*}_i}{\rho^{*}_i}$	\\
\hhline{|:=::=::=::=::=::=::=:|}
\cite{PadillaVelasco97}	&3	&$256$	&$4.50$		&$0.290$	&$0.299$	&$0.031$	\\
\cite{DeMiguelDelRio01}	&3	&$500$	&$4.92$		&$0.299$	&$0.304$	&$0.019$	\\
this study		&3	&$1000$	&$4.975$	&$0.299$	&$0.309$	&$0.033$	\\
\hhline{|:=::=::=::=::=::=::=:|}
\cite{PadillaVelasco97}	&5	&$256$	&$0.880$	&$0.113$	&$0.120$	&$0.062$	\\
\cite{DeMiguelDelRio01}	&5	&$500$	&$0.996$	&$0.119$	&$0.127$	&$0.067$	\\
this study		&5	&$1000$	&$1.025$	&$0.122$	&$0.127$	&$0.040$	\\
\hhline{|b:=:b:=:b:=:b:=:b:=:b:=:b:=:b|}
\end{tabular}
\caption{Comparison of the isotropic-nematic transition data for the HGO model and $k=3$ and $5$
with existing results.}
\label{tble:HGOTransitionData_k3_5}
\end{table}

The comparison shows that the results obtained here are fully compatible with those obtained 
in~\cite{PadillaVelasco97} and~\cite{DeMiguelDelRio01}. The slight numerical differences can be
attributed to the differences in system sizes. 
de Miguel~\cite{DeMiguel92} has already shown that the effect of increasing system sizes on the
isotropic-nematic transition of system of Gay-Berne particles is to shift the transition to
higher densities or lower temperatures. This shift can be observed through the shift of \Ptwo
and $\rho^{*}_{i}$ and $\rho^{*}_{n}$ to higher densities or lower temperature with increased
system sizes. Another effect of system-size noticeable in de Miguel's results is a slight
strengthening of the IN transition with bigger systems.
The good agreement shown here validates the accuracy of the Monte Carlo simulation code 
used in this study.


%=================================================================================================
%=================================================================================================
\subsection{Discotic particles}

The Monte Carlo simulation code was subsequently applied to the modeling of discotic particles 
which have not been considered in the literature. These systems were simulated using 
methods adopted with the calamitic
particles, that is using Monte Carlo simulations in the canonical and isothermal-isobaric
ensembles. Systems of $N=1000$ particles with elongations $k=1/3$ and $1/5$ were simulated 
in compression sequences, using similar run lengths to those employed with the prolate elongaions. 
The phase diagrams obtained from these simulations are shown on Figure~\ref{fig:HGO_phaseDia_k0.33_0.2} 
for $k=1/3$ and $k=1/5$.\\

\picW = 7cm
\begin{figure}
	\centering
	\subfigure[$k=1/3$]{\picL{HGO_P-rho_k0.33.ps}\picL{HGO_P2-rho_k0.33.ps}}
	\subfigure[$k=1/5$]{\picL{HGO_P-rho_k0.2.ps}\picL{HGO_P2-rho_k0.2.ps}}
	\caption{Equation of states obtained from Monte Carlo simulations in the constant $NPT$ and
	$NVT$ ensembles of systems of $N=1000$ hard Gaussian overlap particles with
	$k=1/3$(a) and $k=1/5$(b).}
	\label{fig:HGO_phaseDia_k0.33_0.2}
\end{figure}



\picW = 5cm
\begin{figure}
	\centering
	\subfigure[$k=1/3$, $P^{*}=25.0$]{\pic{HGO_box_k0.33_iso.ps}}
	\hspace*{1cm}
	\subfigure[$k=1/3$, $P^{*}=59.0$]{\pic{HGO_box_k0.33_nem.ps}}
	
	\subfigure[$k=1/5$, $P^{*}=14.0$]{\pic{HGO_box_k0.2_iso.ps}}
	\hspace*{1cm}
	\subfigure[$k=1/5$, $P^{*}=30.0$]{\pic{HGO_box_k0.2_nem.ps}}
	\caption{Typical configuration snapshots obtained from constant pressure Monte Carlo
	simulations of systems of $N=1000$ HGO particles with elongation $k=1/3$(a,b) and $1/5$(c,d).}
	\label{fig:HGOSnaps_k0.33_0.2}
\end{figure}

These results show a similar behaviour to that observed with the calamitic particles. Both
$P^{*}(\rho^{*})$ curves show a `plateau' characteristic of a first order phase
transition which corresponds to the typical sharp `S'-shaped increase in $\Ptwo(\rho^{*})$
indicating an isotropic to discotic-nematic phase transition. This is further confirmed by
observation of configuration snapshots, \eg Figure~\ref{fig:HGOSnaps_k0.33_0.2}.
Despite the high pressures used
here, no signs of a transition to a columnar phase have been observed. This is consistent with
the lack of smectic phase for the hard Gaussian overlap model with $k>1$.




