

\chapter{Confined pear shaped particles}
\label{chap:seven}

\introduction

In this Chapter, the knowledge acquired from our earlier studies of confined systems and the
development of the pear shaped PHGO model are brought together. The aim here is 
to construct a model for a liquid crystal display cell in which 
as proposed by Davidson and Mottram~\cite{DavidsonMottram02}, the switching is achieved though a
combination of flexoelectric behaviour and surface bistability. 
This is attempted here using a hybrid anchored slab of confined flexoelectric (pear shaped) 
particles homeotropically anchored on the top 
of the cell and with a homeotropic/planar bistable anchoring on the bottom.\\
This Chapter presents the steps undertaken in order to achieve both the easy and hard switching
routes between  the hybrid aligned nematic and vertical states of such a display.


%====================================
%====================================


\section{The flexoelectric display}

The model proposed by Davidson and Mottram~\cite{DavidsonMottram02} is an idealised 
representation of an existing display cell, the ZBD device~\cite{ZBD}, in which the grating 
morphology of the ZBD device is treated as a planar surface which has homeotropic or planar 
anchoring states. The geometry
considered in~\cite{DavidsonMottram02} is shown in Figure~\ref{fig:bistDisplay}.
In this, a system of flexoelectric mesogens is confined in a slab geometry; 
the top surface induces a monostable homeotropic anchoring while the bottom surface is
bistable and allows both homeotropic and planar arrangements. In the case of an 
homeotropic bottom surface anchoring, the cell is in the so called
vertical state (Figure~\ref{fig:bistDisplay}a) and all particles have a vertical orientation; 
in the case of a bottom planar anchoring, the state is in a so called hybrid aligned 
nematic (HAN) state (Figure~\ref{fig:bistDisplay}b) where the particle
orientations change continuously from homeotropic on the top surface to planar at the bottom.\\

\picW = 5cm
\begin{figure}
	\centering
	\subfigure[The Vertical state]{\pic{bistDisplay_V.ps}}
	\hspace*{10mm}
	\subfigure[The HAN state]{\pic{bistDisplay_HAN.ps}}
	\caption{Schematic representation of the two stable states of the 
	display cell considered in~\cite{DavidsonMottram02}. The top surface induces a
	monostable homeotropic anchoring while the bottom surface is bistable and induces 
	both homeotropic and planar arrangements. }
	\label{fig:bistDisplay}
\end{figure}

Easy switching between these states can be achieved by applying an electric field between the two
substrates which will reorient the particles either parallel or perpendicular to the 
surfaces for respectively a negative or positive molecular dielectric susceptibility. 
Upon removal of this field, because of the strong homeotropic anchoring at the top
surface, particles close to that substrate will recover the homeotropic anchoring whereas 
particles close to the bottom substrate will keep the field induced orientation because of the
bottom surface bistability. As a result both HAN and vertical states can be produced with an
appropriate choice of starting configuration and electric field for a given value (and sign) of
the molecular dielectric susceptibility.\\

The difficulty that then arises is that of achieving the reverse switching, the so called hard 
switching from V to HAN if $\delta\epsilon > 0$ or from HAN to V if $\delta\epsilon<0$.
Davidson and Mottram showed that this `hard switch' can be made
possible by reversing the field orientation and using moderate values of the field strength,
if the liquid crystal particles have flexoelectric properties. Upon application of the reverse field,
competition is created between the dielectric alignment behaviour and the  field-induced splay 
promoted by the flexoelectricity. For appropriate values of the field, this competition 
causes the confined liquid crystal to adopt a distorted state 
which, upon removal of the field, relaxes to the other state thus rendering hard switching
possible. Table~\ref{tble:switchPara} summarizes the parameterisation combinations required 
to achieve easy and hard switching of the cell.\\




\begin{table}
	\centering
	\begin{tabular}{||c||c||c||}
	\hhline{|t:=:t:=:t:=:t|}
	$\delta\epsilon$	&easy switching		&hard switching	\\
	\hhline{|:=::=::=:|}
	$\delta\epsilon>0$	&HAN to V using $E < 0$	&V to HAN using $E > 0$	\\
	\hhline{|:=::=::=:|}
	$\delta\epsilon<0$	&V to HAN using $E > 0$	&HAN to V using $E < 0$	\\
	\hhline{|b:=:b:=:b:=:b|}
	\end{tabular}
\caption{Electric parameterisation given in~\cite{DavidsonMottram02} 
required to performed the `easy' and `hard' switching between
the HAN and vertical states.}
\label{tble:switchPara}
\end{table}

The treatment used by Davidson and Mottram was, however, based on elastic theory approach. 
In this Chapter, the aim is to investigate, using molecular simulations, the validity of 
this and thus get a microscopic picture of the process underlying the switching scheme.\\
In order to model the display proposed by Davidson and Mottram using molecular simulation
methods, a similar slab geometry like that shown on Figure~\ref{fig:bistDisplay} is to be 
used, that is with an homeotropic top surface and a bistable 
bottom surface. The flexoelectric molecules are to be represented using the $k=5$ PHGO 
model for pear  shaped particles developed in Chapter~\ref{chap:six}.

The competition between the dielectric effect and the field-induced flexoelectric splay 
is to be achieved through a particle-field interaction made of dielectric and dipolar 
contributions, as shown in Appendix~\ref{chap:B}. 
In the case of a negative dielectric susceptibility, for example, the particles 
need to experience the competitive effects of the dielectric contribution, which tends to align 
the particles  perpendicular to the field, and the dipolar effect which tends to align 
the particles parallel  to it. The latter has the effect of introducing splay distortions 
due to the preferred packing arrangement of pear shaped particles.\\

The system electric energy $U_e$  is given by (see Appendix~\ref{chap:B})~:
%
\begin{equation}
	U_e = \sum_{i=1}^{N}\left\{ 
	-\frac{1}{2}\epsilon_0\delta\epsilon\lp \dotProd{\vect{E}}{\ui} \rp^2 
	- \mu \dotProdP{\vect{E}}{\ui}
	\right\}
\end{equation}
%
where $\epsilon_0$ is the unit of energy, $\delta\epsilon$ is the dielectric susceptibility,
$\mu$ is the dipole moment, $\ui$ is the molecular orientation and $\vect{E} = E\vecth{E}=
E\vecth{z}$ is
the applied electric field.








\section{Molecular models}
\label{s:GBP_RSUP_model}

To achieve a simulation of the switching in this
display cell requires the modeling of pear shaped particles in a
confined systems and the presence of bistability between the surface induced 
homeotropic and planar arrangements.\\
The modeling of the pear shaped particles is achieved using the PHGO model as described in
Chapter~\ref{chap:six}. Particles with an elongation $k=5$ are to be used as 
these particles were found to form a stable nematic phase.\\
Particle surface interactions are to be based on the approach as developed in Chapters~\ref{chap:four}
and~\ref{chap:five}; the particles will not interact directly with the substrate, rather an object 
embedded into the molecules controls the surface interaction through an appropriate steric potential. 
It has been shown in the preceeding
Chapters that in the case of ellipsoidal particles, surface bistability between the homeotropic
and planar arrangements can be achieved 
using either the HNW or the RSUP potentials; since the bistability regions for the latter are
wider and stronger, this happens to be the better candidate.\\

\picW = 10cm
\begin{figure}
	\centering
	\pic{GBP_RSUPConfig.ps}
	\caption{Representation of the configuration used for the surface interaction between a
	pear shaped particle and the substrate. The interaction is performed by the ellipsoid
	embedded into the pear~; the former is  shifted by an amount $s$ along the molecular
	orientation.}
	\label{fig:PSUConfig}
\end{figure}


However, $\mathcal{V}^{RSUP}$ represents the interaction between a Gaussian ellipsoid and a
surface and its extension to the accurate description of the interaction between a pear 
having the shape of the B\'ezier pear of Chapter~\ref{chap:six} and a surface is not
straightforward.\\
%
Therefore, the RSUP potential was chosen to describe the surface interaction here. 
In order to prevent the inner ellipsoid from overlapping the surface of 
the pear shaped particle, and to add a more realistic behaviour to the model, the refinement 
depicted in Figure~\ref{fig:PSUConfig} has been made. This involves shifting the position of the 
inner ellipsoid along the molecular long axis towards the bulky end of the pear so that the ends of the 
two objects are coincident. This mimics a situation where the particles can embed their tails but
not their heads into a coated substrate.\\
The pear-surface interaction is described by $\mathcal{V}^{PSU}$ as~:
%
\begin{equation}
	\mathcal{V}^{PSU} = \left\{	%}
	\begin{array}{ccc}
		0	&\mathrm{if}	&|z_\mrm{obj} - z_0| \geq \sigma^{PSU}_w	\\
		\infty	&\mathrm{if}	&|z_\mrm{obj} - z_0| < \sigma^{PSU}_w	
	\end{array}
	\right.
\end{equation}
%
where $z_0$ is the position of the substrate and $z_\mrm{obj}$ is the height of the inner
ellipsoid of elongation $k_S$~:
\begin{equation}
	z_\mrm{obj} = z_i - \frac{1}{2}\lp k - k_S\rp\cos\theta
\end{equation}
where $z_i$ the position of the particle. $\sigma^{PSU}_w$ gives the contact distance 
between the inner ellipsoid and the surface, that is~:
\begin{equation}
	\sigma^{PSU}_w = \so\sqrt{\frac{1-\chi_S\sin^2\theta}{1-\chi_S}}
\end{equation}
and $\chi_S$ is the shape anisotropy of the inner ellipsoid~:
\begin{equation}
	\chi_S = \frac{k^2_S-1}{k^2_S+1}.
\end{equation}

\picW = 5cm
\begin{figure}
	\centering
	\subfigure[$k^\prime_S = 0.4$]{\fbox{\pic{GBP_RSUPConfig_kS040.ps}}}
	\subfigure[$k^\prime_S = 0.8$]{\fbox{\pic{GBP_RSUPConfig_kS080.ps}}}
	\caption{Representation of the configuration of the PSU potential in the limiting cases
	of $k^\prime_S = 0.4$(a) and $k^\prime_S = 0.8$(b). The black region shows the overlap
	between the two objects.}
	\label{fig:PSU_limitCase}
\end{figure}

The use of molecular visualisation tools shows that for particles with $k=5$, the surfaces of 
the two objects do not overlap provided $k_S \in [2.00:3.92]$. As a result, all simulations of this 
model have been performed with reduced ellipsoid elongation $k_S/k \in [0.4:0.8]$ (see
Figure~\ref{fig:PSU_limitCase},) the upper limit inducing only a very small overlap 
tangent to the surface of the pear.\\
%
With this model, a homeotropic arrangement is expected in the case of small $k_S$ due to the
ability of the tails of the particles to be absorbed by the surface. A planar arrangement is 
expected in the case of
long $k_S$ because, although some volume could be absorbed at the tail of the particle with
$\theta=0$, it was shown in Chapter~\ref{chap:five}  that the natural tendency of hard rods to adopt
planar alignment becomes dominant as $k_S/k$ increases.\\
%
The lack of a parametric expression for the B\'ezier pear shape and, therefore, the lack of an
analytical expression for the absorbed volume of a pear into the surface prevented the
development of an analytical
prediction for the stability of the different surface arrangements. Here, therefore information
on the preferred anchoring arrangement can only be obtained using computer simulations.







\section{Symmetric systems}
\label{s:GBP_RSUP_symmetric}

Here, the surface arrangements induced by the PSU model are studied using Monte Carlo simulations in
the canonical ensemble of systems of $N=1000$ PHGO particles of elongation $k=5$ confined in a slab
geometry of height $L_z=4k\so$ and symmetric anchoring.\\

%=====================================================
%=====================================================
\picW = 14cm
\begin{figure}
	\centering
	\pic{GBP_RSUP_Sconf_typeProf_k3_d0.15.ps}
	\caption{Typical profiles for a confined system of PHGO particles with $k=5$ symmetric
	anchored with $k^\prime_S = 0.4$ and $k^\prime_S = 0.8$ and $\rho^{*} = 0.15$.}
	\label{fig:GBP_RSUP_typeProf_d0.15}
\end{figure}

%=====================================================
%=====================================================
\picW = 6cm
\begin{figure}
	\centering
	\subfigure[]{\pic{GBP_box_symm_kS040_d0.15.ps}}
	\hspace*{5mm}
	\subfigure[]{\pic{GBP_box_symm_kS080_d0.15.ps}}
	\caption{Typical snapshots for confined systems of $N=1000$ PHGO particles
	with $k=5$ at $\rho^{*} = 0.15$ symmetrically anchored with $k^\prime_S = 0.4$(a) and
	$0.8$. The particle surface interactions are controlled by $\mathcal{V}^{PSU}$.}
	\label{fig:GBP_RSUP_typeSnaps}
\end{figure}
%=====================================================


Typical $z$-profiles obtained from those simulations are shown on 
Figure~\ref{fig:GBP_RSUP_typeProf_d0.15}  for a density of $\rho^{*}=0.15$ which corresponds 
to a bulk nematic phase and two values for $k^\prime_S$ corresponding to the lower 
and upper limits of the range available for this parameter.\\
For $k^\prime_S = 0.4$, the profiles show the features typical of homeotropic anchoring, that is
large peak separations in $\rho^{*}_\ell(z)$ and $\langle P_2(z)\rangle$, compatible with end to 
end layering. Also,
$Q_{zz}(z)$ displays the typical positive values in regions corresponding to high local
density. In the case $k^\prime_S = 0.8$, by contrast, the profiles show the characteristics of planar
anchoring~; $\rho^{*}_\ell(z)$ and $\langle P_2(z)\rangle$ have a much shorter peak separation
compatible  with side by side alignment, and $Q_{zz}(z)$ displays negative values throughout the slab.\\
Observation of the configuration snapshots (\eg Figure~\ref{fig:GBP_RSUP_typeSnaps}) confirms
the conclusions made from the profiles.
This surface behaviour confirms the assertions made in Section~\ref{s:GBP_RSUP_model} and
confirms the suitability of the PSU potential for the modeling of the HAN and vertical states 
of the display cell.\\


%=====================================================
%=====================================================
\picW = 12cm
\begin{figure}
	\centering
	\picL{GBP_RSUP_QzzWa_symm_d0.15.ps}
	\caption{Representation of $\overline{Q}^{Su}_{zz}$ as a function of the reduced inner object
	molecular elongation $k_S/k$ for a confined system of pear shaped PHGO particles in a
	symmetric anchored confined geometry. The surface interaction is controlled by 
	$\mathcal{V}^{SU}$.  }
	\label{GBP_RSUP_symm_QzzWaSu}
\end{figure}
%=====================================================

The homeotropic to planar anchoring transition has been located using Monte Carlo simulations in
the canonical ensemble of symmetric 
systems at a constant density of $\rho^{*} = 0.15$ and varying the surface anchoring using 
series of simulations with increasing and decreasing $k^\prime_S$ in the range $[0.4:0.8]$. 
The location of the anchoring transition and the identification of possible bistable regions 
have been achieved through the observation of the evolution of $\overline{Q}^{Su}_{zz}(k^\prime_S)$
as shown in Figure~\ref{GBP_RSUP_symm_QzzWaSu}.\\
%
This shows that the transition between the two surface arrangements takes place in the range 
$k^\prime_S \in [0.64:0.74]$. Also in this region, the differences between the 
$\overline{Q}^{Su}_{zz}$  data
obtained from the two series indicate a region of bistability which is strongest for 
$k^\prime_S = 0.7$~; with a bistability value of $1.09$.\\

This result achieves the requirement of a bistable anchoring transition 
between the homeotropic and planar surface arrangements for this system. 
However, as only symmetric anchored systems have been used, the presence of bistability
here does not guarantee a similar behaviour in the case of hybrid anchored systems. This
question is addressed in the next Section.








\section{Hybrid systems}


The study of hybrid anchored systems of HGO particles presented in Chapter~\ref{chap:five} 
showed that the presence of a top surface with strong homeotropic anchoring can cause a 
bottom surface with competing anchoring to lose the bistability of its homeotropic and planar surface
arrangements. The bistability behaviour can, however, be recovered by two complementary means. The
first of these is to increase the slab height, which has the effect of
creating a smoother transition between the homeotropic and planar arrangements. The second
solution is to reduce the anchoring strength at the top surface in order to reduce the elastic
forces this imposes on the particles anchored at the bottom surface.\\

\picW = 13.5cm
\begin{figure}
	\centering
	\pic{GBP_RSUP_QzzWa_hyb_d0.15_3g.ps}
	\caption{Comparison between $\overline{Q}^{Sb}_{zz}(k^\prime_{Sb})$ (dashed lines) from
	simulations of hybrid anchored systems of PHGO particles and
	$\overline{Q}^{Su}_{zz}(k^\prime_{S})$ from simulations of symmetric systems described
	in
	Section~\ref{s:GBP_RSUP_symmetric}(solid lines). The arrows indicate whether the
	simulations have been performed with increasing ($\bigtriangleup$) or 
	decreasing ($\bigtriangledown$) values of $k^\prime_S$}
	\label{fig:GBP_RSUP_hybQzzWaSb}
\end{figure}

Here, therefore, we
study the effect of the top surface anchoring parameterisation on the
bistability behaviour of the bottom surface in an hybrid anchored system of pear shaped
particles interacting with the surfaces through the $PSU$ potential. For this, systems of
$N=1000$ PHGO particles of density $\rho^{*}=0.15$ with $k=5$ confined in an hybrid anchored slab 
have been studied using Monte Carlo simulations in
the canonical ensemble. Series of increasing and decreasing $k^\prime_{Sb}$ in the range
$[0.56:0.74]$ have been used; this range corresponding to the bistability region of $k^\prime_S$
for symmetric systems.
Three values for the top anchoring strength were considered~: 
$k^\prime_{St} = 0.4$, $0.5$ and $0.6$. The transition between the homeotropic and
planar arrangements at the bottom surface was studied through the computation of $\overline{Q}^{Sb}_{zz}$
as a function of $k^\prime_{Sb}$.
A comparison between the $\overline{Q}^{Sb}_{zz}(k^\prime_{Sb})$ data obtained for these hybrid 
systems and the data obtained from equivalent symmetric systems in
Section~\ref{s:GBP_RSUP_symmetric} is shown on Figure~\ref{fig:GBP_RSUP_hybQzzWaSb}.
Configuration snapshots from those simulations with $k^\prime_{Sb} = 0.64$ and $0.74$ are shown
on Figure~\ref{fig:GBP_RSUP_hyb_typSnaps}. We note that, as in the case of the equivalent HGO
systems, the HAN state found for this film thickness suggest discontinuous director profiles.\\

\picW = 4.5cm
\begin{figure}
	\centering
	\subfigure[$k^\prime_{St}=0.4$, $k^\prime_{Sb}=0.64$]{\pic{GBP_box_hyb_kSt40_kSb064_d0.15.ps}}
	\subfigure[$k^\prime_{St}=0.4$, $k^\prime_{Sb}=0.74$]{\pic{GBP_box_hyb_kSt40_kSb074_d0.15.ps}}
	
	\subfigure[$k^\prime_{St}=0.5$, $k^\prime_{Sb}=0.64$]{\pic{GBP_box_hyb_kSt50_kSb064_d0.15.ps}}
	\subfigure[$k^\prime_{St}=0.5$, $k^\prime_{Sb}=0.74$]{\pic{GBP_box_hyb_kSt50_kSb074_d0.15.ps}}
	
	\subfigure[$k^\prime_{St}=0.6$, $k^\prime_{Sb}=0.64$]{\pic{GBP_box_hyb_kSt60_kSb064_d0.15.ps}}
	\subfigure[$k^\prime_{St}=0.6$, $k^\prime_{Sb}=0.74$]{\pic{GBP_box_hyb_kSt60_kSb074_d0.15.ps}}
	\caption{Typical configuration snapshots obtained from simulations of confined systems of
	$N=1000$ PHGO particles with $k=5$ at $\rho^{*}=0.15$, hybrid anchoring and different 
	values of $k^\prime_{St}$ and $k^\prime_{Sb}$.}
	\label{fig:GBP_RSUP_hyb_typSnaps}
\end{figure}


\picW = 4.8cm
\begin{figure}
	\centering
	\subfigure[increasing $k^\prime_{Sb}$]{\pic{GBP_box_hyb_kSt60_kSb070_d0.15_S3.0.ps}}
	\subfigure[decreasing $k^\prime_{Sb}$]{\pic{GBP_box_hyb_kSt60_kSb070_d0.15_S1.0.ps}}
	\caption{Configuration snapshots showing the HAN and V states of a hybrid anchored slab
	of $N=1000$ PHGO particles with $k=5$ at $\rho^{*} = 0.15$ with $k^\prime_{St} = 0.6$ 
	and $k^\prime_{Sb} = 0.7$. Those configuration have been obtained from series of 
	simulations with 
	increasing (a) and decreasing (b) values of $k^\prime_{Sb}$}
	\label{fig:GBP_RSUP_hyb_kSt060_kSt070_snaps}
\end{figure}


For the two lower values of $k^\prime_{St}$ (representing stronger homeotropic anchoring), 
although the two 
series of simulations lead to hysteresis in the values of $\overline{Q}^{Sb}_{zz}(k^\prime_{Sb})$, no
bistability can be observed. None of the state points considered here correspond to a situation 
where the values of $\overline{Q}^{Sb}_{zz}$ obtained from the two series are both significantly
different from zero and of opposite signs.\\
%
With $k^\prime_{St} = 0.6$, however, a small bistable region  is recovered around
$k^\prime_{Sb}= 0.7$; 
the values of $\overline{Q}^{Sb}_{zz}$ are different and of opposite signs. The bistability value for 
$k^\prime_{Sb} = 0.7$ is $0.914$ which is very close to that obtained with symmetric anchored 
systems. Configuration snapshots corresponding to the HAN and V states
of the cell at this state point are given on Figure~\ref{fig:GBP_RSUP_hyb_kSt060_kSt070_snaps}.\\
%
These results show that reducing the strength of the anchoring at the top surface allows to
recover the bistability region by increasing the hysteresis in $\overline{Q}^{Sb}_{zz}$.
The value $k^\prime_{St} = 0.6$ seems to be the highest reasonable that can be used,
as according to $\overline{Q}^{Su}_{zz}(k^\prime_S)$, the use of a higher value would not lead to homeotropic
anchoring at the top surface.\\


The results from these simulations are reasonably encouraging for the application of the model to
the HAN to V switching since, despite the very narrow bistability region for $k^\prime_{Sb}$, 
the difference
between the $\overline{Q}^{Sb}_{zz}$ values obtained from the two series with $k^\prime_{St}=0.6$ 
appears sufficient for the model to be used in the display modeling. Also, the snapshots show 
encouraging  HAN and V states which should be further improved by the use of wider systems.










%\clearpage


\section{Flexoelectric switching}

Here the possibility of two-way switching between the HAN and vertical states using the
flexoelectric properties of the model and the bistable nature
of the bottom surface is investigated. This is performed in three steps. First, the
stability of the HAN and vertical states are assessed by performing the easy switching
between the two states, and considering only the dielectric term in the electric energy 
(\ie $\mu=0$). Then, the possibility of hard switching is investigated by including a
dipolar term in the electric energy. This requires finding an
appropriate combination of values for the electric field magnitude, dielectric anisotropy and the 
dipole moment. Finally, keeping the same parameterisation as that used to achieve hard switching, 
the possibility of easy switching is investigated again so as to check that the newly introduced 
dipolar contribution does not hinder the reverse switching.\\
In order to use the most favorable conditions for achieving switching and obtaining a smooth
structural transition from homeotropic to planar in an HAN cell, all simulations used the Monte
Carlo method in the canonical ensemble and 
used systems of $N=2000$  particles in a cell of width $8\so$. Hybrid anchoring with the
parameterisation  corresponding to the maximum hysteresis obtained in the previous section, 
that is $k^\prime_{St} = 0.6$ and $k^\prime_{Sb} = 0.7$, was employed.

%=================================================================================================
%=================================================================================================
\subsection{Easy switching}
\label{ss:easySwitch}

The aim of the simulations performed here is mainly to test for the stability of the HAN and
vertical states found in the previous Section. Therefore, easy switching has been attempted
between those two states by considering only the dielectric term in the particle-field interaction
(\ie $\delta\epsilon \neq 0, \mu = 0$).\\
Two series of simulations have been performed using an hybrid anchored slab with the surface
parameterisation given at the beginning of this Section. Starting with an HAN configuration, the
first simulation attempted to switch to the vertical state by application and subsequent removal
of an electric field
and considering the particles to have a positive dielectric anisotropy. Then, taking the particles to
have a negative dielectric anisotropy, a second series of simulations was used to switch back to
the HAN state starting from the vertical state obtained from the first series.\\
Each series consisted of a first run of $2.0.10^6$ sweeps, performed to equilibrate the
starting configuration with the field off. This was followed by a simulation of
$0.5.10^6$ sweeps with the field on, which primed the switching. The resulting system
was equilibrated in the new state for another $2.10^6$ sweeps with the electric field removed.
The electric parameterisation used here was $E=1.0$, $\delta\epsilon = \pm 1.0$ and $\mu = 0$.



\picW = 10cm
\begin{figure}
	\centering
	\subfigure[]{\picL{QzzWa_easyHANtoV.ps}}
	\subfigure[]{\picL{QzzWa_easyVtoHAN.ps}}
	\caption[Evolution of $\overline{Q}^{Sb}_{zz}$ with sweep number in the
	simulations of the HAN to vertical(a) and vertical to HAN (b) `easy switching'
	of the hybrid anchored slab described at the beginning of this Section.]
	{Evolution of $\overline{Q}^{Sb}_{zz}$ with sweep number in the
	simulations of the HAN to vertical(a) and vertical to HAN (b) `easy switching'
	of the hybrid anchored slab described at the beginning of this Section. 
	Here only dielectric interactions between the particles and the field are 
	considered. The data for the last and first field off series of (a) and (b) respectively
	have been
	obtained from the same simulation.}
	\label{fig:easySwitch_QzzWa}
\end{figure}


\picW = 4cm
\begin{figure}
	\centering
	\subfigure[field off]{\pic{GBP_box_01.ps}}
	\subfigure[field on]{\pic{GBP_box_02.ps}}
	\subfigure[field off]{\pic{GBP_box_03.ps}}
	\caption{Configuration snapshots corresponding to three phases (a) to (c) 
	of the HAN to vertical `easy switching' of the hybrid anchored slab.}
	\label{fig:easySwitch_HANtoVsnaps}
\end{figure}

\picW = 4cm
\begin{figure}
	\centering
	\subfigure[field off]{\pic{GBP_box_03.ps}}
	\subfigure[field on]{\pic{GBP_box_04.ps}}
	\subfigure[field off]{\pic{GBP_box_05.ps}}
	\caption{Configuration snapshots corresponding to three phases (a) to (c) 
	of the vertical to HAN `easy switching' of the hybrid anchored slab.}
	\label{fig:easySwitch_VtoHANsnaps}
\end{figure}

The evolution of $\overline{Q}_{zz}$ at the bottom surface as a function of the number of sweeps is
shown on Figure~\ref{fig:easySwitch_QzzWa}. The snapshots corresponding to the final
configurations from phase of the simulation sequence are shown on 
Figures~\ref{fig:easySwitch_HANtoVsnaps} and~\ref{fig:easySwitch_VtoHANsnaps}.\\
These results, along with the corresponding profile data (not shown) confirm both the stability
of the HAN and vertical states for this system and the ability of the easy switching mechanism
to switch between them.


%=================================================================================================
%=================================================================================================
%\clearpage
\subsection{Hard switching}
\label{ss:hardSwitch}

We now turn to the possibility of achieving hard switching between the HAN and vertical states of 
the slab, following the theoretical treatment of~\cite{DavidsonMottram02}. Here both the
dielectric and dipolar terms in the particle-field interaction are required (\ie
$\delta\epsilon \neq 0$ and $\mu \neq 0$). Only the case of particles with a negative dielectric
susceptibility is considered. 
The aim here is to find an appropriate parameterisation which allows for switching from the HAN to
the vertical states after application of an electric field along $\vecth{z}$. 
Reference~\cite{DavidsonMottram02} shows that such switching can be achieved using 
$E < 0$.

\subsubsection{Choice of $\vect{E}$}

The first step is to find an appropriate value for the electric field. This needs to be strong
enough to allow the dipolar contribution to distort the director profile and bring the system to
an intermediate state that will relax into the vertical state after removal of the field. 
If the field is too strong, however, the dielectric effect (which scales as $E^2$) will
dominate, causing the HAN state to be
stabilised and thus, rendering the switch to the vertical state impossible.\\
In order to find an appropriate value for $E$, a slab in a HAN state has been simulated using
$\delta\epsilon = -1.0$, $\mu = 1.0$ and different values for the electric field in the range
[-0.02:-6.0]. For each value
of $E$, the system was subject to an equilibration run of $0.25.10^6$ sweeps followed by a
production run of the same length. Figure~\ref{fig:hardSwitch_QzzEinfl} shows the $Q_{zz}$
profiles obtained from the production runs for a selection of these field strengths.

%-----------------------------------------------------
\picW = 12cm
\begin{figure}
	\centering
	\picL{hardSwitch_QzzProf_Einfl.ps}
	\caption{$Q_{zz}(z)$ profiles for an HAN slab subject to an applied electric field along 
	$\vecth{z}$ and different values of $E$.}
	\label{fig:hardSwitch_QzzEinfl}
\end{figure}
%-----------------------------------------------------

These results show that, given the chosen parameterisation adopted for $\delta\epsilon$ and $\mu$,
values less negative than $E = -0.2$ do not induce any significant distortions near the bistable
surface and that, despite the
applied electric field, the system always remains in the HAN state. For fields of strength more
negative than
$-0.4$, in contrast, the dielectric contribution dominates and stabilises the HAN state. 
For the highest absolute
values of $E$, all particles are forced to be parallel to the surfaces, even those subject to the
homeotropic anchoring of the top surface. Figure~\ref{fig:hardSwitch_QzzEinfl} however,
indicates
promising behaviour for $E=-0.2$, for which the dipolar coupling seems to be strong enough 
to induce a slight distortion of the profile without there being too strong a dielectric effect. 
This raises the prospect that by increasing the
value of $\mu$ and keeping all other parameters constant, this distortion can
be increased to the extend that switching to the vertical state can be
achieved.


\subsubsection{Choice of $\mu$}

Here, an attempt is made to identify an appropriate value of $\mu$ so that, upon application of the
electric field with $E=-0.2$, the dipolar effect induces enough of a distortion to cause a 
HAN configuration system to equilibrate into a vertical state upon removal of the field.


%-----------------------------------------------------
\picW = 14cm
\begin{figure}
	\centering
	\pic{hSwitch_muInfl_Qzz01.ps}
	\caption{$Q_{zz}$ profiles for an HAN slab subject to an applied electric field with
	$E=-0.2\vecth{z}$ and $\delta\epsilon = -1.0$ and different values of the 
	dipolar moment $\mu \in [1.0:2.0]$.}
	\label{fig:hardSwitch_QzzmuInfl01}
\end{figure}
%-----------------------------------------------------

%-----------------------------------------------------
\picW = 14cm
\begin{figure}
	\centering
	\pic{hSwitch_muInfl_Qzz02.ps}
	\caption{$Q_{zz}$ profiles for a slab in the HAN configuration and 
	subject to an applied electric field with
	$E=-0.2\vecth{z}$ and $\delta\epsilon = -1.0$ and different values of the 
	dipolar moment $\mu \in [2.5:3.5]$.}
	\label{fig:hardSwitch_QzzmuInfl02}
\end{figure}
%-----------------------------------------------------

\picW = 4cm
\begin{figure}
	\centering
	\subfigure[start]{\pic{GBP_hardSwitch_start.ps}}
	\subfigure[field on]{\pic{GBP_box_Em0.2_mu2.5on.ps}}
	\subfigure[field off]{\pic{GBP_box_Em0.2_mu2.5off.ps}}
	\caption{Configuration snapshots corresponding to the hard switching of a slab in an
	initial HAN state with $E=-0.2\vecth{z}$ and $\delta\epsilon = -1.0$ and $\mu = 2.5$.}
	\label{fig:hardSwitch_snaps2.5}
\end{figure}

\begin{figure}	
	\centering
	\subfigure[start]{\pic{GBP_hardSwitch_start.ps}}
	\subfigure[field on]{\pic{GBP_box_Em0.2_mu3.0on.ps}}
	\subfigure[field off]{\pic{GBP_box_Em0.2_mu3.0off.ps}}
	\caption{Configuration snapshots corresponding to the hard switching of a slab in an
	initial HAN state with $E=-0.2\vecth{z}$ and $\delta\epsilon = -1.0$ and $\mu = 3.0$.}
	\label{fig:hardSwitch_snaps3.0}
\end{figure}

\begin{figure}
	\centering
	\subfigure[start]{\pic{GBP_hardSwitch_start.ps}}		
	\subfigure[field on]{\pic{GBP_box_Em0.2_mu3.5on.ps}}
	\subfigure[field off]{\pic{GBP_box_Em0.2_mu3.5off.ps}}
	\caption{Configuration snapshots corresponding to the hard switching of a slab in an
	initial HAN state with $E=-0.2\vecth{z}$ and $\delta\epsilon = -1.0$ and $\mu = 3.5$.}
	\label{fig:hardSwitch_snaps3.5}
\end{figure}

%-----------------------------------------------------

In order to achieve this, simulations of the slab have been carried out taking the HAN 
configuration as an initial state. For each $\mu$ value the simulation sequence performed
consisted of two runs (one for equilibration and
one for production) with an applied electric field followed by two runs (equilibration and
production) where the field was removed. Each run comprised $0.25.10^6$ sweeps and the 
parameterisation $E=-0.2$ and $\delta\epsilon = -1.0$ was
used. The first two runs were used to establish the `field-on' intermediate state while the last
to generated the state to which the system subsequently relaxed. 
The series of simulations described above was performed with six values of $\mu$ in the range
$[1.0:3.5]$. 


The $Q_{zz}(z)$ profiles corresponding to the obtained  field `on' and `off' 
configurations are shown on Figure~\ref{fig:hardSwitch_QzzmuInfl01}
and~\ref{fig:hardSwitch_QzzmuInfl02} along with the profiles corresponding to the HAN and
vertical states which are shown for comparison. Configuration snapshots of the field -on and
field-off  structures for $\mu = 2.5$, $3.0$ and $3.5$ are shown, respectively, in 
Figures~\ref{fig:hardSwitch_snaps2.5}, \ref{fig:hardSwitch_snaps3.0} 
and~\ref{fig:hardSwitch_snaps3.5}\\

From the $Q_{zz}(z)$ data, it appears that the switching from the HAN state to the vertical
state is possible using values of $\mu \geq 2.5$. As $\mu$ is increased, so does the 
distortion  induced by the dipolar term in $U_e$; more specifically, the profile at the bottom
surface is changed, so inducing the bulk part of the cell to modify its orientation
co-operatively.
Upon removal of the field, for $\mu$ values at which this distortion is sufficient, 
the cell equilibrates into the
vertical state, thus confirming the results of~\cite{DavidsonMottram02}.


%=================================================================================================
%=================================================================================================
%\clearpage
\subsection{Reverse switching}

In Sections~\ref{ss:easySwitch} and~\ref{ss:hardSwitch}, respectively, it was shown that easy 
switching between the HAN and vertical states
can be achieved using only the dielectric effect and that,
using an appropriate parameterisation  ($E=-0.2$, $\delta\epsilon = -1.0$ and $\mu \in
[2.5:3.5]$), hard switching from the HAN to the vertical state can be achieved. However, 
easy switching from vertical to HAN state is not necessarily possible using the same
parameterisation as that used to achieve hard switching but with $E>0$ since the dipolar
contribution
might be too strong to permit the formation of the HAN state.\\


%-----------------------------------------------------
\picW = 14cm
\begin{figure}
	\centering
	\pic{revSwitch_Qzz.ps}
	\caption{$Q_{zz}$ profiles for a slab in the vertical configuration and 
	subject to an applied electric field with
	$E=0.2\vecth{z}$ and $\delta\epsilon = -1.0$ and different values of the 
	dipolar moment $\mu \in [2.5:3.5]$.}
	\label{fig:revSwitch_QzzmuInfl}
\end{figure}
%-----------------------------------------------------

\picW = 4cm
\begin{figure}
	\centering
	\subfigure[start]{\pic{GBP_box_Em0.2_mu2.5off.ps}}
	\subfigure[field on]{\pic{GBP_box_mu2.5on.ps}}
	\subfigure[field of]{\pic{GBP_box_mu2.5off.ps}}
	\caption{Configuration snapshots corresponding to the hard switching of a slab in an
	initial vertical state with $E=0.2\vecth{z}$ and $\delta\epsilon = -1.0$ and $\mu = 2.5$.}
	\label{fig:revSwitch_snaps2.5}
\end{figure}

\picW = 4cm
\begin{figure}
	\centering
	\subfigure[start]{\pic{GBP_box_Em0.2_mu3.0off.ps}}
	\subfigure[field on]{\pic{GBP_box_mu3.0on.ps}}
	\subfigure[field off]{\pic{GBP_box_mu3.0off.ps}}
	\caption{Configuration snapshots corresponding to the hard switching of a slab in an
	initial vertical state with $E=0.2\vecth{z}$ and $\delta\epsilon = -1.0$ and $\mu = 3.0$.}
	\label{fig:revSwitch_snaps3.0}
\end{figure}

\picW = 4cm
\begin{figure}
	\centering
	\subfigure[start]{\pic{GBP_box_Em0.2_mu3.5off.ps}}
	\subfigure[field on]{\pic{GBP_box_mu3.5on.ps}}
	\subfigure[field off]{\pic{GBP_box_mu3.5off.ps}}
	\caption{Configuration snapshots corresponding to the hard switching of a slab in an
	initial vertical state with $E=0.2\vecth{z}$ and $\delta\epsilon = -1.0$ and $\mu = 3.5$.}
	\label{fig:revSwitch_snaps3.5}
\end{figure}



This issue is addressed here by
attempting to perform the easy switching again but this time with both the dielectric and dipolar 
term included in the particle-field interaction. A similar parameterisation as that used in
Section~\ref{ss:hardSwitch} is applied here, the difference being that the electric field
director is taken to be positive. As a result the parameterisation $E=0.2$, $\delta\epsilon = -1.0$ and
$\mu\in[2.5:3.5]$ is used. Several values of $\mu$ are considered so as to also investigate the
effect of increasing $\mu$. These simulations were performed using a similar sequence as 
that used in~\ref{ss:hardSwitch}, the main difference being that the initial configuration for each
series with different $\mu$ was the final configuration obtained for the vertical state 
from the  hard switching simulations with the appropriate $\mu$ value.\\
The $Q_{zz}(z)$ profiles corresponding to the `field-on' and `field-off' states obtained for each value
of $\mu$ are shown on Figure~\ref{fig:revSwitch_QzzmuInfl} and the corresponding 
configuration snapshots for $\mu = 2.5$, $3.0$ and $3.5$ are shown, respectively, on
Figures~\ref{fig:revSwitch_snaps2.5}, \ref{fig:revSwitch_snaps3.0}
and~\ref{fig:revSwitch_snaps3.5}.\\

The $Q_{zz}(z)$ data show that, upon application of the field, most of the vertical slab
arrangement remains 
undistorted, expect for a
region near the bottom surface which adopts a planar arrangement. Upon removal of the field,
this small interfacial distortion proves sufficient to seed this orientation into the bulk part of
the cell. Because of the homeotropic top surface influence, the slab then recovers the HAN state.
These results also show that the distance from the bottom surface over which the cell's vertical
alignment is
distorted in the field-on state decreases with increased $\mu$. For the run lengths used here,
this has the effect of producing 
HAN states of reducing quality as $\mu$ is increased. This trend suggests that with
$\mu>3.5$, although the hard switching is possible, easy switching might be inhibited by
high values of the dipolar coupling term. As a result it can be concluded that switching 
between the HAN and vertical states of the hybrid anchored slab can be achieved if the dipolar
term to the electric field is included, but that the 
parameterisation should be compatible with a window of electric field and
dipolar coupling strength and that if $\delta\epsilon =< 0.0$, $E \sim 0.2\delta\epsilon$ 
and $\mu \sim -2.5\delta\epsilon$.













%====================================
%====================================

\conclusion

In this Chapter, two issues have been addressed. First, the surface induced structural changes in
confined systems of PHGO pear shaped particles interacting with the surface through the RSUP
model have been studied, and bistability regions between the two surfaces arrangements have been
found using a surface parameter in the range $k_s \sim 0.7$. Also, it has been shown that 
this bistability behaviour can
be recovered at the bottom surface of an hybrid anchored slab with a top surface homeotropic
anchoring provided the latter is not too strong (\ie $k_{St} = 0.6$.)\\
Following this, switching between the HAN and vertical states of such a cell has been
investigated using a particle-field interaction containing both dielectric and dipolar
contributions. It has been shown that both easy and hard switching can be performed provided the
energy parameterisation is compatible with a window in both the electric field strength and
dipolar constant. Both switching direction can be achieved using $E = \pm 0.2\delta\epsilon$ and
$\mu = 2.5\delta\epsilon$.\\
%
Having achieved the two switching modes presented in~\cite{DavidsonMottram02}, with a molecular
model, it is apparent that there are several possible mechanisms underlying this switching. The
particle-particle and particle-field interactions used have been developed to allow the bulk
flexoelectric effects, considered in~\cite{DavidsonMottram02}, to play a role in the hard
switching. There is also, however, an implicit dipolar symmetry to the particle-substrate
interaction used in this work, which may also have played an important role in the simulations
presented in Sections~\ref{ss:easySwitch} and~\ref{ss:hardSwitch}. Resolving which of these
mechanisms is the dominant effect present in both these simulations and the real devices they
attempt to model, is beyond the scope of this Thesis; we do, however, comment in the general
conclusion Chapter possible steps to be taken to this end.












