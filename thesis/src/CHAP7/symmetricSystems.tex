

\section{Symmetric systems}
\label{s:GBP_RSUP_symmetric}

Here, the surface arrangements induced by the PSU model are studied using Monte Carlo simulations in
the canonical ensemble of systems of $N=1000$ PHGO particles of elongation $k=5$ confined in a slab
geometry of height $L_z=4k\so$ and symmetric anchoring.\\

%=====================================================
%=====================================================
\picW = 14cm
\begin{figure}
	\centering
	\pic{GBP_RSUP_Sconf_typeProf_k3_d0.15.ps}
	\caption{Typical profiles for a confined system of PHGO particles with $k=5$ symmetric
	anchored with $k^\prime_S = 0.4$ and $k^\prime_S = 0.8$ and $\rho^{*} = 0.15$.}
	\label{fig:GBP_RSUP_typeProf_d0.15}
\end{figure}

%=====================================================
%=====================================================
\picW = 6cm
\begin{figure}
	\centering
	\subfigure[]{\pic{GBP_box_symm_kS040_d0.15.ps}}
	\hspace*{5mm}
	\subfigure[]{\pic{GBP_box_symm_kS080_d0.15.ps}}
	\caption{Typical snapshots for confined systems of $N=1000$ PHGO particles
	with $k=5$ at $\rho^{*} = 0.15$ symmetrically anchored with $k^\prime_S = 0.4$(a) and
	$0.8$. The particle surface interactions are controlled by $\mathcal{V}^{PSU}$.}
	\label{fig:GBP_RSUP_typeSnaps}
\end{figure}
%=====================================================


Typical $z$-profiles obtained from those simulations are shown on 
Figure~\ref{fig:GBP_RSUP_typeProf_d0.15}  for a density of $\rho^{*}=0.15$ which corresponds 
to a bulk nematic phase and two values for $k^\prime_S$ corresponding to the lower 
and upper limits of the range available for this parameter.\\
For $k^\prime_S = 0.4$, the profiles show the features typical of homeotropic anchoring, that is
large peak separations in $\rho^{*}_\ell(z)$ and $\langle P_2(z)\rangle$, compatible with end to 
end layering. Also,
$Q_{zz}(z)$ displays the typical positive values in regions corresponding to high local
density. In the case $k^\prime_S = 0.8$, by contrast, the profiles show the characteristics of planar
anchoring~; $\rho^{*}_\ell(z)$ and $\langle P_2(z)\rangle$ have a much shorter peak separation
compatible  with side by side alignment, and $Q_{zz}(z)$ displays negative values throughout the slab.\\
Observation of the configuration snapshots (\eg Figure~\ref{fig:GBP_RSUP_typeSnaps}) confirms
the conclusions made from the profiles.
This surface behaviour confirms the assertions made in Section~\ref{s:GBP_RSUP_model} and
confirms the suitability of the PSU potential for the modeling of the HAN and vertical states 
of the display cell.\\


%=====================================================
%=====================================================
\picW = 12cm
\begin{figure}
	\centering
	\picL{GBP_RSUP_QzzWa_symm_d0.15.ps}
	\caption{Representation of $\overline{Q}^{Su}_{zz}$ as a function of the reduced inner object
	molecular elongation $k_S/k$ for a confined system of pear shaped PHGO particles in a
	symmetric anchored confined geometry. The surface interaction is controlled by 
	$\mathcal{V}^{SU}$.  }
	\label{GBP_RSUP_symm_QzzWaSu}
\end{figure}
%=====================================================

The homeotropic to planar anchoring transition has been located using Monte Carlo simulations in
the canonical ensemble of symmetric 
systems at a constant density of $\rho^{*} = 0.15$ and varying the surface anchoring using 
series of simulations with increasing and decreasing $k^\prime_S$ in the range $[0.4:0.8]$. 
The location of the anchoring transition and the identification of possible bistable regions 
have been achieved through the observation of the evolution of $\overline{Q}^{Su}_{zz}(k^\prime_S)$
as shown in Figure~\ref{GBP_RSUP_symm_QzzWaSu}.\\
%
This shows that the transition between the two surface arrangements takes place in the range 
$k^\prime_S \in [0.64:0.74]$. Also in this region, the differences between the 
$\overline{Q}^{Su}_{zz}$  data
obtained from the two series indicate a region of bistability which is strongest for 
$k^\prime_S = 0.7$~; with a bistability value of $1.09$.\\

This result achieves the requirement of a bistable anchoring transition 
between the homeotropic and planar surface arrangements for this system. 
However, as only symmetric anchored systems have been used, the presence of bistability
here does not guarantee a similar behaviour in the case of hybrid anchored systems. This
question is addressed in the next Section.





