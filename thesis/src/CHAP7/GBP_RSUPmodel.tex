

\section{Molecular models}
\label{s:GBP_RSUP_model}

To achieve a simulation of the switching in this
display cell requires the modeling of pear shaped particles in a
confined systems and the presence of bistability between the surface induced 
homeotropic and planar arrangements.\\
The modeling of the pear shaped particles is achieved using the PHGO model as described in
Chapter~\ref{chap:six}. Particles with an elongation $k=5$ are to be used as 
these particles were found to form a stable nematic phase.\\
Particle surface interactions are to be based on the approach as developed in Chapters~\ref{chap:four}
and~\ref{chap:five}; the particles will not interact directly with the substrate, rather an object 
embedded into the molecules controls the surface interaction through an appropriate steric potential. 
It has been shown in the preceeding
Chapters that in the case of ellipsoidal particles, surface bistability between the homeotropic
and planar arrangements can be achieved 
using either the HNW or the RSUP potentials; since the bistability regions for the latter are
wider and stronger, this happens to be the better candidate.\\

\picW = 10cm
\begin{figure}
	\centering
	\pic{GBP_RSUPConfig.ps}
	\caption{Representation of the configuration used for the surface interaction between a
	pear shaped particle and the substrate. The interaction is performed by the ellipsoid
	embedded into the pear~; the former is  shifted by an amount $s$ along the molecular
	orientation.}
	\label{fig:PSUConfig}
\end{figure}


However, $\mathcal{V}^{RSUP}$ represents the interaction between a Gaussian ellipsoid and a
surface and its extension to the accurate description of the interaction between a pear 
having the shape of the B\'ezier pear of Chapter~\ref{chap:six} and a surface is not
straightforward.\\
%
Therefore, the RSUP potential was chosen to describe the surface interaction here. 
In order to prevent the inner ellipsoid from overlapping the surface of 
the pear shaped particle, and to add a more realistic behaviour to the model, the refinement 
depicted in Figure~\ref{fig:PSUConfig} has been made. This involves shifting the position of the 
inner ellipsoid along the molecular long axis towards the bulky end of the pear so that the ends of the 
two objects are coincident. This mimics a situation where the particles can embed their tails but
not their heads into a coated substrate.\\
The pear-surface interaction is described by $\mathcal{V}^{PSU}$ as~:
%
\begin{equation}
	\mathcal{V}^{PSU} = \left\{	%}
	\begin{array}{ccc}
		0	&\mathrm{if}	&|z_\mrm{obj} - z_0| \geq \sigma^{PSU}_w	\\
		\infty	&\mathrm{if}	&|z_\mrm{obj} - z_0| < \sigma^{PSU}_w	
	\end{array}
	\right.
\end{equation}
%
where $z_0$ is the position of the substrate and $z_\mrm{obj}$ is the height of the inner
ellipsoid of elongation $k_S$~:
\begin{equation}
	z_\mrm{obj} = z_i - \frac{1}{2}\lp k - k_S\rp\cos\theta
\end{equation}
where $z_i$ the position of the particle. $\sigma^{PSU}_w$ gives the contact distance 
between the inner ellipsoid and the surface, that is~:
\begin{equation}
	\sigma^{PSU}_w = \so\sqrt{\frac{1-\chi_S\sin^2\theta}{1-\chi_S}}
\end{equation}
and $\chi_S$ is the shape anisotropy of the inner ellipsoid~:
\begin{equation}
	\chi_S = \frac{k^2_S-1}{k^2_S+1}.
\end{equation}

\picW = 5cm
\begin{figure}
	\centering
	\subfigure[$k^\prime_S = 0.4$]{\fbox{\pic{GBP_RSUPConfig_kS040.ps}}}
	\subfigure[$k^\prime_S = 0.8$]{\fbox{\pic{GBP_RSUPConfig_kS080.ps}}}
	\caption{Representation of the configuration of the PSU potential in the limiting cases
	of $k^\prime_S = 0.4$(a) and $k^\prime_S = 0.8$(b). The black region shows the overlap
	between the two objects.}
	\label{fig:PSU_limitCase}
\end{figure}

The use of molecular visualisation tools shows that for particles with $k=5$, the surfaces of 
the two objects do not overlap provided $k_S \in [2.00:3.92]$. As a result, all simulations of this 
model have been performed with reduced ellipsoid elongation $k_S/k \in [0.4:0.8]$ (see
Figure~\ref{fig:PSU_limitCase},) the upper limit inducing only a very small overlap 
tangent to the surface of the pear.\\
%
With this model, a homeotropic arrangement is expected in the case of small $k_S$ due to the
ability of the tails of the particles to be absorbed by the surface. A planar arrangement is 
expected in the case of
long $k_S$ because, although some volume could be absorbed at the tail of the particle with
$\theta=0$, it was shown in Chapter~\ref{chap:five}  that the natural tendency of hard rods to adopt
planar alignment becomes dominant as $k_S/k$ increases.\\
%
The lack of a parametric expression for the B\'ezier pear shape and, therefore, the lack of an
analytical expression for the absorbed volume of a pear into the surface prevented the
development of an analytical
prediction for the stability of the different surface arrangements. Here, therefore information
on the preferred anchoring arrangement can only be obtained using computer simulations.




