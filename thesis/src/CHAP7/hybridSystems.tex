

\section{Hybrid systems}


The study of hybrid anchored systems of HGO particles presented in Chapter~\ref{chap:five} 
showed that the presence of a top surface with strong homeotropic anchoring can cause a 
bottom surface with competing anchoring to lose the bistability of its homeotropic and planar surface
arrangements. The bistability behaviour can, however, be recovered by two complementary means. The
first of these is to increase the slab height, which has the effect of
creating a smoother transition between the homeotropic and planar arrangements. The second
solution is to reduce the anchoring strength at the top surface in order to reduce the elastic
forces this imposes on the particles anchored at the bottom surface.\\

\picW = 13.5cm
\begin{figure}
	\centering
	\pic{GBP_RSUP_QzzWa_hyb_d0.15_3g.ps}
	\caption{Comparison between $\overline{Q}^{Sb}_{zz}(k^\prime_{Sb})$ (dashed lines) from
	simulations of hybrid anchored systems of PHGO particles and
	$\overline{Q}^{Su}_{zz}(k^\prime_{S})$ from simulations of symmetric systems described
	in
	Section~\ref{s:GBP_RSUP_symmetric}(solid lines). The arrows indicate whether the
	simulations have been performed with increasing ($\bigtriangleup$) or 
	decreasing ($\bigtriangledown$) values of $k^\prime_S$}
	\label{fig:GBP_RSUP_hybQzzWaSb}
\end{figure}

Here, therefore, we
study the effect of the top surface anchoring parameterisation on the
bistability behaviour of the bottom surface in an hybrid anchored system of pear shaped
particles interacting with the surfaces through the $PSU$ potential. For this, systems of
$N=1000$ PHGO particles of density $\rho^{*}=0.15$ with $k=5$ confined in an hybrid anchored slab 
have been studied using Monte Carlo simulations in
the canonical ensemble. Series of increasing and decreasing $k^\prime_{Sb}$ in the range
$[0.56:0.74]$ have been used; this range corresponding to the bistability region of $k^\prime_S$
for symmetric systems.
Three values for the top anchoring strength were considered~: 
$k^\prime_{St} = 0.4$, $0.5$ and $0.6$. The transition between the homeotropic and
planar arrangements at the bottom surface was studied through the computation of $\overline{Q}^{Sb}_{zz}$
as a function of $k^\prime_{Sb}$.
A comparison between the $\overline{Q}^{Sb}_{zz}(k^\prime_{Sb})$ data obtained for these hybrid 
systems and the data obtained from equivalent symmetric systems in
Section~\ref{s:GBP_RSUP_symmetric} is shown on Figure~\ref{fig:GBP_RSUP_hybQzzWaSb}.
Configuration snapshots from those simulations with $k^\prime_{Sb} = 0.64$ and $0.74$ are shown
on Figure~\ref{fig:GBP_RSUP_hyb_typSnaps}. We note that, as in the case of the equivalent HGO
systems, the HAN state found for this film thickness suggest discontinuous director profiles.\\

\picW = 4.5cm
\begin{figure}
	\centering
	\subfigure[$k^\prime_{St}=0.4$, $k^\prime_{Sb}=0.64$]{\pic{GBP_box_hyb_kSt40_kSb064_d0.15.ps}}
	\subfigure[$k^\prime_{St}=0.4$, $k^\prime_{Sb}=0.74$]{\pic{GBP_box_hyb_kSt40_kSb074_d0.15.ps}}
	
	\subfigure[$k^\prime_{St}=0.5$, $k^\prime_{Sb}=0.64$]{\pic{GBP_box_hyb_kSt50_kSb064_d0.15.ps}}
	\subfigure[$k^\prime_{St}=0.5$, $k^\prime_{Sb}=0.74$]{\pic{GBP_box_hyb_kSt50_kSb074_d0.15.ps}}
	
	\subfigure[$k^\prime_{St}=0.6$, $k^\prime_{Sb}=0.64$]{\pic{GBP_box_hyb_kSt60_kSb064_d0.15.ps}}
	\subfigure[$k^\prime_{St}=0.6$, $k^\prime_{Sb}=0.74$]{\pic{GBP_box_hyb_kSt60_kSb074_d0.15.ps}}
	\caption{Typical configuration snapshots obtained from simulations of confined systems of
	$N=1000$ PHGO particles with $k=5$ at $\rho^{*}=0.15$, hybrid anchoring and different 
	values of $k^\prime_{St}$ and $k^\prime_{Sb}$.}
	\label{fig:GBP_RSUP_hyb_typSnaps}
\end{figure}


\picW = 4.8cm
\begin{figure}
	\centering
	\subfigure[increasing $k^\prime_{Sb}$]{\pic{GBP_box_hyb_kSt60_kSb070_d0.15_S3.0.ps}}
	\subfigure[decreasing $k^\prime_{Sb}$]{\pic{GBP_box_hyb_kSt60_kSb070_d0.15_S1.0.ps}}
	\caption{Configuration snapshots showing the HAN and V states of a hybrid anchored slab
	of $N=1000$ PHGO particles with $k=5$ at $\rho^{*} = 0.15$ with $k^\prime_{St} = 0.6$ 
	and $k^\prime_{Sb} = 0.7$. Those configuration have been obtained from series of 
	simulations with 
	increasing (a) and decreasing (b) values of $k^\prime_{Sb}$}
	\label{fig:GBP_RSUP_hyb_kSt060_kSt070_snaps}
\end{figure}


For the two lower values of $k^\prime_{St}$ (representing stronger homeotropic anchoring), 
although the two 
series of simulations lead to hysteresis in the values of $\overline{Q}^{Sb}_{zz}(k^\prime_{Sb})$, no
bistability can be observed. None of the state points considered here correspond to a situation 
where the values of $\overline{Q}^{Sb}_{zz}$ obtained from the two series are both significantly
different from zero and of opposite signs.\\
%
With $k^\prime_{St} = 0.6$, however, a small bistable region  is recovered around
$k^\prime_{Sb}= 0.7$; 
the values of $\overline{Q}^{Sb}_{zz}$ are different and of opposite signs. The bistability value for 
$k^\prime_{Sb} = 0.7$ is $0.914$ which is very close to that obtained with symmetric anchored 
systems. Configuration snapshots corresponding to the HAN and V states
of the cell at this state point are given on Figure~\ref{fig:GBP_RSUP_hyb_kSt060_kSt070_snaps}.\\
%
These results show that reducing the strength of the anchoring at the top surface allows to
recover the bistability region by increasing the hysteresis in $\overline{Q}^{Sb}_{zz}$.
The value $k^\prime_{St} = 0.6$ seems to be the highest reasonable that can be used,
as according to $\overline{Q}^{Su}_{zz}(k^\prime_S)$, the use of a higher value would not lead to homeotropic
anchoring at the top surface.\\


The results from these simulations are reasonably encouraging for the application of the model to
the HAN to V switching since, despite the very narrow bistability region for $k^\prime_{Sb}$, 
the difference
between the $\overline{Q}^{Sb}_{zz}$ values obtained from the two series with $k^\prime_{St}=0.6$ 
appears sufficient for the model to be used in the display modeling. Also, the snapshots show 
encouraging  HAN and V states which should be further improved by the use of wider systems.









