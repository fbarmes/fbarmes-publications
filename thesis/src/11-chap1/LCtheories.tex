\section{Theoretical approach to liquid crystals.}


As interest in liquid crystal grows, the number of theories used to describe their
complicated phase behaviour increases similarly. Here the focus is brought on to bear on to molecular
theories that, using statistical mechanics, take the intermolecular potential as a starting point 
from which to deduce the macroscopic phase behaviour.\\
%
Theoretical studies of simple atomic fluids~\cite{theorySimpleLiquids} show that the liquid 
phase can be described effectively using intermolecular potentials of the Lennard-Jones form,
\ie containing both long range attractive and short range repulsive component. Mesogens 
can be represented with a similar class of potentials, though account has to be taken of the
inherent in their elongated shape. One significant question considered by these theories is
which of the repulsive and attractive components of the interaction are of greater importance in
mesophase formation. As a result, theories have been developed which consider both of these
contributions to the anisotropic potentials and thus quantify their respective influences. The two main
and complementary approaches are described in the following Sections, namely Onsager and Maier-Saupe
theories.


%===============================================================================================
%===============================================================================================
\subsection{Density functional theory}

The phase structure of simple atomistic fluids can be described successfully using hard spheres as
models, that is considering only the repulsive part of their pair
potential~\cite{BarkerHenderson76}. The extension of this principle to the mesoscale was
first achieved by Lars Onsager in 1949~\cite{Onsager49} in a study of colloidal particles
(tobacco virus). The main idea underlying this seminal work is that the mechanism for spontaneous 
ordering in a system of hard molecules is based on competition between the orientational entropy that
destroys nematic order and the positional entropy that favors it~\cite{deGennes}.\\

Onsager theory is derived from the cluster approach~\cite{greenBook} and is a density functional theory 
in which the free energy $\mathcal{F}$ is expressed as a density virial expansion~\cite{PadillaVelasco97}.
\begin{eqnarray}
	\mathcal{F} &=& \mathcal{F}^{id} + \mathcal{F}^{ex}	\\
	\mathcal{F}^{id} &=& N(\log\rho - 1) + N \int f(\vecth{u})\log\lp 4\pi f(\vecth{u}) \rp d\vecth{u} \\
	\mathcal{F}^{ex} &=& \rho\mathcal{B}_2 \lp f(\vecth{u}) \rp
			+ \frac{1}{2}\rho^{2}\mathcal{B}_3 \lp f(\vecth{u}) \rp
			+ \frac{1}{3}\rho^{3}\mathcal{B}_4 \lp f(\vecth{u}) \rp
			+ \ldots	\label{exess_F}
\end{eqnarray}
%
Where $\mathcal{F}^{id} $ and $\mathcal{F}^{ex} $ are the ideal and excess parts of $\mathcal{F}$ and
$f(\vecth{u}) $ is the orientational distribution function that depends on the particle orientation
vector $\vecth{u}$. Each $\mathcal{B}_i$ represents the excluded volume in a cluster of $i$ 
particles.

Onsager made the following assumptions~:
\begin{itemize}
	\item The molecules (spherocylinders of length $L$ and diameter $D$) interact only through steric
	repulsion (no interpenetration).
	%
	\item The volume fraction is much smaller than $1$.
	%
	\item The rods are very long ($L \gg D$).
\end{itemize}

Within these assumptions, Onsager showed that the virial expansion can be truncated after the second
coefficient; he showed that the third one vanishes and assumed the same for the the following
coefficients. This makes his approach the simplest form of density functional theory.

The implication of his results is that, in the limits considered, the N-I transition 
can be explained exclusively using short range repulsive forces. However, the approximations made
in this theory worsen
considerably as the particle elongation is reduced because the density of the transition is not
vanishing. Therefore this theory can not be applied to mesogens with a standard elongation of 3 to
5. Early computer simulations of hard prolate particles by Vieillard-Baron~\cite{VieillardBaron72} and 
later Frenkel \etal~\cite{FrenkelMulder81,FrenkelMulder85} showed qualitative but not 
quantitative agreement with Onsager's theory\\
This does not imply that density functional theory can not be applied to liquid crystalline
behaviours, only that, for the accurate description of mesogens, more virial coefficients are
needed. The most obvious approach, then is to calculate higher order coefficients, as
been done in~\cite{Vega97,VelascoPadila94}, but the difficulty of the task increases
significantly with the order of the coefficients. A better approach is the use of resummation
techniques such as the y-expansion~\cite{MulderFrenkel85,TjiptoMargoEvans90} that allow the indirect 
inclusion of high order coefficients. Some other resummation methods have been
used successfully on single component~\cite{Parsons79,Williamson95} and 
mixture systems~\cite{StroobantsLekkerkerker84,CampAllen96} leading, recently, to considerable
improvements in the description of anisotropic fluids~\cite{CinacchiSchmid02}.



%===============================================================================================
%===============================================================================================
\subsection{Maier Saupe theory.}


In the early $20^\mathrm{th}$ Century, Born showed that the anisotropic component of the pair potential
was responsible for the order-disorder transitions in nematic phases~\cite{greenBook}.
This was later expanded by Maier and Saupe to give rise to the so called Maier Saupe (MS)
theory~\cite{MaierSaupe58,MaierSaupe59,MaierSaupe60}.\\
The configurational partition function of a fluid is expressed as~:
\begin{equation}
	Q_N = \frac{1}{N!}\int e^{-\beta U(\vect{X}^N) }d\vect{X}^N
\end{equation}
where $\vect{X}^N$ represents the full set of positional and orientational coordinates for the $N$
particles of the fluid. In the case of a perfect gas, as every particle's behaviour  can be
taken to be independent of all others~\cite{structMatter}, the partition function can be 
transformed into the
product of $N$ single particle partition functions, each of which is easily solvable. However,
in the case
of a fluid phase, and even more so a mesophase, the relatively high density implies that each 
particle's energy is dependent on several other particles' coordinates so that the previous 
simplification is no longer valid.\\
MS theory aims to resolve this difficulty using the so called molecular field
approximation. This approximation removes the need for the consideration of each individual 
pair potential; rather each particle is taken to reside in a field which mimics the presence of
all the other particles. Effectively each particle is assumed to be was moving in an 
energy continuum. As a result $U(\vect{X}^N)$ can be written as the sum of $N$ energies, each of 
which is a function of the coordinates of a single particle~:
\begin{equation}
	U(\vect{X}^N) = \sum_{i=1}^N U(\vect{X}_i)
\end{equation}

The problem, then, is to find an expression for the mean-potential experienced by the particles. 
Several approaches have been used for this. The most intuitive approach is to average the anisotropic 
pair potential over the coordinates of one  particle~\cite{HumphriesJames71} while one of the most
simple and rigorous is to start from the singlet distribution function and solve the
Bogoliubov-Born-Green-Kirkwood-Yvon hierarchy of equations as described in~\cite{greenBook}. 
In a translationally invariant situation, the general form of the final mean potential depends only
on $\beta$, the angle between the particle under consideration and the director $\vecth{n}$, so:
\begin{equation}
	U(\beta) = -\epsilon \overline{P_2} P_2(cos^2\beta)
\end{equation}
where $\overline{P_2}$ is the nematic order parameter, $P_2(x)$ is the second order Legendre
polynomial of $x$ and $\epsilon$ is a scaling constant which is given by~:
\begin{equation}
	\epsilon = -V\lp\frac{\partial \epsilon}{\partial V}\rp_T
\end{equation}
Another form for $\epsilon$ is given in~\cite{Intro_LC} by the identification~:
\begin{equation}
	\epsilon = \frac{A}{V^2}
\end{equation}
where the value of $A$ is determined by the interaction properties of the molecules.
In any case the general approach for the implementation of Maier Saupe theory is as follows~:
\begin{itemize}
	\item Choose the anisotropic term of the pair potential
	\item Implement the mean field approximation
	\item Determine the mean potential
	\item Deduce the singlet distribution function
	\item Use this function to calculate the entropy, Helmholtz free energy and order parameters.
\end{itemize}
The two major elements of the theory that influence its accuracy are the mean field 
approximation and the form of the mean potential.\\

Maier Saupe theory can be tested using both computer simulation and real experiment, though
 the former has an advantage since it can test the validity of the mean field approximation 
as the mean-potential can be specified in the `computer experiment'. The theory is reasonably
successful in describing, qualitatively, the behaviour of mesogens, showing a first order IN
transition. The temperature dependence of the order 
parameters is also well described qualitatively. The limits of the Maier Saupe theory become 
apparent in the quantitative predictions, errors being attributed, in part, to the use of 
the mean field approximation (although some improvement can be made by improving the form of the 
potential.) A more fundamental weakness of the mean field approximation is that it neglects
spacial and orientational correlations between molecules.\\

The main conclusion that can be drawn from this is that while MS theory is most effective at 
long range it can successfully describe liquid crystalline behaviour. This implies that short-range 
repulsive forces have little role to play, whereas short range potentials have been shown to
be responsible for both the ordering of nematics and the structure of normal liquids. This apparent
discrepancy can be resolved by appreciating that the long range attractive 
part of the potential used in MS theory  can be regarded  as describing the interactions 
between clusters of highly ordered particles. 







