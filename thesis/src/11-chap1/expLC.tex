\section{Experimental study of liquid crystals.}

%===============================================================================================
%===============================================================================================
\subsection{Optical polarising microscopy.}

Optical polarising microscopy is the main technique used for liquid crystalline phase
characterization~\cite{Intro_LC}. Historically it was also the first technique used by Lehmann
when he studied the liquid crystalline samples provided by Reinitzer.\\
The technique consists of observing, under a microscope, a sample sandwiched between crossed polarizers. 
An isotropically liquid phase does not affect the light and, therefore, no
light can cross the analyzer. In the case of a liquid crystalline phase, however, the birefringent 
property of the material induces refraction of the light according to the director orientation. 
Since only the component of the refracted light parallel to the analyzer polarization direction is
transmitted, the intensity of the transmitted light varies from white if
$\dotProdP{\vecth{n}}{\vecth{p}}= 0$ to black if $\dotProdP{\vecth{n}}{\vecth{p}}= 1$ where 
$\vecth{p}$ is the polarisation direction direction of the analyser. Moreover, because of defects 
in the structure and disclination lines, the orientation of $\vecth{n}$ typically varies with 
position and, therefore, so does the intensity of transmitted light.
Distinct patterns can be observed for different mesophases. Examples of these patterns can be
found in~\cite{Nature_phase}.\\
Nematic phases induce the so called Schlieren textures where black threads marking the
disclination lines can be observed. These threads lead to the name `nematic' being used for this
phase. \SmA liquid crystals have a  `fan-like' pattern when viewed through crossed polarizers and 
\smC a combination of both. The \smB phase induces an altogether different pattern of `mosaic' or 
broken fan textures. However not every phase can be distinguished clearly. For example, \smI 
and \smF phases induce patterns very similar to that of the \smC phase which makes the task 
of identification very difficult if this is the only technique to be used.



%===============================================================================================
%===============================================================================================
\subsection{Differential scanning calorimetry.}

Differential scanning calorimetry (DSC) is a technique used to complement optical 
polarising microscopy in the phase characterization of liquid crystals~\cite{Intro_LC}. 
This technique measures
enthalpy changes ($\Delta H$) at phase transitions. The phase type of the sample is not 
examined using this technique, but the value of the enthalpy gives some information about the degree 
of molecular order in a mesophase.\\
With this technique, two independently heated furnaces are used. One is empty or contains a 
reference sample (usually gold) and  the second contains the sample under study. Both furnaces 
are linked to control loops which
insure that they are kept at the same temperature as each other. Upon cooling or heating, the heat 
absorbed by or released from the sample in order to keep the furnaces at the same temperature is 
measured. Differences between the heat measurement for the two furnaces indicate phase transitions. 
With this technique, temperatures  ranging from $-180^\circ C$ to $600^\circ C$ can be accessed.\\

Thermodynamics state that there are two types of phase transition discontinuous ($1^\mathrm{st}$ order)
and continuous ($2^\mathrm{nd}$ order) corresponding to discontinuities in, respectively, the first and
second derivatives of the Gibbs free energy $G$~:
%
\begin{gather}
	G = H - TS\\
	\lp\frac{\partial G}{\partial T}\rp = -S	\\
	\lp\frac{\partial^2 G}{\partial T^2}\rp = -\frac{C_p}{T}
\end{gather}
%
Thus, a first order transition induces a discontinuity in the entropy and a peak of the DSC baseline
can be observed, whereas a second order transition is indicated by  inflexion of the baseline.\\
Thus a DSC trace can reveal phase transitions that would be missed by optical 
polarising microscopy because of the smallness of changes in the optical properties. Conversely phase
transitions with small enthalpy changes but rather different optical properties can be missed
with the DSC but are easily detected with optical polarising microscopy.


%===============================================================================================
%===============================================================================================
\subsection{X-ray and neutron diffraction.}

X-ray diffraction is one of the most effective techniques for liquid crystalline phase
characterization~\cite{introSoftMatter, greenBook}. Here the mesophase is characterised by analysis of
the diffraction pattern of an X-ray beam incident upon a sample in which the molecules are
aligned with the beam. According to  Bragg's law, diffraction
is obtained at an angle $\theta$ when $\lambda = 2d\sin\theta$ where $\lambda$ is the light
wavelength and $d$ the intermolecular spacing.\\
%
Liquid crystal diffraction patterns are characterized by two lateral vertical clear areas that account 
for the vertical alignment of the molecules. In the case of a nematic, horizontal clear areas,
corresponding to diffuse low intensity peaks, can be observed above and below the centre of the
pattern. In the case of a smectic, the latter are replaced by points corresponding to sharp high
intensity peaks induced by the smectic layering. In the case of a \smA, those points
are located on the vertical axis whereas they lie at an angle in the case of a tilted
smectic such as a monodomain \smC.

