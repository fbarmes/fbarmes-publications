\section{Applications}



%===============================================================================================
%===============================================================================================
\subsection{Liquid Crystal Displays}

Since the discovery of electro-optical effects in liquid crystals in about 1968, their main
applications have been in the display technology~\cite{Shanks82}. The first
device that could be used industrially is the Twisted Nematic 
device (see Figure~\ref{fig:TNcell}).\\
%
This device uses a liquid crystal sandwiched between two electrically conductive glass plates 
rubbed so as to induce a planar surface arrangement. This cell is placed between crossed polarizers. 
The direction of rubbing on the glass plates is made parallel to the local polarizing direction, thus 
inducing a twist of $90^\circ$  from the top to the bottom of the cell. Due to the optical 
anisotropy of the liquid crystal, the direction of polarization of the light in the cell is twisted 
(following the director,) as a result of which the light is transmitted through the analyzer. 
This correspond to a light cell.\\
Upon application of an electric field $\vect{E}$ between the glass plates, the 
molecules reorient to be approximately parallel to $\vect{E}$; in this state, the polarization 
of the light is left unchanged by the liquid crystal molecules and therefore no light can be 
transmitted. This corresponds to the dark state.

\picW = 7cm
\begin{figure}
	\centering
	\pic{TNCell.ps}
	\caption{Schematic representation of the Twisted Nematic display cell. The `on' state
	without applied field is shown on the left and the `off' state with applied electric
	field represented by the black arrow is shown on the right.}
	\label{fig:TNcell}
\end{figure}

The TN cell is mostly used in the wrist-watch type of display~\cite{Geelhaar98} and can employ
different types of compound~\cite{Eidenschink85,Eidenschink89}, the most simple of which is the 
5CB. In order to meet the requirements for more sophisticated displays, such as those used in 
cellular phones or  laptop computers, more advanced display cell have been designed. The first 
such improvement is the Supertwisted Nematic Cell (STN). This is a direct refinement of the TN 
cell but with more advanced mesogens allowing a twist of $270^\circ$ instead of $90^\circ$. This 
results in a sharper and faster transition between the light and dark states.\\
Further refinements have been achieved by improving the addressing of the display's individual
pixels. This lead to the development of the active matrix TFT display which are, to date, the most
used screens for laptop computers.\\
The current trend is towards the development of bistable  
displays~\cite{KreuzerTschudi92,GiocondoLelidis99,DavidsonMottram02}
where, by having two stable arrangements corresponding to the light and dark states of the cell, the
electric field no longer needs to be applied to maintain the dark state. Rather an
electric pulse is used to switch the cell. The main advantages of such display are their much
reduced power consumption, which is of crucial importance in portable devices. Additionally, such
displays can be used as optical storage devices. Further developments in this area include 
tristable nematics displays, which, have the potential to yield extremely versatile
functionality~\cite{KimYoneya02}.

%===============================================================================================
%===============================================================================================
\subsection{Other applications.}

There are numerous applications for liquid crystals other than display. In laser optics, for
instance, a laser can produce grid patterns in a nematic phase which, in turn, can be used as
switches for another laser. Polymer dispersed liquid crystal (PDLC) sheet can be formed into
big panels which, upon application of an electric field, can be made opaque or transparent. A
rather versatile bathroom window can be produced that way, as shown in~\cite{Intro_LC}.\\
Liquid crystals have also been proposed for applications in 
engineering~\cite{Eidenschink88,Eidenschink89a,EidenschinkKonrath99} that exploit the variation of
the viscosity coefficients of mesogens in different phases. This may lead to the development of
very efficient lubricants or bearings which can act as breaks if their temperature exceeds some
threshold.\\
Finally, one possibly surprising area for the application of liquid crystal science is the human body
itself, but many living cells and viruses display and utilise liquid crystalline 
prospectives~\cite{Goodby98}.

