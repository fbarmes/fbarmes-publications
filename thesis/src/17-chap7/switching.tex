

\section{Flexoelectric switching}

Here the possibility of two-way switching between the HAN and vertical states using the
flexoelectric properties of the model and the bistable nature
of the bottom surface is investigated. This is performed in three steps. First, the
stability of the HAN and vertical states are assessed by performing the easy switching
between the two states, and considering only the dielectric term in the electric energy 
(\ie $\mu=0$). Then, the possibility of hard switching is investigated by including a
dipolar term in the electric energy. This requires finding an
appropriate combination of values for the electric field magnitude, dielectric anisotropy and the 
dipole moment. Finally, keeping the same parameterisation as that used to achieve hard switching, 
the possibility of easy switching is investigated again so as to check that the newly introduced 
dipolar contribution does not hinder the reverse switching.\\
In order to use the most favorable conditions for achieving switching and obtaining a smooth
structural transition from homeotropic to planar in an HAN cell, all simulations used the Monte
Carlo method in the canonical ensemble and 
used systems of $N=2000$  particles in a cell of width $8\so$. Hybrid anchoring with the
parameterisation  corresponding to the maximum hysteresis obtained in the previous section, 
that is $k^\prime_{St} = 0.6$ and $k^\prime_{Sb} = 0.7$, was employed.

%=================================================================================================
%=================================================================================================
\subsection{Easy switching}
\label{ss:easySwitch}

The aim of the simulations performed here is mainly to test for the stability of the HAN and
vertical states found in the previous Section. Therefore, easy switching has been attempted
between those two states by considering only the dielectric term in the particle-field interaction
(\ie $\delta\epsilon \neq 0, \mu = 0$).\\
Two series of simulations have been performed using an hybrid anchored slab with the surface
parameterisation given at the beginning of this Section. Starting with an HAN configuration, the
first simulation attempted to switch to the vertical state by application and subsequent removal
of an electric field
and considering the particles to have a positive dielectric anisotropy. Then, taking the particles to
have a negative dielectric anisotropy, a second series of simulations was used to switch back to
the HAN state starting from the vertical state obtained from the first series.\\
Each series consisted of a first run of $2.0.10^6$ sweeps, performed to equilibrate the
starting configuration with the field off. This was followed by a simulation of
$0.5.10^6$ sweeps with the field on, which primed the switching. The resulting system
was equilibrated in the new state for another $2.10^6$ sweeps with the electric field removed.
The electric parameterisation used here was $E=1.0$, $\delta\epsilon = \pm 1.0$ and $\mu = 0$.



\picW = 10cm
\begin{figure}
	\centering
	\subfigure[]{\picL{QzzWa_easyHANtoV.ps}}
	\subfigure[]{\picL{QzzWa_easyVtoHAN.ps}}
	\caption[Evolution of $\overline{Q}^{Sb}_{zz}$ with sweep number in the
	simulations of the HAN to vertical(a) and vertical to HAN (b) `easy switching'
	of the hybrid anchored slab described at the beginning of this Section.]
	{Evolution of $\overline{Q}^{Sb}_{zz}$ with sweep number in the
	simulations of the HAN to vertical(a) and vertical to HAN (b) `easy switching'
	of the hybrid anchored slab described at the beginning of this Section. 
	Here only dielectric interactions between the particles and the field are 
	considered. The data for the last and first field off series of (a) and (b) respectively
	have been
	obtained from the same simulation.}
	\label{fig:easySwitch_QzzWa}
\end{figure}


\picW = 4cm
\begin{figure}
	\centering
	\subfigure[field off]{\pic{GBP_box_01.ps}}
	\subfigure[field on]{\pic{GBP_box_02.ps}}
	\subfigure[field off]{\pic{GBP_box_03.ps}}
	\caption{Configuration snapshots corresponding to three phases (a) to (c) 
	of the HAN to vertical `easy switching' of the hybrid anchored slab.}
	\label{fig:easySwitch_HANtoVsnaps}
\end{figure}

\picW = 4cm
\begin{figure}
	\centering
	\subfigure[field off]{\pic{GBP_box_03.ps}}
	\subfigure[field on]{\pic{GBP_box_04.ps}}
	\subfigure[field off]{\pic{GBP_box_05.ps}}
	\caption{Configuration snapshots corresponding to three phases (a) to (c) 
	of the vertical to HAN `easy switching' of the hybrid anchored slab.}
	\label{fig:easySwitch_VtoHANsnaps}
\end{figure}

The evolution of $\overline{Q}_{zz}$ at the bottom surface as a function of the number of sweeps is
shown on Figure~\ref{fig:easySwitch_QzzWa}. The snapshots corresponding to the final
configurations from phase of the simulation sequence are shown on 
Figures~\ref{fig:easySwitch_HANtoVsnaps} and~\ref{fig:easySwitch_VtoHANsnaps}.\\
These results, along with the corresponding profile data (not shown) confirm both the stability
of the HAN and vertical states for this system and the ability of the easy switching mechanism
to switch between them.


%=================================================================================================
%=================================================================================================
%\clearpage
\subsection{Hard switching}
\label{ss:hardSwitch}

We now turn to the possibility of achieving hard switching between the HAN and vertical states of 
the slab, following the theoretical treatment of~\cite{DavidsonMottram02}. Here both the
dielectric and dipolar terms in the particle-field interaction are required (\ie
$\delta\epsilon \neq 0$ and $\mu \neq 0$). Only the case of particles with a negative dielectric
susceptibility is considered. 
The aim here is to find an appropriate parameterisation which allows for switching from the HAN to
the vertical states after application of an electric field along $\vecth{z}$. 
Reference~\cite{DavidsonMottram02} shows that such switching can be achieved using 
$E < 0$.

\subsubsection{Choice of $\vect{E}$}

The first step is to find an appropriate value for the electric field. This needs to be strong
enough to allow the dipolar contribution to distort the director profile and bring the system to
an intermediate state that will relax into the vertical state after removal of the field. 
If the field is too strong, however, the dielectric effect (which scales as $E^2$) will
dominate, causing the HAN state to be
stabilised and thus, rendering the switch to the vertical state impossible.\\
In order to find an appropriate value for $E$, a slab in a HAN state has been simulated using
$\delta\epsilon = -1.0$, $\mu = 1.0$ and different values for the electric field in the range
[-0.02:-6.0]. For each value
of $E$, the system was subject to an equilibration run of $0.25.10^6$ sweeps followed by a
production run of the same length. Figure~\ref{fig:hardSwitch_QzzEinfl} shows the $Q_{zz}$
profiles obtained from the production runs for a selection of these field strengths.

%-----------------------------------------------------
\picW = 12cm
\begin{figure}
	\centering
	\picL{hardSwitch_QzzProf_Einfl.ps}
	\caption{$Q_{zz}(z)$ profiles for an HAN slab subject to an applied electric field along 
	$\vecth{z}$ and different values of $E$.}
	\label{fig:hardSwitch_QzzEinfl}
\end{figure}
%-----------------------------------------------------

These results show that, given the chosen parameterisation adopted for $\delta\epsilon$ and $\mu$,
values less negative than $E = -0.2$ do not induce any significant distortions near the bistable
surface and that, despite the
applied electric field, the system always remains in the HAN state. For fields of strength more
negative than
$-0.4$, in contrast, the dielectric contribution dominates and stabilises the HAN state. 
For the highest absolute
values of $E$, all particles are forced to be parallel to the surfaces, even those subject to the
homeotropic anchoring of the top surface. Figure~\ref{fig:hardSwitch_QzzEinfl} however,
indicates
promising behaviour for $E=-0.2$, for which the dipolar coupling seems to be strong enough 
to induce a slight distortion of the profile without there being too strong a dielectric effect. 
This raises the prospect that by increasing the
value of $\mu$ and keeping all other parameters constant, this distortion can
be increased to the extend that switching to the vertical state can be
achieved.


\subsubsection{Choice of $\mu$}

Here, an attempt is made to identify an appropriate value of $\mu$ so that, upon application of the
electric field with $E=-0.2$, the dipolar effect induces enough of a distortion to cause a 
HAN configuration system to equilibrate into a vertical state upon removal of the field.


%-----------------------------------------------------
\picW = 14cm
\begin{figure}
	\centering
	\pic{hSwitch_muInfl_Qzz01.ps}
	\caption{$Q_{zz}$ profiles for an HAN slab subject to an applied electric field with
	$E=-0.2\vecth{z}$ and $\delta\epsilon = -1.0$ and different values of the 
	dipolar moment $\mu \in [1.0:2.0]$.}
	\label{fig:hardSwitch_QzzmuInfl01}
\end{figure}
%-----------------------------------------------------

%-----------------------------------------------------
\picW = 14cm
\begin{figure}
	\centering
	\pic{hSwitch_muInfl_Qzz02.ps}
	\caption{$Q_{zz}$ profiles for a slab in the HAN configuration and 
	subject to an applied electric field with
	$E=-0.2\vecth{z}$ and $\delta\epsilon = -1.0$ and different values of the 
	dipolar moment $\mu \in [2.5:3.5]$.}
	\label{fig:hardSwitch_QzzmuInfl02}
\end{figure}
%-----------------------------------------------------

\picW = 4cm
\begin{figure}
	\centering
	\subfigure[start]{\pic{GBP_hardSwitch_start.ps}}
	\subfigure[field on]{\pic{GBP_box_Em0.2_mu2.5on.ps}}
	\subfigure[field off]{\pic{GBP_box_Em0.2_mu2.5off.ps}}
	\caption{Configuration snapshots corresponding to the hard switching of a slab in an
	initial HAN state with $E=-0.2\vecth{z}$ and $\delta\epsilon = -1.0$ and $\mu = 2.5$.}
	\label{fig:hardSwitch_snaps2.5}
\end{figure}

\begin{figure}	
	\centering
	\subfigure[start]{\pic{GBP_hardSwitch_start.ps}}
	\subfigure[field on]{\pic{GBP_box_Em0.2_mu3.0on.ps}}
	\subfigure[field off]{\pic{GBP_box_Em0.2_mu3.0off.ps}}
	\caption{Configuration snapshots corresponding to the hard switching of a slab in an
	initial HAN state with $E=-0.2\vecth{z}$ and $\delta\epsilon = -1.0$ and $\mu = 3.0$.}
	\label{fig:hardSwitch_snaps3.0}
\end{figure}

\begin{figure}
	\centering
	\subfigure[start]{\pic{GBP_hardSwitch_start.ps}}		
	\subfigure[field on]{\pic{GBP_box_Em0.2_mu3.5on.ps}}
	\subfigure[field off]{\pic{GBP_box_Em0.2_mu3.5off.ps}}
	\caption{Configuration snapshots corresponding to the hard switching of a slab in an
	initial HAN state with $E=-0.2\vecth{z}$ and $\delta\epsilon = -1.0$ and $\mu = 3.5$.}
	\label{fig:hardSwitch_snaps3.5}
\end{figure}

%-----------------------------------------------------

In order to achieve this, simulations of the slab have been carried out taking the HAN 
configuration as an initial state. For each $\mu$ value the simulation sequence performed
consisted of two runs (one for equilibration and
one for production) with an applied electric field followed by two runs (equilibration and
production) where the field was removed. Each run comprised $0.25.10^6$ sweeps and the 
parameterisation $E=-0.2$ and $\delta\epsilon = -1.0$ was
used. The first two runs were used to establish the `field-on' intermediate state while the last
to generated the state to which the system subsequently relaxed. 
The series of simulations described above was performed with six values of $\mu$ in the range
$[1.0:3.5]$. 


The $Q_{zz}(z)$ profiles corresponding to the obtained  field `on' and `off' 
configurations are shown on Figure~\ref{fig:hardSwitch_QzzmuInfl01}
and~\ref{fig:hardSwitch_QzzmuInfl02} along with the profiles corresponding to the HAN and
vertical states which are shown for comparison. Configuration snapshots of the field -on and
field-off  structures for $\mu = 2.5$, $3.0$ and $3.5$ are shown, respectively, in 
Figures~\ref{fig:hardSwitch_snaps2.5}, \ref{fig:hardSwitch_snaps3.0} 
and~\ref{fig:hardSwitch_snaps3.5}\\

From the $Q_{zz}(z)$ data, it appears that the switching from the HAN state to the vertical
state is possible using values of $\mu \geq 2.5$. As $\mu$ is increased, so does the 
distortion  induced by the dipolar term in $U_e$; more specifically, the profile at the bottom
surface is changed, so inducing the bulk part of the cell to modify its orientation
co-operatively.
Upon removal of the field, for $\mu$ values at which this distortion is sufficient, 
the cell equilibrates into the
vertical state, thus confirming the results of~\cite{DavidsonMottram02}.


%=================================================================================================
%=================================================================================================
%\clearpage
\subsection{Reverse switching}

In Sections~\ref{ss:easySwitch} and~\ref{ss:hardSwitch}, respectively, it was shown that easy 
switching between the HAN and vertical states
can be achieved using only the dielectric effect and that,
using an appropriate parameterisation  ($E=-0.2$, $\delta\epsilon = -1.0$ and $\mu \in
[2.5:3.5]$), hard switching from the HAN to the vertical state can be achieved. However, 
easy switching from vertical to HAN state is not necessarily possible using the same
parameterisation as that used to achieve hard switching but with $E>0$ since the dipolar
contribution
might be too strong to permit the formation of the HAN state.\\


%-----------------------------------------------------
\picW = 14cm
\begin{figure}
	\centering
	\pic{revSwitch_Qzz.ps}
	\caption{$Q_{zz}$ profiles for a slab in the vertical configuration and 
	subject to an applied electric field with
	$E=0.2\vecth{z}$ and $\delta\epsilon = -1.0$ and different values of the 
	dipolar moment $\mu \in [2.5:3.5]$.}
	\label{fig:revSwitch_QzzmuInfl}
\end{figure}
%-----------------------------------------------------

\picW = 4cm
\begin{figure}
	\centering
	\subfigure[start]{\pic{GBP_box_Em0.2_mu2.5off.ps}}
	\subfigure[field on]{\pic{GBP_box_mu2.5on.ps}}
	\subfigure[field of]{\pic{GBP_box_mu2.5off.ps}}
	\caption{Configuration snapshots corresponding to the hard switching of a slab in an
	initial vertical state with $E=0.2\vecth{z}$ and $\delta\epsilon = -1.0$ and $\mu = 2.5$.}
	\label{fig:revSwitch_snaps2.5}
\end{figure}

\picW = 4cm
\begin{figure}
	\centering
	\subfigure[start]{\pic{GBP_box_Em0.2_mu3.0off.ps}}
	\subfigure[field on]{\pic{GBP_box_mu3.0on.ps}}
	\subfigure[field off]{\pic{GBP_box_mu3.0off.ps}}
	\caption{Configuration snapshots corresponding to the hard switching of a slab in an
	initial vertical state with $E=0.2\vecth{z}$ and $\delta\epsilon = -1.0$ and $\mu = 3.0$.}
	\label{fig:revSwitch_snaps3.0}
\end{figure}

\picW = 4cm
\begin{figure}
	\centering
	\subfigure[start]{\pic{GBP_box_Em0.2_mu3.5off.ps}}
	\subfigure[field on]{\pic{GBP_box_mu3.5on.ps}}
	\subfigure[field off]{\pic{GBP_box_mu3.5off.ps}}
	\caption{Configuration snapshots corresponding to the hard switching of a slab in an
	initial vertical state with $E=0.2\vecth{z}$ and $\delta\epsilon = -1.0$ and $\mu = 3.5$.}
	\label{fig:revSwitch_snaps3.5}
\end{figure}



This issue is addressed here by
attempting to perform the easy switching again but this time with both the dielectric and dipolar 
term included in the particle-field interaction. A similar parameterisation as that used in
Section~\ref{ss:hardSwitch} is applied here, the difference being that the electric field
director is taken to be positive. As a result the parameterisation $E=0.2$, $\delta\epsilon = -1.0$ and
$\mu\in[2.5:3.5]$ is used. Several values of $\mu$ are considered so as to also investigate the
effect of increasing $\mu$. These simulations were performed using a similar sequence as 
that used in~\ref{ss:hardSwitch}, the main difference being that the initial configuration for each
series with different $\mu$ was the final configuration obtained for the vertical state 
from the  hard switching simulations with the appropriate $\mu$ value.\\
The $Q_{zz}(z)$ profiles corresponding to the `field-on' and `field-off' states obtained for each value
of $\mu$ are shown on Figure~\ref{fig:revSwitch_QzzmuInfl} and the corresponding 
configuration snapshots for $\mu = 2.5$, $3.0$ and $3.5$ are shown, respectively, on
Figures~\ref{fig:revSwitch_snaps2.5}, \ref{fig:revSwitch_snaps3.0}
and~\ref{fig:revSwitch_snaps3.5}.\\

The $Q_{zz}(z)$ data show that, upon application of the field, most of the vertical slab
arrangement remains 
undistorted, expect for a
region near the bottom surface which adopts a planar arrangement. Upon removal of the field,
this small interfacial distortion proves sufficient to seed this orientation into the bulk part of
the cell. Because of the homeotropic top surface influence, the slab then recovers the HAN state.
These results also show that the distance from the bottom surface over which the cell's vertical
alignment is
distorted in the field-on state decreases with increased $\mu$. For the run lengths used here,
this has the effect of producing 
HAN states of reducing quality as $\mu$ is increased. This trend suggests that with
$\mu>3.5$, although the hard switching is possible, easy switching might be inhibited by
high values of the dipolar coupling term. As a result it can be concluded that switching 
between the HAN and vertical states of the hybrid anchored slab can be achieved if the dipolar
term to the electric field is included, but that the 
parameterisation should be compatible with a window of electric field and
dipolar coupling strength and that if $\delta\epsilon =< 0.0$, $E \sim 0.2\delta\epsilon$ 
and $\mu \sim -2.5\delta\epsilon$.












