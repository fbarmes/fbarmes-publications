

\section{The flexoelectric display}

The model proposed by Davidson and Mottram~\cite{DavidsonMottram02} is an idealised 
representation of an existing display cell, the ZBD device~\cite{ZBD}, in which the grating 
morphology of the ZBD device is treated as a planar surface which has homeotropic or planar 
anchoring states. The geometry
considered in~\cite{DavidsonMottram02} is shown in Figure~\ref{fig:bistDisplay}.
In this, a system of flexoelectric mesogens is confined in a slab geometry; 
the top surface induces a monostable homeotropic anchoring while the bottom surface is
bistable and allows both homeotropic and planar arrangements. In the case of an 
homeotropic bottom surface anchoring, the cell is in the so called
vertical state (Figure~\ref{fig:bistDisplay}a) and all particles have a vertical orientation; 
in the case of a bottom planar anchoring, the state is in a so called hybrid aligned 
nematic (HAN) state (Figure~\ref{fig:bistDisplay}b) where the particle
orientations change continuously from homeotropic on the top surface to planar at the bottom.\\

\picW = 5cm
\begin{figure}
	\centering
	\subfigure[The Vertical state]{\pic{bistDisplay_V.ps}}
	\hspace*{10mm}
	\subfigure[The HAN state]{\pic{bistDisplay_HAN.ps}}
	\caption{Schematic representation of the two stable states of the 
	display cell considered in~\cite{DavidsonMottram02}. The top surface induces a
	monostable homeotropic anchoring while the bottom surface is bistable and induces 
	both homeotropic and planar arrangements. }
	\label{fig:bistDisplay}
\end{figure}

Easy switching between these states can be achieved by applying an electric field between the two
substrates which will reorient the particles either parallel or perpendicular to the 
surfaces for respectively a negative or positive molecular dielectric susceptibility. 
Upon removal of this field, because of the strong homeotropic anchoring at the top
surface, particles close to that substrate will recover the homeotropic anchoring whereas 
particles close to the bottom substrate will keep the field induced orientation because of the
bottom surface bistability. As a result both HAN and vertical states can be produced with an
appropriate choice of starting configuration and electric field for a given value (and sign) of
the molecular dielectric susceptibility.\\

The difficulty that then arises is that of achieving the reverse switching, the so called hard 
switching from V to HAN if $\delta\epsilon > 0$ or from HAN to V if $\delta\epsilon<0$.
Davidson and Mottram showed that this `hard switch' can be made
possible by reversing the field orientation and using moderate values of the field strength,
if the liquid crystal particles have flexoelectric properties. Upon application of the reverse field,
competition is created between the dielectric alignment behaviour and the  field-induced splay 
promoted by the flexoelectricity. For appropriate values of the field, this competition 
causes the confined liquid crystal to adopt a distorted state 
which, upon removal of the field, relaxes to the other state thus rendering hard switching
possible. Table~\ref{tble:switchPara} summarizes the parameterisation combinations required 
to achieve easy and hard switching of the cell.\\




\begin{table}
	\centering
	\begin{tabular}{||c||c||c||}
	\hhline{|t:=:t:=:t:=:t|}
	$\delta\epsilon$	&easy switching		&hard switching	\\
	\hhline{|:=::=::=:|}
	$\delta\epsilon>0$	&HAN to V using $E < 0$	&V to HAN using $E > 0$	\\
	\hhline{|:=::=::=:|}
	$\delta\epsilon<0$	&V to HAN using $E > 0$	&HAN to V using $E < 0$	\\
	\hhline{|b:=:b:=:b:=:b|}
	\end{tabular}
\caption{Electric parameterisation given in~\cite{DavidsonMottram02} 
required to performed the `easy' and `hard' switching between
the HAN and vertical states.}
\label{tble:switchPara}
\end{table}

The treatment used by Davidson and Mottram was, however, based on elastic theory approach. 
In this Chapter, the aim is to investigate, using molecular simulations, the validity of 
this and thus get a microscopic picture of the process underlying the switching scheme.\\
In order to model the display proposed by Davidson and Mottram using molecular simulation
methods, a similar slab geometry like that shown on Figure~\ref{fig:bistDisplay} is to be 
used, that is with an homeotropic top surface and a bistable 
bottom surface. The flexoelectric molecules are to be represented using the $k=5$ PHGO 
model for pear  shaped particles developed in Chapter~\ref{chap:six}.

The competition between the dielectric effect and the field-induced flexoelectric splay 
is to be achieved through a particle-field interaction made of dielectric and dipolar 
contributions, as shown in Appendix~\ref{chap:B}. 
In the case of a negative dielectric susceptibility, for example, the particles 
need to experience the competitive effects of the dielectric contribution, which tends to align 
the particles  perpendicular to the field, and the dipolar effect which tends to align 
the particles parallel  to it. The latter has the effect of introducing splay distortions 
due to the preferred packing arrangement of pear shaped particles.\\

The system electric energy $U_e$  is given by (see Appendix~\ref{chap:B})~:
%
\begin{equation}
	U_e = \sum_{i=1}^{N}\left\{ 
	-\frac{1}{2}\epsilon_0\delta\epsilon\lp \dotProd{\vect{E}}{\ui} \rp^2 
	- \mu \dotProdP{\vect{E}}{\ui}
	\right\}
\end{equation}
%
where $\epsilon_0$ is the unit of energy, $\delta\epsilon$ is the dielectric susceptibility,
$\mu$ is the dipole moment, $\ui$ is the molecular orientation and $\vect{E} = E\vecth{E}=
E\vecth{z}$ is
the applied electric field.





