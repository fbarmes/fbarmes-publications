

\chapter{Absorbed volume of an HGO into a substrate.}
\label{chap:A}

%===============================================================================================
%===============================================================================================
\section{Introduction}


The situation considered here is that of an Hard Gaussian Overlap particle close to a 
substrate and interacting
with it via a given particle-surface potential so that the particle is allowed to partially
absorb the surface. The problem is to find an expression for $V_e$ the volume absorbed into the
surface. Let us consider the setup shown in Fig~\ref{fig:scalingVe-Vs}
\picW = 10cm
\begin{figure}[h]
	\centering
	\pic{APPENDIX/scalingVe-Vs.ps}
	\caption{Schematic representation of the geometrical configuration considered to
	calculate the absorbed volume of the ellipsoid into the substrate.}
	\label{fig:scalingVe-Vs}
\end{figure}

The approximation made here is to solve the problem with an ellipsoid of revolution, instead of
a Gaussian Overlap since the much simpler expression for the equation of an ellipsoid makes the
problem easier.\\
The idea here is that the problem is easily solved in the case of a unit sphere. 
This, in turn, can be transformed into the same problem for an ellipsoid of elongation k by scaling the 
space along one  of the axis by a factor $k$. The volume $V_e$ in that case is equal to:
%
\begin{equation}
	V_e = kV_s
\end{equation}
where $V_s$ is the volume of the sphere absorbed into the substrate.


%===============================================================================================
%===============================================================================================
\section{Case of a sphere.}

$V_s$ is given by computing the volume of the sphere of radius $a$ between the $z$ coordinates
 $z_0=d_1$ to $z_1= a$.\\
The equation of an ellipsoid of semi axis $a,b,c$ along $\vecth{x},\vecth{y}$ and $\vecth{z}$
is given by~:
\begin{equation}
	\frac{x^2}{a^2} + \frac{y^2}{b^2} + \frac{z^2}{c^2} = 1
\end{equation}
$V_s$ is then given by~:
\begin{equation}
	V_s = \int^{z_1}_{z_0}\int^{y_1}_{y_0}\int^{x_1}_{x_0} dxdydz
\end{equation}
with~:
\begin{eqnarray*}
	x_1 &=& a\sqrt{1-\frac{z^2}{c^2} - \frac{y^2}{b^2} }	\\
	y_1 &=& b\sqrt{1-\frac{z^2}{c^2}}	\\
	x_0 &=& -x_1	\\
	y_0 &=& -y_1
\end{eqnarray*}

Since a sphere is considered here, $a=b=c$ and $V_s$ reads~:
\begin{eqnarray}
	V_s &=& \frac{ab\pi}{c^2}\int^{z_1}_{z_0} \left\{  c^2 - z^2  \right\}dz \nonumber \\
	V_s &=& \frac{\pi}{3}(a-d_1)^2(2a+d_1)
\end{eqnarray}

The distance $d_1$ can be obtained by considering the triangle $O A_1 B_1$ in the
$\vecth{x}-\vecth{z}$ plane. The coordinates of
$A_1$ and $B_1$ are equal, respectively, to those of $A$ and $B$ rescaled by $\frac{1}{k}$ along
$z$. Hence
\begin{eqnarray*}
	A_1 &=& \lp x_A, \frac{z_A}{k} \rp	\\
	B_1 &=& \lp x_B, \frac{z_B}{k} \rp
\end{eqnarray*}

and therefore
\begin{equation}
	A_1B_1 = \sqrt{ (x_B-x_A)^2 + \frac{(z_B-z_A)^2}{k^2} }
\end{equation}

since
\begin{eqnarray}
	O_1B_1 &=& a	\\
	D_1B_1 &=& \frac{A_1B_1}{2}
\end{eqnarray}

we get $d_1$ as~:
\begin{eqnarray}
	d_1 &=& \sqrt{ O_1B_1^2 - D_1B_1^2 }	\nonumber	\\
	d_1 &=& \sqrt{a^2 - \frac{1}{4}\lp   (x_A-x_B)^2 + \frac{1}{k^2}(z_A-z_B)^2 \rp  }
	\label{eqn:d1Initial}
\end{eqnarray}


%===============================================================================================
%===============================================================================================
\section{Coordinates of $A$ and $B$}


$A$ and $B$ are defined as being the coordinates of the contact points between the ellipse and
the plane. Since, these points satisfy both the equation of the ellipse and the plane, they
can be found by solving~:
%
\begin{equation}
	\left\{	%}
	\begin{array}{cccc}
		\frac{x^2+z^2}{a^2} &= 	&1					&(1)\\
		x		    &=	&\frac{z\cos\theta - d}{\sin\theta}	&(2)\\
	\end{array}
	\right.
\end{equation}

Inserting (2) into (1) gives
\begin{equation}
	z^2\lp k^2\cos^2\theta + \sin^2\theta \rp - 2zk^2d\cos\theta + d^2k^2 -
	a^2k^2\sin^2\theta.
\end{equation}

the real roots of which are~:
\begin{equation}
	z_{A,B} = \frac{dk^2\cos\theta \pm \sqrt{k^2\sin^2\theta 
	\left[ -d^2 + a^2(k^2\cos^2\theta + \sin^2\theta)\right]}}
	{k^2\cos^2\theta + \sin^2\theta}
\end{equation}


%===============================================================================================
%===============================================================================================
\section{Expression for $d_1$}

Equation~\ref{eqn:d1Initial} can now be rewritten using~:

\begin{eqnarray*}
	(x_B-x_A)^2 &=& \frac{\cos^2\theta}{\sin^2\theta}\lp z_B-z_A \rp^2	\\
	(z_B - z_A)^2 &=& 4k^2\sin^2\theta
	\left[ -d^2 + a^2\lp k^2\cos^2\theta + \sin^2\theta  \rp^2  \right]
\end{eqnarray*}

which after full simplification gives~:

\begin{equation}
	d_1 = \frac{d}{\sqrt{k^2\cos^2\theta + \sin^2\theta} }
\end{equation}


%===============================================================================================
%===============================================================================================
\section{Expression for $V_e$}

The absorbed volume of the sphere can now be written as~:

\begin{equation}
	V_s = \frac{\pi}{3}
	\lp  a - \sqrt{\frac{d^2}{k^2\cos^2\theta + \sin^2\theta} } \rp^2
	\lp 2a + \sqrt{\frac{d^2}{k^2\cos^2\theta + \sin^2\theta} } \rp
\end{equation}


which gives the final result for the absorbed volume of the ellipsoid~:

\begin{equation}
	V_e = \frac{k\pi}{3}
	\lp  a - \sqrt{\frac{d^2}{k^2\cos^2\theta + \sin^2\theta} } \rp^2
	\lp 2a + \sqrt{\frac{d^2}{k^2\cos^2\theta + \sin^2\theta} } \rp
\end{equation}
