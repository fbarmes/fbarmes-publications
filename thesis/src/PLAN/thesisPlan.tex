

\documentclass[a4paper]{article}

%===============================================================================================
%===============================================================================================
\voffset=-1in
\hoffset=-0.5in
\textwidth 450pt
\textheight 650pt
\headheight 6pt
\footskip 12pt
\oddsidemargin 6pt
\marginparsep 1pt
\parindent 0pt
\parskip 6pt 

\marginparwidth 0pt
\addtolength{\textwidth}{+0.5in}
\addtolength{\textwidth}{10mm}
\addtolength{\textheight}{+1in}

%===============================================================================================
%===============================================================================================
\parindent = 0pt
\renewcommand{\bf}[1]{\textbf{#1}}
\newcommand{\mkChap}[1]
	{
	\large\textbf{#1}
	\vspace{2mm}\\
	}

\newcommand{\mkChapBig}[1]
	{
	\newpage
	\hrule
	\begin{center}
	\LARGE\textbf{#1}
	\end{center}
	\hrule
	\vspace{5mm}
	}


\newcommand{\mkSectionBig}[1]
	{
	\Large\textbf{#1}\\
	}

\newcommand{\mkSubSectionBig}[1]
	{
	\hspace*{10mm}\large\textbf{#1}\\
	}

\newcommand{\mkSection}[1]
	{
	\hspace*{25mm}#1\\
	}


%===============================================================================================
%===============================================================================================

\begin{document}

\begin{center}
	\LARGE \textbf{Fred's thesis contents} \\
\end{center}
\vspace{1cm}


%======================================================================================================
%
%			General Contents
%
%======================================================================================================
\mkChap{Chapter 1 : The liquid crytalline phase (Review)}
	\mkSection{ 1 :  Statistical mechanics}
	\mkSection{ 2 :  The LC phases and applications}
	\mkSection{ 3 :  Experiments with LC's (phase characterisation)}
	\mkSection{ 4 :  Surface effects in LC systems (anchoring, bistability) }
	\mkSection{ 5 :  Flexoelectric behaviour (molecules, measurement)}
	\mkSection{ 6 :  Theories of LC phase transitions}\\

%======================================================================================================
%======================================================================================================
\mkChap{Chapter 2 : Computer simulation of liquid crystals (Review)}
	\mkSection{ 1 :  Hard models (model/phase behaviour relationship)}
	\mkSection{ 2 :  Soft models (model/phase behaviour relationship)}
	\mkSection{ 3 :  Confined systems}
	\mkSection{ 4 :  Flexoelectric/Ferroelectric particles}\\

%======================================================================================================
%======================================================================================================
\mkChap{Chapter 3 : Computer simulations of hard particles (Review/Results)} 
	\mkSection{ 1 :  modeling techniques (MC, MD, Lattices?)}
	\mkSection{ 2 :  Computation of the observables}
	\mkSection{ 3 :  Bulk ellipoids (HGO, k=3,5)}
	\mkSection{ 4 :  Bulk Plates (HGO, k=1/3, 1/5)}\\
	

%======================================================================================================
%======================================================================================================
\mkChap{Chapter 4 : Surface influence on liquid crystalline systems (Results)} 
	\mkSection{ 1 :  Symmetric confined HNW potential}
	\mkSection{ 2 :  Hybrid confined HNW potential}
	\mkSection{ 4 :  Changes with RS potential}
	\mkSection{ 5 :  Changes with RSU potential}\\


%======================================================================================================
%======================================================================================================
\mkChap{Chapter 5 : Bulk simulations of pear shaped particles (Results)} 
	\mkSection{ 1 :  Bulk Italian Pears (phase dia, surf potentials)}
	\mkSection{ 2 :  Bulk Sheffield Pears (method, acuracy, bulk results)}
	\mkSection{ 3 :  GBP, phase characterisation (???)}
	\mkSection{ 4 :  Computation of elastic and flexoelectric constants (???)}


%======================================================================================================
%======================================================================================================
\mkChap{Chapter 6 : Confined simulations of pear shaped particles (Results)} 
	\mkSection{ 1 :  Confined Sheffield Pears, HNW potential }
	\mkSection{ 2 :  Confined Sheffield Pears, Realistics potential (???)}
	\mkSection{ 3 :  Field effect (???)}\\

%======================================================================================================
%======================================================================================================
\mkChapBig{Chapter 1 : Chapter 1 : The liquid crytalline phase (Review)}


%======================================================================================================
%======================================================================================================
\mkChapBig{Chapter 2 : Computer simulation of liquid crystals (Review)}


%======================================================================================================
%======================================================================================================
\mkChapBig{Chapter 3 : Computer simulations of hard particles (Review/Results)}

\mkSectionBig{1. Modeling techniques~:}
Lattice techniques ?
MonteCarlo
Molecular Dynamics

\mkSectionBig{2. Observable computation~:}
g(r)
order parameters
\begin{itemize}
       \item P1($\cos\theta$)
       \item P2($\cos\theta$)
\end{itemize}
QzzWa
SWa

\mkSectionBig{Bulk behaviour of prolate HGO~:}
k=3, phase dia, snaps
k=5, phase dia, snaps

\mkSectionBig{Bulk behaviour of oblate HGO~:}
k=1/3, phase dia, snaps
k=1/5, phase dia, snaps

%======================================================================================================
%======================================================================================================
\mkChapBig{Chapter 4 : Surface influence on liquid crystalline systems (Results)}

\mkSectionBig{Aims :}

\large
\begin{itemize}
	\item Study surface induced structural changes on Liquid Crystalline systems
		\begin{itemize}
		\item Layering
		\item Ordering
		\end{itemize}
	\item Try to recreate these effects with a simple model.
	\item Identify main variables responsible for effect.
	\item Characterise the P-H anchrong transition.
	\item See effects on hybridly anchored systems.
	\item Make the surface potential more realistic while keeping the physics.
\end{itemize}
\hrule
\vspace{5mm}

%-----------------------------------------------------------------------------------------------
\mkSectionBig{1. Surface induced stuctural changes.}

\mkSubSectionBig{1. The surface potential.}
Description of the potential.\\
Expected anchoring orientations.\\
Theoretical treatment for the reltive stability of H and P.\\
Optimum Ln for bistability as a function of k.\\

\mkSubSectionBig{1. Simulation results.}
Descrition of simulation setup.\\
Results : typical profiles, snapshots.\\
Need to illustrate the two anchorings and the variable responsible for it.\\

%-----------------------------------------------------------------------------------------------
\mkSectionBig{2. The anchoring transition.}

\mkSubSectionBig{1. The method}
Method of obtaining QzzWa, SWa.\\
Explain Simulation series.\\
Phase diagrams, up, down Bist for k=3,5\\

%-----------------------------------------------------------------------------------------------
\newpage
\mkSectionBig{3. Hybridly anchored systems.}
Setup\\
Phase diagrams\\
Snapshots\\

%-----------------------------------------------------------------------------------------------
\mkSectionBig{4. The rod sphere potential.}

\mkSubSectionBig{1. The potential}
Description of the potential.\\
History of the potential.\\

\mkSubSectionBig{1. Phase behaviour}
phase diagrams, down\\
snapshots\\

%-----------------------------------------------------------------------------------------------
\mkSectionBig{5. The rod surface potential.}

\mkSubSectionBig{1. The potential}
Description of the potential, derivation of it.\\

\mkSubSectionBig{1. Phase behaviour}
phase diagrams up, down, diff\\
snapshots\\

%-----------------------------------------------------------------------------------------------
\mkSectionBig{6. Conclusions.}


%======================================================================================================
%======================================================================================================
\mkChapBig{Chapter 5 : Bulk simulations of pear shaped particles (Results)} 


%======================================================================================================
%======================================================================================================
\mkChapBig{Chapter 6 : Confined simulations of pear shaped particles (Results)} 


%======================================================================================================
%======================================================================================================

\end{document}
