

\section{Modeling of flexoelectric particles}


Flexoelectricity is an important property to be considered in the design of materials for use in 
liquid crystal devices.  It has been shown, theoretically, that flexoelectricity can be used
as the driver for the switching in new generation LCDs~\cite{DavidsonMottram02}.
Although thoroughly studied experimentally and theoretically (see Chapter~\ref{chap:one}), 
computer simulation studies of flexoelectric particles are relatively scarce, mainly
due to the difficulty of modeling the shape anisotropy if Meyer's principle is to be considered.
Models showing ferroelectric behaviour, have, however been well 
studied~\cite{WeiPatey92,WeiPatey92a}.\\

One attempt at modeling flexoelectric particles was performed by Neal
\etal\cite{NealParker97} in their study of molecules formed from rigid assemblies of 
three Gay-Berne sites. One
of these models was a triangular arrangement of parallel particles whose overall shape
resembled that of a pear. This system exhibited an isotropic to smectic ordering transition in
which the particles adopted anti-parallel orientations in adjacent layers. A subsequent
attempt at modeling pear shaped particles was performed by Stelzer \etal\cite{StelzerBerardi99}
using Gay-Berne sites to one end of each was connected a Lennard-Jones sphere.
Isotropic, nematic and smectic phases were found for this model. The computation of the
flexoelectric coefficients gave a non-zero splay coefficient and, within error estimates, a zero
bend  coefficient in accordance with Meyer's theory. Subsequent simulations by Billeter and
Pelcovits~\cite{BilleterPelcovits00}, using a slightly different energy parameterisation and a
different method for the computation of the flexoelectric coefficient showed results in
agreement with~\cite{StelzerBerardi99}.\\
Berardi~\etal\cite{BerardiRicci01} subsequently developed 
a single-site potential for pear shaped particles using Zewdie's Stone expansion
approach~\cite{Zewdie98a,Zewdie98b}. This study was rather successful as this computationally
efficient model showed
isotropic, nematic and smectic phases, the latter two of which, upon application of an appropriate
energy parameterisation, exhibited net polar order. Further details regarding this potential are
provided in Chapter~\ref{chap:six}



