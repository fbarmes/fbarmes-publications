\chapter{Conclusions and future work}



In this Thesis, the study of confined liquid crystalline systems and the development of a model 
for tapered pear shaped particles were addressed. These two seemingly
very different lines of work have subsequently been united in the study of confined pear shaped
particles which lead to towards the final aim of the Thesis: the development of a model for a 
liquid crystal display cell where switching between the two
stable states may be flexoelectricity induced. The development of this latter model required two 
conditions to be met, the first being a surface potential inducing both planar and homeotropic 
anchoring and, with  appropriate tuning of the model parameter, a bistability
region. The second requirement was that of a molecular model for flexoelectric (\ie pear shaped) 
mesogens displaying a stable nematic phase.\\
Here, the conclusions that have been drawn from this study are summarized and various avenues 
for future work are discussed.



\section{Conclusions}

%==============================================================
%		SYMMETRIC CONFINED SYSTEMS
%==============================================================
The study of confined hard particle liquid crystalline systems has been addressed 
in Chapters~\ref{chap:four} and~\ref{chap:five}. The Surface induced structural 
changes in symmetric anchored systems have
been studied using three surface interaction models, namely the hard needle wall (HNW),
rod-sphere (RSP) and rod-surface (RSUP) potentials.  For all three of these models, the surface
interaction was not mediated directly by the HGO particles, but by other objects embedded
within them. This approach allowed the study to be extended to include consideration of 
the effect of varying particle
absorption into the surface. Through this, it has been shown that the HNW and RSUP potentials 
induce planar
and homeotropic anchoring for, respectively, long and short elongations of the inner object
($k_S$) whereas with the RSP potential and long $k_S$, the planar arrangement was replaced by 
tilted anchoring. These results have proved to be consistent with a theoretical
treatment based on the geometrical characteristics of the surface interaction models.
The implications of these results are twofold. The behaviour found for the HNW and RSUP models
showed that the anchoring transition between
homeotropic and planar arrangements can be controlled by the molecular volume made available to
be absorbed into the substrates. Secondly, the tilted phase obtained with the RSP model lead to
a revision of the
explanation of tilted surface arrangements which had previously always been attributed to
attractive forces. The observation of such a structure in this study showed that it can also be
obtained with a purely steric potential.\\
Anchoring and order phase diagrams have been computed for all models. For the HNW
and RSUP potentials, using series of simulations at constant densities and either increasing or
decreasing $k_S$, regions of bistability between the planar and homeotropic surface arrangement
have been found. Those regions proved to be stronger and wider for the RSUP model. 
This shows that even in very simple model systems, bistable surface behaviour can be introduced
by tuning the competition between two locally stables states. This suggests that similar tuning
of real liquid crystal substrate systems may offer a viable route to relatively simple bistable
surfaces for low energy device applications.\\

Hybrid anchored systems with a top homeotropic surface have been studied in
Chapter~\ref{chap:five} using the HNW as a surface potential in an attempt to create a hybrid
anchored slab with a top surface homeotropic anchoring and bistable anchoring at bottom
substrate. It has been shown that the bistability behaviour of the model at the bottom surface
is lost if the top homeotropic anchoring is made too strong. However, by reducing the latter
(\ie reducing $k_S$), the bistability at the bottom surface can be recovered. This has been
explained in terms of the elastic forces transmitted from the top surface particles onto those at the
bottom surface. Reducing the top anchoring strength reduces these forces and, thus, 
allows bistability to be recovered. The point has also been addressed in Chapter~\ref{chap:seven}.
Also, the behaviour of the homeotropic to planar structural transition has been shown to be
dependent upon the slab height. For short slabs (\ie small systems), there is a discontinuous
structural transition between the two arrangements, whereas this becomes continuous for thicker
slabs (\ie bigger systems). These findings are consistent with both
experimental and theoretical studies of the effect of thickness on the bent director
structure of a hybrid anchored system. Also, this result has shown that the first prerequisite
for the modeling of a bistable LCD cell can be met: it is possible to simulate a hybrid
anchored slab with a homeotropically anchored top surface and bistable anchoring at the bottom
surface using an appropriate parameterisation for the surface model. In addition, provided large
enough systems are used, a continuous structural transition from homeotropic to planar alignment
can be obtained if the slab is in the HAN configuration.\\


%==============================================================
%		PEAR SHAPED PARTICLES
%==============================================================
Chapter~\ref{chap:six} has addressed the study of tapered pear-shaped particles. The first model to
be used for this was the truncated Stone expansion model; this, however, did not meet the 
requirement of
displaying a stable nematic phase and showed only two phases: an isotropic phase at low density
and glassy domain-ordered phase at high density. This behaviour proved to be very surprising as
an equivalent soft model used by the Bologna group showed isotropic, nematic and \smA phases. The
lack of ordered fluid phases in the steric model was attributed to concave features in
the particles' contact surfaces which had the effect of preventing the particles from sliding along
one another.
As an alternative, the PHGO model, a generalisation of the HGO model to non-centrosymmetric particles
has been developed. This does not show the concave features mentioned above. For the two shortest
elongations used, $k=3$ and $4$, the PHGO model only showed isotropic and fluid domain-ordered
phases at, respectively, low and high densities. However, upon increasing the elongation to $k=5$,
the model phase behaviour became much richer, with transitions from isotropic to nematic and
then to a bilayered smectic $\mrm{A_2}$ phase. The study of this latter
phase proved to be very interesting, showing anisotropic compressibility behaviour. Thus the
PHGO model, with its nematic phase, met the second requirement for the bistable LCD cell
model.\\


%==============================================================
%		FLEXOELECTRIC SWITCHING
%==============================================================
In Chapter~\ref{chap:seven}, confined systems of PHGO particles interacting with the substrates
through a variant of the RSUP potential have been studied. This surface potential has been shown to
exhibit both planar and homeotropic arrangements with a narrow region of bistability around
$k_S/k=0.7$. In the case of hybrid anchored systems of $N=2000$ particles with a top homeotropic
surface, bistability at the bottom surface could be recovered using $k_{St}/k=0.6$. Due
to the large system size used, a continuous structural transition was found in the case of the bent
director state.\\ 
Using this, a molecular model for the LCD cell was simulated and both hard and easy 
switching between the HAN and vertical states were attempted using both dielectric and dipolar
contributions in the particle-field interactions. Both switching directions could be achieved 
using a narrow window of electric field strength $E$ and dipole moment $\mu$. A successful
parameterisation was $E/\delta\epsilon = \pm 0.2$ and $\mu/\delta\epsilon = 2.5$. However the
question as to  whether the mechanism underlying this switching behaviour is direct surface
effect, indirect flexoelectric behaviour or a combination of the two has not been resolved; 
this forms, in part, the basis of future work.










\section{Future work}


The work performed in this Thesis has lead to the development of a model for a novel bistable
LCD cell where two way switching is achieved by the application of a directional
electric field pulse. Much of the work that will directly follow this Thesis concerns the study and
improvement of this model. The first task is certainly the implementation of a method
for the calculation of the slay flexoelectric coefficient as this would allow a good quantitative
measurement of the effects of subsequent alterations to the model. Also, the use of the
molecular dynamics method would provide interesting results by giving access to the true 
dynamics of relaxation properties of the model systems.\\
%
The cell model itself should also be improved, for instance by the use of mixtures of ellipsoidal
shaped (HGO) and pear shaped particles (PHGO) which would allow independent tuning and/or
enhancement of the bulk
flexoelectric properties of the cell and the field-direction-dependence of its surface 
behaviour. This would also make the model a better description of a real liquid crystal cell,
since mixtures of several different components are commonly used.\\
Finally, incorporation of attractive forces into the molecular models should be
considered so as to render them more realistic and give access to better control mechanisms for
tuning phase and anchoring behaviour. This could include addition of quadrupolar contributions 
in the particle-particle interaction as it has been shown that quadrupole interactions are 
predominant in the origins of flexoelectricity in real mesogens.\\
All of these refinements should lead to the development of, more realistic modeling of both the
LCD cell considered in Chapter~\ref{chap:seven} and the more fundamental behaviour needed to
make such cell. The work presented in this Thesis therefore represent a step along the path
towards providing more efficient display cells as well as an academic study into the development
of models for confined and flexoelectric liquid crystal behaviour.


