

\section{Overview}

%------------------------------------------------------
%	Importance of liquid crystals
%------------------------------------------------------
The liquid crystalline phases which were discovered in 1888, represent a state of matter which shares
properties of both its neighbouring isotropic liquid and crystal solid phases.
As a result of this, liquid crystals have found many industrial applications in fields as
different as biology, rheology, laser optics or tribology. However the greatest application of
liquid crystals is, without any doubt, in the field of electro-optic displays.\\

%------------------------------------------------------
%	Why surface are so important
%------------------------------------------------------
In the most common liquid crystal display cells, the so called super-twisted nematic displays,
surface interactions play a significant role in device operation; switching
between the `on' and `off' states is achieved by changing the molecular orientations from a
field aligned to a surface aligned arrangement. The surface treatment of the cell surfaces,
therefore, 
plays a significant role in the cell performance characteristics such as switching
speed, viewing angle and contrast. Surprisingly, the surface treatments used industrially 
are often applied following empirical rules and a full understanding at a molecular scale of 
the surface-induced structural changes near the surfaces is still lacking. The reasons for this 
lie in the very complexity of liquid crystalline phases which renders a full theoretical treatment
extremely difficult while most experimental approaches are unable to achieve the much needed 
molecular resolution. 
Consequently, the behaviour of liquid crystals at interfaces has become a particular
focus for numerical simulations in which the study of generic models, based on statistical 
mechanics, can be used to gain an in-depth insight into molecular behaviour.\\

%------------------------------------------------------
%	Introducing the flexo display
%------------------------------------------------------
The latest developments in liquid crystal display technology have lead to bistable displays
which are thought to rely on the properties of flexoelectric mesogen molecules for their
operation. The advantage of these
displays lies in their reduced power consumption, leading to a battery lives about one thousand
times longer than those super-twisted nematic displays.  Again, bistable devices rely crucially
on surface effects for their successful operation, hence the importance of a sound knowledge 
of the interfacial region properties. Also, the use of flexoelectric particles requires a 
good understanding of this particular class of liquid crystal.



