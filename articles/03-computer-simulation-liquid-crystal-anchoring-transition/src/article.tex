

%%\documentclass[a4paper,12pt,onecolumn]{article}
\documentclass[aps,10pt,twocolumn]{revtex4}

%===============================%
%                               %
%   packages                    %
%                               %
%===============================%
\usepackage{graphics}		% for graphics
\usepackage{graphicx}		% for graphics
\usepackage{calc}			% allow calculations
\usepackage{subfigure}      % allow subfigure
\usepackage{amssymb}        % font
\usepackage{amstext}        % ams latex package
\usepackage{amsmath}        % font
\usepackage{amsfonts}       % font
\usepackage{amsbsy}			% font
\usepackage{float}			% figure HERE
\usepackage{color}			% allow use of color
\usepackage{fancybox}       % nice boxes
\usepackage{graphpap}       % to use graphpaper in pictures
\usepackage{hhline}			% nice lines is tables
\usepackage{xspace}			% add space at the end of macros
%%\usepackage{cite}			%compress down long lists of refs

%%\usepackage{ifthen}		% implementation of \ifthenelse{}{}
%%\usepackage{caption2}     % to change the caption font

%===============================%
%                               %
%   new commands                %
%                               %
%===============================%
\newcommand{\remark}[1]{\textbf{\textcolor{red}{#1}}}
\newcommand{\remove}[1]{}
\newcommand{\linespacing}[1]{\renewcommand{\baselinestretch}{#1}\large\normalsize}
\newcommand{\mrm}[1]{\mathrm{#1}}
\newcommand{\eqn}[1]{\mbox{$#1$} }
\newcommand{\vect}[1]{ \mathbf{#1} }
\newcommand{\vecth}[1]{ \mathbf{\hat{#1} } }
\newcommand{\gvect}[1]{\forcebold{#1}}
\newcommand{\gvecth}[1]{\forcebold{\hat{#1}}}

\newcommand{\dotproduct}[2]{\vect{#1} \cdot \vect{#2}   }

\newlength{\tabul}
\newcommand{\TAB}{\hspace*{\tabul}}

\newcommand{\lp}{\left(}
\newcommand{\rp}{\right)}
\newcommand{\lcurl}{\left\{}    %}
\newcommand{\rcurl}{\right\}} %{


%============================================================================
%           Contact distance related macros
%============================================================================

\newcommand{\so}{\sigma_0}
\newcommand{\sel}{\sigma_\ell}

\newcommand{\rij}{\vecth{r}_{ij}}
\newcommand{\ui}{\vecth{u}_i}
\newcommand{\uj}{\vecth{u}_j}
\newcommand{\sijr}{\sigma(\ui,\uj,\rij)}
\newcommand{\dotProdP}[2]{ \left( #1 \cdot #2 \right) }
\newcommand{\dotProd}[2]{ #1 \cdot #2 }
\newcommand{\half}{\frac{1}{2}}
\newcommand{\lc}{\left[}
\newcommand{\rc}{\right]}

\newcommand{\sir}{\sigma(\ui,\vect{r}_{ij})}
\newcommand{\drui}{d_i\lc\dotProdP{\ui}{\rij}\rc}
\newcommand{\lrui}{\ell_i\lc\dotProdP{\ui}{\rij}\rc}

\newcommand{\druj}{d_j\lc\dotProdP{\uj}{\rij}\rc}
\newcommand{\lruj}{\ell_j\lc\dotProdP{\uj}{\rij}\rc}

\newcommand{\druiSq}{d_i^2\lc\dotProdP{\ui}{\rij}\rc}
\newcommand{\lruiSq}{\ell_i^2\lc\dotProdP{\ui}{\rij}\rc}

\newcommand{\drujSq}{d_j^2\lc\dotProdP{\uj}{\rij}\rc}
\newcommand{\lrujSq}{\ell_j^2\lc\dotProdP{\uj}{\rij}\rc}

\newcommand{\eijr}{\epsilon(\ui,\uj,\rij)}
\newcommand{\eo}{\epsilon_0}
\newcommand{\eOne}{\epsilon_1}
\newcommand{\eTwo}{\epsilon_2}

\newcommand{\ijr}{\ui,\uj,\rij}
\newcommand{\uo}{\vecth{u}_o}
\newcommand{\uof}{\vecth{u}_o^{f}}
\newcommand{\uofp}{\vecth{u}_o^{f^\prime}}

\newcommand{\un}{\vecth{u}_n}
\newcommand{\unf}{\vecth{u}_n^{f}}
\newcommand{\unfp}{\vecth{u}_n^{f^\prime}}

\newcommand{\kSp}{k_S^{\prime}}

\newcommand{\etal}{\emph{et al.}\@\xspace}
\newcommand{\ie}{\emph{i.e.}\@\xspace}
\newcommand{\eg}{\emph{e.g.}\@\xspace}
\newcommand{\etc}{etc.\@\xspace}


\parindent = 0pt    %no indentation

%===============================%
%                               %
%   document parameter          %
%                               %
%===============================%


%%\parskip 12pt

%line spacing report requires double spacing
%%\def\dsp{\def\baselinestretch{2.0}\large\normalsize}


%===============================%
%                               %
%   page style and size         %
%                               %
%===============================%
%%\oddsidemargin = 0.6in                 % 40mm left margin
%%\textwidth = 5.75in                    % 20mm right margin
%%\textheight = 9.1in                    % 20mm from page number to bottom
%%\voffset = -0.5in                      % lift up




%=======================================%
%                                       %
%   beginning of the document           %
%                                       %
%=======================================%

\begin{document}

\graphicspath
{
{../imgs/}
}

%=======================================================================================
%           TITLE
%=======================================================================================
\title{Computer simulation of a liquid crystal anchoring transition.}

\author{F. Barmes}
\affiliation{Centre Europ\'een de Calcul Atomique et Mol\'eculaire, 46 All\'{e}e
d'Italie, 69007 Lyon, France}


\author{D.J. Cleaver}

\affiliation{Materials Research Institute, Sheffield Hallam University,
Sheffield S1 1WB, United Kingdom}



\pacs{61.30.-v 64.70Md 61.30.Cz68.08.-p}

\date{\today}



%=======================================================================================
%           ABSTRACT
%=======================================================================================
\begin{abstract}
We present a study of the effects of confinement on a system of hard Gaussian overlap particles interacting with
planar substrates through the hard-needle--wall potential. Using geometrical arguments to calculate the molecular
volume absorbed at the substrates, we show that both planar and homeotropic arrangements can be obtained using
this model. Monte Carlo simulations are then used to perform a systematic study of the model's behaviour as a
function of the system density and the hard-needle--wall interaction parameter. As well as showing the homeotropic
to planar anchoring transition, the anchoring phase diagrams computed from these simulations indicate regions of
bistability. This bistable behaviour is examined further through the explicit simulation of field-induced two-way
switching between the two arrangements.
\end{abstract}
\maketitle
%=======================================================================================
%           INTRODUCTION
%=======================================================================================

\section{Introduction}

Confinement of a liquid crystal has a symmetry breaking effect; this induces both positional layering and
orientational coupling which is transmitted to the bulk alignment through a mechanism called
anchoring~\cite{Jerome91}. While the former effect is a universal consequence of
confinement~\cite{Schoen96,Schoen96a}, the latter is specific to mesogenic systems. For these, three main
arrangements for the particles close to the surface can be observed, namely homeotropic, planar and tilted. A
range of azimuthal anchoring states are also possible. Upon change of experimental conditions, modification of the
surface arrangement can be observed to lead to a change in the bulk alignment; such an event is called an
anchoring transition~\cite{Jerome91}.

Experimental studies of confined liquid crystals have reported that anchoring transitions can be achieved by
various means such as change in temperature~\cite{PatelYokoyama93,JagemalmKomitov1997,BarberiGiocondo98} or the
conformation of the aligning agent~\cite{ZhuLu94,ZhuWei94}. Incident radiation can also induce anchoring
transitions by selectively switching the conformations of substrate molecules and, thus, modifying the interfacial
interactions~\cite{BarberoPopaNita00}. Alternatively, the absorption behaviour of a liquid crystal at a solid
substrate, either directly~\cite{AlkhairallaAllison99,AlkhairallaBoden02}, or through the introduction of a second
species~\cite{Jerome93}, can lead to anchoring transitions if the density of absorbed particles or the nature of
the absorption is changed. One last example is that in which a multistable anchored system which, preferentially
adopts one of its possible conformation due to its treatment
history~\cite{Jerome91,StoenescuMartinotLagarde99}, is switched into an alternative state by, e.g., an appropriate
applied field.

Although a number of mechanisms underlying anchoring transitions have been raised (see~\cite{Sluckin95} for a
review) rather few theoretical analyses have been performed. Teixeira and
Sluckin~\cite{TeixeiraSluckin92,TeixeiraSluckin92a} used a Landau-de Gennes free energy  functional to study the
planar to homeotropic anchoring transition in liquid crystal systems confined by  different substrates. They found
a rich anchoring behaviour which helped in the identification of mechanisms responsible for the anchoring
transitions, specifically, the compositions of binary mixtures of liquid crystals and the amount of absorption at
the surface. Subsequently Teixeira~\etal~\cite{TeixeiraSluckin93} used a Landau-de Gennes formalism to observe a
temperature driven anchoring transition at the interface between a liquid crystal and smooth solid surface, thus
confirming the experimental findings. The effect of non-uniform substrates (\ie microtexture) has also been
studied using a Landau-de Gennes formalism~\cite{ZhengQianSheng96,ZhengQianSheng97}. This work found
temperature-driven phase-transitions between states with different tilt angles.

In the field of molecular simulations, the effects of confinement on liquid crystalline systems has been
increasingly well studied, and all of the surface arrangements listed above have been obtained through appropriate
choices of particle-surface interaction potential. Using various parameterizations of the Gay-Berne
model~\cite{GayBerne81} and taking the particle-wall contact distance to be that between a rod and a sphere, both
tilted and planar anchoring states have been observed~\cite{ZhangChakrabarti96, WallCleaver97, LathamCleaver00,
WallCleaver03}. If, alternatively, the surface is represented using a monolayer of spheres, the particle-substrate
interaction can be designed to induce either homeotropic or planar anchoring~\cite{GruhnSchoen97, GruhnSchoen98,
GruhnSchoen98a}. Using hard ellipsoids confined so that their centres of mass interacted sterically with smooth
substrates, Allen~\cite{Allen99} observed homeotropic surface arrangement. Subsequently van
Roij~\etal\cite{VanRoijDijkstra00, VanRoijDijkstra00a, DijkstraVanRoij01} investigated the behaviour of hard
spherocylinders at a hard smooth wall and observed surface-induced wetting and planar ordering. These studies also
showed that the planar arrangement is the natural state of hard-rod nematic phase in contact with a flat surface.
Chrzanowska~\etal and Cleaver~\etal\cite{Chrzanowska_Teixera_01, Cleaver_Teixeira_01} used the hard Gaussian
overlap (HGO) model~\cite{BernePechukas72} (\ie a hard version of the Gay-Berne model) to investigate confined
symmetric and hybrid anchored films using the hard-needle--wall (HNW) potential as a surface model. Here,
simulations of symmetrically anchored systems showed that, with appropriate tuning, this surface model can induce
either homeotropic or planar anchoring, a finding which we expand upon in the current article.

Liquid crystal adsorption has also been studied using simulations of all-atom models, investigating, for
example, the behaviour of 8CB on various substrates~\cite{CleaverTildesley94, YoneyaIwakabe95, CleaverCallaway95}.
These studies gave results which were largely consistent with scanning tunneling microscopy investigations with
respect to the structure of the observed planar arrangements. A more systematic series of simulation was performed
subsequently by Binger and Hanna~\cite{BingerHanna99, BingerHanna00, BingerHanna01} who simulated the adsorption
of several liquid crystals (\eg 5CB, 8CB, MBF). Systems ranging from single molecules up to two monolayers
anchored on different polymeric substrates (\eg PE, PVE, Nylon 6) were investigated. From this, the authors found
that, for most substrates, the liquid crystals adopted planar arrangements with some specific conformations being
favoured; conversely, Doerr and Taylor~\cite{DoerrTaylor99, DoerrTaylor99a} reported preferential homeotropic
anchoring from their simulations of 5CB on amorphous PE.

Comparatively, the
literature on computer simulations of anchoring transitions is extremely scarce. Using HGO and HNW potentials for,
respectively, intermolecular and surface interactions, Cleaver and Teixeira~\cite{Cleaver_Teixeira_01} have found
a density-driven homeotropic to planar anchoring transition at one wall of a model cell with hybridly-set
boundaries. Also, more recently, Lange and Schmid~\cite{LangeSchmid02,LangeSchmid02a,LangeSchmid02c} have observed
an anchoring transition in a system of ellipsoidal Gay-Berne particles confined by grafted polymer chains. Here,
the transition between tilted and homeotropic arrangements was induced by changes in the grafting density.

In this paper, we use Monte Carlo simulations to study the effects of confinement on a system of model mesogenic
particles and so gain a microscopic understanding of their anchoring transition. This paper is organised as
follows: the models used for this study are described in Section~\ref{s:models}. The model's surface induced
structural changes are studied in Section~\ref{s:simRes} through observation of typical profiles obtained from the
simulations and the dependence of these profiles on the surface potential and the system density. From this, in
Section~\ref{s:anchoringPhaseDia}, we obtain a comprehensive mapping of the model's anchoring behaviour, including
the identification and localisation of its planar to homeotropic anchoring transition. This Section also contains
an explicit study of the anchoring bistability found to be associated with this transition. The conclusions drawn
from this work and a description of future studies are given in Section~\ref{s:ccl}.


%=======================================================================================
%           THE MODEL
%=======================================================================================
\section{The model}
\label{s:models}

Here, surface induced structural changes are studied using Monte Carlo simulations of rod-shaped
particles that  interact with one another through the hard Gaussian overlap (HGO)
potential~\cite{Rigby89} and with the confining  substrates via the hard needle wall (HNW)
potential. The latter was used since it  provides a simple and
intuitive steric interaction which can be tuned so as to induce either homeotropic or planar
anchoring arrangements~\cite{Chrzanowska_Teixera_01,Cleaver_Teixeira_01}. It also lends itself
to the straightforward development of
geometry-based predictions with  which to compare simulation results. Despite this simplicity,
however, the HNW potential is able to exhibit both  anchoring transitions and bistability.

The HGO model is a steric model in which the contact distance is the shape parameter determined
by Berne and Pechukas~\cite{BernePechukas72} when they considered the overlap of two ellipsoidal
Gaussian.  Thus, the  interaction potential $\mathcal{V}^HGO$ between
two particles $i$ and $j$ with respective orientations $\ui$  and $\uj$ and intermolecular vector $\vect{r}_{ij} =
r_{ij}\rij$ is defined as
\begin{equation}
    \mathcal{V}^\mrm{HGO} = \left\{  %}
    \begin{array}{ccc}
    0   &\mrm{if}   &r_{ij} \geq \sijr  \\
    \infty  &\mrm{if}   &r_{ij} < \sijr
    \end{array}
    \right.
\end{equation}
where $\sijr$ is the contact distance
\begin{equation}
    \sijr = \so\left\{
    1 - \frac{1}{2}\chi\left[
    \frac{ \lp\dotProd{\rij}{\ui} + \dotProd{\rij}{\uj}\rp^2 }{1 + \chi(\dotProd{\ui}{\uj})}
      + \frac{ \lp\dotProd{\rij}{\ui} - \dotProd{\rij}{\uj}\rp^2 }{1 - \chi(\dotProd{\ui}{\uj})}
    \right] \right\}^{-\frac{1}{2}}.
\end{equation}
Here $\so$, the particle width, sets the unit of distance for this model and the shape anisotropy parameter $\chi
= (k^2-1)/(k^2+1)$ where $k= \sel/\so$ is the particle length to breadth ratio.

The HGO model is the hard-particle equivalent of the much-studied Gay-Berne model~\cite{GayBerne81}.  The phase
behaviour of the HGO model is density driven and fairly simple, comprising only two non-crystalline phases;
isotropic and (for $k\gtrsim 3$) nematic fluids at, respectively, low and high number densities $\rho^{*}$. The
isotropic-nematic phase-coexistence densities have been located for various particle elongations in a series of
previous Monte Carlo simulation studies~\cite{PadillaVelasco97,DeMiguelDelRio01,DeMiguelDelRio03}; for the most
commonly used elongation of $k=3$, the transition occurs for $\rho^{*}\approx 0.30$ with a slight system size
dependence. Our own simulations of 3d bulk systems of $N=1000$ particles~\cite{fbThesis,TeixeiraBarmes03} have
found isotropic (nematic) coexistence densities of $\rho^{*}_I=0.299$ ($\rho^{*}_N=0.309$), in agreement with
these previous studies. Although the HGO model was originally derived using geometrical considerations, an HGO
particle cannot be represented by a fixed solid object. Rather, it is a mathematical abstraction of
the interaction surface between two non-spherical particles~\cite{Rigby89}. For moderate
elongations, however, the properties of
HGO particles are similar to those of an the equivalent hard ellipsoid of revolution~\cite{Rigby89}. Simulation
studies ~\cite{DeMiguelDelRio03} have borne this out, showing that the equation of state of the HGO fluid is
qualitatively equivalent to, but consistently displaced from, that of the hard ellipsoid fluid.

In this paper, we sidestep the issue of how an undefined HGO particle should interact with a planar substrate
(this is addressed elsewhere~\cite{fbThesis,BarmesCleaver03b}) since here the HNW potential has been used for the
particle-wall interaction. In this, the particle surfaces do not interact directly with the substrates; rather the
interaction is mediated by axial needles placed at the centres of the HGO particles (Figure~\ref{fig:HGO_wal}).
The interaction potential between particle $i$ and a planar substrate located at $z=z_0$ is, thus, given
by~\cite{Cleaver_Teixeira_01}
\begin{equation}
\mathcal{V}^{HNW} = \left\{ %}
    \begin{array}{ccc}
    0               &\mrm{if}   &|z_i-z_0| \geq \sigma_w(\ui)  \\
    \infty          &\mrm{if}   &|z_i-z_0| < \sigma_w(\ui)
    \end{array}
    \right.
\end{equation}
where
\begin{equation}
        \sigma_w(\ui) = \frac{1}{2}\so k_S\cos(\theta_i).
    \label{eqn:sigma_w_HNW}
\end{equation}
Here, $k_S$ is the dimensionless needle length and $\theta_i=\arccos(u_{i,z})$ is the angle between the substrate
normal and the particle's orientation vector, which also corresponds to the zenithal Euler angle
(Figure~\ref{fig:HGO_wal}).

The surface behaviour of the HNW model has previously been studied in slab-geometry by Cleaver and
Teixeira~\cite{Cleaver_Teixeira_01} and Chrza\-now\-ska \etal\cite{Chrzanowska_Teixera_01}. Also the model
simulated by Allen~\cite{Allen99}, in which the particle centres of mass were taken to interact sterically with
the substrate, corresponds to the HNW potential with $k_S=0$. For small $k_S$, the homeotropic arrangement has
been shown to be stable, whereas planar anchoring is favoured for long $k_S$. Insight can be gained into this
transition by noting that, in the limit of perfect orientational and positional order, the Helmholtz free energy
of this system is minimized by the arrangement that maximises the particle volume absorbed into the substrates.
For all $k$ and $k_S\neq 0$, the absorbed volume of a single particle is clearly maximal for $\theta=\pi/2$,
suggesting that the planar arrangement should always be stable (for the case $k_S=0$, planar and homeotropic
alignments both allow the absorption of half the particle volume). For many body systems, however, it is also
necessary to consider the relative packing efficiencies of the two arrangements.

To this end, we now consider the behaviour of a system of HGO particles in a fixed volume, one face of which is
bounded by an HNW potential substrate. The aim is to calculate the homeotropic to planar transition needle length
for this system. For simplicity, each particle close to the surface is approximated to be an ellipsoid of
revolution~\cite{Rigby89} with elongation $k=\sel/\so$ and semi-axis lengths $a=b=\so/2$ and $c=\sel/2$. The
homeotropic to planar transition needle length can then be determined by equating the ratio of the volume absorbed
per unit area of the substrate for the two key arrangements. In the limit of perfect order, symmetry details of
the packing can be ignored in this calculation since they must be the same for both anchoring alignments; the two
arrangements will map onto each other via suitable affine transformations.

In the case of planar alignment, the adsorbed volume and occupied area are independent of $k_S$ and their ratio is
simply $\so/3$. The homeotropic case is rather more involved, since although the projected area of each particle
on to the surface is constant, the particle-substrate distances are now dependent on $k_S$. In the
ellipsoidal-particle approximation, the escaped volume for such a particle can be shown to be
\begin{equation}
    V_e = \frac{\pi\so^2}{4}
    \left[
    \frac{k_S}{2} \lp \frac{k_S^2}{3\sel^2} -1  \rp + \frac{\sel}{3}
    \right]
    \label{eqn:V_H(Ln)}.
\end{equation}
Using this result, the needle length corresponding to the planar-homeotropic transition is then given by the root
$k_S^T$ of
\begin{equation}
        \frac{1}{6k^2\so^2}k_S^3 - \frac{1}{2}k_S + \frac{\so}{3}\lp k-1\rp = 0
        \label{eqn:LnTfinalPoly}
\end{equation}
satisfying $k_S^T \in ]0:k]$. The result is $k-$dependent; the variation of the transition $k_S^T/k$ as a function
of $k$ is shown in Figure~\ref{fig:LnT-k}. For the two particle elongations used in the simulation part of this
study, that is $k=3$ and $5$, the predicted transition needle lengths are $k_S^T/k = 0.48$ and 0.61, respectively.

%SIMULATIONS
%=======================================================================================
\section{Simulation of symmetrically anchored systems}
\label{s:simRes}

Here, we present the results from a comprehensive Monte Carlo simulation study of HGO systems confined between
symmetrical HNW substrates. All of the simulations were performed in the canonical ensemble on systems of $N=1000$
particles. Particle elongations $k=3$ and $5$ were both studied, but for reasons of space we restrict ourselves,
in most cases, to showing results for the $k=3$ systems only. The substrates were separated by a distance
$L_z=4k\so$ and were located at $z=\pm L_z/2$, the system being periodic in the $x$ and $y$ directions. The
simulation box lengths in these other directions were determined by the imposed number density, $\rho^*$, through
the relationship $L_x=L_y=\sqrt{N/(\rho^{*}L_z)}$. In each simulation, the same substrate potential (i.e., $k_s$
value) was applied at each wall, so that all of the results presented relate to symmetric anchoring situations.
That restriction apart, a systematic study of the $k_S$ and $\rho^*$ dependence of these systems has been
undertaken; Figures~\ref{fig:simsDia_HNW}(a) and (b) show the state points at which simulations were performed as
well as the directions of the various simulation series. The simulation series directions are given since, at some
state points, the anchoring adopted by the system was found to be dependent on that of the initial configuration
employed. Typical run-lengths at each state point were $0.5\times 10^6$ MC sweeps (where one sweep represents one
attempted move per particle) of equilibration followed by a production run of $0.5\times 10^6$ sweeps.

%%
%% Change after referee #2
%%
The relatively modest system size of $N=1000$ has been used here in order to enable a comprehensive
mapping of the relevant phase space to be achieved. From De Miguel's study of system size effects in 3d bulk
systems of Gay-Berne particles~\cite{DeMiguel92}, it is apparent that any $N$-dependence of bulk behaviour should
be negligible for $N=\mathcal{O}(10^3)$. This conclusion does not transfer automatically to confined systems,
however, since the surface extrapolation lengths can become comparable with the substrate-substrate
separation~\cite{Priezjev03}. For the systems studied here, in which the surface conditions were symmetrical, we
have found that doubling the slab thickness (\ie running with $N=2000$ particles) does not have a significant
effect on the anchoring behaviour observed. However, in equivalent simulations of hybrid anchored systems, in
which the two surface extrapolation length regions can promote competing effects, we have found that the slab
thickness becomes a significant simulation parameter~\cite{fbThesis,BarmesCleaver04c}.
%%
%%
%%

We present the simulation results by first describing the typical behaviours exhibited by these systems. We then
go on to assess the global phase and structural dependence on the imposed number density, $\rho^*$, and internal
needle length $k_S \so$ before, in the next Section, going on to construct anchoring maps and make an
explicit examination of the bistability displayed by one system.

Initial analysis of the surface-induced structural changes has been performed using profiles of the number
density, $\rho^{*}_\ell(z)$, and the orientational order measured with respect to the substrate normal
\begin{equation}
Q_{zz}(z)=\frac{1}{N(z)}\sum_{i=1}^{N(z)} \left( \frac{3}{2}u_{i,z}^2 - \frac{1}{2} \right)
\end{equation}
where $N(z)$ is the instantaneous occupancy of the layer. Typical profiles for bulk isotropic and nematic
densities are shown for homeotropic ($k_S<k^T_S$), planar ($k_S>k^T_S$) and competing ($k_S\sim k^T_S$) anchoring
arrangements in Figures~\ref{fig:typicalProfile_k3_homeo}-\ref{fig:typicalProfile_k3_bist}.

Homeotropic alignment was observed when the HNW potential was characterised by a short needle length. Thus,
Figures~\ref{fig:typicalProfile_k3_homeo} show that with $k=3$ and $k_S/k = 0.2$, $\rho^{*}_\ell(z)$ was dominated
by surface-layer peaks at $|z-z_0|\sim k_S \so / 2$. Small secondary features, displaced by $k\so$ from the
surface peaks, are apparent at the nematic density, consistent with an arrangement which involved smectic-like
layers templated by the substrates. The corresponding $Q_{zz}(z)$ profiles confirm this homeotropic arrangement,
showing positive values in regions of high $\rho^{*}_\ell(z)$, that is the interfacial regions for both densities
and the bulk region at the nematic density. The oscillations in $Q_{zz}(z)$ correlate closely with those in
$\rho^{*}_\ell(z)$ due to both the intrinsic coupling between density and order parameter and the homeotropic
director-pinning imposed by the smectic-like layers. Additionally, regardless of the number density, $Q_{zz}(z)$
displays negative values very close to the substrates ($|z-z_0|\leq k_S \so/{2}$); these features occur because
the HNW potential dictates that any particles in these regions must be tilted away from the substrate normal.
These regions with negative $Q_{zz}(z)$ values have very low number densities, however, and so are of little
practical significance.

For planar alignment situations, such as that observed for $k=3$ and $k_S/k = 0.8$, the density and order profiles
adopted the forms shown on Figures~\ref{fig:typicalProfile_k3_planar}. Here, because the surface layers were
dominated by planar-aligned particles, the main peaks in $\rho^{*}_\ell$ are located at $|z-z_0|\sim 0$. The
peak-peak separations in the nematic-density $\rho^{*}_\ell(z)$ are also much smaller than those seen in the
homeotropic case (approximately $\so$ rather than $k_S\so$), since here the particles in neighbouring strata had a
side-by-side arrangement (i.e. they were {\em not} in smectic-like layers). Again, the features in the $Q_{zz}(z)$
profiles correspond closely with those in $\rho^{*}_\ell(z)$, but this time the maxima in $\rho^{*}_\ell$ relate
to minima in $Q_{zz}(z)$ since planar order results in negative values of $Q_{zz}(z)$.

For some $k_S$ values intermediate between the two cases just presented, situations were found in which the
relative stability of the planar and homeotropic arrangements was unclear. Generally this situation arose in
nematic systems for which $k_S \sim k_S^{T}$. Typical profiles corresponding to this competing anchoring situation
are shown in Figure~\ref{fig:typicalProfile_k3_bist} for $k=3$ and $k_S/k=0.48$, {\em i.e.} a needle length very
close to the $k_S^{T}$ value given by Equation.(\ref{eqn:LnTfinalPoly}). Here, at high number densities, two sets of
profiles (corresponding to different surface arrangements) were obtained depending on the history of the
simulation sequence. Since, in each case, these profiles were obtained from runs equilibrated over $\sim 10^6$ MC
sweeps, bistability is suggested.

The $\rho^*=0.28$ profiles shown in Figure~\ref{fig:typicalProfile_k3_bist} are noteworthy, since they contain
features suggesting both planar and homeotropic influences. For example, the interfacial regions are characterized
by two peaks, of comparable height, corresponding to substrate distances of $|z-z_0|=0$ and $k_S \so / 2$,
respectively. This double peaked behaviour can be observed in both the $\rho^{*}_\ell(z)$ and the $Q_{zz}(z)$
profiles; in the latter it is manifested by the positive and negative regions corresponding to the peaks in the
local density.

The two sets of $\rho^*=0.34$ profiles shown in Figure~\ref{fig:typicalProfile_k3_bist} are very different from
one another, and indicate that the two surface arrangements shown at this density in
Figures~\ref{fig:typicalProfile_k3_homeo} and \ref{fig:typicalProfile_k3_planar} were both accessible for this
$k_S$ value. The planar state was obtained when, in a series of runs performed at fixed density, the $k_S$ value
was decremented between runs, whereas the homeotropic arrangement was formed in an equivalent sequence performed
with increasing $k_S$ values. An interesting sign change is apparent in the $Q_{zz}(z)$ profile for the planar
$\rho^*=0.34$ state. Here, while the surface and bulk alignment were both unquestionably planar, positive $Q_{zz}$
values were obtained for $|z-z_0|\simeq k_S \so / 2$, indicating, rather surprisingly, localized regions with net
homeotropic alignment. This behaviour was also seen for systems whose $k_S$ values were sufficiently large for the
planar aligned state to be unambiguously stable, and so is unrelated to the possible bistability noted above.
Instead, we attribute this behaviour to the effect of particle-particle interactions on the $\theta$-dependence of
the orientational distribution function at these $z$ values. For example, we note that at these wall separations,
a particle orientation corresponding to modest tilt away from the substrate normal is disfavored since the
resulting gap opened up between the tilted particle and the substrate would be too small to be occupied by a
second particle. Consequently, the effect of the interactions between the tilted particle and its neighbours would
be either to close the gap (by shifting the particle towards the substrate or rotating its orientation closer to
the substrate normal) or to make the gap big enough for a second particle to enter (by making the particle more
planar). This relative instability of moderate tilt angles at these wall separations is, we consider, the primary
cause of the positive $Q_{zz}$ values noted above.

%==============================================================================================

Having described the behaviours exhibited by these systems, we now consider their $\rho^{*}$- and
$k_S$-dependences by presenting the results from a more comprehensive investigation. In order to study the
influence of density, several series of simulations were carried out at constant $k_S$ values and increasing and
decreasing densities. From these, $\rho^{*}_\ell(z,\rho^{*})$ and $Q_{zz}(z,\rho^{*})$ surfaces were computed for
a range of $k_S$ values. Results for the cases $k_S/k = 0.0$ and $k_S/k = 1.0$ are shown on
Figures~\ref{fig:rhoInfl_k3_L000} and~\ref{fig:rhoInfl_k3_L100} for $k=3$. Because little hysteresis was found
between the series performed with increasing and decreasing densities, only the data obtained
from the latter series are shown.

The surfaces shown in Figures~\ref{fig:rhoInfl_k3_L000} and~\ref{fig:rhoInfl_k3_L100} largely confirm the
observations made previously. At low densities, the surface induced effects are limited to the surface regions.
With increase in density, however, the number of peaks in $\rho^{*}_\ell(z)$ increases steadily with $\rho^*$ due
to enhanced layering of the particles. Surface-induced order can be seen to extend further into the cell with
increase in $\rho^*$, largely reflecting the features apparent in the $\rho^{*}_\ell(z)$ surfaces. In the case of
extreme homeotropic anchoring ($k_S/k = 0.0$), regardless the density, the central region of the cell failed to
adopt nematic order. This was because, for this reduced needle length, the particle volume absorbed by the
substrates reduced the bulk region density sufficiently to shift the I-N transition to $\rho^{*}$ values outside
the range considered here. This does not, however, question the existence of uniform alignment in the case $k_S =
0.0$. For the $k_S/k = 1.0$ surface, orientational ordering of approximately the central 50\% of the system can be
seen to have taken place very uniformly. Relatively long runs were needed to establish this behaviour since, at
densities with isotropic bulk regions, there was little azimuthal coupling between the two ordered surface
regions.

%==============================================================================================

In order to generate equivalent information on the influence of $k_S/k$ on the cell's behaviour, further series of
simulations were performed at constant densities with both increasing and decreasing needle lengths. Results from
the series with decreasing $k_S/k$, $k=3$ and densities $\rho^{*}=0.28$ and 0.34 are shown as surfaces of
$\rho^{*}_\ell(z,k_S/k)$ and $Q_{zz}(z,k_S/k)$ in Figures~\ref{fig:LnInfl_k3_d0.28} and~\ref{fig:LnInfl_k3_d0.34},
respectively. The differences found between the series performed with increasing and decreasing $k_S/k$ are
discussed in the next Section.

From Figures~\ref{fig:LnInfl_k3_d0.28} and~\ref{fig:LnInfl_k3_d0.34}, the features discussed previously for strong
homeotropic and planar anchoring arrangements can be seen at low and high values of $k_S/k$, respectively. A clear
distinction can be made between the two anchoring regions from each of these surfaces, and an homeotropic to
planar anchoring transition is again evident. As the transition region is approached from either high or low
$k_S$, the peak heights of the $\rho^{*}_\ell$ surface decrease quite rapidly. At low density, remnants of
these peaks remain at the transition, leading to bimodal surface layers which have features corresponding to both
anchoring states. At nematic densities, in contrast, the crossover from one structure to the other is quite sharp.
For the $\rho^{*}_\ell(z,k_S/k)$ surfaces shown in Figures~\ref{fig:LnInfl_k3_d0.28}
and~\ref{fig:LnInfl_k3_d0.34}, the $z$-values of the peaks on the two sides of the transition are clearly not
coincident. The associated $Q_{zz}(z,k_S/k)$ surfaces give an obvious signature of the homeotropic to planar
transition due to the sign change they exhibit. This change is relatively smooth in the surface regions, but is
very sharp in the bulk region of the nematic density system. The gradient of this change was also found to
increase with $k$; the transition was much sharper for $k=5$ than for $k=3$.

%=======================================================================================
%           ANCHORING MAPS
%=======================================================================================
\section{Anchoring maps and bistability}
\label{s:anchoringPhaseDia}


A phase-space mapping of the planar to homeotropic anchoring transition region needs to be expressed in terms of
some quantitative indicator of the arrangement displayed by a given confined system. A useful measure for the
characterization of the surface arrangement would be a scalar capable of indicating both the type and strength of
the anchoring for a given $(\rho^{*}, k_S/k)$ state point. To this end, we introduce $\overline{Q}_{zz}$, a
density-profile-weighted average of $Q_{zz}(z)$ taken over a given region of interest. We define
$\overline{Q}_{zz}$ as~:
\begin{equation}
    \overline{Q}_{zz} = \frac{\sum_{z_i} Q^{n}_{zz}(z_i) \rho^{*}_\ell(z_i)}
            {\sum_{z_i} \rho^{*}_\ell(z_i)}
\end{equation}
where the $z_i$ considered are restricted to a given region of interest ({\em i.e.} bulk or surface). Here
$Q^{n}_{zz}(z)\in[-1:1]$ is a pragmatically rescaled version of $Q_{zz}$ defined by~:
\begin{equation}
    Q^{n}_{zz} = \left\{    %}
    \begin{array}{ccc}
        Q_{zz}      &\text{if}  &Q_{zz} \geq 0  \\
        2.Q_{zz}    &\text{if}  &Q_{zz} < 0.\\
    \end{array}
    \right.
\end{equation}
In order for $\overline{Q}_{zz}$ to be an objective measure, a criterion is needed with which to set the range of
$z_i$ to be included in its evaluation for, say, a surface region. Clearly, this limiting $z_i$ needs to be
located at a point at which the surface has no direct influence on the molecules. An apparently attractive choice
for this would, therefore be the distance at which the particles can rotate freely without direct interaction with
the surface, {\em i.e.} $|z_i-z_0| = k_S \so/2$. This approach is flawed, however, since, in the limit of zero
needle length, it implies $|z_i-z_0| \sim 0$, whereas the surface layers clearly have finite thicknesses for all
needle lengths.

Instead, the following approach was adopted in defining this limit; the surface region width was made a function
of the needle length and density by making the limiting $z_i$ value dependent on features of the measured density
profiles. Explicitly, where the anchoring was found to be planar (with the first local maximum of
$\rho^{*}_\ell(z)$ at $|z_i-z_0|\sim 0$), the surface region was taken to extend from the substrate to the
distance corresponding to the second maximum in $\rho^{*}_\ell(z)$. If however, the anchoring was homeotropic
(with the first local maximum of $\rho^{*}_\ell(z)$ at $|z_i-z_0|\sim k_S \so/2$), the surface region was taken to
extend from the substrate to the first local minimum in $\rho^{*}_\ell(z)$. In those cases with ambiguous,
double-peaked density profiles, the former scheme was adopted. In all cases, the bulk region was taken to be that
part of the system not included in the surface regions.

%===============================================================================================
%===============================================================================================

In what follows, maps of $\overline{Q}_{zz}$ plotted as a function of $\rho^{*}$ and $k_S/k$ are used to construct
anchoring maps. Such diagrams have been determined for systems with elongations $k=3$ and $k=5$ using
data from the series of simulations performed at constant densities and both increasing and decreasing needle
lengths (recall Figure~\ref{fig:simsDia_HNW}). Results for both series in the surface ({\em Su}) and bulk ({\em
Bu}) regions are shown in Figures~\ref{fig:QzzPhaseDia_k3} and~\ref{fig:QzzPhaseDia_k5} for $k=3$ and $k=5$,
respectively.

The results obtained for the two elongations are qualitatively similar. In the surface region diagrams, the
anchoring transitions occur at $k_S/k$ values close to those predicted in Section~\ref{s:models}, as can be
observed from the lines of $\overline{Q}^{Su}_{zz} = 0$. This agreement can be seen to improve with increase in
density. Also the contour line spacing around $k_S^T$ becomes tighter with increasing density, indicating a
possible discontinuous transition between the planar and homeotropic anchoring states. In the bulk region
diagrams, by comparison, little surface influence can be observed at low densities, because the values of
$\overline{Q}^{Bu}_{zz}$ are limited by these systems' orientational isotropy. At number densities corresponding
to bulk nematic order, however, the surface influence extends into the bulk region and sharp anchoring transitions
become apparent at needle lengths similar to those suggested by the surface region anchoring diagrams. It is
noteworthy that this effect is generally seen at global number densities significantly greater than the I-N
transition densities of the equivalent bulk systems. This indicates that the local densities found in the bulk
regions were lower than the imposed $\rho^{*}$ values due to the absorbing nature of the substrates.

The anchoring maps are asymmetric in that bulk planar ordering develops at lower densities than its
homeotropic counterpart. This is due, in part, to the $k_S$-dependence of the volume absorption by the substrates,
leading to the approximately bi-linear dependence of $\rho^{*}_{NI}$ on $k_S$ seen in the $\overline{Q}^{Bu}_{zz}$
anchoring maps. As noted previously, in the limit $k_S=0$, this absorption has proved sufficient to prevent the
onset of bulk nematic for the range of $\rho^{*}$ considered here. In addition to this bilinear $k_S$ dependence,
a further increase in $\rho^{*}_{NI}$ can be observed at $k_S\simeq k_S^T$. We attribute this anomalous behaviour
to the competing anchoring effects experienced by these systems. As was noted previously, the profiles obtained
for such systems contain features associated with both of the anchoring arrangements. This, we suggests, leads to
relatively disordered surface regions in these systems which, in turn, causes a delay in the onset of bulk-region
nematic order.

%==============================================================================================

Another interesting feature of Figures~\ref{fig:QzzPhaseDia_k3} and~\ref{fig:QzzPhaseDia_k5} comes from the
comparison of the diagrams for increasing and decreasing needle lengths. At high densities, this confirms the
earlier observation that, in conditions corresponding to competing  anchoring, the structure adopted becomes
dependent on system history. This suggests possible bistable behaviour, on the `timescales' of our simulations at
least, for state points close to the anchoring transition. The extent of this bistability has been evaluated by
computation of the absolute value of the difference between the results obtained with series of increasing and
decreasing needle lengths. These bistability maps, shown on Figures~\ref{fig:QzzPhaseDia_k3}(c)
and~\ref{fig:QzzPhaseDia_k5}(c), indicate distinct bistable regions at nematic densities for both $k=3$ and $k=5$.

In order to examine this behaviour more explicitly, a final series of simulations has been performed in an attempt
to switch the cell from planar to homeotropic and back again. For this, a previously equilibrated planar anchored
system of $N=1000$ particles with $k=3$, $\rho^{*}=0.34$ and $k_S/k = 0.5$ was used as the initial configuration.
The switching was then performed through the series of simulations $R_1$ to $R_5$ listed in
Table~\ref{tble:bistFieldConditions}, \ie by applying and removing an electric field $\vect{E} = E\vecth{z}$ and
taking the dielectric anisotropy $\chi_e$ to be, alternately, positive and negative. The effect of this was to
align the particles parallel or perpendicular to $\vect{E}$ for positive and negative values of $\chi_e$,
respectively. While this setup is admittedly somewhat unrealistic, it can be related to a putative experimental
system in which the mesogen's dielectric coupling varies according to the frequency of an applied AC field.

Configuration snapshots corresponding to the initial and final states from each run in this series are shown in
Figures~\ref{fig:HGOk3BistSnaps}(a) to (f). The corresponding behaviours of $\overline{Q}^{Su}_{zz}$ and
$\overline{Q}^{Bu}_{zz}$, as a function of the number of MC sweeps, are shown in Figures~\ref{fig:QzzWak3Bist}(a)
and~\ref{fig:QzzWak3Bist}(b). Also, for comparison, the values of $\overline{Q}_{zz}$ observed in our previous
runs at this state point are shown as horizontal lines.

The results confirm that, for the `timescales' accessible to these simulations, the bistability implied by
Figure~\ref{fig:QzzPhaseDia_k3}(c) is fully realizable. Run $R_1$ shows that the initial system remained stable in
its planar arrangement; after reorientation of the particles along $\vecth{z}$ by the applied field (run $R_2$),
the  system equilibrated naturally to a homeotropic arrangement (run $R_3$) although the final values adopted by
$\overline{Q}_{zz}$ proved higher than those obtained from the previous runs performed at this state point. This
discrepancy may have been a consequence of the relatively high field strength employed in run $R_2$ which induced
almost perfect homeotropic alignment in the near-substrate region. Thus, while the bulk of the system equilibrated
towards the stable homeotropic state upon removal of the field, packing constraints in the surface region may have
inhibited particles from adopting planar orientation close to the surface (as was seen for a small number of
particles in the previous simulations). We do not think that this brings into question the equilibration of the
state observed here. Reapplication of the field with a negative molecular dielectric anisotropy (run $R_4$) then
generated a strongly planar arrangement which relaxed to the original stable state upon field removal (run $R_5$).
In this case, the system equilibrated to the same values of $\overline{Q}_{zz}$ as those obtained previously. It
should be noted also, that the `response times' of the systems were different in the bulk and interfacial regions;
no precise information can be inferred from this, however, since the true time evolution of the system cannot be
determined from our MC simulations.

%=======================================================================================
%            CONCLUSIONS
%=======================================================================================
\section{Conclusions}
\label{s:ccl}

In this paper we have studied the behaviour of liquid crystalline systems confined in slab geometry between
symmetric walls. For the simple HNW potential used here, we have shown that the preferred anchoring direction is
controlled by the particle-substrate interaction needle length, $k_S$. At nematic densities, all of the systems
simulated have exhibited a homeotropic to planar anchoring transition on increase in $k_S$. As well as having
orthogonal bulk alignments, the homeotropic and planar states have been shown to involve distinct surface
arrangements, with little configurational overlap.

The anchoring transition observed here appears to be first order, because of both the associated discontinuity in
the anchoring orientation and the configurational hysteresis (or bistability) observed. This bistability has been
found over a relatively narrow parameter window, but has proved highly reproducible with both states being very
long lived. Presumably, this longevity is related to the configurational distinctiveness of the two surface states
and the anchoring-orientation-conserving effect of the overlying nematic fluid. It should be possible to
characterize this bistability more quantitatively by using reweighting techniques to determine the form of
free-energy barrier separating the stable states; the height of this $\rho^*$-dependent barrier height sets the
time-scale of any spontaneous switching between the states and, so, knowledge of it would enable some sort of
comparison to be made with bistable experimental systems.

Rather surprisingly, the anchoring behaviour of this system has shown very little dependence on the system
density; the limits of the bistability regions are, essentially, isochores. One consequence of this has been that
the approximate transition needle lengths, $k_S^T$, calculated in the high packing limit, have proved reasonably
accurate for all nematic densities. Put another way, this suggests that even at $\rho_{NI}^*$, this system was
aware of the two states' particle-volume absorption efficiencies in the high packing limit. This counterintuitive
result is presumably a consequence of the simplicity of the HNW model used here: we will revisit this issue, and
the routes it offers for manipulation of the bistability region, in future work employing models with less
idealized particle-substrate interactions~\cite{BarmesCleaver03b}. Specifically, focus will be brought to bear on
the influence of the particle-surface contact function on the surface anchoring behaviour and the size and shape
of the bistability region.



%=======================================================================================
\section*{Acknowledgments}

We wish to acknowledge useful discussions with Dmytro Antipov, Chris Care and Paulo Teixeira and Sheffield Hallam
University's Materials Research Institute for financial support.


%=======================================================================================
%           Bibliography
%=======================================================================================
\linespacing{1.0}

\begin{thebibliography}{10}

\bibitem{Jerome91}
{B. J\'erome}.
\newblock Surface effects and anchoring in liquid crystals.
\newblock {\em Physics reports}, 54:391--451, 1991.

\bibitem{Schoen96}
{M. Schoen}.
\newblock On the uniqueness of stratification-induced structural tranformations
  in confined films.
\newblock {\em Ber. Bunsenges. Phys. Chem.}, 100:1355, 1996.

\bibitem{Schoen96a}
{M. Schoen}.
\newblock The impact of discrete wall structure on stratification-induced
  structural phase transitions in confined films.
\newblock {\em Journal of Chemical Physics}, 105:2910--2918, 1996.

\bibitem{PatelYokoyama93}
{J.S. Patel and H. Yokoyama}.
\newblock Continuous anchoring transition in liquid crystals.
\newblock {\em Nature}, 362:525, 1993.

\bibitem{JagemalmKomitov1997}
{P. Jagemalm and L. Komitov}.
\newblock Temperature induced anchoring transition in nematic liquid crystals
  with two-fold degenerate alignment.
\newblock {\em Liquid Crystals}, 23:1, 1997.

\bibitem{BarberiGiocondo98}
{R. Barberi, M. Giocondo, M. Iovane, I. Dozov and E. Polossat}.
\newblock {Nematic anchoring transitions on bistable SiO films driven by
  temperature and impurities}.
\newblock {\em Liquid Crystals}, 25:23, 1998.

\bibitem{ZhuLu94}
{Y-M Zhu, Z-H. Lu, X.B Jia, Q.H. Wei, D. Xiao, Y. Wei, Z.H. Wu, Z.L. Hu and
  M.G. Xie}.
\newblock Anchoring transition of liquid crystals on crown ether monolayers.
\newblock {\em Physical Review Letters}, 72:2573, 1994.

\bibitem{ZhuWei94}
{Y-M. Zhu and Y. Wei}.
\newblock {Conformation change-driven anchoring transition of liquid crystals
  on crown ether liquid crystals Langmuir-Blodgett films}.
\newblock {\em Journal of Chemical Physics}, 101:10023, 1994.

\bibitem{BarberoPopaNita00}
{G. Barbero and V. Popa-Nita}.
\newblock Model for the planar-homeotropic anchoring transition induced by
  trans-cis isomerisation.
\newblock {\em Physical Review E}, 61:6696, 2000.

\bibitem{AlkhairallaAllison99}
{B. Alkhairalla, H. Allinson, N. Boden, S. D. Evans, and J. R. Henderson }.
\newblock Anchoring and orientational wetting of nematic liquid crystals on
  self-assembled monolayer substrates~: an evanescent wave ellipsometric study.
\newblock {\em Physical Review E}, 59:3033, 1999.

\bibitem{AlkhairallaBoden02}
{Alkhairalla B, Boden N, Cheadle E, Evans SD, Henderson JR, Fukushima H,
  Miyashita S, Schonherr H, Vancso GJ, Colorado R, Graupe M, Shmakova OE and
  Lee TR}.
\newblock Anchoring and orientational wetting of nematic liquid crystals on
  semi-fluorinated self-assembled monolayer surfaces.
\newblock {\em Europhysics Letters}, 59:410, 2002.

\bibitem{Jerome93}
{B. Jerome J. O'Brien, Y. Ouchi, C. Stanners and Y.R. Shen}.
\newblock {Bulk reorientation driven by orientational transition in a liquid
  crystal monolayer}.
\newblock {\em Physical Review Letters}, 71:758, 1993.

\bibitem{StoenescuMartinotLagarde99}
{D.N. Stonescu, P. Martinot-Lagarde and I. Dozov}.
\newblock Anchoring transition dominated by surface memory effect.
\newblock {\em Molecular Crystals And Liquid Crystals}, 329:339, 1999.

\bibitem{Sluckin95}
{T.J. Sluckin}.
\newblock Anchoring transitions at liquid crystal surfaces.
\newblock {\em Physica A}, 213:105, 1995.

\bibitem{TeixeiraSluckin92}
{P.I.C. Teixeira and T.J. Sluckin}.
\newblock {Microscopic theory of anchoring transitions at the surfaces of pure
  liquid crystals and their mixtures. I. The Fowler approximation}.
\newblock {\em Journal of Chemical Physics}, 97:1498, 1992.

\bibitem{TeixeiraSluckin92a}
{P.I.C. Teixeira and T.J. Sluckin}.
\newblock {Microscopic theory of anchoring transitions at the surfaces of pure
  liquid crystals and their mixtures. II. The effect of surface adsoption}.
\newblock {\em Journal of Chemical Physics}, 97:1510, 1992.

\bibitem{TeixeiraSluckin93}
{P.I.C Teixeira, T.J. Sluckin and D.E. Sullivan}.
\newblock Landau-de gennes theory of anchoring transition at a nematic liquid
  crystal-substrate interface.
\newblock {\em Liquid Crystals}, 14:1243, 1993.

\bibitem{ZhengQianSheng96}
{T. Zheng Qian and P. Sheng}.
\newblock Liquid-crystal phase transitions induced by microtextured substrates.
\newblock {\em Physical Review Letters}, 77:4564, 1996.

\bibitem{ZhengQianSheng97}
{T. Zheng Qian and P. Sheng}.
\newblock Orientational states and phase transitions induced by microtextured
  substrates.
\newblock {\em Physical Review E}, 55:7111, 1997.

\bibitem{GayBerne81}
{J.G. Gay and B.J. Berne}.
\newblock Modification of the overlap potential to mimic a linear site-site
  potential.
\newblock {\em Journal of Chemical Physics}, 74:3316, 1981.

\bibitem{ZhangChakrabarti96}
{Z. Zhang, A. Chakrabarti, O.G. Mouristen and M.J. Zuckermann}.
\newblock Substrate-induced bulk alignment of liquid crystals.
\newblock {\em Physical Review E}, 53:2461, 1996.

\bibitem{WallCleaver97}
{G.D. Wall and D.J. Cleaver}.
\newblock Computer simulation studies of confined liquid-crystal films.
\newblock {\em Physical Review E}, 56:4306--4316, 1997.

\bibitem{LathamCleaver00}
{R. Latham and D.J. Cleaver}.
\newblock Substrate-induced demixing in a confined liquid crystal film.
\newblock {\em Chemical Physics Letters}, 330:7--14, 2000.

\bibitem{WallCleaver03}
{G.D. Wall and D.J. Cleaver}.
\newblock Computer simulations of adsorbed liquid crystal films.
\newblock {\em Molecular Physics}, 101:1105, 2003.

\bibitem{GruhnSchoen97}
{T. Gruhn and M. Schoen}.
\newblock Microscopic structure of molecularly thin confined liquid-crystal
  films.
\newblock {\em Physical Review E}, 55:2861, 1997.

\bibitem{GruhnSchoen98}
{T. Gruhn and M. Schoen}.
\newblock {Grand canonical Monte Carlo simulations of confined `nematic'
  Gay-Berne films}.
\newblock {\em Thin Solid Films}, 330:46, 1998.

\bibitem{GruhnSchoen98a}
{T. Gruhn and M. Schoen}.
\newblock {A grand canonical ensemble Monte Carlo study of confined planar and
  homeotropically anchored Gay-Berne films}.
\newblock {\em Molecular Physics}, 93:681, 1998.

\bibitem{Allen99}
{M.P. Allen}.
\newblock Molecular simulation and theory of liquid crystal surface anchoring.
\newblock {\em Molecular Physics}, 96:1391, 1999.

\bibitem{VanRoijDijkstra00}
{R. van Roij, M. Dijkstra and R. Evans}.
\newblock Orientational wetting and capillary nematization of hard-rod fluids.
\newblock {\em Europhysics Letters}, 49:350, 2000.

\bibitem{VanRoijDijkstra00a}
{R. van Roij, M. Dijkstra and R. Evans}.
\newblock {Interfaces, wetting and capillary nematization of a hard-rod fluid:
  Theory for the Zwanzig model}.
\newblock {\em Journal of Chemical Physics}, 117:7689, 2000.

\bibitem{DijkstraVanRoij01}
{M. Dijkstra, R. van Roij and R. Evans}.
\newblock Wetting and capillary nematization of a hard-rod fluids~: a
  simulation study.
\newblock {\em Physical Review E}, 63:051703--1, 2001.

\bibitem{Chrzanowska_Teixera_01}
{A. Chrzanowska, P.I.C. Teixeira, H. Ehrentraut and D.J. Cleaver}.
\newblock Ordering of hard particles between hard walls.
\newblock {\em Journal of Physics : Condensed Matter}, 13:1--13, 2001.

\bibitem{Cleaver_Teixeira_01}
{D.J. Cleaver and P.I.C Teixeira}.
\newblock Discontinuous structural transition in a thin hybrid liquid crystal
  film.
\newblock {\em Chemical Physics Letters}, 338:1--6, 2001.

\bibitem{BernePechukas72}
{B.J. Berne and P. Pechukas}.
\newblock Gaussian model potentials for molecular interactions.
\newblock {\em Journal of Chemical Physics}, 56:4213--4216, 1972.

\bibitem{CleaverTildesley94}
{D.J. Cleaver and D.J. Tildesley}.
\newblock {Computer modelling of the structure of 4-n-octyl-4'-cyanobiphenyl
  adsorbed on graphite}.
\newblock {\em Molecular Physics}, 81:781, 1994.

\bibitem{YoneyaIwakabe95}
{M. Yoneya and Y. Iwakabe}.
\newblock {Molecular dynamics simulations of liquid crystal molecules adsorbed
  on graphite}.
\newblock {\em Liquid Crystals}, 18:45, 1995.

\bibitem{CleaverCallaway95}
{D.J. Cleaver, M.J. Callaway, T. Forester, W. Smith and D.J. Tildesley}.
\newblock {Computer modelling of the 4-n-alkyl-4'-cyanobiphenyls adsorbed on
  graphite: energy minimizations and molecular dynamics of periodic systems}.
\newblock {\em Molecular Physics}, 86:613, 1995.

\bibitem{BingerHanna99}
{D.R. Binger and S. Hanna}.
\newblock {Computer simulation of interactions between liquid crystal molecules
  and polymer surfaces. I. Alignment of nematic and smectic A phases}.
\newblock {\em Liquid Crystals}, 26:1205, 1999.

\bibitem{BingerHanna00}
{D.R. Binger and S. Hanna}.
\newblock {Computer simulation of interactions between liquid crystal molecules
  and polymer surfaces. II. Alignment od smectic C forming mesogens}.
\newblock {\em Liquid Crystals}, 27:89, 2000.

\bibitem{BingerHanna01}
{D.R. Binger and S. Hanna}.
\newblock {Computer simulation of interactions between liquid crystal molecules
  and polymer surfaces. III. use of pseudopotentials to represent the surface}.
\newblock {\em Liquid Crystals}, 28:1215, 2001.

\bibitem{DoerrTaylor99}
{T.P. Doerr and P.L. Taylor}.
\newblock {Simulations of liquid crystal anchoring at an amorphous polymer
  surface from various initial configurations.}
\newblock {\em Molecular Crystals And Liquid Crystals}, 330:1735, 1999.

\bibitem{DoerrTaylor99a}
{T.P. Doerr and P.L. Taylor}.
\newblock {Molecular dynamics simulations of liquid crystal anchoring at an
  amorphous polymer surface.}
\newblock {\em International Journal of Mordern Physics C}, 10:415, 1999.

\bibitem{LangeSchmid02}
{H. Lange and F. Schmid}.
\newblock Surface anchoring on layers of grafted liquid-crystalline chain
  molecules~: a computer simulation.
\newblock {\em Journal of Chemical Physics}, 111:362, 2002.

\bibitem{LangeSchmid02a}
{H. Lange and F. Schmid}.
\newblock An anchoring transition at surfaces with grafted liquid-crystalline
  chain molecules.
\newblock {\em European Physical Journal E}, 7:175, 2002.

\bibitem{LangeSchmid02c}
{H. Lange and F. Schmid}.
\newblock Surface anchoring on liquid crystalline polymer brushes.
\newblock {\em Computer physics communications}, 147:276, 2002.

\bibitem{Rigby89}
{M. Rigby}.
\newblock Hard gaussian overlap fluids.
\newblock {\em Molecular Physics}, 68:687, 1989.

\bibitem{PadillaVelasco97}
{P. Padilla and E. Velasco}.
\newblock The isotropic-nematic transition for the hard gaussian overlap fluid
  : testing the decoupling approximation.
\newblock {\em Journal of Chemical Physics}, 106:10299, 1997.

\bibitem{DeMiguelDelRio01}
{E. de Miguel, E. Mart\'{\i}n del R\'{\i}o}.
\newblock {The isotropic-nematic transition in hard gaussian overlap fluids}.
\newblock {\em Journal of Chemical Physics}, 115:9072--9082, 2001.

\bibitem{DeMiguelDelRio03}
{E. de Miguel, E. Mart\'{\i}n del R\'{\i}o}.
\newblock {Equation of state for hard gaussian overlap fluids}.
\newblock {\em Journal of Chemical Physics}, 118:1852, 2003.

\bibitem{fbThesis}
F.~Barmes.
\newblock {\em Computer simulation of confined and flexoelectric liquid
  crystalline systems}.
\newblock PhD thesis, Sheffield Hallam University, June 2003.

\bibitem{TeixeiraBarmes03}
{P.I.C. Teixeira, F. Barmes and D.J. Cleaver}.
\newblock {Symmetric alignment of the nematic matrix between close penetrable
  colloidal particles}.
\newblock {\em Journal of Physics : Condensed Matter}, accepted, 2004.

\bibitem{BarmesCleaver03b}
{F. Barmes and D.J. Cleaver}.
\newblock {Surface interaction potential effect on the anchoring behaviour of
  liquid crystals.}
\newblock in preparation, 2004.

\bibitem{DeMiguel92}
{E. de Miguel}.
\newblock {System size effects at the isotropic-nematic transition from
  computer simulations}.
\newblock {\em Physical Review E}, 47:3334, 1992.

\bibitem{Priezjev03}
{N.V. Priezjev, G. Ska\u{c}ej, R.A. Pelkovits and S. \u{Z}umer}.
\newblock External and intrinsic anchoring in nematic liquid crystals: A monte
  carlo study.
\newblock {\em Physical Review E}, 68:041709, 2003.

\bibitem{BarmesCleaver04c}
{F. Barmes and D.J. Cleaver}.
\newblock {Anchoring transitions and bistability in hybrid anchored liquid
  crystal films}.
\newblock {\em in preparation}, 2004.

\end{thebibliography}

%=======================================================================================================%
%           tables
%=======================================================================================================%
\clearpage
\newpage

\begin{table}
\centering
\begin{tabular}{||c||c||c||c||c||}
    \hhline{|t:=:t:=:t:=:t:=:t:=:t|}
    Run     &   $\vecth{E}$ &$E$    & $\chi_e$  &run length\\
    \hhline{|:=::=::=::=::=:|}
    $R_1$   &   (0,0,0)     &0.0    &0.0    &$0.25.10^6$\\
    $R_2$   &   (0,0,1)     &6.0    &0.5    &$0.25.10^6$\\
    $R_3$   &   (0,0,0)     &0.0    &0.0    &$1.00.10^6$\\
    $R_4$   &   (0,0,1)     &6.0    &-0.5   &$0.25.10^6$\\
    $R_5$   &   (0,0,0)     &0.0    &0.0    &$0.50.10^6$\\
    \hhline{|b:=:b:=:b:=:b:=:b:=:b|}
\end{tabular}
\caption{Parameterizations used to perform the switching between the planar and homeotropic states of the bistable
system. Run lengths are given in Monte Carlo sweeps.} \label{tble:bistFieldConditions}
\end{table}


%=======================================================================================================%
%           figures                                                                     %
%=======================================================================================================%

\pagebreak
\newlength{\picH}   % picture height
\newlength{\picW}   % picture width
\newcommand{\picA}{270} % picture angle

\picW = 10cm
\newcommand{\pic}[1]{\fbox{\includegraphics[width=\picW]{#1}}}
\newcommand{\picL}[1]{\fbox{\includegraphics[height=\picW, angle=\picA]{#1}}}



%=======================================================================================
\picW = 8cm
\begin{figure}
\centering \pic{fig_01.ps} \caption{(Color online) Schematic representation of the geometry used for the hard
needle wall particle-surface interaction} \label{fig:HGO_wal}
\end{figure}

%=======================================================================================
\picW = 8cm
\begin{figure}
        \centering
        \picL{fig_02.ps}
    \caption{(Color online) Predicted variation of $k_S^T/k$ with $k$ in the high packing limit.}
    \label{fig:LnT-k}
\end{figure}

%=======================================================================================
\picW = 10cm
\begin{figure}
    \centering
    \subfigure[$k=3$]{\picL{fig_03a.ps}}
    \subfigure[$k=5$]{\picL{fig_03b.ps}}
    \caption{(Color online) Representation of the state points considered in the simulations presented in this
    paper. The arrows represent the direction in which the simulations series were performed.}
    \label{fig:simsDia_HNW}
\end{figure}

%=======================================================================================
\picW = 8cm
\begin{figure}
        \centering
    \pic{fig_04.ps}
    \caption{(Color online) Typical profiles corresponding to homeotropic anchoring for $k=3$
    and $k_S/k = 0.20$ as obtained from simulation series at constant density and decreasing
    needle length.}
    \label{fig:typicalProfile_k3_homeo}
\end{figure}

%=======================================================================================
\picW = 8cm
\begin{figure}
        \centering
        \pic{fig_05.ps}
    \caption{(Color online) Typical profiles corresponding to a planar arrangement with $k=3$ and $k_S/k =
    0.80$ as obtained from simulation series at constant density and decreasing
    needle length.}
    \label{fig:typicalProfile_k3_planar}
\end{figure}

%=======================================================================================
\picW = 8cm
\begin{figure}
        \centering
        \pic{fig_06.ps}
    \caption{(Color online) Typical profiles for $k=3$ and $k_S/k = 0.48$. The term $k_S$ up or down in the key
    of this Figure, refers to whether the profiles has been obtained from simulation with
    increasing (up) or decreasing (down) needle length.}
    \label{fig:typicalProfile_k3_bist}
\end{figure}
\clearpage
\pagebreak
%=======================================================================================
\picW = 7cm
\begin{figure}
        \centering
        \subfigure[]{\pic{fig_07a.ps}}
    \subfigure[]{\pic{fig_07b.ps}}
    \caption{(Color online) Surfaces showing the influence of $\rho^{*}$ on systems with $k=3$ and $k_S/k =
    0.0$ (strongly homeotropic anchoring). These data are extracted from simulation series with
    decreasing needle length.}
    \label{fig:rhoInfl_k3_L000}
\end{figure}

\picW = 7cm
\begin{figure}
        \centering
        \subfigure[]{\pic{fig_08a.ps}}
    \subfigure[]{\pic{fig_08b.ps}}
    \caption{(Color online) Surfaces showing the influence of $\rho^{*}$ on systems with $k=3$ and $k_S/k = 1.0$ (strongly planar anchoring).
    These data are extracted from simulation series with decreasing needle length.}
    \label{fig:rhoInfl_k3_L100}
\end{figure}


%=======================================================================================
\picW = 7cm
\begin{figure}
        \centering
        \subfigure[]{\pic{fig_09a.ps}}
    \subfigure[]{\pic{fig_09b.ps}}
    \caption{(Color online) Surfaces showing the influence of $k_S/k$ on systems with $k=3$ and $\rho^{*} = 0.28$.
    These data are extracted from simulation series with decreasing needle length.}
    \label{fig:LnInfl_k3_d0.28}
\end{figure}

\picW = 7.0cm
\begin{figure}
        \centering
        \subfigure[]{\pic{fig_10a.ps}}
    \subfigure[]{\pic{fig_10b.ps}}
    \caption{(Color online) Surfaces showing the influence of $k_S/k$ on systems with $k=3$ and $\rho^{*} = 0.34$.
    These data are extracted from simulation series with decreasing needle length.}
    \label{fig:LnInfl_k3_d0.34}
\end{figure}

%=======================================================================================

\picW = 7cm
\begin{figure}
    \centering
    \subfigure[Anchoring maps for series with increasing $k_S/k$.]{
    \pic{fig_11al.ps}
    \pic{fig_11ar.ps}}

    \subfigure[Anchoring maps for series with decreasing $k_S/k$.]{
    \pic{fig_11bl.ps}
    \pic{fig_11br.ps}}

    \subfigure[Bistability diagrams (\ie difference between (a) and (b)).]{
    \pic{fig_11cl.ps}
    \pic{fig_11cr.ps}}
    \caption{Anchoring maps of $\overline{Q}_{zz}$ for $k=3$ for the surface
    and bulk regions of the cell. Diagrams on the l.h.s are relative to the
    interfacial region and those on the r.h.s are relative to the bulk region.}
    \label{fig:QzzPhaseDia_k3}
\end{figure}


%=======================================================================================
\picW = 7cm
\begin{figure}
    \centering
    \subfigure[Series with increasing $k_S/k$.]{
    \pic{fig_12al.ps}
    \pic{fig_12ar.ps}}

    \subfigure[Series with decreasing $k_S/k$.]{
    \pic{fig_12bl.ps}
    \pic{fig_12br.ps}}

    \subfigure[Bistability phase diagram (\ie difference between (a) and (b)).]{
    \pic{fig_12cl.ps}
    \pic{fig_12cr.ps}}
    \caption{Anchoring maps of $\overline{Q}_{zz}$ for $k=5$ for the surface
    and bulk regions of the cell. Diagrams on the l.h.s are relative to the
    interfacial region and those on the r.h.s are relative to the bulk region.}
    \label{fig:QzzPhaseDia_k5}
\end{figure}


%=======================================================================================
\picW = 6cm
\begin{figure}
    \centering
    \subfigure[start $R_1$]{\pic{fig_13a.ps}}
    \subfigure[end $R_1$]{\pic{fig_13b.ps}}
    \subfigure[end $R_2$]{\pic{fig_13c.ps}}
    \subfigure[end $R_3$]{\pic{fig_13d.ps}}
    \subfigure[end $R_4$]{\pic{fig_13e.ps}}
    \subfigure[end $R_5$]{\pic{fig_13f.ps}}
    \caption{Configuration snapshots corresponding to the initial (start) and final
    configurations of runs $R_1$ to $R_5$ described in Section~\ref{s:anchoringPhaseDia}.}
    \label{fig:HGOk3BistSnaps}
\end{figure}

%=======================================================================================
\picW = 12cm
\begin{figure}
    \centering
    \subfigure[$\overline{Q}_{zz}^{Su}(n)$]{\picL{fig_14a.ps}}
    \subfigure[$\overline{Q}_{zz}^{Bu}(n)$]{\picL{fig_14b.ps}}
    \caption{(Color online) Evolution of $\overline{Q}_{zz}$ in the (a)interfacial and (b)bulk regions as a
    function of the number of Monte Carlo sweeps.}
    \label{fig:QzzWak3Bist}
\end{figure}

\end{document}
