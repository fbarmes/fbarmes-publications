	
%\documentclass[aps,pre,10pt,onecolumn]{revtex4}
%\documentclass[aps,12pt,preprint,superscriptaddress]{revtex4}
\documentclass[aps,10pt,twocolumn]{revtex4}

%============================================================================
%		packages
%============================================================================
\usepackage[final]{graphics}	% for graphics
\usepackage[final]{graphicx}	% for graphics
\usepackage{subfigure}		% allow subfigure
\usepackage{amssymb}		% font
\usepackage{amstext}		% ams latex package
\usepackage{amsmath}		% font
\usepackage{amsfonts}		% font
\usepackage{amsbsy}		% font
\usepackage{hhline}		% nice lines is tables
\usepackage{xspace}		% add space at the end of macros
\usepackage{color}
\usepackage{calc}
\usepackage{hhline}     % nice lines in tables

%--------------------------
%	newcommands
%--------------------------
\newcommand{\etal}{\emph{et al.}\@\xspace}
\newcommand{\ie}{\emph{i.e.}\@\xspace}
\newcommand{\eg}{\emph{e.g.}\@\xspace}

\newcommand{\cf}{\emph{c.f.}\@\xspace}
\newcommand{\mrm}[1]{\ensuremath{\mathrm{#1}}\xspace}
\newcommand{\SiOTwo}{\ensuremath{\mrm{SiO_2}}\xspace}

\newlength{\picH}	% picture height
\newlength{\picW}	% picture width
\newcommand{\picA}{270}	% picture angle

\picW = 8cm

\newcommand{\pic}[1]{\includegraphics[width=\picW]{#1}}
\newcommand{\picL}[1]{\includegraphics[height=\picW, angle=\picA]{#1}}
\newcommand{\picB}[1]{\fbox{\includegraphics[width=\picW]{#1}}}
\newcommand{\picLB}[1]{\fbox{\includegraphics[height=\picW, angle=\picA]{#1}}}

\newcommand{\e}[1]{\ensuremath{\times 10^{#1}}}
\newcommand{\angstrom}{\ensuremath{\mrm{\AA}}\xspace}

\newcommand{\vect}[1]{ \mathbf{#1} }
\newcommand{\vecth}[1]{ \mathbf{\hat{#1} } }


%--------------------------
%	page layout
%--------------------------
\parindent=0pt

\begin{document}


\graphicspath
{
{../imgs/}
}


%%=============================
%%	Titre
%%=============================
\title{Molecular dynamics of shock-wave induced\\ structural changes in silica glasses}
\author{F. Barmes}

\affiliation
{
Centre Europ\'een de Calcul Atomique et Mol\'eculaire,
46, all\'ee d'Italie,
69007 Lyon, France
}


\author{L. Soulard}
\affiliation
{
CEA-DAM Ile-de-France,
Boite Postale 12,
91680 Bruy\`eres-le-Ch\^atel,
France
}



\author{M. Mareschal}
\affiliation
{
Universit\'e Libre de Bruxelles,
Facult\'e des Sciences, CP231,
Boulevard du Triomphe
B1050, Brussels, Belgium
}


%%=============================
%%	Abstract
%%=============================
\begin{abstract}
%%------------------------------------------------------------------------------------------------------
%%	What
%%------------------------------------------------------------------------------------------------------
We seek to model the shock wave induced structural changes in silicate glass at the atomic
scale. We use both direct shock propagation with non-equilibrium molecular dynamics (NEMD) and
bulk  simulations in the Hugoniot ensemble to characterize the structure and topology of the
shocked glass. 
%%------------------------------------------------------------------------------------------------------
%%	Bulk
%%------------------------------------------------------------------------------------------------------
Despite the lack of long-range interactions in our model, the close agreement between our
structures and those obtained by experimental and simulation studies alike, underlines the
importance of the role played by first neighbor interactions on the structure of silicate glass.  
%%------------------------------------------------------------------------------------------------------
%%	Shock
%%------------------------------------------------------------------------------------------------------
The results obtained from this study show that, in agreement with experimental work, the
structure and topology of the shock-induced densified phase is unique in its structure as can be
revealed by medium-range order measurements. The modifications include a reduction of the
average tetrahedra size and an increase in the proportion of 3-4 and 8-10 membered Si-rings. 
%%------------------------------------------------------------------------------------------------------
%%	Hug
%%------------------------------------------------------------------------------------------------------
Application of a Hugoniostat method based on constraint dynamics shows near-perfect agreement
with the NEMD results. Besides validating the former method, this opens the prospect of studying
shock-induced effects at a fraction of the cost required to run large scale shock simulations
while using much more complicated potentials and setups.\\
\end{abstract}




\maketitle

%%=========================================================================================================================
%% Section : Introduction
%%=========================================================================================================================
\section{Introduction}


%%------------------------------------------------------------------------------------------------------
%%	Shock waves damaging lenses in high power laser experiments
%%------------------------------------------------------------------------------------------------------
High power lasers are used to achieve extreme temperature and pressure conditions. During such
experiments, deterioration of the optics due to the passage of the laser has been observed in
the form of small craters at the back of the lenses~\cite{FeitCampbell97,FeitHrubesh00}. Is is
believed that these craters originate from the absorption of the laser by defects which create a
dense absorbing plasma.  The induced temperature rise in turn triggers the propagation of a
mechanical shock wave creating regions of permanently densified glass~\cite{DemosKoslowski00}.\\
%%
%%------------------------------------------------------------------------------------------------------
%%	results on shock through SiO2
%%------------------------------------------------------------------------------------------------------
Experimental work on shock compression shows the creation of a densified
phase upon propagation of shock wave through silica glass~\cite{SugiuraIkeda96}.
Kubota~\etal~\cite{KubotaCaturla01,KubotaCaturla03} have used molecular 
dynamics to propagate shock waves with up to $2.0$ \mrm{Km/s} piston velocity through slabs
of up to $240\e{3}$ atoms. The authors show drastic shock induced modifications in the structure
and topology of the \SiOTwo network. It has also been shown that the structure of the resulting
densified phase is unique as can be revealed by medium-range order measurements such as the ring
size distribution~\cite{SusmanVolin91,TrachenkoDove02,DavilaCaturla03}.\\ 
%%
%%------------------------------------------------------------------------------------------------------
%%	Hugoniostat simulations
%%------------------------------------------------------------------------------------------------------
Despite the major increase in the available computer power which allows simulation of several
hundred million atoms, non-equilibrium molecular dynamics (NEMD) simulations of shocked materials
still represent a computationally major undertaking. This is mainly due to the large system
sizes needed to achieve a steady shock front; additionally a non-negligible part of the total
simulation time is spent calculating the trajectories of particles in the unshocked region. 
Another NEMD method has been used in~\cite{ZhakhovskiiZybin99} where the simulation box
consists of a moving window which follows the shock front. The method achieves accurate description
of the shock induced effects provided steady shock waves are obtained.
Alternatively, to address the system size problem, several groups have attempted to construct an
alternative method where equilibrium simulations are performed on bulk systems in the Hugoniot
ensemble. Such a method 
allows usage of much smaller systems to achieve results equivalent to those obtained with NEMD.
The first such method (GNVHug) was proposed by Soulard~\cite{Soulard99} and uses the Gauss
principle of least constraints to integrate the equations of motions and constrain the system on
the Hugoniot. This method presents the double advantage of reaching equilibrium very rapidly
($\mathcal{O}(10^{-13})\mrm{s}$) and guaranteeing that the system obeys the Hugoniot relations
at all times. The method was applied to the simulation of shocked liquid argon and
nitromethane~\cite{Soulard99,Soulard01} and has shown good agreement with NEMD simulations. An
alternative approach was proposed by Maillet~\etal~\cite{MailletMareschal00} which uses a
Nos\'e-Hoover like method to couple the system, instantaneously, uniaxially and homogeneously
compressed to the shock density, with a thermostat to achieve the final Hugoniot state. Applied
to the shock deformation of Lennard-Jones crystals along the $\langle 100 \rangle$ direction,
the method has shown to reproduce both the 
Hugoniot curve and the shock induced defect structure. This method was later extended to
\textit{ab initio} molecular-dynamics to obtain the shock Hugoniot of tin up to 200
GPa~\cite{BernardMaillet02}. Answering the criticism that the instantaneous compression of the 
system can lead to unrealistically large temperature and stresses,
Ravelo~\etal~\cite{RaveloHolian04} improved the uniaxial Hugoniostat by adding an strain-rate
dynamical variable which acts as a piston. The method allows gradual compression of the system
to the shock density and therefore a more natural evolution of both the  temperature and stress.
Applied to the calculations of the Hugoniot of Lennard-Jones crystals equivalent to those
performed in~\cite{MailletMareschal00}, the method improves the agreement with NEMD, especially
for large compressions.\\  
%%
%%------------------------------------------------------------------------------------------------------
%%	In this paper ...
%%------------------------------------------------------------------------------------------------------
In this paper, we seek to model the shock induced structural changes in silicon dioxide at the
atomic scale. To this end, after validating that our model achieves an accurate description of
the glassy state of silicon dioxide, we use NEMD simulations to characterize precisely the structure
induced by a mechanical shock wave. We then apply the GNVHug method and show this gives very good agreement
with the NEMD results. The remainder of this paper is organized as follows. In
Section~\ref{s:potential} we present the potential used to describe the inter-atomic
interactions in silicon dioxide. In
Section~\ref{s:bulk}, we describe the methods and results  obtained to create and analyze a
bulk system of silicon dioxide in the glass phase in order to check our potential leads to an
appropriate structure before shocking it. In Section~\ref{s:shock}, we turn on to the study of
the structural changes induced by the propagation of a mechanical shock wave using NEMD simulations, while
Section~\ref{s:hug} concentrates on obtaining similar results with the Hugoniostat. 
Section~\ref{s:CCL} concludes this work with a discussion of the results drawn from this work
and proposes some directions for future developments.

%%=========================================================================================================================
%% Section : Model
%%=========================================================================================================================
\section{The intermolecular potential}
\label{s:potential}

Accurate description of silicate glass is not in principle an easy task. Ideally the
inter-atomic potential should include Coulombic interactions, and an appropriate method  to take the long-range forces into account (\eg
Ewald summation) and also include dynamic charges.\\
%%------------------------------------------------------------------------------------------------------
%%	The main potentials
%%------------------------------------------------------------------------------------------------------
Several intermolecular potentials that successfully describe the behavior of silicon dioxide in
the glass phase have been built over the years~\cite{Limoge01}. Among the most widely used, are
the so-called BKS potential developed by van Beest~\etal~\cite{vanBeestKramer90} and the
three body potential of Vashishta~\etal~\cite{VashishtaKalia90} which we will refer
to as the VKRE potential hereforth. We choose to use the BKS potential, as it
allows for faster simulation than the more complicated three-body form of the VKRE
potential while leading to similar results; the three body information necessary to
create the tetrahedra network in silicon dioxide is not lost with the BKS potential, but is 
implicitly present by the choice of the potential parameters derived from the cluster model of
van Beest~\etal~\cite{vanBeestKramer90}.\\ 

%%------------------------------------------------------------------------------------------------------
%%	Our BKS potential
%%------------------------------------------------------------------------------------------------------
The BKS potential is a two-body potential based on the Buckingham potential with an added Coulombic
term. The potential describes the
interaction between two atoms $i$ and $j=\{\mrm{Si},\mrm{O}\}$ separated by $\vect{r}_{ij} =
r_{ij}\vecth{r}_{ij}$, the interatomic vector as:
%%
%%
\begin{equation}
\mathcal{V}^{\mrm{BKS}} =  \frac{A_0q_i q_j}{r_{ij}} + A_{ij}e^{-B_{ij}r_{ij}} -\frac{C_{ij}}{r^6_{ij}}
\label{eqn:BKS}
\end{equation}
with $A_0 = 1/(4\pi\epsilon_0) = 2.307\e{-28}$ \mrm{J/m}. Values for the $A_{ij},B_{ij}$ and $C_{ij}$
parameters are taken from~\cite{vanBeestKramer90} and recalled here in Table~\ref{tble::BKS_par}.\\
%%
%%

\begin{table}
\centering
\begin{tabular}{|cccc|}
\hline
	&$A_{ij}$ \mrm{(J)}		&$B_{ij}$ \mrm{(m^{-1})}	&$C_{ij}$ \mrm{(Jm^6)}\\
\hline
O-O		&2.225\e{-16}		&2.760\e{10}				&2.804\e{-77}	\\
Si-O	&2.884\e{-15}		&4.873\e{10}				&2.139\e{-77}	\\
Si-Si	&0.0				&0.0						&0.0			\\
\hline
\end{tabular}
\caption{Numerical values of the BKS potential parameters as given by van Beest~\etal~\cite{vanBeestKramer90}}
\label{tble::BKS_par}
\end{table}

%%
%%
\picW=8cm
\begin{figure}
\picL{fig01.ps}
\caption{Graphical representation of the inter-atomic potential. The dashed lines
represent the unphysical behavior at short inter-atomic separations of the original BKS potential.}
\label{fig:potBKS}
\end{figure}
%%

Buckingham-like potentials have the unphysical property of diverging to minus infinity at
small interatomic distances which can be disastrous at high temperatures where atoms
have the kinetic energy to overcome the potential barrier and fuse together. Several methods
have been employed to correct for this behavior. Vollamyr~\etal~\cite{VollmayrKob96} replace the
BKS potential with a harmonic form for $r^0_{\mrm{SiO}} \leq 1.1936\, \angstrom$ and $r^0_{\mrm{OO}} \leq
1.439\, \angstrom$. Another method is that of Guissani and Guillot~\cite{GuissaniGuillot95} or Coluzzi and
Verrocchio~\cite{ColuzziVerrocchio02} where a 18-6 and 24-6 Lennard-Jones potential are
respectively added to the BKS form. Here we choose to follow the former method and replace the
BKS potential with a second order polynomial $p(r)$ for $r<r^0$.\\
We also make the assumption that the structure of \SiOTwo is essentially
controlled by the interactions within the shell of first neighbors and as a result we cut and shift
the potential for $r>r_c$. This assumption comes from the argument that since silicate glass is
essentially a random continuous network of \SiOTwo tetrahedra, controlling the shape of an individual
tetrahedra and its relative position with respect to its first neighbors is sufficient to recover
medium and long-range order.\\
%%
%%------------------------------------------------------------------------------------------------------
%% The long-range effect you don't expect
%%------------------------------------------------------------------------------------------------------
It should be noted that the effects on the structural and dynamic properties of cutting the
Coulombic part of the potential are different than those described by
Jund~\etal~\cite{JundRarivo00}. Here we simply cut and shift the potential for
$r>r_c$ but keep all contributions to the potential, including the Coulombic term, below the
cutoff distance which is 3.6 times greater than the average Si-O distance. Jund~\etal, on the other hand,
studied the  effect of gradually removing the Coulombic part of the potential at \emph{all
distances}. A quantitative study on the effects of the long range forces on the stucture of
silicate glasses is under way and will be the subject of a future publication~\cite{HalversonBarmes07}.
In the light of the preliminary results obtained (not shown here), we are however confident that
the long-range forces effect is negligible in the structural properties of silicate glass.


Preliminary results (not shown here) however seem to confirm our approximation for the stuctural
properties.\\ 


The final form of the  potential used here is expressed as:
%%
%%
\begin{equation}
\mathcal{V}= 
	\left\{
	\begin{array}{cc}
	p(r) = a_{ij}r^2 + b_{ij}r + c_{ij}					&r_{ij} \leq r^0_{ij}\\
	%
	\mathcal{V}^{\mrm{BKS}}_c 					 		& r^0_{ij} < r_{ij} \leq r_c\\
	%
	0													& r_{ij} > r_c
	\end{array}
	\right.
\label{eqn::BKSFB}
\end{equation}
%%
with 
\begin{equation*}
\mathcal{V}^{\mrm{BKS}}_c = \mathcal{V}^{\mrm{BKS}}(r) - \mathcal{V}^{\mrm{BKS}}(r_c)
-(r-r_c)\frac{d\mathcal{V}^{\mrm{BKS}}(r_c)}{dr},
\end{equation*}
%%
the cut and shifted version of the potential. We set $r_c=6\,\angstrom$, $r^0_{\mrm{SiO}}=1.5\,\angstrom$ and
$r^0_{\mrm{OO}}=2.0\,\angstrom$. The coefficients of $p(r)$ are calculated so that the zero,
first and second order derivatives of $\mathcal{V}^{\mrm{BKS}}_c$ and $p(r)$ are equal. The
values used for the polynomial coefficients are given in Table~\ref{tble:poly_coef}. A graphical
representation of this potential is given in Figure~\ref{fig:potBKS}. 

\begin{table}
\centering
\begin{tabular}{|c|c|c|c|}
\hline
		&$a_{ij}$ \mrm{(J/m^{2})}		& $b_{ij}$ \mrm{(J/m)}		&$c_{ij}$ \mrm{(J)}	\\
\hline
O-O		&1.510\e{2}						&-7.925\e{-8}					& 1.100\e{-17}	\\
Si-O	&3.413\e{2}  	 				&-9.361\e{-8}					& 3.925\e{-18}	\\
Si-Si	&0.0			   				&0.0							& 0.0			\\
\hline
\end{tabular}
\caption{Numerical values used for the coefficient of the polynomial functions.}
\label{tble:poly_coef}
\end{table}


%%=========================================================================================================================
%%	Section : Bulk behavior
%%=========================================================================================================================
\section{Bulk behavior of \SiOTwo}
\label{s:bulk}

\subsection{Sample preparation method}

%%------------------------------------------------------------------------------------------------------
%% A few words about silicate glass structure 
%%------------------------------------------------------------------------------------------------------
The structure of pure silica in the glass phase is that of a disordered network of corner
sharing tetrahedra where each silicon is surrounded by four oxygens. The structure of vitreous
silica is well known from experimental
work~\cite{MozziWarren69,Bruckner70,JohnsonWright83,Wright94}.
%%------------------------------------------------------------------------------------------------------
%% General method - cooling with high rate
%%------------------------------------------------------------------------------------------------------
The general method for creating a sample of glassy \SiOTwo for use in molecular simulation involves
gradual cooling of a liquid phase past the liquid to glass transition
point~\cite{VashishtaKalia90,HorbackKob99}. The cooling rate is an important parameter as
it has been shown that the resulting glass structure is heavily dependent on its creation
history~\cite{VollmayrKob96}. Still, even the slowest cooling rates achievable with today's
computers are several orders of magnitude greater than the fastest cooled experimental glasses;
as a result much higher glass transition temperatures are
expected~\cite{VollmayrKob96,YuanCormack01,ColuzziVerrocchio02}. 
%%------------------------------------------------------------------------------------------------------
%% Details of our method
%%------------------------------------------------------------------------------------------------------
In this study, we have used a method analog to that of~\cite{VashishtaKalia90} to create a
system in the glass phase. The method starts from a $\beta$-cristobalite crystal~\cite{Wickoff} whose 
density is set to $\rho=2.2\, \mrm{g/cm^{3}}$, the  typical density of glassy
\SiOTwo~\cite{Weast}. The initial system used here consists of $N=1536$ atoms organized in
$4\times4\times4$ crystal cells along the $x,y$ and $z$ directions.
This initial configuration is melted at high temperature  ($T=6\e{3}\mrm{K}$) until all
memory from the crystal structure is lost. The fluid phase is then cooled down to $T=300 \mrm{K}$ using 
a series of 13 simulations grouped in sets of three for thermalization, equilibration and
production. Each simulation was 25 ps in length using a timestep $\delta t=0.5\e{-15}$ s. During
the equilibration and production simulations, the temperature was kept constant while during the
thermalisation simulation, the system was subjected to a temperature gradient where the
temperature of the heat bath was decreased every hundred time steps.
%%
%% temperature gradient : 40e9 K/s
%%

\subsection{Analysis}

%%------------------------------------------------------------------------------------------------------
%% What we analyse
%%------------------------------------------------------------------------------------------------------
The structure of the glass thus created was assessed by calculating short, medium and long range
structural properties. The long range structure of the glass is obtained
through calculation of the pair correlation function as shown in Figure~\ref{fig:mkGlass_gr}.
Short to medium range order is obtained using bond-angle distribution (BAD) and ring size
analysis. 
While we have calculated all six BADs, we restrict ourselves on
discussing only the OSiO, SiOSi and SiSiSi angles distribution (which we will refer to as
$\theta_\mrm{OSiO},\theta_\mrm{SiOSi}$ and $\theta_\mrm{SiSiSi})$ as these carry the most
information about the structure of the network but also because only $\theta_\mrm{OSiO}$ and
$\theta_\mrm{SiOSi}$ are available with experimental techniques.  Comparison of the results
obtained here with those from both experimental and other simulation studies shows good
agreement.\\  

\picW=8cm
\begin{figure}
\picL{fig02.ps}
\caption{Representation of the pair correlation functions for the Si-Si(top), Si-O(middle) and
O-O(bottom) pairs in the glass phase at $T=300K$.}
\label{fig:mkGlass_gr}
\end{figure}

\picW=8cm
\begin{figure}
\picL{fig03.ps}
\caption{Bond angle distribution obtained for the Si-O-Si($\blacksquare$), O-Si-O($\blacktriangle$) and Si-Si-Si($\bullet$) angles from
simulation of the bulk system at both high(left) and low(right) temperatures.}
\end{figure}

%%------------------------------------------------------------------------------------------------------
%% g(r) : long range order and alpha-beta distances
%%------------------------------------------------------------------------------------------------------
Measurements of the g(r) first peak and the location of the bond length distribution main peak (not
shown here) shows that for the O-O, Si-O and Si-Si pairs we obtain respectively
$r_\mrm{OO}=2.639\,\angstrom$, $r_\mrm{SiO}=1.640\,\angstrom$ and $r_\mrm{SiSi}=3.067\,\angstrom$. 
This compares well with the experimental values 
$r_\mrm{OO}=2.63\,\angstrom$ and $r_\mrm{SiO}=1.61\,\angstrom$~\cite{MozziWarren69,JohnsonWright83} but also
with simulation data~\cite{YuanCormack01,VashishtaKalia90}. Using the BKS potential, Yuan and
Cormack~\cite{YuanCormack01} found $r_\mrm{OO}=2.621\,\angstrom$, $r_\mrm{SiO}=1.615\,\angstrom$ and
$r_\mrm{SiSi}=3.129\,\angstrom$ while Vashishta~\etal~\cite{VashishtaKalia90} obtained
$r_\mrm{OO}=2.65\,\angstrom$, $r_\mrm{SiO}=1.62\,\angstrom$ and $r_\mrm{SiSi}=3.05\,\angstrom$ using their
three-body potential.\\

%%------------------------------------------------------------------------------------------------------
%% O-Si-O angle : tetrahedra structure
%%------------------------------------------------------------------------------------------------------
The structure of the tetrahedra is measured by the $\theta_\mrm{OSiO}$ distribution. For a network of perfect
tetrahedra, the ideal $\theta_\mrm{SiOSi}$ is $109.47^\circ$ close to the  experimental value
of $109.7^\circ$~\cite{GrimleyWright90}. Simulation data usually reproduce this property well with
a value of $108.6^\circ$\cite{JundRarivo00,YuanCormack01}. Similarly we obtain $108.9^\circ$.\\
%%------------------------------------------------------------------------------------------------------
%% Si-O-Si and Si-Si-Si
%%------------------------------------------------------------------------------------------------------
Measurement of the $\theta_\mrm{SiOSi}$ distribution gives information of the structure beyond the
basic tetrahedral configuration. While variations exist, experimental
data~\cite{MozziWarren69,NemilovKhim82,CoombsDeNatale85,Galeener85}  suggest a
value of $144^\circ$ for the $\theta_\mrm{SiOSi}$ distribution peak. Our value of $152^\circ$ clearly
overestimates this, but this seems to be a feature intrinsic to the BKS potential, as the 
value of $152^\circ$ and $\sim 150^\circ$ were reported by other simulation
studies~\cite{YuanCormack03,JundRarivo00,JundJulien99}. According
to~\cite{YuanCormack03}, this can be explained by the lack of Si-O-Si interactions in the cluster
model used to derive the BKS potential. In Table~\ref{tble:mkGlass}, we present a summary of the
above comparisons.\\

%%------------------------------------------------------------------------------------------------------
%% comparison summary
%%------------------------------------------------------------------------------------------------------
\begin{table}
\centering
\begin{tabular}{|c|c|c|c|c|c|}
\hline
& $r_{\mrm{OO}}[\angstrom]$		&$r_{\mrm{SiO}}[\angstrom]$	&$r_{\mrm{SiSi}}[\angstrom]$ &$\theta_\mrm{OSiO}$	& $\theta_\mrm{SiOSi}$	\\
\hline
this study								&2.639	&1.640	&3.067	& $108.9^\circ$		&$152^\circ$		\\
\cite{MozziWarren69,JohnsonWright83}	&2.63	&1.61	&		&					&$144^\circ$		\\
\cite{GrimleyWright90}					&		&		&		&$109.47^\circ$		&					\\
\cite{YuanCormack01,YuanCormack03}		&2.621	&1.615	&3.129	&$108.6^\circ$		&$152^\circ$		\\
\cite{VashishtaKalia90}					&2.65	&1.62	&3.05	&					&					\\
\hline
\end{tabular}\\
\caption{Comparison of the structural properties of the silicate glass obtained with this study and published
simulation~\cite{YuanCormack01,YuanCormack03,VashishtaKalia90} and
experimental~\cite{MozziWarren69,JohnsonWright83,GrimleyWright90} results.}
\label{tble:mkGlass}
\end{table}




%%------------------------------------------------------------------------------------------------------
%% Measuring the rings
%%------------------------------------------------------------------------------------------------------
An alternative method to analyse medium range order is the calculation of the n-sized rings
distributions. This measures the topology of the \SiOTwo network over distances that extend to
several Si-O bond distances. Ring size statistics is considered as the generally accepted
measure of medium range order in random-continuous network such as those formed by silicate
glasses~\cite{WeaireWooten80,TadrosKlenin85,GrenecheTeillet85,LuedtkeLandman89}. This 
allows the identification of topologically different structures. For instance,  the ring size
distribution is similar for $\alpha$ and $\beta$ cristobalite or $\alpha$ and $\beta$ quartz.
The distributions however differ  between the two crystal structures, quartz and cristobalite
showing respectively single peaks at sizes 6 and 8.\\
%%------------------------------------------------------------------------------------------------------
%%	General idea
%%------------------------------------------------------------------------------------------------------
In silicon dioxide glass, rings are closed paths where the vertices and edges are respectively
the Si atoms and the Si-O-Si bonds. Measurement of the ring size distribution is performed using shortest
path analysis~\cite{WrightDesa78} which involves identification of
the shortest-path linking each triplet of first neighbors Si atoms. The number of triplets to consider is
related to the Si atom coordination; a four-coordinated Si atom for instance belongs to six different rings.
The difficulty of the ring size analysis lies in defining a unique criteria able to identify the
primitive rings, \ie those rings which are not formed by the combination of smaller rings. It has been
shown that simply counting all rings up to a given size leads to meaningless
results~\cite{WeaireWooten80,TadrosKlenin85,LuedtkeLandman89}. The number of rings 
increasing monotonously with ring size as small rings are counted several times as part of
bigger rings. A correct ring analysis must therefore identify exclusively primitive rings.\\


%%------------------------------------------------------------------------------------------------------
%% Our method - building to connectivity
%%------------------------------------------------------------------------------------------------------
In this study we choose to use the criteria based on those proposed by
Franzblau~\cite{Franzblau91} and Marians and Hobbs~\cite{MariansHobbs90} for the definition of a
primitive ring.\\
The first step is to build the network connectivity. The common approach to this is to count as
a bond all distances shorter than a given threshold. As we plan to study highly
densified shocked phases, this method can however lead to an unrealistically large number of
bonds and therefore we have chosen an alternative method.\\
A neighbor list is built for each atom to
identify all \mrm{Si_i-O_j} bonds. Care must be taken that this method is reversible, \ie
the path \mrm{Si_1-O_2-Si_2} must exist both in the \mrm{Si_1} and \mrm{Si_2} neighbor
lists.
A successful method to build those lists is to loop over all \mrm{O_i} and find \mrm{Si_{i,1}}
and \mrm{Si_{i,2}}, the two closest \mrm{Si} neighbors. \mrm{O_i} then registers itself in the
neighbor list of \mrm{Si_{i,1}} and \mrm{Si_{i,2}}. Although this method assumes
that each oxygen has a twofold coordination, it can be shown that the
connectivity obtained with this method is consistent with the network topology and allows to
recover an accurate Si coordination.\\
%%
%%------------------------------------------------------------------------------------------------------
%% Our method - candidate rings
%%------------------------------------------------------------------------------------------------------
The next step is to identify all candidate rings. This is
done by triple looping over all \mrm{Si_i} atoms and all their first \mrm{Si_{1}} and
\mrm{Si_{2}} neighbors. For each (\mrm{Si_i},\mrm{Si_{1}}) pairs, we build a tree of all
successive neighbors of \mrm{Si_1} down to the eleventh level and identify the position of
\mrm{Si_2}. The candidate rings are the \mrm{Si_i}-\mrm{Si_1}-\ldots-\mrm{Si_2} closed paths in
the tree.\\ 
%%
%%------------------------------------------------------------------------------------------------------
%% Our method - SP criteria
%%------------------------------------------------------------------------------------------------------
Each candidate ring is then tested against our shortest-path criteria. This criteria follows the
definition given by Franzblau~\cite{Franzblau91}. A n-member ring is considered as primitive if
by following the \mrm{Si_0},\mrm{Si_1},\ldots,\mrm{Si_i},\ldots,\mrm{Si_n} atoms which form the
ring, the \mrm{Si_0-Si_i} distance sequence is unimodal. This criteria allows to remove all rings whose
shape is not convex such as eight-shaped or tennis racket-shaped rings. The shortest ring
isolated as such is counted in the statistics.\\  

\picW=8cm
\begin{figure}
\subfigure[]{\picL{fig04a.ps}}
\subfigure[]{\picL{fig04b.ps}}
\caption{Ring size distribution measured from the bulk simulations at high(a) and low(b)
temperatures. The points ($\blacksquare$) represent the results obtained by Rino and
Ebbsj\"o~\cite{RinoEbbsjo93} at $T=2500$K and $T=310$K using the VKRE potential.}
\label{fig:mkGlass_SPA}
\end{figure}

The method has been tested against both  $\beta$ and $\alpha$-cristobalite crystals and lead to
a distribution with only 6-membered rings. Figure~\ref{fig:mkGlass_SPA} shows the ring size
distribution obtained from the bulk simulations at both high and low temperatures along with a
comparison with the results obtained by Rino and Ebbsj\"o~\cite{RinoEbbsjo93} using the VKRE
model. Unfortunately they did not provide any details regarding the shortest path criteria used
in their study. Save for a discrepancy for 8-membered rings at high temperature,
Figure~\ref{fig:mkGlass_SPA} shows good agreement between the two studies.



%%=========================================================================================================================
%%	Section : Shock induced structural changes
%%=========================================================================================================================

\section{\SiOTwo under shock conditions}
\label{s:shock}

\subsection{Method}


%%------------------------------------------------------------------------------------------------------
%%	The system
%%------------------------------------------------------------------------------------------------------
The analysis of shock-induced structural changes have been investigated using a non-equilibrium
molecular dynamics method. The initial system of size $N=30\e{3}$ atoms consisted of
$50\times5\times5$
$\beta$-cristobalite cells along the $x,y$ and $z$ directions.  Periodic boundary conditions were
applied along the $y$ and $z$ axes while free surfaces were used along $x$. The
method described in the previous section was applied on this initial configuration to obtain a
glass phase. All shock simulation started from a well equilibrated glass system at $T=300$K.\\

%%------------------------------------------------------------------------------------------------------
%% Simulations procedure
%%------------------------------------------------------------------------------------------------------
At $t=t_0$, the system was hit by two pistons represented by $(y,z)$ planes of infinite mass on
the left and right hand surfaces. The left piston has a velocity $u \neq 0$ along positive $x$ while
the piston on the right-hand side had a zero velocity. Displacement of the left-hand side piston
induced the propagation of a mechanical shock wave through the sample whose velocity $D$ is a
function of $u$ and of the equation of state of the system. The shock wave divides the system in
two parts: the non-shocked region, which is located downstream of the shock wave, is referred by the
subscript $0$ while the shocked region, located upstream of the shock wave, is referred by the 
subscript $1$. The aim here is to analyse the shock-induced structural changes in the shocked
region.\\ 


\picW=8cm
\begin{figure}
\picL{fig05.ps}
\caption{Evolution of the shock velocity ($D$) as a function of
piston velocity ($u$) as obtained from the non-equilibrium shock simulations.  The dotted line
represents the linear fit of the simulation data. Experimental data are taken from~\cite{SugiuraKondo81}}
\label{fig:shock_D-u}
\end{figure}



\subsection{Analysis}

%%------------------------------------------------------------------------------------------------------
%% What we analyse
%%------------------------------------------------------------------------------------------------------
Analysis of the shocked region is performed using a moving analysis window whose size and position
changes with time so as to match that of region 1. The dimensions of this 
window are calculated using the profile of particle velocity along $x$, the 
shock propagation direction.The shock front is located using the point of maximum inflexion on
the profile. The shocked and non-shocked regions are located respectively up and downstream of
this point and separated by a transition region where the system undergoes the
near-instantaneous transition from unshocked to shocked states. This transition region is
identified using a numerical algorithm that analyses local variations in the profile and excluded
from the analysis window. The thermodynamic, structural and topological properties are then
computed independently in each regions.\\ 

%%In order to exclude this transition region from the analysis window, we take region 1 to extend
%%from the position of the piston to the last local-maximum upstream of the shock front. Likewise,
%%the region 0 is taken to extend from the first local minimum downstream of the shock front  to
%%the right hand system surface.


\picW=8cm
\begin{figure}
\subfigure[]{\picL{fig06a.ps}}
\subfigure[]{\picL{fig06b.ps}}
\subfigure[]{\picL{fig06c.ps}}
\caption{Shock-induced modifications of the $\theta_\mrm{OSiO}$(a), $\theta_\mrm{SiOSi}$(b) and
$\theta_\mrm{SiSiSi}$(c) bond angle
distributions. In the main graphs the dashed curves represent the distributions of the
unshocked materials and the bold line the distributions for the highest shock velocities
($v_p=4.5$ \mrm{ Km/s}). 
The solid lines represent the distributions for 
$v_p = 1,2,3.5,4,4.5$  \mrm{Km/s}.
The insert shows a comparison between the high shock velocity distribution (bold line) and those
obtained from an unshocked bulk system with similar temperature (T=2000K). }
\label{fig:shock_BAD}
\end{figure}


%%------------------------------------------------------------------------------------------------------
%% D-u
%%------------------------------------------------------------------------------------------------------
Measurement of the shock front position as a function of time and for each piston velocity allows
to measure $D(u)$ as shown on Figure~\ref{fig:shock_D-u}.
In the plastic regime, the curve obtained here is very similar to that
of~\cite{KubotaCaturla03}. For $D>1$ \mrm{Km/s} the system undergoes a transformation from
an elastic to plastic regime where $u$ is a linear function of $D$. In the latter regime, we
observe shock induced structural changes and a permanent densification of the system.
Additionally, our curve compares well with the experimental data obtained by
Sugiura~\etal~\cite{SugiuraKondo81} using flier plate experiment.\\
We however note a difference between our data and those of~\cite{KubotaCaturla03} in the elastic
regime. This is a consequence of our analysis algorithm which uses the velocity 
profile to locate the shock front position. In the elastic regime, the small changes induced to
the velocity profile lead to a very noisy measurement of the shock front position as a function
of time; hence the difference in the shock velocity. This behavior disappeared however as
the plastic regime, which is of interest here, is entered.\\ 

%%------------------------------------------------------------------------------------------------------
%% BADs
%%	-> OSiO : smaller tetrahedra
%%	-> SiOSi : more small rings
%%	-> SiSiSi : also more big rings
%%------------------------------------------------------------------------------------------------------
Figure~\ref{fig:shock_BAD} shows the shock-induced modifications of the bond angle
distributions. Upon increasing the shock velocity, the distributions gradually change from that
of an unshocked material to that of a heavily shocked material. For
the $\theta_\mrm{OSiO}$ distribution, this involves displacement of the main peak to lower angles, indicating
a reduction of the average tetrahedra size which is compatible with the shock-induced
densification. The stronger the shock, the more dense the phase and therefore the smaller the
average tetrahedra. This is consistent with experimental observation of the variation of
$\theta_\mrm{OSiO}$ under pressure~\cite{DevineArndt87,DevineDupree87}.\\
%%
%%
Variations of the $\theta_\mrm{SiOSi}$ distribution obeys to the same rule, stronger shock waves
induce smaller angles which corresponds to a denser phase. We however note the creation of a
local maxima at $\theta_{\mrm{SiOSi}}=105.30^\circ$ for intermediate shock strengths. Again, the
same phenomenon has been observed experimentally~\cite{HemleyMao86,OkunoReynard99}.
%%
%%
The $\theta_\mrm{SiSiSi}$ distributions of the unshocked material shows two peaks at
$\theta^1_{\mrm{SiSiSi}}=60.30^\circ$ and $\theta^2_{\mrm{SiSiSi}}=103.50^\circ$ with a much higher
probability for $\theta^2_{\mrm{SiSiSi}}$
($\theta^2_{\mrm{SiSiSi}}/\theta^1_{\mrm{SiSiSi}}=12.03$). In the shocked materials, while the 
positions of the peaks shift only slightly to lower and greater angles for respectively
$\theta^1_{\mrm{SiSiSi}}$ and $\theta^2_{\mrm{SiSiSi}}$, their ratios nears one for the fastest
shock velocities used here. The shift of the $\theta^1_{\mrm{SiSiSi}}$ peak to lower
angles is understandable from the densification of the materials, which induces the
inter-tetrahedra angle to decrease. The shift of the $\theta^2_{\mrm{SiSiSi}}$ peak to higher
angles is however counter-intuitive as this corresponds to an opening of the inter-tetrahedra
angle, which is not \emph{a priori} compatible with the phase densification. In terms of the
system's topology, the shifts to lower angles of the $\theta^1_{\mrm{SiOSi}}$ and
$\theta^1_{\mrm{SiSiSi}}$ distribution peaks suggests an increase of small ring while the shift of
$\theta^2_{\mrm{SiSiSi}}$ to higher angles also suggests an increase in the proportion of large 
rings.\\
%%
%%
The inserts of Figure~\ref{fig:shock_BAD} compare the distributions obtained for the fastest
shock velocity used here with those obtained from bulk simulations at a temperature equal to that found in
the shocked region. This shows that the structure of the shocked state
is much different than that of a high temperature system. The modifications observed here
are therefore induced by the propagation of the shock wave and not merely the consequence of the
associated temperature increase. The $\theta_\mrm{OSiO}$ distribution suggests a reduction of the average tetrahedra
size  whereas the modifications of the $\theta_\mrm{SiOSi}$ and $\theta_\mrm{SiSiSi}$
distributions suggest an increase in the proportion of both small and large rings.\\

\picW=8cm
\begin{figure}
\subfigure[]{\picL{fig07a.ps}}
\subfigure[]{\picL{fig07b.ps}}
\caption{Ring size distribution in the shocked materials. Subfigure (a) compares the
distribution in the unshocked region with those in the shocked region with slow and fast piston
velocity. Subfigure(b) represents the ring size proportion in the shocked 
region for different $u$ divided by the proportion in the unshocked region.}
\label{fig:shock_rings}
\end{figure}

%%------------------------------------------------------------------------------------------------------
%%	Rings
%%	-> small and big ring increase
%%	-> increase in same proportions
%%	-> IMPROVE A BIT, REMOVE REPETITIONS
%%------------------------------------------------------------------------------------------------------
In order to further assess the medium-range structure modification induced by the shock wave, we
compare the ring size histograms in the shocked and unshocked regions.
In agreement with the results of~\cite{KubotaCaturla01}, Figure~\ref{fig:shock_rings}(a)
confirms the results suggested by the bond angle distributions. 
As the shock velocity increases, the ring size distribution in the shocked regions flattens out
suggesting a structural reorganization which translates into an increase of both small (3-4) and
large (7-10) rings. Observation of the ratio of the rings size proportion in the shocked region
to that in the unshocked region shows that the relative increase is similar for both sizes.
While the increase in the proportion of small rings is easily understandable considering the
shock induced densification of the material, the increase of large ring is less so. One would
assume that reducing the average ring size and the average $\theta_\mrm{SiOSi}$ would favor
small rings only. The increase in the proportion of large ring can be related to
the $\theta_\mrm{SiSiSi}$ distribution. This indicates that the system favors two
configurations with inter-tetrahedra angle of  about $60^\circ$ and $100^\circ$. The low angle
peak accounts for the small rings while the high angle peak accounts for large rings.
Experimentally, it has been shown that the presence of 3 and 4 membered Si rings in silicate
glass can be associated to the so-called \mrm{D_2} and \mrm{D_1} peaks respectively in Raman
spectra~\cite{PasquarelloCar98}. The shock induced intensity increase of the \mrm{D_2} and \mrm{D_1} peaks in
the Raman spectra of Okuno~\etal~\cite{OkunoReynard99} thus confirms our simulation data.
Unfortunately, a similar connection can not be made for large rings as they can not be precisely
isolated experimentally.

%%=========================================================================================================================
%%	Section : Hugoniostat
%%=========================================================================================================================
\section{Modeling \SiOTwo using the Hugoniostat}
\label{s:hug}

\subsection{Method}

As an alternative method to non-equilibrium molecular dynamics which involves the simulation of
large systems, we seek here to simulate the shock induced structural changes using bulk
simulations in the Hugoniot ensemble. The advantage of using such a method is that it
requires fewer atoms ($\mathcal{O}(10^{3})$) which allows to run much longer simulations and
achieve better statistics in shorter simulation times. Here we use the GNVHug Hugoniostat of
Soulard~\cite{Soulard99} as it allows to reach equilibrium very quickly and obeys the Hugoniot
relations at all times. The method has been applied successfully to the
computation of the Hugoniot in the solid and liquid phases; the behavior of the method in the
glassy state remains unknown however. According to its original phase, a material uses very
different thermodynamic paths in the near-instantaneous transition to the shocked state and therefore
successful application of the GNVHug method in one phase does not translate to another. We
therefore also aim to clarify the issue of the applicability of the GNVHug method to glassy phases.\\
%%
The GNVHug has been described  in details in previous publications~\cite{Soulard99,Soulard01}.
We recall here the general idea.\\ 

A cubic system whose size $L$ and volume $V$ are allowed to vary with time is considered with
three dimensional periodic boundary conditions applied. The system consists of $N$ atoms $i$ 
with individual masses $m_i$ and $E_p$ and $E_k$ the total potential and kinetic energies. Using the Gauss
principle~\cite{hoo01,eva01ST,mor03ST}, the time evolution of the system is such that 
%%
\begin{equation}
\frac{1}{2}\sum_{i=1}^{N}\left( m_{i}\left[ \frac{f_{i}}{m_{i}}-\frac{d^2r_i}{dt^2}\right]^{2} \right)  \label{Eq2}
\end{equation}
%%
with $f_{i}=-\partial E_p/\partial r_i$, is minimum at all time.\\

Given a volume $V$, the system is constrained on the thermodynamic hypersurface $H\left(
E,P\right)$ which corresponds to a Hugoniot state at thermodynamic equilibrium:
%%
\begin{eqnarray}
H &=&E-\frac{1}{2}P\left( V_{0}-V\right) \nonumber \\ &=& E_{0}+\frac{1}{2}P_{0}\left(
V_{0}-V\right)  \label{Eq3} 
\end{eqnarray}
with $P_{0}$, $E_{0}$ and $V_{0}$ the initial pressure, energy and volume and $P$ and $E$
the pressure and the energy behind the shock. The pressure is defined using the virial
expression: 
%%
\begin{equation}
P=\frac{Nk_{B}T}{V}+\frac{\Phi}{3V}  \label{Eq4b}
\end{equation}
where $\Phi$ is the virial and $k_{B}$ the Boltzmann constant.\\

Since $V$ is taken to be constant (\ie $\frac{dV}{dt}=0)$ we get:
\begin{equation}
\frac{dE}{dt}-\frac{1}{2}\frac{dP}{dt}\left( V_{0}-V\right) =0  \label{Eq4}
\end{equation}

Combining Equations \ref{Eq2} and \ref{Eq4} \textit{via} a Lagrange multiplier $\lambda $, the
following equation of motions is obtained:
%%
%%
\begin{eqnarray}
\frac{d^2r_i}{dt^2}&=&\frac{f_i}{m_i}-\lambda\frac{dr_i}{dt}\label{Eq5}\\
\lambda&=&\frac{V-V_0}{4V-V_0}\frac{\frac{1}{2}\frac{d\Phi}{dt}+\sum_i^Nf_{i}\frac{dr_i}{dt}}{2E_k}.\label{Eq6}
\end{eqnarray}\\
%%


The first stage of the Hugoniot calculation involves the determination of $P_{0}$ and $E_{0}$ by
means of standard molecular dynamics simulations in the canonical ensemble.  
The GNVHug algorithm is then started by scaling the system to a chosen volume $V$. The
initial positions of atoms  are deduced from the initial structure and the appropriate scale factor
$\left(V/V_0\right)^{(1/3)}$. Knowing the virial and the potential energy, the relevant
kinetic energy is calculated so as to satisfy Equation~\ref{Eq3}. 
If a negative value is obtained, the system is relaxed for a few hundred time steps until the
kinetic energy becomes positive.



\subsection{Results}


In order to test the GNVHug method, a glass system was prepared using the method described
in Section~\ref{s:bulk}. The initial configuration was a $\beta$-cristobalite crystal of
$N=1536$ atoms ($4\times4\times4$ crystal cells) with three dimensional periodic boundary
conditions applied. All simulations in the Hugoniot ensemble started from the same, well
equilibrated glass system. The pole parameters ($E_0$,$P_0$,$V_0$) which are required as input
parameters for the simulations were averaged over the last production simulation of the glass
preparation procedure. Shock densities in the range $\rho_1\in[3:5]\, \mrm{ g/cm^{3}}$ were used.\\ 

\picW=8cm
\begin{figure}
\subfigure[]{\picL{fig08a.ps}}
\subfigure[]{\picL{fig08b.ps}}
\caption{Comparisons of the thermodynamic properties obtained with the NEMD ($\square$) and
Hugoniostat ($\bullet$) methods. }
\label{fig:cmp_thedy}
\end{figure}



%%------------------------------------------------------------------------------------------------------
%%	What we measure - how
%%------------------------------------------------------------------------------------------------------
The first comparison between the NEMD and GNVHug methods involves the calculation of the shock wave
velocity as a function of piston velocity ($D(u)$) and the calculation of the Hugoniot equation
of state ($P_1(\rho_1)$). Direct comparison between the two methods is however not possible as
some parameters of one method (\eg $D$) are not directly accessible in the second. We therefore
make use of the Hugoniot relations to be able to compare the results obtained from the two
methods. From the fundamental Hugoniot relations~\cite{Thouvenin} one can derive:
\begin{eqnarray}
	\rho_1	&=& (\rho_0D)/(D-u)	\label{eqn:Hug_1}\\
	P_1		&=&	\rho_0Du + P_0	\label{eqn:Hug_2}\\
	D		&=&	\rho_0^{-1}\sqrt{(P_1-P_0)/(\rho_0^{-1} - \rho_1^{-1})}	\label{eqn:Hug_3}\\
	u		&=& \sqrt{(P_1-P_0)(\rho_0^{-1}-\rho_1^{-1})}					\label{eqn:Hug_4}
\end{eqnarray}
%%
%%
Using Equations~\ref{eqn:Hug_1}-\ref{eqn:Hug_4}, Figure~\ref{fig:cmp_thedy} shows the comparison
of $D(u)$ and $P_1(\rho_1)$ obtained using the two methods. This shows that the thermodynamic
quantities are in very close agreements in the plastic regime. The elastic regime is in principle
also available to the Hugoniot method as it obeys the Hugoniot relations. However, for small
density increments, the  shock-induced effects are reversible and the shocked state differs
little from the unshocked-state; hence $H\sim 0$. Constraining the system to $\frac{\partial H}{\partial
t}=0$ is therefore a numerically challenging  task which explains why the method fails to match
the NEMD results in the elastic regime. This does not question the efficiency of the method, but
is merely a numerical flaw of the glassy systems used.
Additionally, as an amorphous glassy phase is studied here, it is assumed that anisotropic effects are
negligible and therefore we have only computed the pressure and not the shear stress as the
approximation of isostatic should hold.\\



\picW=4cm
\begin{figure}
\subfigure[equivalence (A)]{\picL{fig09a}}
\subfigure[equivalence (B)]{\picL{fig09b}}

\subfigure[equivalence (C)]{\picL{fig09c}}
\subfigure[equivalence (D)]{\picL{fig09d}}
\caption{Comparison of the bond angle distributions measured using the NEMD and the Hugoniostat method
for several equivalent densities and shock velocities in the plastic regime. The points
represent the Hugoniostat results while the lines and histograms show the shock data.}
\label{fig:cmp_BAD}
\end{figure}


\picW=4cm
\begin{figure}
\subfigure[equivalence (A)]{\picL{fig10a}}
\subfigure[equivalence (B)]{\picL{fig10b}}

\subfigure[equivalence (C)]{\picL{fig10c}}
\subfigure[equivalence (D)]{\picL{fig10d}}
\caption{Comparison of the ring-size distributions measured using the NEMD and the Hugoniostat method
for several equivalent densities and shock velocities in the plastic regime. The points
represent the Hugoniostat results while the lines and histograms show the shock data.}
\label{fig:cmp_ring}
\end{figure}


%%\picW=8cm
%%\begin{figure}
%%\subfigure[]{\picL{FIGS/cmp_BAD.ps}}
%%\subfigure[]{\picL{FIGS/cmp_SPA.ps}}
%%\caption{Comparison of the structural quantities measured using the NEMD and Hugoniostat method
%%for several equivalent densities and shock velocities in the plastic regime. The points
%%represent the Hugoniostat results while the lines and histograms show the shock data. In both graphs the
%%equivalence A-D are shown left to right and top to bottom ($\frac{A|B}{C|D}$).}
%%\label{fig:cmp_struct}
%%\end{figure}


Figure~\ref{fig:cmp_BAD} and~\ref{fig:cmp_ring}  show the comparison of the structural properties between the NEMD
and Hugoniot methods. Using the results of Figure~\ref{fig:cmp_thedy}, the following parameters
are taken to lead to equivalent conditions:
\begin{eqnarray*} 
\mrm{(A):}\rho=3.6\, \mrm{ g/cm^{3}} &\leftrightarrow& v_p=2\, \mrm{ Km/s}\\
\mrm{(B):}\rho=4.1\, \mrm{ g/cm^{3}} &\leftrightarrow& v_p=3\, \mrm{ Km/s}\\
\mrm{(C):}\rho=4.6\, \mrm{ g/cm^{3}} &\leftrightarrow& v_p=4\, \mrm{ Km/s}\\
\mrm{(D):}\rho=4.8\, \mrm{ g/cm^{3}} &\leftrightarrow& v_p=5\, \mrm{ Km/s}\\
\end{eqnarray*}
%%
Computation of both the bond angle distributions and the ring size distributions show very
close agreement in the structures obtained with the equilibrium and non-equilibrium
simulations. For simulations equivalent to piston velocities greater than $3\,\mrm{ Km/s}$,
the two sets of results become virtually indistinguishable. This shows that past the plastic
regime limit, the shock-induced structural properties are available using the equilibrium
method.\\

The main advantage of using the GNVHug method here lies in the speed with which the results were
obtained. Starting from a $\beta$-cristobalite crystal, the creation of the (large) glass sample
and the shock simulations required more than a month of computation using parallel simulations
with six processors. Using  the same resources and single processor calculations,
the Hugoniostat method required a mere week of simulation to recover the same results. Besides
the clear financial advantage of 
using short simulations, this opens up the prospect of using more realistic methods and
potentials. The small system sizes and three dimensional periodicity associated with the GNVHug
method make usage of long-range Coulombic interactions with dynamic charges and many-body
potentials a standard exercise. Application of a similar method with direct shock propagation,
while possible, is much more complicated. The system size and complicated boundaries
require more complicated techniques for treatment of the long-range forces (\ie Fast
Multipole Method) in addition to large computational overheads.\\ 



%%=========================================================================================================================
%%	Section : Conclusion
%%=========================================================================================================================
\section{Conclusion}
\label{s:CCL}

%%------------------------------------------------------------------------------------------------------
%%	What we did
%%------------------------------------------------------------------------------------------------------
In this paper we have used atomistic simulations to model the shock-induced structural changes in silicate glass.
%%
%%------------------------------------------------------------------------------------------------------
%%	Long range not needed ?
%%------------------------------------------------------------------------------------------------------
Our model is a modified version of the BKS potential where the long-range part of the
interaction is cut and shifted, based on the assumption that the glass structure
is mainly dictated through interactions within the first neighbor shell. We show that the glass
produced with this potential is structurally equivalent to those generated using the BKS
potential with long range interaction and the three-body VKRE potential. In addition, our 
structural parameters are consistent with those found in experimental studies. This
therefore suggests the validity of the assumption made here. The short range interactions are
responsible for the average tetrahedra size and shape as well as their relative positions. The
glass being a continuous random network of connected tetrahedra, the long range order is
controlled by the cumulative effects of setting the individual tetrahedra structure and their
relative positions with respect to their immediate neighbors.\\
%%
%%------------------------------------------------------------------------------------------------------
%%	NEMD results
%%------------------------------------------------------------------------------------------------------
We have subsequently used non-equilibrium molecular dynamics to propagate mechanical
shock waves through our glassy system. Comparison of the structures measured in the shocked and
unshocked regions reveal that, above the plastic regime limit, the shock wave induces
profound and irreversible structural changes. These are manifested by the reduction
in the average tetrahedra size associated with the material densification. The relative
positions of the tetrahedra is distributed with equal probabilities among two configurations
where the angle made by connected tetrahedra equals $60^\circ$ or $103^\circ$. Analysis of the ring size
distributions shows that this is connected with an increase in the proportion of both 3-4 and
7-10 membered rings. Thus, in addition to the material densification manifested by smaller
tetrahedra and narrow inter-tetrahedra angles, the shock wave also stabilizes the formation of
large rings in the network created by wide inter-tetrahedra angles.\\ 
%%
%%------------------------------------------------------------------------------------------------------
%%	GNVHug
%%------------------------------------------------------------------------------------------------------
We have then applied the GNVHug Hugoniostat method to perform equilibrium bulk
simulations of the shock-induced structural changes. The method is shown to reproduce well the
$D(u)$ and $P_1(\rho_1)$ curves in the plastic regime. Analysis of the structures shows very good
agreement with those obtained using NEMD simulations; the correspondence improving with
increased shock wave velocities. As the Hugoniostat method allows use of much smaller systems,
these results open up the prospect of achieving modeling of the shock wave induced effects using
complicated potentials at a fraction of the cost required to run large scale shock simulations.\\
%%
%%------------------------------------------------------------------------------------------------------
%%	What now
%%------------------------------------------------------------------------------------------------------
This will form the basis of future work where the Hugoniostat method will be used to explore
more complicated potentials including long-range forces and dynamics charges but also more
complicated setups in an attempt to obtain a more realistic model of the mechanisms underlying
damage of the optics by high-power lasers. Additionally, the results obtained from the bulk
simulation will be completed by a more refined study to quantify the influence of long range
forces on the structural properties of the glass comparing the glass structure obtained with and without
treatment of the long range forces.

%%=========================================================================================================================
%%	Bibliography
%%=========================================================================================================================
\begin{thebibliography}{10}

\bibitem{FeitCampbell97}
M.D. Feit, J.H. Campbell, D.R. Faux, F.Y. Gening, M.R. Kozlowski, A.M.
  Rubenchik, R.A. Riddle, A.~Salleo, and J.~Yoshiyama.
\newblock In {\em SPIE proceedings}, volume 3244, page 350, 1997.

\bibitem{FeitHrubesh00}
M.D. Feit, L.W. Hrubesh, A.M. Rubenchik, and J.N. Wong.
\newblock In {\em SPIE proceedings}, volume 4347, page 316, 2000.

\bibitem{DemosKoslowski00}
S.G. Demos, M.R. Koslowski, M.~Staggs, L.L. Chase, A.~Burnham, and H.B.
  Radousky.
\newblock In {\em Laser Induced damage in optical materials}, 2000.

\bibitem{SugiuraIkeda96}
H.~Sugiura, R.~Ikeda, K.~Kondo, and T.~Yamadaya.
\newblock {\em Journal of Applied Physics}, 81:1651, 1996.

\bibitem{KubotaCaturla01}
A.~Kubota, M.-J. Caturla, J.S. Stolken, and M.D. Feit.
\newblock {\em Optics Express}, 8:611, 2001.

\bibitem{KubotaCaturla03}
A.~Kubota, M.-J. Caturla, J.S. Stolken, B.~Sadigh, A.~Quong, A.~Rubenchik, and
  M.D. Feit.
\newblock In {\em SPIE proceedings}, 2003.

\bibitem{SusmanVolin91}
S.~Susman, K.J. Volin, D.L. Price, M.~Grimsditch, J.P. Rino, R.K. Kalia,
  P.~Vashishta, G.~Gwanmesia, Y.~Wang, and R.C. Liebermann.
\newblock {\em Physical Review B}, 43:1194, 1991.

\bibitem{TrachenkoDove02}
K.~Trachenko and M.~Dove.
\newblock {\em Journal of Physics : Condensed Matter}, 14:7449, 2002.

\bibitem{DavilaCaturla03}
L.P. D\'avila, M.J. Caturla, A.~Kubota, B.~Sadigh, T.D. de~la Rubia, J.F.
  Shackelford, S.H. Risbud, and S.H. Garofalini.
\newblock {\em Physical Review Letters}, 91:205501, 2003.

\bibitem{ZhakhovskiiZybin99}
V.V. Zhakhovski\u{\i}, S.V. Zybin, K.~Nishihara, and S.I. Anisimov.
\newblock {\em Physical Review Letters}, 83:1175, 1999.

\bibitem{Soulard99}
L.~Soulard.
\newblock In {\em Shock Compression of Condensed Matter}, 1999.

\bibitem{Soulard01}
L.~Soulard.
\newblock In {\em Shock Compression of Condensed Matter}, 2001.

\bibitem{MailletMareschal00}
J.-B. Maillet, M.~Mareschal, L.~Soulard, R.~Ravelo, P.S. Lomdahl, T.C. Germann,
  and B.L. Holian.
\newblock {\em Physical Review E}, 63:016121, 2000.

\bibitem{BernardMaillet02}
S.~Bernard and J.-B. Maillet.
\newblock {\em Physical Review B}, 66:012103, 2002.

\bibitem{RaveloHolian04}
R.~Ravelo, B.L. Holian, T.C. Germann, and P.S. Lomdahl.
\newblock {\em Physical Review B}, 70:014103, 2004.

\bibitem{Limoge01}
Y.~Limoge.
\newblock {\em C.R. Acad. Sci. Paris t. 2}, 4:263, 2001.

\bibitem{vanBeestKramer90}
G.J.~Kramer B.W.H.~van Beest and R.A. van Santen.
\newblock {\em Physical Review Letters}, 64:1955, 1990.

\bibitem{VashishtaKalia90}
J.P.~Rino P.~Vashishta, R.K.~Kalia and I.~Ebbsj\"o.
\newblock {\em Physical Review B}, 41:12197, 1990.

\bibitem{VollmayrKob96}
W.~Kob K.~Vollmayr and K.~Binder.
\newblock {\em Physical Review B}, 54:15808, 1996.

\bibitem{GuissaniGuillot95}
Y.~Guissani and B.~Guillot.
\newblock {\em Journal of Chemical Physics}, 104:7633, 1996.

\bibitem{ColuzziVerrocchio02}
B.~Coluzzi and P.~Verrocchio.
\newblock {\em Journal of Chemical Physics}, 116:3789, 2002.

\bibitem{JundRarivo00}
P.~Jund, M.~Rarivomanantsoa, and R.~Julien.
\newblock {\em Journal of Physics : Condensed Matter}, 12:8777, 2000.

\bibitem{HalversonBarmes07}
J.D. Halverson and F.~Barmes.
\newblock {}.
\newblock {\em in preparation}, 2007.

\bibitem{MozziWarren69}
R.L. Mozzi and B.E. Warren.
\newblock {\em Journal of Applied Physics}, 2:164, 1969.

\bibitem{Bruckner70}
R.~Br\"uckner.
\newblock {\em Journal of Non-Crystalline Solids}, 5:123, 1970.

\bibitem{JohnsonWright83}
P.A.V. Johnson, A.C. Wright, and R.N. Sinclair.
\newblock {\em Journal of Non-Crystalline Solids}, 58:109, 1983.

\bibitem{Wright94}
A.C. Wright.
\newblock {\em Journal of Non-Crystalline Solids}, 179:84, 1994.

\bibitem{HorbackKob99}
J.~Horbach and W.~Kob.
\newblock {\em Physical Review B}, 60:3169, 1990.

\bibitem{YuanCormack01}
X.~Yuan and A.N. Cormack.
\newblock {\em Journal of Non-Crystalline Solids}, 283:69, 2001.

\bibitem{Wickoff}
W.G. Wickoff.
\newblock {\em Crystal structures 2nd ed.}
\newblock Wiley, 1963.

\bibitem{Weast}
R.C.~Weast ed.
\newblock {\em Handbook of chemistry and physics}.
\newblock The Chemical Rubber Co., 1970.

\bibitem{GrimleyWright90}
D.J. Grimley, A.C. Wright, and R.C. Sinclair.
\newblock {\em Journal of Non-Crystalline Solids}, 119:49, 1990.

\bibitem{NemilovKhim82}
S.V. Nemilov and Fiz. Khim.
\newblock {\em Stekla}, 8:385, 1982.

\bibitem{CoombsDeNatale85}
P.G. Coombs, J.F.~De Natale, P.J. Hood, D.K. McElfresh, R.S. Wortman, and J.F.
  Shackelford.
\newblock {\em Philos. Mag. B}, 51:L39, 1985.

\bibitem{Galeener85}
F.L. Galeener.
\newblock {\em Philos. Mag. B}, 51:L1, 1985.

\bibitem{YuanCormack03}
X.~Yuan and A.N. Cormack.
\newblock {\em Journal of Non-Crystalline Solids}, 319:31, 2003.

\bibitem{JundJulien99}
P.~Jund and R.~Julien.
\newblock {\em Phil. Mag. A}, 39:37, 1999.

\bibitem{WeaireWooten80}
{ D. Weaire and F. Wooten }.
\newblock {\em Journal of Non-Crystalline Solids}, 35-36:495, 1980.

\bibitem{TadrosKlenin85}
{ A. Tadros, M.A. Klenin and G. Lucovsky }.
\newblock {\em Journal of Non-Crystalline Solids}, 75:407, 1985.

\bibitem{GrenecheTeillet85}
J.M. Greneche, J.~Teillet, and J.M.D. Coey.
\newblock {\em Journal of Non-Crystalline Solids}, 83:27, 1985.

\bibitem{LuedtkeLandman89}
{ W.D. Luedtke and U. Landman }.
\newblock {\em Physical Review B}, 40:1164, 1989.

\bibitem{WrightDesa78}
{A.C. Wright and J.A.E. Desa}.
\newblock {\em Phys. Chem. Glasses}, 19:140, 1978.

\bibitem{Franzblau91}
{ D.S. Franzblau }.
\newblock {\em Physical Review B}, 44:4925, 1991.

\bibitem{MariansHobbs90}
{ C.S. Marians and L.W. Hobbs }.
\newblock {\em Journal of Non-Crystalline Solids}, 124:242, 1990.

\bibitem{RinoEbbsjo93}
{J.P. Rino, I. Ebbsj\"o, R.K. Kalia, A. Nakano and P. Vashishta}.
\newblock {\em Physical Review B}, 47:3053, 1993.

\bibitem{SugiuraKondo81}
H.~Sugiura, K.~Kondo, and A.~Sawaoka.
\newblock {}.
\newblock {\em Journal of Applied Physics}, 52:3375, 1981.

\bibitem{DevineArndt87}
R.A.B. Devine and J.~Arndt.
\newblock {}.
\newblock {\em Physical Review B}, 35:9376, 1987.

\bibitem{DevineDupree87}
R.A.B. Devine, R.~Dupree, I.~Farnan, and J.J. Capponi.
\newblock {}.
\newblock {\em Physical Review B}, 35:2560, 1987.

\bibitem{HemleyMao86}
R.J. Hemley, H.K. Mao, P.M. Bell, and B.O. Mysen.
\newblock {}.
\newblock {\em Physical Review Letters}, 57:747, 1986.

\bibitem{OkunoReynard99}
M.~Okuno, B.~Reynard, Y.~Shimada, Y.~Syono, and C.~Willaime.
\newblock {}.
\newblock {\em Phys. Chem. Minerals}, 26:304, 1999.

\bibitem{PasquarelloCar98}
A.~Pasquarello and R.~Car.
\newblock {}.
\newblock {\em Physical Review Letters}, 80:5145, 1998.

\bibitem{hoo01}
W.~G. Hoover.
\newblock {\em Molecular Dynamics}.
\newblock Springer-Verlag, 1986.

\bibitem{eva01ST}
D.J. Evans, W.G. Hoover, B.H. Failor, B.~Moran, and A.J.C. Ladd.
\newblock {\em Physical Review A}, 28(2):1016, 1983.

\bibitem{mor03ST}
G.P. Morris and C.P. Dettmann.
\newblock {\em Chaos}, 8(2):321, 1998.

\bibitem{Thouvenin}
J.~Thouvenin.
\newblock {\em D\'etonique}.
\newblock Eyrolles, 1997.

\end{thebibliography}


\end{document}

