\documentclass[aps,pre,twocolumn,groupedaddress,showpacs]{revtex4}

%============================================================================
%		packages
%============================================================================
\usepackage[final]{graphics}	% for graphics
\usepackage[final]{graphicx}	% for graphics
\usepackage{subfigure}		% allow subfigure
\usepackage{amssymb}		% font
\usepackage{amstext}		% ams latex package
\usepackage{amsmath}		% font
\usepackage{amsfonts}		% font
\usepackage{amsbsy}		% font
\usepackage{hhline}		% nice lines is tables
\usepackage{xspace}		% add space at the end of macros
\usepackage{color}

%============================================================================
%		Macros - figures
%============================================================================
\newlength{\picH}	% picture height
\newlength{\picW}	% picture width
\newcommand{\picA}{270}	% picture angle

\picW = 8cm
\newcommand{\picB}[1]{\fbox{\includegraphics[width=\picW]{#1}}}
\newcommand{\picLB}[1]{\fbox{\includegraphics[height=\picW, angle=\picA]{#1}}}

\newcommand{\pic}[1]{\includegraphics[width=\picW]{#1}}
\newcommand{\picL}[1]{\includegraphics[height=\picW, angle=\picA]{#1}}

\newlength{\htab}
\htab=1mm
\newcommand{\hsp}{\hspace*{\htab}}

\newlength{\vtab}
\vtab=3mm
\newcommand{\vsp}{\vspace*{\vtab}}

\hyphenation {qua-dru-pole mol-e-cules thermal-ly}

%============================================================================
%============================================================================
\begin{document}

\graphicspath
{
{../imgs/}
}


%============================================================================
%	Title infos
%============================================================================
\title{Symmetric alignment of the nematic matrix \\
between close penetrable colloidal particles}


\author{P.I.C. Teixeira}
\affiliation{Faculdade de Engenharia, Universidade Cat\'{o}lica Portuguesa,
Estrada de Tala\'{\i}de, P-2635-631 Rio de Mouro, Portugal}


\author{F.Barmes and D.J. Cleaver}

\affiliation{Materials Research Institute, Sheffield Hallam University \\
Pond Street, Sheffield S1 1WB, United Kingdom}


\date{25 November 2003}


\pacs{61.30Gd, 61.30Cz}

%============================================================================
%	Abstract
%============================================================================
\begin{abstract}
A simple model is proposed for the liquid crystal matrix surrounding 
`soft' colloidal particles whose separation is much smaller than their 
radii. We use our implementation of the Onsager approximation of 
density-functional theory [A. Chrzanowska, P. I. C. Teixeira, 
H. Ehrentraut and D. J. Cleaver, J. Phys.: Condens.\ Matter {\bf 13}, 
4715 (2001)] to calculate the structure of a nanometrically thin film
of hard Gaussian overlap particles of elongations $\kappa=3$  and
$\kappa=5$, confined between two solid walls. The penetrability of 
either substrate can be tuned independently to yield symmetric or 
hybrid alignment. Comparison with Monte Carlo simulations of the same 
system [D. J. Cleaver and P. I. C. Teixeira, Chem.\ Phys.\ Lett.\ 
{\bf 338}, 1 (2001); F. Barmes and D. J. Cleaver (unpublished)] reveals 
good agreement in the symmetric case.

\vspace{1cm}
\end{abstract}

%============================================================================
%	Make title
%============================================================================
\maketitle



%============================================================================
%============================================================================
\section{Introduction}
\label{sect-intro}


A {\it nematic colloid}, sometimes called {\it inverted nematic emulsion}, 
is a dispersion of isotropic particles, either rigid (grains) or deformable 
(e.g., water droplets), in a nematic liquid crystal (LC) \cite{Stark:2001}.
Because the nematic director is anchored on the surface of the (nearly always 
quasi-spherical) inclusions, distortions of the director field are induced 
which in turn give rise to effective interactions between the inclusions 
themselves. These can be manipulated by treating the walls of the container, 
applying fields, or coating the surfaces of the colloidal particles, in order
to produce various states of aggregation, e.g., strings along the director
lines \cite{Poulin:1997}. Nematic colloids are therefore ideal model systems 
to study topological defects, and one important practical application appears
to be the suspension of abrasive particles in lyotropic mesophases. 
In a newer variant, inclusions of size exceeding the cholesteric pitch (radius
$\sim 1$ $\mu$m) are dispersed in a well-aligned cholesteric sample. Here the 
colloidal particles stabilise a network of defects by residing at its nodes, 
thereby transforming the cholesteric liquid into a new type of material 
exhibiting gel-like rheological properties \cite{Zapotocky:1999}. 
This study was later extended to the simpler system consisting of smaller 
(radii $150-200$ nm) particles in a nematic host: upon quenching the initial 
homogeneously-mixed colloid from the isotropic (I) to the nematic (N) state, 
the particles were seen to phase-separate and aggregate into thin walls, 
bounding domains of practically pure nematic LC \cite{Val}. The 
resulting metastable, but relatively long-lived, cellular structure exhibited 
dramatically enhanced mechanical strength, with an elastic modulus $G^{\prime}
\ge 10^5$ Pa and a well-defined yield stress, which are functions of particle
concentration. 
\par
In all the above, the colloidal particles are much larger than the LC 
molecules, which can then be regarded as a continuum background. Effective 
interaction potentials between colloidal particles have been derived, 
analytically in some limits, more generally numerically, using either 
Frank theory (below the I--N transition) \cite{Poulin:1997,Terentjev} 
or Landau-de Gennes (LdG) theory (above the I--N transition) \cite{Galatola}; 
more recent work considered Van der Waals, steric, elctrostatic and 
LC-mediated contributions \cite{Stark:2000}. However, continuum-based 
approaches are expected to fail when the interparticle separation 
becomes of the order of a few LC molecular sizes, as in fully-formed 
aggregates. Furthermore, especially in the case of LdG theory, the solution 
method is heavy, and affords little insight into the physics of the problem. 
Finally, evaluation of the three- and more-particle contributions
(these effective potentials being in general not pairwise additive) 
becomes prohibitively complicated \cite{Galatola}: one would like to 
be able to obtain the different terms as functions of the microscopic 
potential parameters, and in a systematic, well-controlled manner. 
\par
The complexity of the task at hand requires that we start our attack from 
the very beginning: take the simplest molecular model of a LC and squeeze 
it to nanometric thickness; then find the free energy dependence on particle
separation, and hence the effective interaction. Here we carry out only 
the first part of this programme, leaving the second for future work.
In a previous paper \cite{Chrzanowska:2001} we showed how the simple  
Onsager approximation of density-functional theory could provide a 
semi-quantitatively accurate description of the structure of a fluid
of hard rods confined between two hard, impenetrable walls, provided 
allowance was made, in a phenomenological way, for the incorrect  
prediction of the location of the isotropic--nematic (I--N) transition. 
In the present paper we apply the same strategy to symmetric films confined 
between flat substrates of variable penetrability, in order to mimic different 
anchoring conditions. This is not unreasonable as a first approach, 
in view of the large disparity of the typical lengthscales of LC 
molecules -- a few nm -- and inclusions -- hundereds of nm (but 
see \cite{Groh} for an attempt to consider curved hard surfaces).
\par
This paper is organised as follows: in section \ref{sect-theory} we 
recapitulate the theory of \cite{Chrzanowska:2001} and extend it to
the case of unequal anchorings at the confining walls. Then in section 
\ref{sect-results} we present results for the density and order parameter 
profiles of LC films subject to symmetrical anchoring conditions, and 
compare them with those obtained by Monte Carlo (MC) simulation 
\cite{Cleaver:2001,Barmes:2003}. Finally in section \ref{sect-concl} 
we discuss the potential and limitations of our approach, and outline some 
directions for future research. 


%============================================================================
%============================================================================
\section{Theory}
\label{sect-theory}

As before, we take as a representation of uniaxial rod-like particles
the hard Gaussian overlap (HGO) potential used previously in 3d bulk 
simulations and Onsager theory of lyotropic LC behaviour 
\cite{Kike:1997,deMiguel:2001} (i.e., where the LC phase transitions 
are driven by density, rather than temperature, changes): 

\begin{equation}
\label{eq:gbp}
U_{12}({\bf r}_{12},\omega_1,\omega_2) = \left\{\begin{array}{ll}
0 &\mbox{if $r_{12} \geq \sigma(\hat{\bf r}_{12},\omega_1,\omega_2)$} \\
\infty & \mbox{if $r_{12} < \sigma(\hat{\bf r}_{12},\omega_1,\omega_2)$}
\end{array} \right. ,
\end{equation}

where $\omega_i=(\theta_i,\phi_i)$ are the polar and azimuthal angles 
describing the orientation of the long axis of particle $i$, and 
$\hat{\bf r}_{12}={\bf r}_{12}/r_{12}$ is a unit vector along the line
connecting the centres of the two particles. The range parameter function 
is given by
%%
%%
\begin{multline}
\label{eq:closest}
%
\sigma(\hat{\bf r}_{12},\omega_1,\omega_2) = 
\sigma_0  \left[ 1 - \frac{1}{2}\chi
\left\{ { {(\hat{\bf r}_{12}\cdot \hat u_1 +\hat{\bf r}_{12}\cdot\hat u_2)^2} \over{1+\chi(\hat u_1\cdot\hat u_2)}}+
\right. \right.\\
\left. \left.
{{(\hat{\bf r}_{12}\cdot\hat u_1 -\hat{\bf r}_{12}\cdot\hat u_2)^2} \over{1-\chi(\hat u_1\cdot\hat u_2)}} \right\} 
\right]^{-{1\over 2}},
\end{multline}
%%
%%
where $\hat u_i = (\cos\phi_i\,\sin\theta_i,\sin\phi_i\,
\sin\theta_i,\cos\theta_i)$ and $\chi = (\kappa^2 - 1)/(\kappa^2 + 1)$,
$\kappa$ being the particle length to breadth ratio, $\sigma_L/ \sigma_0$.
For moderate $\kappa$, the HGO model is a good approximation to the
hard ellipsoid (HE) contact function \cite{Perram:19XX,Allen:1993}; 
furthermore, their virial coefficients (and thus their equations of state,
at least at low to moderate densities) are very similar \cite{Steele:1987}. 
However, this is no longer true of highly non-spherical particles
\cite{Rigby}, for which the behaviour of the two models differs appreciably
\cite{deMiguel:2001}: here we restrict ourselves to $\kappa=3$ and $\kappa=5$.
Finally, HGOs have the considerable computational advantage over HEs that 
$\sigma(\hat{\bf r}_{12},\omega_1,\omega_2)$, the distance 
of closest approach between two particles, is given in closed form.

Particle--substrate interactions have been modelled, as in 
\cite{Cleaver:2001,Barmes:2003}, by a hard needle--wall potential
%%
%%
\begin{equation}
\label{eq:surf}
U_{wall}(z,\theta) = \left\{ \begin{array}{ll}
0 & \mbox{if $\left|z-z_0 \right| \geq
\frac{L}{2}\cos\theta$}\\
\infty & \mbox{if $\left|z-z_0\right| < \frac{L}{2}\cos\theta$}      
\end{array} \right.
\end{equation}
%%
%%
where the $z$-axis has been chosen to be perpendicular 
to the substrate, located at $z_0$, and $0\le L\le \sigma_L$
sets the length of the needle with which the substrate interacts. Here, $L$
affords us a degree of control over the anchoring properties: physically, 
$0< L< \sigma_L$ corresponds to a system where the molecules are able to 
embed their side groups, but not the whole length of their cores, into the
bounding walls. Varying $L$ between 0 and $\sigma_L$ is therefore equivalent 
to changing the degree of penetrability of the substrates in an experimental
situation, e.g., by manipulating the density or the orientation of an 
adsorbed surface layer. This can be done independently at either substrate,
whereby symmetric or hybrid anchoring conditions can be obtained. 
\par
The equilibrium density distribution of a HGO film is that which 
minimises its grand-canonical functional \cite{Evans:1979}:
%%
%%
%\begin{eqnarray}
%\label{eq:grand}
%\beta\Omega\left[\rho({\bf r},\omega)\right]&=&
%\beta {\cal F}\left[\rho({\bf r},\omega)\right]
%+\beta\int \left[\sum_{\alpha=1}^2 U_{wall}(\theta,|z-z_0^{\alpha}|)
%-\mu\right]\rho({\bf r},\omega)\,d{\bf r}d\omega \nonumber \\
%&=&\int \rho({\bf r},\omega)\left[\log \rho({\bf
%r},\omega) -1\right]\,d{\bf r}d\omega\nonumber \\
%& &\mbox{}-{1\over 2}\int\rho({\bf r}_1,\omega_1)
%f_{12}({\bf r}_1,\omega_1,{\bf r}_2,\omega_2)
%\rho({\bf r}_2,\omega_2)\,d{\bf r}_1 d\omega_1 d{\bf r}_2 d\omega_2
%\nonumber \\
%& &\mbox{}+\beta\int \left[\sum_{\alpha=1}^2
%U_{wall}(|z-z_0^{\alpha}|,\theta)
%-\mu\right]\rho({\bf r},\omega)\,d{\bf r}d\omega,
%\end{eqnarray}
%%
%%
\begin{widetext}
\begin{eqnarray}
\label{eq:grand}
\beta\Omega\left[\rho({\bf r},\omega)\right]&=&
\beta {\cal F}\left[\rho({\bf r},\omega)\right]
%%
%%
+\beta\int \left[\sum_{\alpha=1}^2 U_{wall}(\theta,|z-z_0^{\alpha}|)
-\mu\right]\rho({\bf r},\omega)\,d{\bf r}d\omega\nonumber\\
%%
%%
&=&\int \rho({\bf r},\omega)\left[\log \rho({\bf
r},\omega) -1\right]\,d{\bf r}d\omega\nonumber
-{1\over 2}\int\rho({\bf r}_1,\omega_1)
f_{12}({\bf r}_1,\omega_1,{\bf r}_2,\omega_2)
\rho({\bf r}_2,\omega_2)\,d{\bf r}_1 d\omega_1 d{\bf r}_2 d\omega_2\\
%%
&+&\beta\int \left[\sum_{\alpha=1}^2
U_{wall}(|z-z_0^{\alpha}|,\theta)
-\mu\right]\rho({\bf r},\omega)\,d{\bf r}d\omega,
\end{eqnarray}
\end{widetext}
%%
%%
where ${\cal F}\left[\rho({\bf r},\omega)\right]$ is the intrinsic
Helmholtz free energy of the inhomogeneous fluid, $f_{12}({\bf 
r}_1,\omega_1,{\bf r}_2,\omega_2)=\exp\left[-\beta U_{12}({\bf
r}_1,\omega_1,{\bf r}_2,\omega_2)\right]-1$ is its Mayer function, 
$\mu$ is the chemical potential, $z_0^{\alpha}$ ($\alpha=1,2$) are the 
positions of the two substrates, and, because we are dealing with hard-body
interactions only, for which the temperature is an irrelevant variable, we
can set $\beta=1/k_{\rm B}T=1$ in all practical calculations (we retain it
in the formulae for generality). $\rho({\bf r},\omega)$ is the
density-orientation profile in the presence of the external potential
$U_{wall}(z,\theta)$; it is normalised 
to the total number of particles $N$,
%%
%%
\begin{equation}
\label{eq:rhonorm}
\int \rho({\bf r},\omega)\,d{\bf r}d\omega = N,
\end{equation}
%%
%%
and is related to the probability that a particle positioned at 
${\bf r}$ has orientation between $\omega$ and $\omega+d\omega$
From equation (\ref{eq:gbp}) it follows
that the interaction term in equation (\ref{eq:grand}) is just the
excluded volume of two HGO particles, weighted by the density-orientation
distributions $\rho({\bf r},\omega)$. This equation constitutes the
Onsager approximation to the free energy of the confined HGO fluid.
\par
Because the particle-substrate interaction, equation (\ref{eq:surf}), 
only depends on $z$ and $\theta$, it is reasonable to assume that there 
is no in-plane structure, so that all quantities are functions of $z$ only. 
Then equation (\ref{eq:grand}) simplifies to 
%%
%%
\begin{multline}
\label{eq:grandz}
%%
{{\beta\Omega\left[\rho(z,\omega)\right]}\over{S_{xy}}}=
\int \rho(z,\omega)\left[\log \rho(z,\omega) -1\right]\,dz 
d\omega\\
%%
%%
-{1\over 2}\int\rho(z_1,\omega_1)
\Xi(z_1,\omega_1,z_2,\omega_2)\\
%%
%%
\times\rho(z_2,\omega_2)\,dz_1 d\omega_1 dz_2 d\omega_2
\\
%%
%%
\mbox{}+\beta\int \left[\sum_{\alpha=1}^2
U_{wall}(|z-z_0^{\alpha}|,\theta)
-\mu\right]\rho(z,\omega)\,dz d\omega,
\end{multline}
%%
%%
where $S_{xy}$ is the interfacial area. 
$\Xi(z_1,\omega_1,z_2,\omega_2)$ is now the area of a slice
(cut parallel to the bounding plates) of the excluded volume of two HGO
particles of orientations $\omega_1$ and $\omega_2$ and centres at $z_1$
and $z_2$ \cite{Poniewierski:1993}, for which an analytical expression has 
been derived \cite{Kike:1998}. Note that each surface particle experiences 
an environment that has both polar {\it and} azimuthal anisotropy, as a 
consequence of the excluded-volume interactions between the particles in 
addition to the `bare' wall potential.
\par
Minimisation of the grand canonical functional, equation (\ref{eq:grandz}),
\begin{equation}
\label{eq:domegadrho}
{{\delta \Omega\left[\rho(z,\omega)\right]}\over
{\delta\rho(z,\omega)}}=0,
\end{equation}
yields the Euler-Lagrange equation for the density-orientation profile,
\begin{equation}
\label{eq:rhozomega}
\log\rho(z,\omega)=\beta\mu - \int^{\prime} 
\Xi(z,\omega,z^{\prime},\omega^{\prime})
\rho(z^{\prime},\omega^{\prime})\,dz^{\prime}d\omega^{\prime},
\end{equation}
where the effect of the wall potentials, given by equation
(\ref{eq:surf}), has been incorporated through restriction of the 
range of integration over $\theta$:
\begin{equation}
\label{eq:angint}
\int^{\prime}d\omega = \int_0^{2\pi}d\phi \int^{\theta_m}_{\pi-\theta_m} 
\sin\theta\,d\theta = \int_0^{2\pi}d\phi
\int^{\cos\theta_m}_{-\cos\theta_m}\,dx,
\end{equation}
with 
\begin{equation}
\cos\theta_{m}= \left\{ \begin{array}{ll}
1 & \mbox{if $|z-z_0| \ge {L\over 2}$} \\
{{|z-z_0|}\over{L/2}} & \mbox{if $|z-z_0| < {L\over 2}$} 
\end{array} \right. ,
\end{equation}
$z_0$ being, we recall, the position of a substrate.
\par
It is clear from the structure of equation (\ref{eq:rhozomega}) that $\mu$
is the Lagrange multiplier associated with requiring that the mean number
of particles in the system be $N$. We are therefore at liberty to fix 
either $\mu$ or $N$ (see also discussion in \cite{Allen:1999}): as in 
earlier work we opt for the latter, since it allows closer contact with 
(constant $NVT$) simulation.
\par
Once $\rho(\omega,z)$ has been found, we can integrate out the angular
dependence to get the density profile,
\begin{equation}
\label{eq:rhoz}
\rho(z)=\int \rho(z,\omega)\,d\omega,
\end{equation}
and use this result to define the orientational distribution function 
(ODF) $\hat f(z,\omega)=\rho(z,\omega)/\rho(z)$, from which we can
calculate the orientational order parameters in the laboratory-fixed
frame \cite{MTG:1984}:
\begin{eqnarray}
\label{eq:eta}
\eta(z)&=&\langle P_2(\cos\theta )\rangle = Q_{zz}, \\
\label{eq:epsilon}
\varepsilon(z)&=&\langle\sin 2\theta\,\sin \phi \rangle 
={4\over 3}Q_{yz},\\
\label{eq:nu}   
\nu(z)&=&\langle \sin 2\theta\,\cos\phi \rangle = {4\over 3}Q_{xz}, \\
\label{eq:sigma}
\varsigma(z)&=&\langle \sin^2\theta\,\cos 2\phi \rangle =
{2\over 3}(Q_{xx}-Q_{yy}), \\
\label{eq:tau}
\tau(z)&=&\langle \sin^2\theta\,\sin2\phi \rangle = {4\over 3}Q_{xy},
\end{eqnarray}
where $\langle {\cal A}\rangle =\int {\cal A}\hat f(z,\omega)\,d\omega$.
These are the five independent components of the nematic order
parameter tensor, $Q_{\alpha\beta}=\langle {1\over
2}(3\hat\omega_\alpha\hat\omega_\beta-\delta_{\alpha\beta})\rangle$: they
give the fraction of molecules oriented along the $z$-axis ($Q_{zz}$); along 
the bisectors of the $yz$-, $xz$- and $xy$-quadrants ($Q_{yz}$, $Q_{xz}$ and
$Q_{xy}$, respectively); and the difference between the fractions of molecules
oriented along the $x$- and $y$-axes ($Q_{xx}-Q_{yy}$). 
In the case under study there is no twist, i.e., the director is confined 
to a plane that we can take as the $xz$ plane and $\varepsilon(z)=\tau(z)=0$. 
The three remaining order parameters, $\eta(z)$, $\nu(z)$ and $\varsigma(z)$, 
are in general all non-zero owing to surface-induced biaxiality, see our 
earlier work for $L=\sigma_L$ \cite{Chrzanowska:2001}. This effect has not 
been neglected in the present treatment, but in what follows we show results 
for $\eta(z)=Q_{zz}$ only, as we wish to concentrate on the 
planar-to-homeotropic transition.



\section{Results}

\label{sect-results}


In earlier work \cite{Chrzanowska:2001} we found that that the second-virial 
approximation does not give an accurate prediction for the location or the 
width of the I--N transition of particles of moderate 
elongation, as the contribution of higher virial coefficients is then
substantial. Therefore it does not make much sense to perform comparisons 
of theory and simulation at the same density. To circumvent this difficulty 
we have, as in a previous paper, resorted to extending a phenomenological 
scaling originally proposed by McDonald {\it et al.} \cite{Friederike:2000}
in the context of the I--N interface: comparison is instead effected 
between state points from theory and simulation characterised by the same 
$d_{bulk}\equiv (\rho_{bulk}-\rho_I)/(\rho_N-\rho_I)$. Table \ref{table1}
lists the I--N coexistence densities we have used, obtained from the theory 
of \cite{Chrzanowska:2001}, and from thermodynamic integration of a 
500-particle system \cite{deMiguel:2001}. The latter results are in reasonably 
good agreement with those of two MC studies: the earlier constant-$NpT$ 
simulation of a 256-particle system by Padilla and Velasco \cite{Kike:1997}, 
and the later constant-$NVT/NpT$ simulations of a 1000-particle system by 
two of us \cite{Barmes:2003}.\\

\begin{table}
\begin{center}
\begin{tabular}{||ccccc||}
	\hhline{|t:=====:t|}
	$\kappa$	&$\rho^{*,th}_I$ & $\rho^{*,th}_N$ &$\rho^{*,sim}_I$   &$\rho^{*,sim}_N$\\
	\hhline{||-----||}
	\hsp$3$\hsp	&\hsp$0.830100$\hsp	&\hsp$0.864356$\hsp	&\hsp$0.2990$\hsp	&\hsp$0.3046$\hsp\\
	\hhline{||-----||}
	$5$		&$0.210731$		&$0.234746$		&$0.1192$		&$0.1275$ \\
	\hhline{|b:=====:b|}
	\end{tabular}
\end{center}

\caption{Bulk I--N coexistence densities of HGOs from theory ({\it th}) 
\protect\cite{Chrzanowska:2001} and simulation ({\it sim}) 
\protect\cite{deMiguel:2001}.}
\label{table1} 
\end{table}

We emphasise that there is no fundamental 
reason why such a scaling should work, and that a proper validation would
require both a more sophisticated theoretical treatment (along the lines
of, e.g., \cite{Somoza:1989} or \cite{Friederike:2002}) and a more reliable 
location by MC of the I--N transition of the confined HGO fluid, to allow 
for a possible shift relative to the bulk (`capillary nematisation') 
\cite{Sluckin:1987,VanRoij:1999,Barmes:2003}. Still, we regard the procedure 
adopted as the best working tool for comparison of these systems that is 
available at the moment. The unscaled densities at which simulation runs and 
theory calculations have been performed are collected in Table \ref{table2}.\\

\begin{table}
\begin{center}
\begin{tabular}{||c|c|c|c||}
	\hhline{|t:====:t|}
	$\kappa$   &$d_{bulk}$   &$\rho^{*,th}$   &$\rho^{*,sim}$ \\
	\hhline{||----||}
	$3$   &$-3.39285298$   &$0.713874491$   &$0.28$ \\
	$3$   &$7.32142242$   &$1.08090251$   &$0.34$ \\
	$5$   &$-1.10843372$   &$0.18411197$   &$0.11$ \\
	$5$   &$2.50602458$   &$0.270913166$   &$0.14$ \\
	\hhline{|b:====:b|}
\end{tabular}
\end{center}
\caption{Scaled densities $d_{bulk}$ used in figures 
(\protect\ref{fig1})--(\protect\ref{fig4}) for comparison of theory
({\it th}) and simulation ({\it sim}) results: $d_{bulk}$ is positive
(negative) in the bulk nematic (isotropic) phase. $\rho^{*,th}$ and 
$\rho^{*,sim}$ are the unscaled densities at which the theoretical 
calculations and the simulations have been performed.} 
\label{table2} 
\end{table}

Equation (\ref{eq:rhozomega}) was solved iteratively for $\rho(z,\omega)$ 
by the Picard method, with an admixture parameter of 0.9 (i.e., 90\% of `old' 
solution in each iteration), starting from a uniform and isotropic profile. 
Following Chrzanowska \cite{Chrzanowska:unpub}, integrations were performed 
by Gauss-Legendre quadrature, using 64 $z$-points (the minimum necessary to
resolve the structure of profiles at the higher densities considered) and
$16\times 16$ $\omega$-points (for consistence with the bulk calculation
reported in \cite{Chrzanowska:2001}). Note that the range of $\omega^{\prime}$ 
depends on $z^{\prime}$: the closer a particle is to a substrate, the fewer 
orientations are accessible. In order to achieve good accuracy it is 
nevertheless crucial to include the same number of points in the angular 
integrations for all $z^{\prime}$ \cite{Chrzanowska:unpub}. Convergence 
was deemed to have been achieved when the error, defined as sum of the 
absolute values of differences between consecutive iterates at 
$64\times 16\times 16= 16384$ points), was less than $10^{-1}$.
The density and order parameter profiles were then calculated from 
equations (\ref{eq:rhoz}) and (\ref{eq:eta})--(\ref{eq:tau}), respectively. 
Details of the simulations have been published elsewhere 
\cite{Cleaver:2001,Barmes:2003}.\\

\picW=8cm
\begin{figure}
	\picB{bigbridfig1.eps}
	%%
	\caption{Scaled density $d(z)$ (top) and order parameter $Q_{zz}(z)$ (bottom)
	profiles for a symmetrical film of HGO particles of elongation $\kappa=5$ and 
	$L^*=3.0$ ($L/\sigma_L=0.6$), for $d_{bulk}=-1.10843372$ (solid line and 
	circles) and $2.50602458$ (dashed line and squares). Lines are from theory, 
	symbols are from simulation. At the higher density, which lies in the nematic 
	region of the bulk phase diagram, alignment is planar throughout the film, 
	as shown by the fact that $Q_{zz}(z)<0$ everywhere. See the text and table 
	\protect\ref{table2} for details.}
\label{fig1}
\end{figure}


In Figure \ref{fig1} we plot $d(z)$ and $Q_{zz}(z)$ profiles for a symmetrical 
film of HGO particles of elongation $\kappa=5$ and $L^*=L/\sigma_0=3.0$, 
which exhibits uniform planar alignment \cite{Cleaver:2001,Barmes:2003}. 
Equilibrium density peak formation is captured reasonably well by theory, 
which however underestimates the height of the peaks and the depth of the 
troughs, especially at the higher density. For $d_{bulk}=-1.10843372$, 
$Q_{zz}(z)$ is everywhere zero except in the surface layers (the high-density
peaks at $|z-z_0^{\alpha}|\sim 0$, $\alpha=1,2$), by which we mean 
the regions where the rotational freedom of a particle is restricted by the 
presence of the walls. Because the walls are impenetrable to a `needle' of 
length $L$, see equation (\ref{eq:surf}), the thickness of these layers is 
$\sim L/2= 1.5\sigma_0$: in order to reside there a particle has to be 
parallel to the wall, hence $Q_{zz}(z)<0$.
We speculate that the $Q_{zz}(z)$ peaks at $|z-z_0^{\alpha}|\sim L/2$
are due to excluded volume effects: particles are unable to minimise their 
free energy by retaining full rotational freedom, and thus give rise to 
a zero order parameter, by the presence of the adsorbed surface layers at 
$|z-z_0^{\alpha}|\le L/2$. It is then advantageous for them to align 
homeotropically so as to stick their ends through the wall and thereby
decrease the total excluded volume inside the system.
%Furthermore, note that the $Q_{zz}(z)$ peaks at 
% $\sim L/2$  correspond to the nearest particles can get to a wall and still 
%retain full rotational freedom, hence boost both their orientational and 
%translational entropies by being weakly ordered in a fairly dense region.
At the higher density $d_{bulk}=2.50602458$ there is again good agreement 
between theory and simulation for $Q_{zz}(z)$, presumably because the degree 
of order is now quite large and close to saturation.\\

\picW=8cm
\begin{figure}
	\picB{bigbridfig2p.eps}
	\caption{Same as figure \protect\ref{fig1}, but for $L^*=2.0$ 
	($L/\sigma_L=0.4$). Now alignment at the higher density is homeotropic 
	throughout most of the film, as can be seen from the fact that $Q_{zz}(z)>0$ 
	except in regions of width $\sim L/2=1.25\sigma_0$ close to the walls.}
	\label{fig2}
\end{figure}


Figure \ref{fig2} shows the same quantities but for $L^*=2.0$, for which 
alignment is perpendicular to the walls \cite{Cleaver:2001,Barmes:2003}. 
The density peaks, the highest of which now occur at 
$|z-z_0^{\alpha}|\sim L/2=\sigma_0$, 
are even more pronounced than for $L^*=3.0$, and likewise 
underestimated by theory. In particular, simulation appears to show the 
formation of some 5 layers, whereas theory bears out only 4. The $Q_{zz}(z)$ 
profiles from both theory and simulation are very similar to those for 
$L^*=3.0$ at the lower density, but theory overestimates the degree of 
order at the higher density.\\

\picW=8cm
\begin{figure}
	\picB{bigbridfig3.eps}
	%%
	\caption{Scaled density $d(z)$ (top) and order parameter $Q_{zz}(z)$ (bottom)
	profiles for a symmetrical film of HGO particles of elongation $\kappa=3$ 
	and $L^*=1.8$ ($L/\sigma_L=0.6$), for $d_{bulk}=-3.39285298$ (solid line and 
	circles) and $7.32142242$ (dashed line and squares). Lines are from theory, 
	symbols are from simulation. At the higher density, which lies in the nematic 
	region of the bulk phase diagram, alignment is planar throughout the film,
	as in figure \protect\ref{fig1}. Now the Onsager approximation severely 
	underestimates the degree of structure of $d(z)$. See the text 
	and table \protect\ref{table2} for details.}
	\label{fig3}
\end{figure}

\picW=8cm
\begin{figure}
	\picB{bigbridfig4.eps}
	%%
	%%
	\caption{Same as figure \protect\ref{fig3}, but for $L^*=1.2$ 
	($L/\sigma_L=0.4$). Now alignment at the higher density is homeotropic 
	throughout most of the film but planar in regions of width 
	$\sim L/2=0.6\sigma_0$ close to the walls, as in figure \protect\ref{fig2}.}
	\label{fig4}
\end{figure}


Figures \ref{fig3} and \ref{fig4} illustrate planar and homeotropic alignment 
for the smaller elongation $\kappa=3$: now the Onsager approximation is 
expected to fare worse, and indeed it underestimates the structure of the
density profiles $d(z)$ rather severely: neither the heights nor the positions 
of most peaks are predicted with even semi-quantitative accuracy. Still, the 
mean values of $Q_{zz}(z)$ are reasonably faithfully reproduced, as well as
the fact that the density peaks are farther apart for homeotropic than for
planar anchoring, as expected.\\
%%
It is a nearly universal rule that the behaviour of $Q_{zz}(z)$ follows that
of $d(z)$, i.e., regions of high density are also more highly ordered. The
exceptions are the surface layers of homeotropically-anchored films, where 
there are very few but very strongly aligned particles (see figures \ref{fig2}
and \ref{fig4}).\\
%%
In table \ref{table3} we present the approximate $L/\sigma_L=L^*/\kappa$ at
which there is a crossover from homeotropic to planar `bulk' alignment. 
Unsurprisingly, the agreement between theory and simulation is better for
the larger elongation, $\kappa=5$. 
Insight can be gained into the anchoring transition by noting that, 
in the limit of perfect orientational and positional order, the
Helmholtz free energy of this system is minimised by the arrangement 
that maximises the particle volume absorbed into the substrates. The
homeotropic-to-planar transition is then given by equating the ratio 
of the volume absorbed into the substrates to the area occupied by the
particle on the substrates (i.e., the projection of the particles onto 
the substrates) for the two key arrangements. In the limit of perfect
order, symmetry details of the packing can be ignored in this calculation 
since they must be the same for both anchorings; the two arrangements 
will map onto each other via suitable affine transformations
These two competing tendencies yield a crossover needle 
length $L/\sigma_L=L^*/\kappa\sim 0.4817$ for $\kappa=3$ and 
$L/\sigma_L=L^*/\kappa\sim 0.6084$ for $\kappa=5$, which are in line with 
our results: in particular, it is an increasing function of $\kappa$. Such 
quantitative discrepancies as there are likely follow from having assumed 
(contrary to observation) that there is close packing at the walls: as we
saw above, theory in particular systematically underestimates the contact 
densities. See \cite{Barmes:2003} for details. As the bulk HGO fluid does
not exhibit any smectic phases, we do not expect the crossover $L$ to depend
on system size, provided the latter is greater than twice the distance to 
which a single wall induces stratification.\\

\begin{table}
\begin{center}
\begin{tabular}{||c|c|c|c||}
	\hhline{|t:====:t|}
	$\kappa$   &$(L/\sigma_L)^{th}$   &$(L/\sigma_L)^{sim}$   &$(L/\sigma_L)^{analyt}$ \\
	\hhline{||----||}
	$3$   &$(0.55,0.56)$   &$(0.45,0.46)$   &0.4817 \\
	$5$   &$(0.56,0.58)$   &$(0.50,0.52)$   &0.6084 \\
	\hhline{|b:====:b|}
\end{tabular}
\end{center}
\caption{Film alignment crosses over from homeotropic to planar for 
$L/\sigma_L=L^*/\kappa$ in each of the intervals shown. From theory 
({\it th}) and simulation ({\it sim}) (this work), and the analytical 
estimates of \protect\cite{Barmes:2003} ({\it analyt}).}
\label{table3} 
\end{table}
\newpage


Interestingly, on the homeotropic side just before (i.e., at $L$ marginally 
below that of) the crossover, the density profiles exhibit very pronounced 
peaks and troughs, which smooth out dramatically once the system has gone 
planar, see figure \ref{fig5}. In particular, note the double-peaked structure
close to the walls at the larger $L$, which suggests that the interface
layers exhibit both planar and homeotropic features. As far as we can tell 
the transition is fairly abrupt, occurring in an interval of $L^*$ of width 
$0.1$ (alternatively, in an interval of $L/\sigma_L$ of width $0.02$ ).
Simulation results suggest \cite{Barmes:2003} that there may be bistability 
close to the transition, i.e., both homeotropic and planar anchoring states
stable for the same $L$. We were able to get the theory to converge to the 
`wrong' anchoring (homeotropic for $L^*=2.9$, with higher free energy than 
planar) by strongly biassing the initial guess: more work is needed to 
establish whether this is real or an artifact of the numerical method.\\
%%
\picW=8cm
\begin{figure}
	\picB{bigbridfig5.eps}
	\caption{Scaled density $d(z)$ (top) and order parameter $Q_{zz}(z)$ (bottom)
	profiles for a symmetrical film of HGO particles of elongation $\kappa=5$ and 
	reduced density $d_{bulk}=2.50602458$ on either side of the crossover from 
	homeotropic to planar anchoring. Theory: $L^*=2.8$ or $L/\sigma_L=0.56$ (solid 
	line), $L^*=2.9$ or $L/\sigma_L=0.58$ (dashed line). Simulation: $L^*=2.5$ or
	$L/\sigma_L=0.5$ (circles), $L^*=2.6$ or $L/\sigma_L=0.52$ (squares). Note the 
	high degree of layering when the homeotropically aligned film is just about 
	to go planar. See the text and table \protect\ref{table3} for details.}
	\label{fig5}
\end{figure}
%%
Finally, attention is drawn to the fact that all comparisons in this section 
have been effected at constant density and varying $L$. This is the natural 
thing to do, as $N$, the particle number, and $|z_0^1-z_0^2|$, the system 
size, are fixed in the MC simulations. Note, however, that because particles
can partially penetrate the substrates, the volume accessible to them is a 
function not just of $|z_0^1-z_0^2|$, but also of $L$, the needle length: the 
smaller $L$, the deeper a particle can sink, and thus effectively the larger
the system. It then follows that the {\it imposed} total density (what we 
referred to above as simply `the density'), found by dividing the number of 
particles by the wall separation times the wall area, is not the {\it actual} 
or {\it true} density. It is possible to estimate the system volume increase 
as a function of the type of anchoring and ultimately of $L$, and thereby 
get the `true' density \cite{Barmes:2003}. Yet here we have elected not to 
do this and used instead the imposed, `uncorrected' density: though it
may not be an absolutely unambiguous descriptor in the microscopic sense,
it is what is set in simulations and measured in experiments. Furthermore,
in order to keep the `true' density constant we would require knowledge of
the actual system volume and therefore of what anchoring would come out of 
a given calculation or simulation prior to running it -- a not impossible, 
but rather unwieldy, iterative task.


\section{Conclusions}

\label{sect-concl}


In this paper we have presented a density-functional treatment of a HGO
fluid confined between parallel walls of tunable penetrability. 
Despite its simplicity, the Onsager approximation can in some cases yield
semi-quantitative results for the density and orientational distribution
of particles of elongation as small as $\kappa=5$ (but not $\kappa=3$).
This simple model for the structure of the nematic matrix squeezed
between tight-packed colloidal particles captures effects missed by the 
more current Frank and LdG theories, namely to do with layering. Moreover, 
the solution procedure also yields the free energy, thus making it possible 
to derive the effective interaction between walls/particles. This will 
be the subject of future work.
\par
When comparing theory and simulation, account
has to be taken of the fact that they yield rather different I--N 
transition densities and widths. This is due to our neglect of 
correlations of order higher than second-virial, which are relevant
in the range of densities of interest. Greater predictive power would 
require a far more sophisticated approach, such as a weighted-density 
\cite{Somoza:1989} or fundamental-measure \cite{Friederike:2002} 
approximation. The development 
and implementation of such a scheme are, however, highly non-trivial.
In keeping with our aim of assessing the validity and usefulness of 
the Onsager approach, we instead adapted a phenomenological scaling of the 
density first proposed by Allen and collaborators \cite{Friederike:2000}.
Agreement for symmetric films is fairly good, in spite of the smallness
of $\kappa$, but its quality is strongly dependent on the accuracy of 
the isotropic and nematic coexistence densities as determined independently 
by either theory or simulation. We have not addressed the fact that these 
are (sometimes dramatically) shifted from their bulk values by both 
confinement and wall penetrability \cite{Barmes:2003}.
\par
The theory can also be applied to hybrid films. It would be particularly 
interesting to see whether it is able to describe: (i) the more common 
uniform and bent-director structures already predicted \cite{Barbero:1983} 
and observed \cite{BlinovMazulla}; (ii) the discontinuous transition between
these two structures found by ourselves \cite{Cleaver:2001}; and (iii) the 
more exotic biaxial structure in which two strata of film, each with a uniform 
director orientation dictated by the nearest wall, are separated by a sharp 
interface \cite{Palffy}. Both (ii) and (iii) depend crucially on the anchoring
strengths at the two substrates being large and dissimilar or large and 
similar, respectively. However, preliminary calculations
suggest that our chosen mechanism of making the walls partially penetrable
to particles produces much stronger homeotropic than parallel anchoring.
This would need to be checked by a direct calculation of the anchoring 
energy, similar to that performed in \cite{Allen:1999}. A more ambitious 
aim would be to be able to shed some light on the observation by Vandenbrouck 
{\it et al.} \cite{Vanden} of spinodal dewetting of the nematogen 5CB 
spun-cast onto silicon wafers, where hybrid anchoring was enforced by 
conflicting boundary conditions: orthogonal at the free surface, and planar 
at the silicon substrate. Such behaviour was initially interpreted in terms 
of a competition between elasticity and van der Waals forces \cite{Vanden},
but subsequent arguments have related it to the fluctuation-induced 
interactions that underlie the pseudo-Casimir effect \cite{Ziherl}.
\par
The present theory can be straightforwardly generalised to more
sophisticated surface interactions, and also to mixtures of two or  
more types of hard body. One can envisage a very rich behaviour of a 
confined binary mixture where the two components have different easy
axes at either substrate.


\section*{Acknowledgements}

We are grateful to C. M. Care, A. Chrzanowska, J. R. Henderson, Y. Mao, 
A. Poniewierski, E. Velasco, P. Ziherl and S. \v{Z}umer for stimulating 
discussions. F. Barmes acknowledges Sheffield Hallam University's Materials 
Research Institute for a research student bursary.
%P. I. C. Teixeira acknowledges partial funding from the
%Funda\c{c}\~{a}o para a Ci\^{e}ncia e Tecnologia (Portugal). 




\begin{thebibliography}{99}


\bibitem{Stark:2001} H. Stark, Phys.\ Rep.\ {\bf 351}, 387 (2001).

\bibitem{Poulin:1997} P. Poulin, V. A. Raghunathan, P. Richetti and D. Roux, 
J. Phys.\ II {\bf 4}, 1557 (1994); P. Poulin, H. Stark, T. C. Lubensky and 
D. A. Weitz, Science {\bf 275}, 1770 (1997).

\bibitem{Zapotocky:1999} M. Zapotocky, L. Ramos, P. Poulin, T. C. Lubensky
and D. A. Weitz, Science {\bf 283}, 209 (1999).

\bibitem{Val} S. P. Meeker, W. C. K. Poon, J. Crain and E. M. Terentjev, 
Phys.\ Rev.\ E {\bf 61}, R6083 (2000); 
V. J. Anderson, E. M. Terentjev, S. P. Meeker, J. Crain and
W. C. K. Poon, Eur.\ Phys.\ J. E {\bf 4}, 11 (2001); V. J. Anderson and 
E. M. Terentjev, {\it ibid.} {\bf 4}, 21 (2001); P. G. Petrov and E. M. 
Terentjev, Langmuir {\bf 17}, 2942 (2001).

\bibitem{Terentjev} R. W. Ruhwandl and E. M. Terentjev, Phys.\ Rev.\ E
{\bf 55}, 2958 (1997); {\it ibid.} {\bf 56}, 5561 (1997); 
T. C. Lubensky, D. Pettey and N. Currier, Phys.\ Rev.\ E {\bf 57}, 610 (1998);
P. Poulin and D. A. Weitz, Phys.\ Rev.\ E {\bf 57}, 626 (1998);
E. M. Terentjev, in {\it Modern Aspects of Colloidal Dispersions}, ed.\ by
R. H. Ottewill and A. R. Rennie (Kluwer, Dordrecht, 1998), p.\ 257.

\bibitem{Galatola} P. Galatola and J.-B. Fournier, Molec.\ Cryst.\ Liq.\
Cryst. {\bf 330}, 535 (1999); A. Bor\v{s}tnik, H. Stark and S. \v{Z}umer,
Phys.\ Rev.\ E {\bf 60}, 4210 (1999);
Phys.\ Rev.\ Lett.\ {\bf 86}, 3915 (2001);
Phys.\ Rev.\ E {\bf 65}, 032702 (2002); P. Galatola, J.-B. Fournier and
H. Stark, Phys.\ Rev.\ E {\bf 67}, 031404 (2003).

\bibitem{Stark:2000} A. Bor\v{s}tnik, H. Stark and S. \v{Z}umer,
Phys.\ Rev.\ E {\bf 61}, 2831 (2000).

\bibitem{Chrzanowska:2001} A. Chrzanowska, P. I. C. Teixeira, H. Eherentraut
and D. J. Cleaver, J. Phys.: Condens.\ Matter {\bf 13}, 4715 (2001).

\bibitem{Groh} B. Groh and S. Dietrich, Phys.\ Rev.\ E {\bf 59}, 4216 (1999);
B. Groh, {\it ibid.} {\bf 59}, 5606 (1999).

\bibitem{Cleaver:2001} D. J. Cleaver and P. I. C. Teixeira, Chem.\ Phys.\
Lett.\ {\bf 338}, 1 (2001).

\bibitem{Barmes:2003} F. Barmes and D. J. Cleaver (unpublished).

\bibitem{Kike:1997} P. Padilla and E. Velasco, J. Chem.\ Phys.\ {\bf 106},
10299 (1997).

\bibitem{deMiguel:2001} E. de Miguel and E. Mart\'{\i}n del Rio, 
J. Chem.\ Phys.\ {\bf 115}, 9072 (2001).

\bibitem{Perram:19XX} J. W. Perram and M. S. Wertheim,
J. Comput.\ Phys.\ {\bf 58}, 409 (1985); J. W. Perram,
J. Rasmussen, E. Praestgaard and J. L. Lebowitz, Phys.\ Rev.\ E {\bf 54},
6565 (1996).

\bibitem{Allen:1993} M. P. Allen, G. T. Evans, D. Frenkel and
B. M. Mulder, Adv.\ Chem.\ Phys.\ {\bf 86}, 1 (1993).

\bibitem{Steele:1987} V. R. Bhethanabotla and W. Steele, Molec.\ Phys.\
{\bf 60}, 249 (1987).

\bibitem{Rigby} M. Rigby, Molec.\ Phys.\ {\bf 68}, 687 (1989);
S. L. Huang and V. R. Bhethanabotla, Int.\ J. Mod.\ Phys.\ C {\bf 10}, 
361 (1999). 

\bibitem{Evans:1979} R. Evans, Adv.\ Phys.\ {\bf 28}, 143 (1979).

\bibitem{Poniewierski:1993} A. Poniewierski, Phys.\ Rev.\ E {\bf 47}, 3396
(1993). 

\bibitem{Kike:1998} E. Velasco and L. Mederos,
J. Chem.\ Phys.\ {\bf 109}, 2361 (1998). 

\bibitem{Chrzanowska:unpub} A. Chrzanowska, J. Comput.\ Phys.\ {\bf 191},
265 (2003).

\bibitem{Allen:1999} M. P. Allen, Molec.\ Phys.\ {\bf 96}, 1391 (1999);
J. Chem.\ Phys.\ {\bf 112}, 5447 (2000).

\bibitem{MTG:1984} M. M. Telo da Gama, Molec.\ Phys.\ {\bf 52}, 585
(1984); {\it ibid.} {\bf 52}, 611 (1984).

\bibitem{Friederike:2000} A. J. McDonald. M. P. Allen and F. Schmid,
Phys.\ Rev.\ E {\bf 63}, 010701(R) (2001). 

\bibitem{Somoza:1989} A. M. Somoza and P. Tarazona, J. Chem.\ Phys.\
{\bf 91}, 517 (1989); E. Velasco, L. Mederos and D. E. Sullivan,
Phys.\ Rev.\ E {\bf 62}, 3708 (2000).  

\bibitem{Friederike:2002} G. Cinacchi and F. Schmid, J. Phys.: Condens.\
Matter {\bf 14}, 12189 (2002).

\bibitem{Sluckin:1987} A. Poniewierski and T. J. Sluckin Liq.\ Cryst.\
{\bf 2}, 281 (1987).

\bibitem{VanRoij:1999} R. van Roij, M. Dijkstra and R. Evans,
Europhys.\ Lett.\ {\bf 49}, 350 (2000); J. Chem.\ Phys.\ {\bf 113}, 7689
(2000); Phys.\ Rev.\ E {\bf 63}, 051703 (2001).

\bibitem{Barbero:1983} G. Barbero and R. Barberi, J. Phys.\ (Paris)
{\bf 44}, 609 (1983).

\bibitem{BlinovMazulla} L. M. Blinov, D. B. Subachyus and S. V. Yablonsky,
J. Phys.\ II (Paris) {\bf 1}, 459 (1991); A. Mazulla, F. Ciuchi and J. R. 
Sambles, Phys.\ Rev.\ E {\bf 64}, 021708 (2001).

\bibitem{Palffy} P. Palffy-Muhoray, E. C. Gartland and J. R. Kelly,
Liq.\ Cryst.\ {\bf 16}, 713 (1994); H. G. Galabova, N. Kothekar and D. W.
Allender, Liq.\ Cryst.\ {\bf 23}, 803 (1997); A. \v{S}arlah and S. \v{Z}umer,
Phys.\ Rev.\ E {\bf 60}, 1821 (1999); I. Rodr\'{\i}guez-Ponce, J. M. 
Romero-Enrique and L. F. Rull, Phys.\ Rev.\ E {\bf 64}, 051704 (2001);
C. Chiccoli, P. Pasini, A. \v{S}arlah, C. Zannoni and S. \v{Z}umer, 
Phys.\ Rev.\ E {\bf 67}, 050703 (2003).

\bibitem{Vanden} F. Vandenbrouck, M. P. Valignat and A. M. Cazabat,
Phys.\ Rev.\ Lett.\ {\bf 82}, 2693 (1999). 

\bibitem{Ziherl} P. Ziherl, R. Podgornik and S. \v{Z}umer,
Phys.\ Rev.\ Lett.\ {\bf 84}, 1228 (2000);
P. Ziherl, F. K. P. Haddadan, R. Podgornik
and S. \v{Z}umer, Phys.\ Rev.\ E {\bf 61}, 5361 (2000).

\end{thebibliography}

\end{document}
