
%%\documentclass[a4paper,12pt,onecolumn]{article}
%%\documentclass[aps,10pt,twocolumn]{revtex4-1}
%%\documentclass[aps,pre,10pt,twocolumn,%superscriptaddress{revtex4-1}

\documentclass[%
reprint,
superscriptaddress,
%groupedaddress,
%unsortedaddress,
%runinaddress,
%frontmatterverbose, 
%preprint,
showpacs,
%preprintnumbers,
%nofootinbib,
%nobibnotes,
%bibnotes,
 amsmath,amssymb,
 aps,
%pra,
%prb,
%rmp,
%prstab,
%prstper,
floatfix,
%onecolumn
]{revtex4-1}

%===============================%
%                               %
%   packages                    %
%                               %
%===============================%

\usepackage{graphicx}           % for graphics
\usepackage{graphics}           % for graphics
\usepackage{calc}               % allow calculations
\usepackage{subfigure}          % allow subfigure
\usepackage{amssymb}            % font
\usepackage{amstext}            % font
\usepackage{amsmath}            % font
\usepackage{amsfonts}           % font
\usepackage{amsbsy}             % font
\usepackage{float}              % figure HERE
\usepackage{color}              % allow use of color
\usepackage{fancybox}           % nice boxes
\usepackage{epsfig}     		% to include eps figs
\usepackage{fancybox}       	% nice boxes
\usepackage{hhline}     		% nice lines in tables

%===============================%
%                               %
%   new commands                %
%                               %
%===============================%
\newcommand{\DiF}[1]{ {\fontsize{18}{22pt}\usefont{U}{pzd}{m}{n}\symbol{'#1} }}
\newcommand{\vect}[1]{ \mathbf{#1} }
\newcommand{\vecth}[1]{ \mathbf{\hat{#1} } }
\newcommand{\dotproduct}[2]{\vect{#1} \centerdot \vect{#2}   }
\newcommand{\TAB}{\hspace*{\tabul}}
\newcommand{\forcebold}[1]{\mbox{\boldmath $#1$ }}
\newcommand{\lp}{\left(}
\newcommand{\rp}{\right)}
\newcommand{\lcurl}{\left\{}    %}
\newcommand{\rcurl}{\right\}} %{
\newcommand{\change}[1]{\textbf{BC} \textit{#1} \textbf{EC} }

\newcommand{\pic}[1]{\includegraphics[width=\picW]{#1}}
\newcommand{\picL}[1]{\includegraphics[height=\picW, angle=270]{#1}}


\newlength{\imageheight}
\newlength{\imagewidth}
\newlength{\picH}
\newlength{\picW}
\newlength{\picA}
\newlength{\tablength}
\newlength{\tabul}


%============================================================================
%   Constants
%============================================================================
\newcommand{\F}[2]{\frac{#1}{#2}}

%============================================================================
%   Contact distance related macros
%============================================================================

\newcommand{\rij}{\vecth{r}_{ij}}
\newcommand{\ui}{\vecth{u}_i}
\newcommand{\uj}{\vecth{u}_j}
\newcommand{\sijr}{\sigma(\ui,\uj,\rij)}
\newcommand{\dotProdP}[2]{ \left( #1 \centerdot #2 \right) }
\newcommand{\dotProd}[2]{ #1 \centerdot #2 }
\newcommand{\half}{\frac{1}{2}}
\newcommand{\lc}{\left[}
\newcommand{\rc}{\right]}

\newcommand{\sir}{\sigma(\ui,\vect{r}_{ij})}
\newcommand{\drui}{d_i\lc\dotProdP{\ui}{\rij}\rc}
\newcommand{\lrui}{\ell_i\lc\dotProdP{\ui}{\rij}\rc}

\newcommand{\druj}{d_j\lc\dotProdP{\uj}{\rij}\rc}
\newcommand{\lruj}{\ell_j\lc\dotProdP{\uj}{\rij}\rc}

\newcommand{\druiSq}{d_i^2\lc\dotProdP{\ui}{\rij}\rc}
\newcommand{\lruiSq}{\ell_i^2\lc\dotProdP{\ui}{\rij}\rc}

\newcommand{\drujSq}{d_j^2\lc\dotProdP{\uj}{\rij}\rc}
\newcommand{\lrujSq}{\ell_j^2\lc\dotProdP{\uj}{\rij}\rc}


\parindent = 0pt
%=======================================%
%                                       %
%   beginning of the document           %
%                                       %
%=======================================%

\begin{document}

\graphicspath
{
{../imgs/}
}

\title{Computer simulations of hard pear-shaped particles}

\author{F. Barmes}
\affiliation{Materials Research Institute, Sheffield Hallam University,
Sheffield S1 1WB, United Kingdom}

\author{M. Ricci}
\author{C. Zannoni}
\affiliation{Dip.di Chimica Fisica ed Inorganica, Universit\`{a} di Bologna,
40136 Bologna, Italy}

\author{D.J. Cleaver}
\affiliation{Materials Research Institute, Sheffield Hallam University,
Sheffield S1 1WB, United Kingdom}





\pacs{61.30 61.20Ja 61.30}

\date{\today}


%=======================================================================================================%
%           ABSTRACT
%=======================================================================================================%
\begin{abstract}
We report results obtained from Monte Carlo simulations
investigating mesophase formation in two model systems of hard
pear-shaped particles. The first model considered is a hard
variant of the truncated Stone-Expansion model previously shown to
form nematic and smectic mesophases when embedded within a 12-6
Gay-Berne-like potential 
[R. Berardi, M. Ricci and Z. Zannoni. \emph{ChemPhysChem}, 7:443, 2001]. 
When stripped of
its attractive interactions, however, this system is found to lose
its liquid crystalline phases. For particles of length to breadth
ratio $k=3$, glassy behaviour is seen at high pressures, whereas
for $k=5$ several bi-layer-like domains are seen, with high
intradomain order but little interdomain orientational
correlation. For the second model, which uses a parametric shape
parameter based on the generalised Gay-Berne formalism, results
are presented for particles with elongation $k=3,4$ and $5$. Here,
the systems with $k=3$ and $4$ fail to display orientationally
ordered phases, but that with $k=5$ shows isotropic, nematic and,
unusually for a hard-particle model, interdigitated smectic ${\rm
A_2}$ phases.
%CHANGE1
\end{abstract}

\maketitle

%=======================================================================================================%
%           INTRODUCTION
%=======================================================================================================%
\section{Introduction}

In recent years, flexoelectricity has become an increasingly important feature in the design of
materials for use in liquid crystal devices. Flexoelectric behaviour, which
leads to field-induced director distortion, results, at a molecular level, from competition
between electric and steric dipolar interactions. As well as leading to modified bulk
properties, flexoelectricity has been mooted to be a possible driver for switching in devices
with bistable anchoring surfaces~\cite{Davidson_Mottram}. Indeed, it has been suggested
that the switching mechanism of the zenithally bistable device~\cite{ZBD} may rely, in part, on
flexoelectric behaviour.

The early studies of Meyer~\cite{Meyer_69} and Prost and Marcerou~\cite{Prost_Marcerou_77},
showed that the mechanisms underlying flexoelectricity can be understood in two ways. In the
original explanation from Meyer, flexoelectric behavior was explained in terms of particles with
a strong anisotropy in their charge repartition. Thus, it was shown that, upon polarization by
an applied field, pear-shaped particles exhibit a splay director distortion, whereas
banana-shaped particles exhibit a bend distortion. Subsequently, Prost and Marcerou showed that
flexoelectricity could also be obtained using particles with a non-zero quadrupole moment. This
did not contradict Meyer's original work, however, since in reality flexoelectric mesogens are
known to possess either one or both of these properties~\cite{ProstMarcerou80}.

Although well studied theoretically~\cite{Osipov83, Osipov_84, DosovLargarde83}, few computer
simulations using flexoelectric particles have been performed to date. Whilst particle based
simulations showing ferroelectric behaviour are reasonably well established (see, {\em e.g.},
\cite{wei92a,wei92b}), models with the dipolar and/or quadrupolar symmetry steric interactions
needed for flexoelectric behaviour are relatively scarce.

Neal and co-workers performed one such study using molecules represented by rigid assemblies of
three Gay-Berne sites~\cite{Neal}. One of the assemblies considered in Ref.~\cite{Neal} was a
triangular arrangement of mutually parallel Gay-Berne sites, leading, overall, to approximately
pear-shaped molecules. On compression, a system of such molecules ordered from an isotropic
liquid to a smectic arrangement in which the molecular orientations in successive layers were
almost perfectly anti-parallel. Subsequently, Stelzer {\em et al.}~\cite{Stelzer_Berardi_99}
investigated the behaviour of pear-shaped molecules using a model with two interaction sites per
particle; each particle comprised a Lennard-Jones site embedded near to one end of a Gay-Berne
site. Isotropic, nematic and smectic phases were observed, local antiparallel alignment being
seen in the nematic phase. Measurements of the splay and bend flexoelectric coefficients gave a
non-zero splay coefficient and, to within error estimates, a zero bend coefficient in accordance
with Meyer's theory. Equivalent simulations by Billeter and
Pelcovits~\cite{Billeter_Pelcovits_00}, using qualitatively the same model but with different
energy parametrisations and an alternative method for the calculation of the flexoelectric
coefficients, confirmed the results of Ref. \cite{Stelzer_Berardi_99}. In this case, however, no
stable nematic phase was found between the isotropic and (locally antiparallel) smectic A
phases.

Whilst the results from these systems proved encouraging, their reliance on multi-site generic
potentials remained a relative inefficiency. This was resolved somewhat in recent work by
Berardi and some of the current authors~\cite{Berardi_Ricci_01}, in which a single-site model
was developed, using Zewdie's generalisation approach~\cite{Zewdie98a,Zewdie98b}, to represent
tapered or pear-shaped particles. Here, using the geometrical shape of a B\'{e}zier curve as a
template for the particle shape, a numerically calculated mesh of contact distance values was
fitted using a truncated Stone expansion which, in turn, was employed in the simulations themselves.
Results from this study were very encouraging, as both nematic and smectic A phases were
found, and, through appropriate manipulation of the well-depth anisotropy terms, equivalent
phases with net {\em polar} order were generated.

In this paper, we seek to explore the fundamental properties of single-site pear-shaped models
such as that used in Ref~\cite{Berardi_Ricci_01}, by investigating mesophase formation
in systems of hard, noncentrosymmetric particles. Hard particle simulations have proved to be an
effective and efficient testbed for many of the theories of liquid crystal
physics~\cite{hard_rev}, and have confirmed that shape anisotropy alone can be sufficient for
the onset of nematic and even smectic order. Two distinct systems are described here. The first
is a hard version of the truncated Stone expansion potential described in~\cite{Berardi_Ricci_01}. The
second employs a novel approach, based on a parametric variant of the generalized Gay-Berne
shape parameter~\cite{CleaverCare96}, which yields an analytical expression for the contact
distance between two pear-shaped objects.

The content of the remainder of this paper is arranged as follows. In subsection \ref{s:Stone}
we give a brief description of the truncated Stone expansion potential before presenting and discussing
results obtained from Monte Carlo simulations of same. In the following subsection, we introduce
the parametric approach for generating shape parameters for non-ellipsoidal particles, and apply
it to generate shape parameters for the B\'{e}zier pears considered in
Ref.~\cite{Berardi_Ricci_01}. Results obtained from Monte Carlo simulations of such systems are
presented in subsection~\ref{s:GBP_results}. Finally, the two sets of simulations are compared
and discussed in Section~\ref{S:GBP_conclusions}.



%=======================================================================================================%
%           RESULTS FROM BOLOGNA
%=======================================================================================================%
\section{Simulations of hard pears.}
\subsection{The truncated Stone expansion model}
\label{s:Stone}
\subsubsection{Model Details}
Our first simulations of hard pear-shaped particles used a steric
version of the potential described in Ref.~\cite{Berardi_Ricci_01}. This gives the interaction
between two particles, $i$ and $j$, as %
\begin{equation}
    \mathcal{V}^{HP} = \left\{  %}
    \begin{array}{ccc}
        \infty  &\mathrm{if\ }  &r_{ij} < \sijr \\
        0   &\mathrm{if\ }  &r_{ij} \geq \sijr
    \end{array}
    \right.
\end{equation}
%
where $\sijr$ represents the contact distance between two pear-shaped particles with
orientations $\vecth{u}_i$ and $\vecth{u}_j$ and $\vecth{r}_{ij} = \frac{\vect{r}_{ij}}{r_{ij}}$
and $\vect{r}_{ij}$ is the intermolecular vector.
Following the approach of Zewdie~\cite{Zewdie98a,Zewdie98b}, this
contact distance was expressed as an expansion of the form
\begin{eqnarray}
\label{eqn:Stone_sigma1}
    \sijr   &\simeq&  \mathcal{L}(\vecth{u}_i,\vecth{u}_j,\vecth{r}_{ij}) \nonumber \\
        &=& \sum_{L_1,L_2,L_3}\sigma_{L_1, L_2, L_3}S^{*L_1, L_2, L_3}
        (\vecth{u}_i,\vecth{u}_j,\vecth{r}_{ij})
\end{eqnarray}
%
where $S^{L_1, L_2, L_3}$ is a Stone function~\cite{Stone}, and the expansion coefficients
$\sigma_{L_1, L_2, L_3}$ are given by

\begin{widetext}
\begin{equation}
\label{eqn:Stone_sigma2}
    \sigma_{L_1, L_2, L_3} = \frac
    {
    \int \mathcal{L}(\vecth{u}_i,\vecth{u}_j,\vecth{r}_{ij})S^{L_1, L_2, L_3}
    (\vecth{u}_i,\vecth{u}_j,\vecth{r}_{ij})d\vecth{u}_id\vecth{u}_jd\vecth{r}_{ij}
    }
    {
    \int S^{*L_1, L_2, L_3}(\vecth{u}_i,\vecth{u}_j,\vecth{r}_{ij})S^{L_1, L_2, L_3}
    (\vecth{u}_i,\vecth{u}_j,\vecth{r}_{ij})d\vecth{u}_id\vecth{u}_jd\vecth{r}_{ij}
    }.
\end{equation}
\end{widetext}

%
Simulations were performed with two parameterisations of this model, with length to breadth
ratios, $k$, of 3 and 5 respectively. In both cases the shape parameter expansion
(\ref{eqn:Stone_sigma1}) was restricted to indices $\{L_1, L_2, L_3\} = 1\ldots 6$; the
expansion coefficients used for $k = 3$ were identical to those given in
Ref.~\cite{Berardi_Ricci_01}, while those for $k=5$ are listed in Table~\ref{tble:HP_sigma_k5}.


\begin{table}
    \centering
    \begin{tabular}{cccccc}
    \hhline{======}
      $[000]$ & $    1.90456 $ & $[011]$ & $    0.51113 $ & $[101]$ & $    0.51113 $  \\
      $[022]$ & $    2.01467 $ & $[202]$ & $    2.01467 $ & $[033]$ & $   -0.11376 $  \\
      $[303]$ & $   -0.11376 $ & $[044]$ & $    0.91479 $ & $[404]$ & $    0.91479 $  \\
      $[055]$ & $   -0.29937 $ & $[505]$ & $   -0.29937 $ & $[066]$ & $    0.41523 $  \\
      $[606]$ & $    0.41523 $ & $[110]$ & $   -0.03942 $ & $[121]$ & $   -0.45400 $  \\
      $[211]$ & $   -0.45400 $ & $[123]$ & $    0.59579 $ & $[213]$ & $    0.59579 $  \\
      $[132]$ & $    0.17137 $ & $[312]$ & $    0.17137 $ & $[143]$ & $   -0.27083 $  \\
      $[413]$ & $   -0.27083 $ & $[220]$ & $   -0.56137 $ & $[222]$ & $   -2.78379 $ \\
      $[224]$ & $    2.41676 $ & $[231]$ & $    0.31104 $ & $[321]$ & $    0.31104 $  \\
      $[233]$ & $    0.45382 $ & $[323]$ & $    0.45382 $ & $[242]$ & $    0.38115 $  \\
      $[422]$ & $    0.38115 $ & $[244]$ & $   -1.69388 $ & $[424]$ & $   -1.69388 $  \\
      $[246]$ & $    1.40664 $ & $[426]$ & $    1.40664 $ & $[330]$ & $   -0.07836 $  \\
      $[440]$ & $   -0.17713 $ & $[442]$ & $   -0.52246 $ &    &   \\
    \hhline{======}
    \end{tabular}
    \caption{The non zero $\sigma_{L_1, L_2, L_3}$ coefficients for the HP model and $k=5$.}
    \label{tble:HP_sigma_k5}
\end{table}

\subsubsection{Simulation Results}

\picW = 6cm
\begin{figure*}
    \centering
    \subfigure[$k=3$]{\picL{HP_P-rho_k3.ps}}
    \subfigure[$k=3$]{\picL{HP_P2-rho_k3.ps}}
    \subfigure[$k=5$]{\picL{HP_P-rho_k5.ps}}
    \subfigure[$k=5$]{\picL{HP_P2-rho_k5.ps}}
    \caption{Results from constant $NPT$ simulations of the truncated Stone expansion model obtained using
    $k=3$(a,b) and $k=5$(c,d). The curves for $k=3$ and $k=5$ correspond, respectively, to system sizes
    of $N=1250$ and $N=1000$.}
    \label{fig:HP_phaseDia}
\end{figure*}


Our first simulations were performed on a system of 1250 particles
with elongation $k=3$ using constant {\em NPT} Monte Carlo (MC)
techniques. This system was chosen since it was shown in
Ref.~\cite{Berardi_Ricci_01} that the attractive version of this
model with elongation $k=3$ has isotropic, nematic and smectic
phases. In addition to the normal positional and orientational MC
moves, one fifth of the attempted particle moves were orientation
inversions, implemented through the reversal of the appropriate
$\vecth{u}_i$ vector. Volume change moves were attempted, on
average, once every two MC sweeps, where one sweep represents one
attempted move per particle in the system.
Within these volume
change moves, each box dimension was allowed to change
independently so as to minimise the influence of the periodic
boundary conditions during the formation of possible smectic
phases~\cite{Dominguez_Velasco_02}.
%CHANGE2
%
Typically, run lengths of $0.5\ \mathrm{to}\ 1 \times 10^6$ MC
sweeps were used for equilibration and production phases, but at the highest densities
considered, equilibration runs were extended up to $5 \times 10^6$ MC sweeps. Two separate
simulation sequences were performed. The first was a compression sequence starting from a low
density phase with a fully isotropic initial distribution of particle orientations. The second
was an expansion sequence, the starting configuration for which was generated by taking a high
density configuration obtained from the compression sequence and inducing the particle
orientations to align with the $(0,0,1)$ direction. This was achieved by applying a uniform
field with this orientation to the system and assuming a strong molecular coupling via a
positive dielectric anisotropy.

\picW = 6cm
\begin{figure}
    \centering
    \pic{HP_box_NPT_k5_b_N1000_P2.5.ps}
    \caption{Configuration snapshot for the truncated Stone expansion model with $k=5$ and
    $P^{*}=2.5$.}
    \label{fig:HP_k5_snap}
\end{figure}

\picW = 6cm
\begin{figure*}
    \centering
    \subfigure[$k=3$]{\picL{HP_dc-t_NVT_N1000_k3_S1.ps}}
    \subfigure[$k=5$]{\picL{HP_dc-t_NVT_N1000_k5_S1.2.ps}}
    \caption{Particle mean square displacement curves obtained for the truncated Stone expansion model with
    $k=3$ and $k=5$. These data were obtained from constant NVT MC simulations.
    One sweep corresponds to $N$ Monte Carlo moves}
    \label{fig:HP_dc}
\end{figure*}

The equation of state and nematic, $<P_2>$, and polar, $<P_1>$ order parameter data~\cite{OP}
obtained from these simulations are shown in Fig.~\ref{fig:HP_phaseDia}a. Surprisingly, the
results obtained from the compression sequence show no spontaneous ordering, and at all
densities $<P_2>$ falls short of the values typical of a nematic phase. In contrast, the expansion
sequence (with a field-aligned initial configuration) performed on this system has some
reasonably high $<P_2>$ values, consistent with nematic order being present at the higher
densities considered. The discrepancy between these two sets of order parameter values is also
seen in the equation of state data and indicates a failure of this system to equilibrate at
densities $\rho^{*} > 0.30$. We return to the causes of this non-equilibration below.


An equivalent compression sequence was performed using a system of 1000 particles with
elongation $k=5$. This system was studied since increasing particle shape anisotropy generally
promotes mesophase formation. While the equation of state data obtained for this system showed a
slight inflection and $<P_2>$ attained values of 0.3 (Fig.~\ref{fig:HP_phaseDia}b), the behaviour
expected for an isotropic-nematic transition was again absent. Configuration snapshots from high
density runs performed using this system ({\em e.g.} Fig.~\ref{fig:HP_k5_snap}) showed that the
modest order parameter values resulted from the formation of numerous bi-layer-like domains.
While the local order within these domains was very high, orientational correlations between the
domains were weak. This multi-domain structure persisted even when run-lengths were extended
significantly.


The failure of these hard-particle systems to reproduce the density-driven nematic-isotropic
transition shown by the equivalent soft particle model is a surprising result; by contrast, the
nematic-isotropic transition densities of hard gaussian overlap model \cite{DeMiguelDelRio01}
are virtually identical to those of the equivalent (soft) Gay-Berne systems
\cite{DeMiguel_Rull_91,DeMiguel_Rull_91a}. Indeed, the failure of our hard pear systems to form
nematic phases could be taken as an indication that particle shape did not contribute
significantly to the mesophase formation processes seen in Ref.~\cite{Berardi_Ricci_01}. To
assess both this and the non-equilibration noted above, we have measured the particle mobility
in our systems by computing the mean square displacement
\begin{equation}
        \langle\delta r^2(n)\rangle  = \langle({\bf r}_n - {\bf r}_0)^2\rangle
\end{equation}

%
where ${\bf r}_n - {\bf r}_0$ is the displacement vector moved by a particle in $n$ consecutive
MC sweeps and the angled brackets indicate an average over all particles and the run length. In
MC simulations with fixed maximum particle displacement, Brownian diffusion dictates that
$\langle\delta r^2(n)\rangle$ should increase linearly with $n$ in a fluid phase. Instead, the
$\langle\delta r^2(n)\rangle$ data for $k=3$ (fig~\ref{fig:HP_dc}a), show that, as the density
was increased, the mobility of the particles decreased dramatically, indicating the onset of
glassy behaviour. This observation is certainly consistent with our earlier conclusion that
equilibration was not achieved at high densities; for both of our simulation sequences for
$k=3$, the sampling of configuration space will have been poor for $\rho^{*} \geq 0.30$. For the
$k=5$ system, the measured mobility again showed a marked decrease with increase in density,
although it did not reach the very low levels found at $k=3$ (fig~\ref{fig:HP_dc}b). We note,
however, that $\langle\delta r^2(n)\rangle$ does not distinguish between single particle
diffusion and en-masse mobility of larger assemblies such as the bilayer domains seen in
Fig.~\ref{fig:HP_k5_snap}.



The low mobilities found at high densities in these systems can be explained by consideration
of details of the shape parameters obtained by truncating the expansion (\ref{eqn:Stone_sigma1})
at $\{L_1, L_2, L_3\} = 1\ldots 6$. To illustrate this, we show, in Figs. \ref{fig:HP_shapes},
sample shape parameters for parallel and anti-parallel particle configurations (i.e.
$\lp\dotproduct{u_i}{u_j}\rp = -1$ and $1$) for both $k=3$ and $k=5$. These reveal that the
contact functions used in our simulations were not purely convex, as had been supposed, but had
significant ridges. We suggest that in our $k=3$ simulations, these non-convex features were
sufficient to prevent particles from sliding past one another and so gave rise to locked
configurations. For the $k=5$ particles, for which strong local ordering was achieved, we note
that the shape parameter for antiparallel particles (illustrated in Fig. \ref{fig:HP_shapes}d)
has an equatorial ridge which presumably leads to the interlocked bilayer structures so prevalent
in Fig.~\ref{fig:HP_k5_snap}.\\
These problems are similar to those encountered in Ref.~\cite{WilliamsonJackson98} where
simulations were performed using a seven site linear hard sphere chain (LHSC) model. This model
was found to form metastable glassy states in the vicinity of the isotropic-nematic transition
due to the non-convex shapes of the particles which inhibited their ability to slide past one
another. The tendency of these systems to become irretrievably interlocked was overcome in
Ref.~\cite{WilliamsonJackson98} by the use of reptation moves. This solution was not available
to us here, however, since ours is a single-site model.

\picW = 6cm
\begin{figure*}
    \centering
    \subfigure[Parallel particles.]{\pic{HP_3DProf_k3_0.ps}}
    \subfigure[Antiparallel particles.]{\pic{HP_3DProf_k3_PI.ps}}

    \subfigure[Parallel particles.]{\pic{HP_3DProf_k5_0.ps}}
    \subfigure[AntiParallel particles.]{\pic{HP_3DProf_k5_PI.ps}}
    \caption{Contact surfaces for the truncated Stone expansion model
    with $k=3$(a,b) and $k=5$(c,d).}
    \label{fig:HP_shapes}
\end{figure*}

Once in a stable nematic phase, the LHSC model proved to be reasonably well behaved,
exhibiting a stable nematic region and undergoing a reversible nematic-smectic A
transition. This raises the question of whether the glassy behaviour we have observed here is
simply a simulation bottleneck associated with the isotropic-nematic transition or a genuine
pre-empting of the nematic phase by a glass. While we are not in a position to give a
categorical answer to this question, the evidence we do have suggests the latter to be the case.
All of our $k=3$ simulations with $\rho^{*} > 0.30$ ({\em i.e.} those in both the compression
{\em and} expansion sequences) had low particle mobilities, the effective diffusion coefficient
decreasing monotonically with increase in applied pressure. This indicates that if there is
a region of fluid, nematic phase-stability, it lies beyond the pressure values considered here;
we have certainly found no evidence that the nematic phase seen in Ref.~\cite{Berardi_Ricci_01}
is preserved when the $k=3$ model is stripped of its attractive interactions. From this change
of phase behaviour, we infer that, for this system, the presence of attractive interactions
affects the local packing of the particles -- the attractive wells, being located at $r> \sijr$,
provide a means by which the particles can escape from the inter-locked arrangements that
dominate the equivalent hard particle system at high densities. We are not aware of any other
model system for which both shape anisotropy \emph{and} attractive interactions are required to
promote a nematic-isotropic transition. For the $k=5$ hard particle system, while the measured
mobility did decrease with increase in density, it did not drop as far as that found at $k=3$.
That said, the tendency of this system to form local bilayer-like packing arrangements is in
conflict with the usual mechanisms of nematic phase formation ({\em e.g.} diverging
orientational correlations), leading us to conclude that here, too, the nematic phase is
probably never stable.

Faced with this unexpected phase behaviour, we present, in the following subsection, an
alternative, parametric approach to developing non-centrosymmetric single-site models. By
applying this approach to the B\'{e}zier pears used as a basis for the truncated Stone expansion models
used in this subsection, we then derive a series of pear-shaped models for different particle
elongations and perform MC simulations to investigate their ability to form mesophases.

%=======================================================================================================%
%           OUR METHOD
%=======================================================================================================%
\subsection{The parametric hard gaussian overlap model} \label{s:GBP_method}

%=======================================================================================================
%=======================================================================================================
\subsubsection{Computation of the contact distance}

We start with the generalised expression for the shape parameter which governs the interaction
between a pair of uniaxial, but non-identical, ellipsoidal gaussians~\cite{CleaverCare96}. This
expression, which itself is an approximation to the hard ellipsoid contact function for
non-identical ellipsoids~\cite{Rasmussen}, gives the contact distance between particles $i$ and
$j$ of elongations $l_i$, $l_j$ and breadths $d_i$, $d_j$ as
\begin{widetext}
\begin{equation}
\sijr = \sigma_0 \left[ 1 - \chi\left\{
    \frac{\alpha^2\dotProdP{\rij}{\ui}^2 + \alpha^{-2}\dotProdP{\rij}{\uj}
    -2\chi\dotProdP{\rij}{\ui}\dotProdP{\rij}{\uj}\dotProdP{\ui}{\uj}  }{1-\chi^2\dotProdP{\ui}{\uj}^2}
    \right\}\right]^{-\frac{1}{2}}
\end{equation}
with
\begin{eqnarray*}
    \sigma_0 &=& \sqrt{\frac{d^2_i + d^2_j}{2}}               \\
    %
    \alpha^2 &=& \left[ \frac{ (l^2_i-d^2_i)(l^2_j+d^2_i)}
    {(l^2_j-d^2_j)(l^2_i+d^2_j)}\right]^{\half}         \\
    %
    \chi &=& \left[ \frac{(l^2_i-d^2_i)(l^2_j-d^2_j)}
    {(l^2_j+d^2_i)(l^2_i+d^2_j)}\right]^{\half}.         \\
    %
\end{eqnarray*}
If, alternatively, brackets containing the length and breadth values are grouped as
\begin{equation*}
    \begin{array}{cc}
    A = (l^2_i-d^2_i) &B = (l^2_j-d^2_j) \\
    C = (l^2_j+d^2_i) &D = (l^2_i+d^2_j),
    \end{array}
\end{equation*}

the shape parameter can be rewritten as

\begin{equation}
\label{eqn:GGB}
    \sijr = \sigma_0 \left[ 1 - \frac{ AC\dotProdP{\rij}{\ui}^2 + BD\dotProdP{\rij}{\uj}^2
    - 2AB\dotProdP{\rij}{\ui}\dotProdP{\rij}{\uj}\dotProdP{\ui}{\uj} }
    {CD - AB\dotProdP{\ui}{\uj}^2}
    \right]^{-\half}.
\end{equation}
\end{widetext}
In practice, this form, being free of possible division by zero or complex numbers, is better
suited for implementation in computer simulation codes.

The limitation of the expression (\ref{eqn:GGB}) is that it is restricted to particles with
ellipsoidal symmetry. The thrust of this subsection is to illustrate that, since
eqn.(\ref{eqn:GGB}) is valid for any set of particle axis lengths $l_i$, $l_j$, $d_i$, $d_j$, it
can also be used for some situations in which these axis lengths, rather than being held fixed,
are allowed to vary parametrically.

As an illustration of this, we consider the properties of a pear-shaped particle. When its
sharp end interacts, it resembles a particle with a relatively large $l/d$ ratio, whereas its
blunt end corresponds to an $l/d$ ratio rather nearer to unity. In generating smooth variation
between these two limiting cases, a multitude of parametric forms is possible: here we restrict
ourselves to making $l_i$ and $d_i$ simple polynomials of the polar angle
$\dotProdP{\rij}{\ui}$, that is
\begin{eqnarray}
        d_i(\dotProd{\rij}{\ui}) &=& a_{d,0}
        +\ldots + a_{d,n}(\dotProd{\rij}{\ui})^n\\     \label{eqn:drui}
        %
        l_i(\dotProd{\rij}{\ui}) &=& a_{l,0} +\ldots + a_{l,m}(\dotProd{\rij}{\ui})^m.
\end{eqnarray}

Whilst this restriction limits, to some extent, the range of particle shapes available, it has
the advantage that the effect of this parametric approach on $\sijr$ is transparent: through the
coefficients $A$, $B$, $C$ and $D$, it simply introduces higher order dependence of $\sijr$ on
the scalar products $\dotProdP{\rij}{\ui}$ and $\dotProdP{\rij}{\uj}$. On a more practical
level, we note that restricting the shape parameter expansion to polynomials in these dot
products has the benefit of making it readily usable in a molecular dynamics (MD) simulation. To
reflect the generalisation introduced by this approach, we name the resultant class of model the
parametric hard gaussian overlap (PHGO).

As we show in the following subsections, by working through a specific example, this parametric
approach can be used to generalise the HGO model to give shape parameters which approximate
other convex and axially symmetric particle shapes. We note, however, that since the PHGO's
departure from the conventional HGO shape parameter is based, in part, on the interparticle
vector, $\vect{r}_{ij}$, it ignores some possible close contacts and is {\em not}, therefore,
suitable for modelling particles with concave surface regions ({\em e.g} dumbells). For systems
which do satisfy the convexity criterion, however, the approximations inherent in the PHGO
approach are outweighed by the advantages: it yields an analytical form for the shape parameter,
making it suitable for either MC or MD simulation and is easy to embed into a Gay-Berne type
model; it introduces little computational overhead, beyond that required to simulate the
standard HGO model; and, since it is a simple generalisation of the HGO model, it can readily be
used to represent some or all of the particles in a multi-component mixture or a multi-site
model (indeed, extension to polydispersity and/or dynamic particle shape variation is quite
straightforward).

%=======================================================================================================
%=======================================================================================================
\subsubsection{Parameterising B\'{e}zier pears}

\picW = 8cm
\begin{figure}
    \centering
    \picL{k3_GBP.ps}
    \caption{B\'ezier point geometry corresponding to the pear shape used for the PHGO model.
    For an elongation $k=3$, $\sigma_0 = 1.0$ and $h=2.0$.}
    \label{fig:bezierPear_k3}
\end{figure}


In order to test the PHGO approach, we have used it to generate shape parameters for
pear-shaped particles based on B\'{e}zier-curves (i.e. the same target system as was employed in
Ref.~\cite{Berardi_Ricci_01}). For this, the ideal particle shape was first determined
geometrically using a combination of two B\'{e}zier curves as shown in
Fig.~\ref{fig:bezierPear_k3}. This geometry is very similar to that used in
Ref.~\cite{Berardi_Ricci_01}, the exception being that we took points $q_2$ and $q_3$ to be
coincident. This has the advantage of making the B\'{e}zier points' coordinates more easily
scalable with the desired length to breadth ratio $k$. The coordinates of the B\'{e}zier points
used are given in Table~\ref{tble:bezierCoords}.

\tabul = 1cm
\begin{table}
    \centering
    \begin{tabular}{ccc}
    \hhline{===}
    \TAB q\TAB  &\TAB x\TAB &\TAB y\TAB             \\
    \hhline{---}
    $q_1$          &$-\frac{1}{2}\sigma_0$   &0.0                    \\
    $q_2,$$q_3$    &0.0                      &$\frac{2}{3}k\sigma_0$  \\
    $q_4$          &$\frac{1}{2}\sigma_0$    &0.0                    \\
    $q_5$          &1.0                      &$-\frac{2}{3}k\sigma_0$ \\
    $q_6$          &-1.0                     &$-\frac{2}{3}k\sigma_0$ \\
    \hhline{===}
    \end{tabular}
    \caption{Coordinates for the pear shape Bezier points used for the PGHO model.}
    \label{tble:bezierCoords}
\end{table}


From these points, it is possible to extract the coordinates of any point on the
curve~\cite{graphicsMaths}. By taking these B\'{e}zier curves to correspond to the contact
surface between a pear shaped particle $i$ and a point probe $j$ ({\em i.e.} taking $l_j=d_j=0$
in eqn.(\ref{eqn:GGB})), it can readily be shown that these points need to be fitted by the
particle-point shape parameter
\begin{equation}
\begin{split}
    &\sir =\\
    &\frac{d_i(\dotProd{\rij}{\ui})l_i(\dotProd{\rij}{\ui})}
    {\left[l_i^2(\dotProd{\rij}{\ui}) +
    \lp\dotProd{\ui}{\rij}\rp^2\lp d_i^2(\dotProd{\rij}{\ui})- l_i^2(\dotProd{\rij}{\ui})
    \rp\right]^{\frac{1}{2}}}.
    \label{eqn:GBP_sir}
\end{split}
\end{equation}
To achieve this, various polynomial forms were considered for the expansions of
$d_i(\dotProdP{\rij}{\ui})$ and $l_i(\dotProdP{\rij}{\ui})$ in eqns.(\ref{eqn:drui}), these
being fitted numerically using a simplex least squares minimisation
method~\cite{Numerical_Recipes}. Good fits were obtained by taking 10 terms in the particle
breadth polynomial and 2 in the length polynomial. Full sets of the coefficients obtained are
given in Table~\ref{tble:dili} for particles with overall aspect ratios $k=3$, 4 and 5.

\tabul = 1.0cm
\begin{table}
    \centering
    \begin{tabular}{cccc}
    \hhline{====}
    $a_{d,0} $       &   0.501852454     &    0.501377232    &    0.497721868    \\
    $a_{d,1} $       &   -0.141145314    &   -0.129608758    &   -0.123155821    \\
    $a_{d,2} $       &   -0.060542359    &   -0.074219217    &    0.024405876    \\
    $a_{d,3} $       &   0.225813650     &   0.484166441     &    0.723627215    \\
    $a_{d,4} $       &   0.832274021     &   0.923492941     &    0.389831429    \\
    $a_{d,5} $       &   -1.015039575    &   -1.987232902    &   -3.018638148    \\
    $a_{d,6} $       &   -2.504045172    &   -2.943008017    &   -1.951629076    \\
    $a_{d,7} $       &   1.375313426     &   2.808075172     &    4.413215403    \\
    $a_{d,8} $       &   3.196830129     &   3.815344782     &    2.998417509    \\
    $a_{d,9} $       &   -0.699241457    &   -1.426641750    &   -2.241573216    \\
    $a_{d,10}$       &   -1.430400139    &   -1.682476460    &   -1.416614353    \\
    \hhline{====}
    $a_{l,0} $       &   1.498259615     &   1.995906501     &    2.493069403    \\
    $a_{l,1} $       &   -0.002027616    &   -0.004518187    &   -0.008067236    \\
    \hhline{====}
    \end{tabular}
    \caption{Values of the $a_{d,\alpha}$ and $a_{l,\alpha}$ for the PHGO model with $k=3$, 4 and 5}
    \label{tble:dili}
\end{table}

\picW = 7cm
\begin{figure}
    \centering
    \picL{GBP_profile_k5.ps}
    \label{fig:fitProfile}
    \caption{Fit of the PHGO model (line) to the B\'ezier curve (points) for k=$5$. Equivalent
    curves for $k=3$ and $4$ are similar but with better agreement.}
\end{figure}

\picW = 6cm
\begin{figure*}
    \centering
    \subfigure[Parallel particles.]{\pic{GBP_3DProf_k5_0.ps}}
    \subfigure[Anti-parallel particles]{\pic{GBP_3DProf_k5_PI.ps}}
    \caption{Contact surfaces for the PHGO model with an elongation $k=5$. For shorter
    elongations, the shapes are similar but smoother.}
    \label{fig:CD}
\end{figure*}


In order to assess the accuracy of this fitting procedure, we present in
Fig.~\ref{fig:fitProfile}a plot comparing the target B\'{e}zier curve and the corresponding
fitted shape parameter for $k=5$. The strong correspondence between these data sets, whilst
encouraging, does not guarantee that the particle-particle potential will be as required. To
assess this more fully, we computed the contact surfaces between two pears as a function of
$\rij$ uniformly distributed on the unit sphere. For this, the orientations $\ui$ and $\uj$ were
held fixed and the three cases \mbox{$\ui=\uj=(0,0,1)$}, \mbox{$\ui = -\uj = (0,0,1)$} and
\mbox{$\ui = (0,0,1),\uj = (0,1,0)$} were considered. The parallel and anti-parallel surfaces
are shown for $k=5$ in Figs.~\ref{fig:CD}. As required, an approximately ellipsoidal contact
surface is obtained when the two particles are parallel, and a pear shape when they are
anti-parallel. The orthogonal case (which is not shown since it is rather unpreposessing), is a
more severe test since here the point of contact is often well away from the line of centres. We
have found that this case gives an asymmetrical lobe shape which we have found to be consistent
with the equivalent surface given by the models of Ref.~\cite{Berardi_Ricci_01}. Importantly, we
note that all three of the PHGO contact functions considered in this way are almost perfectly
convex, and, so, should not be prone to the locking-up suffered by the truncated Stone function expansion
models simulated in subsection \ref{s:Stone}. In the next subsection, we go on to investigate
both this assertion and the general applicability of our parametric models to molecular
simulation by performing constant {\em NPT} Monte Carlo compression sequences on PHGO pear
systems with $k=3$, 4 and 5.



%=======================================================================================================%
%           RESULTS
%=======================================================================================================%
\subsubsection{Simulation Results} \label{s:GBP_results}

\picW = 6cm
\begin{figure*}
    \centering
    \subfigure[$k=4$]{\picL{GBP_P-rho_k4.ps}}
    \subfigure[$k=4$]{\picL{GBP_P2-rho_k4.ps}}
    \subfigure[$k=5$]{\picL{GBP_P-rho_k5.ps}}
    \subfigure[$k=5$]{\picL{GBP_P2-rho_k5.ps}}
    \caption{Results from constant NPT simulations of the PHGO model obtained with $k=4$(a,b) and
    $k=5$(c,d) and system sizes of $N=1000$ particles.}
    \label{fig:phaseDia}
\end{figure*}

\picW = 6cm
\begin{figure*}
    \centering
    \subfigure[$g_\parallel(r_\parallel)$]{\picL{GBP_grParallel_k4.ps}}
    \subfigure[$g_\perp(r_\perp)$]{\picL{GBP_grPerpLayer_k4.ps}}
    \caption{Pair correlation functions $g^{\rm mol}_\perp(r_\perp)$ and
    $g^{\rm mol}_\parallel(r_\parallel)$ resolved parallel and perpendicular to the molecular
    orientation for PHGO particles with elongation $k=4$.}
    \label{fig:PCF_GBP_k4}
\end{figure*}

\begin{figure*}
    \centering
\picW = 5cm
    \subfigure[isotropic]{\pic{GBP_box_k4_iso.ps}}
\picW = 4.5cm
    \subfigure[domain ordered]{\pic{GBP_box_k4_fun.ps}}
    \caption{Configuration snapshots of systems of $N=1000$ PHGO particles with $k=4$ at
    $P^{*}=1.80$(a) and $5.00$(b).}
    \label{fig:phaseSnaps_k4}
\end{figure*}


In order to test our models, we have examined their phase
behaviour via MC simulations in the isothermal-isobaric ensemble
using $N=1000$ particles and a series of increasing pressures.
Three particle elongations, $k=3,4$ and $5$, have been studied,
their phase behaviour being assessed through the variation of the
number density, $\rho^{*}$, and the polar and nematic order
parameters, $<P_1>$ and $<P_2>$, respectively.
The volume
change scheme used here was the same as that used with the Stone
expansion model and
%CHANGE3
typical run lengths were $0.5\times 10^6$ to $1\times 10^6$ for
equilibration and production.



The improved equilibration behaviour of these
systems meant that the very long runs used previously were
unnecessary here. The results of these simulations are illustrated
by the plots presented in Figs.~\ref{fig:phaseDia}. For the sake
of brevity, results for $k=3$ system are not shown here since they
are qualitatively the same as those found for $k=4$.


The behaviour of the order parameters for the two lower values of $k$ indicates that there was
no long range orientational ordering for these systems. $<P_1>$ remained nearly constant at around
$0.0$ while $<P_2>$ failed to reach the values ($>0.6$) characteristic of nematic order. However,
for both elongations, $P(\rho^{*})$ had an inflection suggesting proximity to a weak phase
transition. These features coincided, approximately, with the broad maxima seen in the
corresponding $<P_2>$ curves. This suggests that even if no nematic phase is shown by these
systems, some other high density phase may have been formed here. We note that for these systems
the particle mobilities, monitored via their mean square displacements, changed little
throughout the density range considered in these simulations.

\picW = 6cm
\begin{figure*}
    \centering
    \subfigure[$g_\parallel(r_\parallel)$]{\picL{GBP_grParallel_k5.ps}}
    \subfigure[$g_\perp(r_\perp)$]{\picL{GBP_grPerpLayer_k5.ps}}
    \caption{Pair correlation functions $g_\perp(r_\perp)$ and
    $g_\parallel(r_\parallel)$ resolved parallel and perpendicular to the director
    for PHGO model particles with elongation $k=5$.}
    \label{fig:PCF_GBP_k5}
\end{figure*}

\begin{figure*}
    \centering
    \picW = 4.5cm
    \subfigure[isotropic]{\pic{GBP_box_k5_iso.ps}}
    \picW = 4.9cm
    \subfigure[nematic]{\pic{GBP_box_k5_nem.ps}}
    \picW = 3.7cm
    \subfigure[smectic]{\pic{GBP_box_k5_smA.ps}}
    \caption{Configuration snapshots of systems of $N=1000$ PHGO particles with $k=5$ and
    $P^{*}=1.00$(a), $1.50$(b) and $2.80$(c).}
    \label{fig:phaseSnaps_k5}
\end{figure*}


More insight into the high density arrangements adopted by these systems has been obtained
through computation of the pair correlation functions resolved parallel
($g^{\rm mol}_\parallel(r_\parallel)$) and perpendicular ($g^{\rm mol}_\perp(r_\perp)$) to the
particle orientations $\vecth{u}_i$ (the superscript mol is used to indicate that molecular,
rather than director orientations were used to calculate these functions). These are shown in
Fig.~\ref{fig:PCF_GBP_k4} for $k=4$, and indicate local smectic-like arrangements with anti-parallel
alignment of nearest neighbours within layers. However, the decay of the oscillations in $g^{\rm
mol}_\parallel(r_\parallel)$, coupled with the low corresponding $<P_2>$ values, indicate the
absence of long-ranged smectic order. Configurations snapshots illustrate these structures more
clearly. As can be seen from Fig.\ref{fig:phaseSnaps_k4}, with $k=4$, upon
compression, these systems formed convoluted, space filling bilayer structures, the bilayers
being planar in some regions and highly curved in others. The presence of these curved regions
makes these systems qualitatively different from those seen in the $k=5$ truncated Stone expansion model
(recall Fig~\ref{fig:HP_k5_snap}), where the orientations of the bilayer domains changed
discontinuously with position.


For the $k=5$ PHGO pear model, a very different situation was
found. While $<P_1>$ remained resolutely at zero for all
densities, confirming an absence of polar order, $<P_2>$ showed
the well known `S' shape characteristic of an isotropic-nematic
transition and reached the values expected for an orientationally
ordered phase. A corresponding plateau in the $P(\rho^{*})$ curve
and a configuration snapshot (Fig.~\ref{fig:phaseSnaps_k5})
confirm this assessment. At higher densities, secondary features
are apparent in both $P_2(\rho^{*})$ and $P(\rho^{*})$ indicating
the presence of a second phase transition. The nature of this
third phase was determined by computation of the pair correlation
functions resolved parallel ($g_\parallel(r_\parallel)$) and
perpendicular ($g_\perp(r_\perp)$) to the director $\vecth{n}$ as
shown in Figs.~\ref{fig:PCF_GBP_k5}(a) and (b). These graphs show
that for pressures above that of the second phase transition,
$g_\parallel(r_\parallel)$ became periodic, indicating the onset
of a smectic phase. Moreover, the decay of the oscillations in
$g_\perp(r_\perp)$ further indicates this to be a smectic A. A
snapshot configuration from this high density region confirms this
identification, a highly interdigitated bilayer smectic ${\rm
A_2}$ phase being seen, in which the molecules in adjacent layers
are almost perfectly anti-parallel. This anti-parallel arrangement
is apparent from the peak splitting observed in
$g_\parallel(r_\parallel)$, the short peaks corresponding to the
distinct natural separations of particles in the two possible
anti-parallel arrangements. Similar behaviour has been observed in
simulations of Gay-Berne systems with longitudinal terminal
molecular dipoles~\cite{BerardiOrlandi03}.
Comparison of
the $g_\parallel(r_\parallel)$ and $g_\perp(r_\perp)$ data
obtained at different pressures in the range $P=[2.4:3.8]$, shows
an interesting compressibility behaviour. Upon increasing the
pressure in this range, the system density rises and intra-layer
particle separations decrease slightly but the bilayer separations
increase (Figs.~\ref{fig:PCF_GBP_k5}(a) and (b)). From the
measured $g_\parallel(r_\parallel)$ data it is found that the
distance between the main peaks, which corresponds to the
separation of the bilayers, increases from $7.38$ to $7.66$. The
distance from the main peak to the first minor peak, which
corresponds to the strongly interdigitating `tail-tail'
configuration increases from $2.49$ to $2.76$, whereas that to the
second minor peak, corresponding to the weakly interdigitating
`head-head' alignment remains effectively constant at $4.85$.
Thus, the in-plane compression induced by this increase in
pressure leads to a 10\% increase in the separation within the
interdigitated bilayers that comprise the smectic ${\rm A_2}$
phase.
%CHANGE4


%=======================================================================================================%
%           CONCLUSION                                                          %
%=======================================================================================================%
\section{Discussion and Conclusions}
\label{S:GBP_conclusions}

In this paper, we have investigated the mesogenic behaviour of two
classes of model hard pear-shaped particles, both based on a
target shape built using a B\'{e}zier curve. The first model
considered used a truncated Stone expansion approach to generate the
particle-particle contact distance numerically. Although the
Gay-Berne version of this model was well behaved, giving nematic
and smectic A mesophases~\cite{Berardi_Ricci_01}, these were
\emph{not} found on removal of the attractive interactions.
Rather, the non-convex regions of the contact surfaces induced the
particles to interlock, leading to the formation of multi-domain
and glassy phases. For this model, therefore, it appears that the
nematic-isotropic transition is not driven by particle shape
alone: long-ranged orientational order is only seen when the shape
is softened somewhat, by the incorporation of attractive
interactions.

The second hard-pear model considered here was based on the PHGO
approach, a route to non-centrosymmetric shape parameters which we
have introduced in this paper. While the PHGO shape parameter is
not determined from a full evaluation of the appropriate gaussian
integral, the approximation it makes, that locally a
non-centrosymmetric particle closely resembles an appropriately
chosen ellipsoid, is intuitively reasonable. Furthermore, the
computational simplicity and ready transferability of the PHGO
model suggest that it may be of considerable utility in the
generic modelling of self assembling systems. Here, we have found
that the smooth, convex contact surfaces of a PHGO hard pear model
yield stable nematic and bilayered smectic ${\rm A_2}$ phases.
Interestingly, these phases are only seen when the particle aspect
ratio is increased to $k=5$, whereas hard ellipsoid systems are
know to form a nematic with $k$ values as low as
2.75~\cite{FrenkelMulder85}. Future work exploring the behaviour
of the PHGO hard pear model will include a more thorough study of
its flexoelectric properties, and an investigation into the
applicability of the PHGO shape parameter in theoretical
approaches commonly used to study liquid crystals.

\section*{Acknowledgements}

FB acknowledges Sheffield Hallam University's Materials Research Institute for a research
student bursary and the INSTM for financial support through contract no. TMR.FMRX CT970121
in respect of extended visits he made to Bologna. DJC thanks Ian Withers for discussions
which stimulated the development of the PHGO approach. %

%=======================================================================================================%
%           Bibliography                                                        %
%=======================================================================================================%

\begin{thebibliography}{10}

\bibitem{Davidson_Mottram}
{A.J. Davidson and N.J. Mottram}.
\newblock {\em Phys. Rev. E}, 65:051710, 2002.

\bibitem{ZBD}
{G.P. Bryan-Brown, C.V. Brown, I.C. Sage and V.C. Hui}.
\newblock {\em Nature}, 392:365, 1998.

\bibitem{Meyer_69}
{R.B. Meyer}.
\newblock {\em Phys. Rev. Letts.}, 22:918, 1969.

\bibitem{Prost_Marcerou_77}
{J. Prost and J.P. Marcerou}.
\newblock {\em Le Journal de Physique}, 38:315--324, 1977.

\bibitem{ProstMarcerou80}
{J.P. Marcerou and J. Prost}.
\newblock {\em Mol. Cryst. Liq. Cryst.}, 58:259--284, 1980.

\bibitem{Osipov83}
{M.A. Osipov}.
\newblock {\em Sov. Phys. JETP}, 58:6, 1983.

\bibitem{Osipov_84}
{M.A. Osipov}.
\newblock {\em Le Journal de Physique Lettres}, 45:823--826, 1984.

\bibitem{DosovLargarde83}
{I. Dozov et al.}
\newblock {\em Le Journal de Physique Lettres}, 44:L817--L822, 1983.

\bibitem{wei92a}
{D. Wei and G.N. Patey}.
\newblock {\em Phys. Rev. Letts.}, 68:2043, 1992.

\bibitem{wei92b}
{D. Wei and G.N. Patey}.
\newblock {\em Phys. Rev. A}, 46:7783, 1992.

\bibitem{Neal}
{M.P. Neal, A.J. Parker and C.M Care}.
\newblock {\em Molec. Phys.}, 91:603, 1997.

\bibitem{Stelzer_Berardi_99}
{J. Stelzer, R. Berardi and C. Zannoni}.
\newblock {\em Chem. Phys. Letts.}, 299:9, 1999.

\bibitem{Billeter_Pelcovits_00}
{J.L. Billeter and R.A. Pelcovits}.
\newblock {\em Liq. Cryst.}, 27:1151, 1997.

\bibitem{Berardi_Ricci_01}
{R.Berardi, M. Ricci and C. Zannoni}.
\newblock {\em ChemPhysChem}, 7:443, 2001.

\bibitem{Zewdie98a}
{H. Zewdie}.
\newblock {\em J. Chem. Phys.}, 108:2117, 1998.

\bibitem{Zewdie98b}
{H. Zewdie}.
\newblock {\em Phys. Rev. E}, 57:1793, 1998.

\bibitem{hard_rev}
{M.P. Allen, G.T. Evans, D. Frenkel and B. Mulder}.
\newblock {\em Adv. Chem. Phys.}, 86:1, 1993.

\bibitem{CleaverCare96}
{D.J. Cleaver, C.M. Care, M.P. Allen and M.P. Neal}.
\newblock {\em Phys. Rev. E}, 54:559, 1996.

\bibitem{Stone}
{A.J. Stone}.
\newblock {\em Molec. Phys.}, 36:241, 1978.

\bibitem{Dominguez_Velasco_02}
{H. Dom\'{\i}nguez, E. Velasco and J. Alejandre}.
\newblock {\em Molec. Phys.}, 100:2739, 2002.

\bibitem{OP}
{P. Pasini, C. Zannoni}.
\newblock {\em Advances in the computer simulations of Liquid crystals, chapter
  2}.
\newblock Kluwer Academic Publisher, Dordrecht, 1998.

\bibitem{DeMiguelDelRio01}
{E. de Miguel, E. Mart\'{i}n del R\'{i}o}.
\newblock {\em J. Chem. Phys.}, 115(19):90729082, 2001.

\bibitem{DeMiguel_Rull_91}
{E. de Miguel et al}.
\newblock {\em Molec. Phys.}, 72(3):593--605, 1991.

\bibitem{DeMiguel_Rull_91a}
{E. de Miguel et al}.
\newblock {\em Molec. Phys.}, 74(2):405--424, 1991.

\bibitem{WilliamsonJackson98}
{D.C. Williamson, G. Jackson}.
\newblock {\em J. Chem. Phys.}, 108:10294, 1998.

\bibitem{Rasmussen}
{J. W. Perram et al.}
\newblock {\em Phys. Rev. E}, 54:6565, 1996.

\bibitem{graphicsMaths}
{D.F. Rogers and J.A. Adams}.
\newblock {\em Mathematical elements for computer graphics, 2nd ed.}
\newblock Mc Graw-Hill, 1990.

\bibitem{Numerical_Recipes}
{W.H. Press, S.A. Teukolsky, W.T. Vetterling and B.P. Flannery}.
\newblock {\em Numerical recipes in C, the art of scientific computing}.
\newblock Cambridge University Press, 1990.

\bibitem{BerardiOrlandi03}
{R. Berardi, S. Orlandi and C. Zannoni}.
\newblock {\em Phys. Rev. E}, in the press, 2003.

\bibitem{FrenkelMulder85}
{D. Frenkel and B.M. Mulder}.
\newblock {\em Molec. Phys.}, 55:1171, 1985.

\end{thebibliography}

\end{document}
