
%\documentclass[aps,pre,11pt,onecolumn,superscriptaddress]{revtex4}
\documentclass[aps,10pt,twocolumn]{revtex4}


%============================================================================
%		packages
%============================================================================
\usepackage[final]{graphics}	% for graphics
\usepackage[final]{graphicx}	% for graphics
\usepackage{subfigure}		% allow subfigure
\usepackage{amssymb}		% font
\usepackage{amstext}		% ams latex package
\usepackage{amsmath}		% font
\usepackage{amsfonts}		% font
\usepackage{amsbsy}		% font
\usepackage{hhline}		% nice lines is tables
\usepackage{xspace}		% add space at the end of macros
\usepackage{color}
\usepackage[french]{babel}
\usepackage{calc}

%--------------------------
%	newcommands
%--------------------------
\newcommand{\etal}{\emph{et al.}\@\xspace}
\newcommand{\cf}{\emph{c.f.}\@\xspace}
\newcommand{\mrm}[1]{\ensuremath{\mathrm{#1}}\xspace}
\newcommand{\SiOTwo}{\ensuremath{\mrm{SiO_2}}\xspace}

\newlength{\picH}	% picture height
\newlength{\picW}	% picture width
\newcommand{\picA}{270}	% picture angle

\picW = 8cm

\newcommand{\pic}[1]{\includegraphics[width=\picW]{#1}}
\newcommand{\picL}[1]{\includegraphics[height=\picW, angle=\picA]{#1}}
\newcommand{\picB}[1]{\fbox{\includegraphics[width=\picW]{#1}}}
\newcommand{\picLB}[1]{\fbox{\includegraphics[height=\picW, angle=\picA]{#1}}}

\newcommand{\e}[1]{\ensuremath{e^{#1}}}

%--------------------------
%	page layout
%--------------------------
%%\textheight=254mm
%%\voffset=-9mm

\parindent=0pt

\begin{document}

\graphicspath
{
{../imgs/}
}

%%=============================
%%	Titre
%%=============================
\title{Structure induite par passage d'ondes de choc dans les verres silicate}
\author{F. Barmes}
\affiliation{Centre Europ\'een de Calcul Atomique et Mol\'eculaire, Lyon}

\author{L. Soulard}
\affiliation{CEA-DAM. Bruy\`ere le Ch\^atel}



\noaffiliation

%%=============================
%%	Abstract
%%=============================
\begin{abstract}
\textbf{R\'esum\'e~:} On d\'ecrit ici une m\'ethode permettant de mod\'eliser et d'analyser la
structure induite par le passage d'une onde de choc dans un verre silicate. On montre aussi que
les r\'esultats obtenus par les simulations sous choc et celles utilisant l'Hugoniostat sont
tr\`es similaires, validant ainsi la seconde m\'ethode qui permet l'utilisation de syst\`emes
beaucoup plus petit.\\ 


\textbf{Abstract~:} Here we describe a method allowing modeling and analysis of the shock
induced structural changes in silicate glasses. We show that the results obtained using
non-equilibrium molecular dyanmics and those obtained using the Hugoniostat ensemble are very
similar, thus validating the latter method which allows usage of much smaller systems.
\end{abstract}

\maketitle


%%=========================================================================================================================
%%	Introduction
%%=========================================================================================================================
\section*{Introduction}

Le but de cette \'etude est de construire et de valider les m\'ethodes et outils n\'ecessaires \`a la
mod\'elisation de verres silicates en conditions de choc \`a l'\'echelle mol\'eculaire.\\
La description r\'ealiste de verres de \SiOTwo n\'ecessiterait le d\'eveloppement d'un potentiel
compliqu\'e comprenant des forces \`a deux et trois corps afin de rendre compte de
l'organisation structurelle du mat\'eriau mais aussi des forces \`a courte et longue port\'ee
(interactions coulombienne). Cette \'etude se concentre sur la validation
des m\'ethodes \`a employer afin de pouvoir mod\'eliser et analyser un verre silicate en
conditions de choc~; on choisit donc, dans un premier temps, de r\'esoudre le probl\`eme en
utilisant un potentiel facilement ma\^{\i}trisable afin de pouvoir se concentrer sur la mise en place
des m\'ethodes de simulation. On montrera toutefois que le potentiel utilis\'e permet d'obtenir
des r\'esultats qualitativement correct.

%%=========================================================================================================================
%%	Potentiel et m\'ethodes
%%=========================================================================================================================
\section{Mod\'elisation d'un verre silicate}


\subsection{Potentiel d'interaction}
Le potentiel d'interaction choisit pour la description des forces inter-atomique, est le
potentiel BKS, d\'evelopp\'e par van Beest, Kramer et van Santen~\cite{BKS} coup\'e \`a
$r=r_c$ et que l'on remplace pour $r<r_0$ par un polyn\^{o}me d'ordre deux afin de corriger le caract\`ere
non-physique du potentiel BKS \`a courte port\'ee. Le potentiel utilis\'e devient donc~:
%%
%%
\begin{equation}
\mathcal{V}= 
	\left\{
	\begin{array}{cc}
	%
	\mathcal{V}^{\mrm{BKS}}_c &r_{ij} > r_0 \\
	%
	p(r) = a_{ij}r^2 + b_{ij}r + c_{ij}					&r_{ij} \leq r_0
	\end{array}
	\right.
\label{eqn::BKSFB}
\end{equation}
avec
\begin{equation}
\mathcal{V}^{\mrm{BKS}}_c = \mathcal{V}^{BKS}(r) - \mathcal{V}^{BKS}(r_c) - (r-r_c)\frac{d\mathcal{V}^{BKS}(r_c)}{dr}\\
\end{equation}
%%
%%
La Figure~\ref{fig::BKSFB} donne une repr\'esentation graphique du potentiel final utilis\'e.
\begin{figure}
\picW=7cm
\picL{potBKS.ps}
\caption{Repr\'esentation du potentiel d'interaction inter-atomique utilis\'e dans cette
\'etude. Les courbes pointill\'ees repr\'esentent le potentiel BKS tronqu\'e sans l'utilisation
de p(r).}
\label{fig::BKSFB}
\end{figure}
%%
%%
Dans l'\'equation~(\ref{eqn::BKSFB}), $\mathcal{V}^{BKS}(r)$ est le potentiel BKS~\cite{BKS} qui d\'efinit l'interaction entre
deux atomes $i,j=\left\{\mrm{Si},\mrm{O}\right\}$ de charges $q_i$ et $q_j$ et s\'epar\'es par
une distance $r_{ij}$. Les valeurs des coefficients du potentiel sont donn\'es dans~\cite{BKS}.
%%
\begin{equation}
\mathcal{V}^{BKS}(r_{ij}) = \frac{q_iq_j}{4\pi \epsilon_0 r} +A_{ij}e^{-B_{ij}r_{ij}} - \frac{C_{ij}}{r^6_{ij}} 
\end{equation}
%%
%%
Les coefficients du polyn\^{o}me $p(r)$ sont calcul\'es tels que~:
\begin{equation}
	\left\{
	\begin{array}{ccc}
	\mathcal{V}^{\rm BKS}_c 							&=&	p(r)				\\
	\frac{d}{dr}\mathcal{V}^{\rm BKS}_c					&=& \frac{d}{dr}p(r)	\\
	\frac{d2}{dr2}\mathcal{V}^{\rm BKS}_c				&=& \frac{d2}{dr2}p(r)
	\end{array}
	\right.
\end{equation}




\subsection{Cr\'eation et analyse d'un verre silicate}

Dans le dioxide de silicium \`a l'\'etat solide, le mat\'eriau adopte une structure o\`u les atomes
d'oxyg\`ene sont dispos\'es aux coins de t\'etra\`edre centr\'es autour des atomes de
silicium.  La m\'ethode utilis\'ee pour cr\'eer une telle configuration s'inspire de celle d\'ecrite par
Vashishta~\etal~\cite{VashishtaKalia90} ou Taraskin et Elliott~\cite{TaraskinElliott97}.\\
On part d'une structure de $\mrm{\beta-cristobalite}$~\cite{Wickoff} avec une maille
\'el\'ementaire telle que la masse volumique $\mu=2.2\mrm{g.cm^{-3}}$. On fait alors fondre ce crystal
\`a haute temp\'erature (T=6000K) afin de faire dispara\^{\i}tre toute m\'emoire de la structure
ordonn\'ee. Le liquide ainsi obtenu est alors lentement refroidi via une s\'erie de 13x3
simulations de 25ps
(thermalisation,\'equilibration,production) afin d'amener progressivement le syst\`eme \`a
T=300K.\\

Afin de v\'erifier la validit\'e de la m\'ethode utilis\'ee, les propri\'et\'es structurelles
du verre (g(r),distribution des angles de liaisons, tailles des anneaux) sont
calcul\'ees (Figure~\ref{fig:mkGlass}). La comparaison entre les r\'esultats obtenus et ceux publi\'es par
d'autres auteurs~\cite{SusmanVolin91,RinoEbbsjo93,HorbachKob99,YanCormack} montre que la
structure de notre verre est tout \`a fait correcte, malgr\'e l'a priori simplicit\'e du
potentiel utilis\'e. 

\begin{figure}
\picW=7cm
\subfigure[g(r)]{\picL{mkGlass_gr.ps}}
\subfigure[Distribution des angles de liaisons]{\picL{mkGlass_BAD.ps}}
\subfigure[Distribution de la taille des anneaux]{\picL{mkGlass_SPA.ps}}
\caption{Repr\'esentation de la structure du syst\`eme de \SiOTwo calcul\'ee \`a haute (T=4000K)
et basse (T=300K) temp\'erature. Dans le cas de la distribution des anneaux, les points
($\blacksquare$) repr\'esentent les donn\'ees obtenues dans~\cite{RinoEbbsjo93} \`a 310 K avec
le potentiel VKRE.}
\label{fig:mkGlass}
\end{figure}

%%=========================================================================================================================
%%	Passage d'ondes de choc
%%=========================================================================================================================
\section{Simulation en conditions de choc}

Les simulations en conditions de choc sont effectu\'ees en utilisant des syst\`emes o\`u les
conditions p\'eriodiques ont \'et\'e enlev\'ees dans la direction de propagation de l'onde, dans
notre cas $x$. A l'instant, $t=0$ deux pistons repr\'esent\'es par des plans de masse infinie
entrent en contact avec les faces de gauche et de droite du syst\`eme. Le piston de droite \`a une
vitesse nulle, et le piston de gauche se d\'eplace vers la droite \`a une vitesse $u\neq 0$.
La mise en mouvement du piston de gauche \`a pour effet d'entra\^{\i}ner dans le syst\`eme la
propagation d'une onde de choc se deplacant vers la droite. L'onde de choc, dont la vistesse $D$
d\'epend de $u$ ainsi  que de l'\'equation d'\'etat du syst\`eme, s\'epare le syst\`eme en deux
r\'egions. La r\'egion en amont du choc est la r\'egion non-choqu\'ee (0) et la r\'egion en aval
est la r\'egion choqu\'ee (1). Le but ici est d'analyser la structure induite dans la r\'egion
(1) par le passage de l'onde.\\

Afin de mener a bien ces analyses, on utilise une fen\^etre dont la position et la taille
change au cours de la simulation de mani\`ere \`a englober la totalit\'e de la r\'egion (1). Les
dimensions de cette fen\^etre sont calcul\'ees \`a partir du profile de vitesses des atomes
selon la direction de propagation de l'onde. Dans un premier temps, on localise la position du
front d'onde, c'est \`a dire le point d'inflexion maximal sur le profile. La r\'egion (1) est
situ\'ee entre la  position du piston et celle du dernier maximum local en aval du
front d'onde. On calcule alors les propri\'et\'es
suivantes~:
\begin{itemize}
	\item[$\bullet$] Vitesse de propagation de l'onde de choc en fonction de la vitesse du piston.
	\item[$\bullet$] Distribution des angles de liaisons SiOSi, OSiO et SiSiSi dans la r\'egion (1)
	\item[$\bullet$] Distribution de la taille des anneaux dans la r\'egion (1).\\
\end{itemize}


\begin{figure}
\picW=7cm
\picL{shockVelocity.ps}
\caption{Repr\'esentation de l'\'evolution de la vitesse du front d'onde $D$ en fonction de la
vitesse du piston $u$. La ligne pointill\'ee sert uniquement \`a guider l'oeil.}
\label{fig::shockVelocity}
\end{figure}

La vitesse de propagation de l'onde de choc en fonction de la vitesse du piston ($D=f(u)$) est
calcul\'ee en mesurant la position du front d'onde en fonction du temps  pour chaque vitesse
de propagation du piston. $D$ est simplement la pente de cette droite. La
Figure~\ref{fig::shockVelocity} montre les r\'esultats obtenus et permet de distinguer
un r\'egion \'elastique pour $u < 1.5\, \mrm{Km.s^{-1}}$ o\`u la vitesse du choc est tr\`es
importante mais ne varie pas avec $u$ ainsi qu'une r\'egion plastique o\`u $D$ est une fonction
lin\'eaire de $u$.\\

La Figure~\ref{fig::BAD_shock} montre les r\'esultats obtenus pour le calcul de la
distribution des angles de liaisons SiOSi, OSiO et SiSiSi dans la r\'egion 1. On remarque que
pour tous les angles mesur\'es, l'augmentation de la vitesse du piston  induit une distribution
d'angle plus large, qui peut \^etre expliqu\'e par l'augmentation de la temp\'erature en aval du
front d'onde. On peut toutefois montrer que les distributions obtenues \`a vitesse de piston
\'elev\'ees et donc \`a temp\'eratures \'elev\'ees sont tr\`es diff\'erentes de celles obtenues dans
le cas de simulations de volume \`a ces m\^emes temp\'eratures (\cf inserts de la
Figure~\ref{fig::BAD_shock}). La structure induite par le passage de l'onde  est donc bien
sp\'ecifique aux conditions de choc et n'est pas le simple r\'esultat de l'augmentation de la
temp\'erature derri\`ere le front d'onde. 
\begin{figure}
\picW=7cm
\subfigure[angle OSiO]{\picL{BAD_shock_OSiO.ps}}
\subfigure[angle SiOSi]{\picL{BAD_shock_SiOSi.ps}}
\subfigure[angle SiSiSi]{\picL{BAD_shock_SiSiSi.ps}}
\caption{Repr\'esentation de la distribution des angles de liaisons OSiO(a), SiOSi(b) et
SiSiSi(c). Dans les cadres majeurs, Les courbes en pointill\'es montrent la distribution au
p\^ole, les courbes continues fines correspondent \`a la distribution pour  diff\'erentes
vitesses $u$ et les courbes en gras  correspondent \`a la vitesse de piston maximale
repr\'esent\'ee ($u=4500 \mrm{m.s^{-1}}$). Dans les inserts, les courbes continues correspondent aux
courbes en gras des cadres majeurs, et les courbes en gras repr\'esentent, la distribution \`a haute
temp\'erature obtenue au cours d'une simulation de volume.}
\label{fig::BAD_shock}
\end{figure}


%%=========================================================================================================================
%%	Hugoniot
%%=========================================================================================================================
\section{Simulation sur l'Hugoniostat}

On se propose ici, d'appliquer la m\'ethode de l'Hugoniostat afin de pouvoir effectuer une
comparaison avec les simulations sous choc direct, et dans le cas o\`u les r\'esultats sont
comparables, de pouvoir disposer d'une m\'ethode permettant d'analyser plus rapidement la
structure de choc et avec de meilleurs statistiques. Plusieurs m\'ethodes ont \'et\'es
developp\'ees afin de pouvoir effectuer des simulations sur l'Hugoniostat~\cite{HUG01,HUG02,HUG03}. Dans notre
cas, on choisit une m\'ethode sous contrainte qui \`a l'avantage d'ob\'eir aux \'equations
d'Hugoniot \`a chaque pas de temps~\cite{HUG03}. On rapelle cependant les \'equations
d'Hugoniot qui permettent de lier les propri\'et\'es thermodynamiques des r\'egions (0) et (1)
avec les vitesses $u$ et $D$~:
%%
\begin{equation}\
	\begin{array}{ccc}
	\mu_1	&=&		\left(\mu_0D\right)/\left(D-u\right)\\
	P_1		&=&		\mu_0Du+P_0\\
	D		&=&		(1/\mu_0)\left[ (P_1-P_0)/(\mu_0^{-1} - \mu_1^{-1})  \right]^{1/2}\\
	u		&=&		\left[  (P_1-P_0)(\mu_0^{-1} - \mu_1^{-1})\right]^{1/2}
	\end{array}
\label{eqn:hug}	
\end{equation}

Le syst\`eme de d\'epart des simulations dans l'Hugoniostat (p\^ole) est un verre de
$N=1536$ atomes \`a $T=300\, K$ et $\rho=2.2 \mrm{g.cm^{-3}}$ pr\'epar\'e selon la m\'ethode d\'ecrite ci-avant.
Les propri\'et\'es thermodynamique du p\^ole ont \'et\'es mesur\'ees au cours de la derni\`ere
simulation de production du syst\`eme.\\ 


Chaque simulation dans l'Hugoniostat part de la m\^eme configuration (le p\^ole) compress\'ee de mani\`ere
instantan\'ee \`a la densit\'e voulue. La plage de densit\'e utilis\'ee est $\rho \in
[3000:5000]\,\mrm{Kg.m^{-3}}$. Pour chaque simulation, on mesure l'\'evolution de la pression
ainsi que les propri\'et\'es structurelle du syst\`eme.\\


\begin{figure}
\picW=7cm
\subfigure[]{\picL{cmp_D-u.ps}}
\subfigure[]{\picL{cmp_P-mu.ps}}
\caption{Comparaisons des caract\'eristiques thermodynamiques obtenues au cours des simulations
sous choc ($\square$) en en utilisant l'Hugoniostat ($\bullet$). La sous-figure (a) utilise les
donn\'ees mesur\'ees sous choc et calcul\'ees \`a partir des simulation dans l'Hugoniostat en
utilisant les relations d'Hugoniot. La sous-figure (b) utilise les donn\'ees mesur\'ees dans
l'Hugoniostat et calcul\'ees \`a partir des simulations sous choc.}
\label{fig::cmp_thedy}
\end{figure}


La diff\'erence entre les simulations sous choc et avec l'Hugoniostat, est que les quantit\'es
impos\'ees dans le cas d'une m\'ethode sont celles qui peuvent etre mesur\'ees avec l'autre.
Dans le cas des simulations dans l'Hugoniostat, la densit\'e est impos\'ee et on mesure la
pression $P_1$. Connaissant $\mu_0$ et $P_0$, les \'equations d'hugoniot permettent calculer $u$
et $D$.  Dans le cas des simulations sous choc, on impose $u$
et on mesure $D$. Connaissant $P_0$ et $\mu_0$ on peut alors calculer $P_1$ et $\mu_1$. Les \'equations
d'Hugoniot~(\ref{eqn:hug}) permettent donc de faire la relation entre les propri\'et\'es mesur\'ees et
impos\'ees dans les deux types de simulations.\\

Afin de comparer les propri\'et\'es thermodynamiques
obtenues avec les deux types de simulations, on mesure les  courbes $D=f(u)$ et $P=d{\mu}$ (\cf
Figure~\ref{fig::cmp_thedy}). On remarque que les deux m\'ethodes de simulations donnent des
r\'esultats tr\`es comparable \`a
ceci pr\`es que la m\'ethode d'Hugoniot ne permet pas d'observer la zone \'elastique.
Dans la zone plastique qui nous int\'eresse ici, les r\'esultats sont cependant tr\`es satisfaisant.\\



\begin{figure}
\picW=7cm
\subfigure
[Distributions des angles de liaisons obtenues \`a partir des
simulation sous choc (lignes continues) et avec l'Hugoniostat ($\blacktriangle$:SiSiSi, 
$\bullet$:OSiO, $\blacksquare$:SiOSi).]
{\picL{cmp_BAD.ps}}
\subfigure[Distributions des tailles des anneaux obtenues \`a partir des
simulation sous choc ($\bullet$) et avec l'Hugoniostat (histogrammes).]
{\picL{cmp_SPA.ps}}
\caption{Comparaison des structures obtenues avec les simulations sous choc et en utilisant
l'Hugoniostat. Dans les deux sous figures, les \'equivalences (1) \`a (4) sont repr\'esent\'e de gauche a
droite et de haut en bas.}

\label{fig::cmp_BAD}
\label{fig::cmp_SPA}
\end{figure}



Afin de confirmer les observations men\'ees \`a l'aide des propri\'et\'ees thermodynamiques. On
mesure pour les deux types de simulation, et chaque s\'eries les propri\'et\'es structurelles
(distribution d'angles  de liaisons et distribution de tailles des anneaux). Afin d'effectuer
les comparaison, on choisit \`a l'aide des r\'esultats pr\'esent\'es dans la
Figure~\ref{fig::cmp_thedy}, l'\'equivalence entre la vitesse du piston dans les simulations sous
choc et la densit\'e impos\'ee dans l'Hugoniostat. On d\'ecide de l'\'equivalence suivante~:
\begin{eqnarray*} 
\mrm{cas (1) :}\mu=3600 \mrm{Kg.m^{-3}} &\equiv& v_p=2000 \mrm{m.s^{-1}}\\
\mrm{cas (2) :}\mu=4100 \mrm{Kg.m^{-3}} &\equiv& v_p=3000 \mrm{m.s^{-1}}\\
\mrm{cas (3) :}\mu=4600 \mrm{Kg.m^{-3}} &\equiv& v_p=4000 \mrm{m.s^{-1}}\\
\mrm{cas (4) :}\mu=4800 \mrm{Kg.m^{-3}} &\equiv& v_p=5000 \mrm{m.s^{-1}}\\
\end{eqnarray*}

Les comparaisons sont pr\'esent\'ees dans les Figures~\ref{fig::cmp_BAD} et~\ref{fig::cmp_SPA}
dans le cas des angles de liaisons et des tailles d'anneaux respectivement. Encore une fois, les
comparaisons montrent une tr\`es grande correspondance entre les r\'esultats obtenus avec
l'Hugoniostat et en conditions de choc. Les comparaisons s'am\'eliorant \`a mesure que 
la force du choc augmente.



%%=========================================================================================================================
%%	Conclusion - perspectives
%%=========================================================================================================================
\section*{Conclusion}

On pr\'esente ici une m\'ethode permettant la simulation de verres silicate en conditions de
choc. Deux m\'ethodes ont \'et\'e utilis\'ees, \`a savoir la simulation avec passage direct
d'ondes de choc et l'utilisation de l'Hugoniostat, un ensemble thermodynamique qui permet de
reproduire les conditions de choc.\\
%%
La premi\`ere m\'ethode, bien que tr\`es ch\`ere en nombre de particules permet de suivre
l'\'evolution du front d'onde au travers du mat\'eriau. L'utilisation d'une fen\^etre
d'analyse se d\'epla\c{c}ant avec le front d'onde pour recouvrir la r\'egion choqu\'ee, permet
l'analyse des propri\'et\'es thermodynamique et structurelle du mat\'eriau juste apr\`es le
passage de l'onde.\\
%%
La m\'ethode de l'Hugoniostat permet d'\'etudier les propri\'et\'es statiques du syst\`eme en
conditions de choc avec des syst\`emes de tailles tr\`es inf\'erieures et donc bien plus
rapide.\\
%%
L'analyse thermodynamique et structurelle des configuration obtenues jointe \`a l'utilisation
des relations d'Hugoniot permet la comparaison des r\'esultats obtenus avec les deux
m\'ethodes. On montre que les deux m\'ethodes permettent d'obtenir des r\'esultats tr\`es
similaires, validant ainsi l'utilisation de la m\'ethode d'Hugoniot en remplacement des
simulation sous choc direct.

%%=========================================================================================================================
%%	Bibliography
%%=========================================================================================================================
\bibliographystyle{unsrt}
\bibliography{refs}


\end{document}

%%Les param\`etres de simulations sont les suivants~:
%%\begin{center}
%%\begin{tabular}{lll}
%%\hline
%%$A_0=\frac{1.0}{4\pi\epsilon_0}$ &= $2.307080\e{-28}$	&[J.m-1]	\\
%%$q_{Si}$	&= 2.4 					&charge partielle du silicium\\
%%$q_O$		&= -1.2 				&charge partielle de l'oxyg\`ene\\
%%$r_c$		&= 6.0					&[m] distance du cutoff \\
%%\hline
%%\end{tabular}
%%\end{center}

%%Les coefficients du potentiel BKS sont donn\'es dans~\cite{BKS}~:

%%\begin{center}
%%\begin{tabular}{cccc}
%%	&$A_{ij}$ \mrm{[J]}		&$B_{ij}$ \mrm{[m^-1]}	&$C_{ij}$ \mrm{[J.m^6]}\\
%%\hline
%%O-O	&2.225061\e{-16}	&2.760000\e{+10}	&2.803810\e{-77}	\\
%%Si-O	&2.884521\e{-15}	&4.873180\e{+10}	&2.139517\e{-77}	\\
%%Si-Si	&0.0			&0.0			&0.0			\\
%%\hline
%%\end{tabular}
%%\end{center}

%%t les coefficients de p(r) sont~:
%%\begin{center}
%%\begin{tabular}{cc}
%%\hline
%%O-O		&	2.0\e{-10} m\\
%%Si-O	&	1.5\e{-10} m\\
%%Si-Si	&	0.0\e{-10} m\\
%%\hline
%%\end{tabular}
%%\end{center}

%%\begin{center}
%%\begin{tabular}{cccc}
%%\hline
%%		&a [$J.m^{-2}$]		&b	[$J.m^{-1}$]	&c [J]				\\
%%O-O		&1.510154e+02		&-7.925215e-08		&1.100414e-17		\\
%%Si-O	&3.413039e+02		&-9.361408e-08		&3.924997e-18		\\
%%Si-Si	&0.000000e+00		&0.000000e+00		&0.000000e+00		\\
%%\hline
%%\end{tabular}
%%\end{center}


%%\begin{figure}
%%\picW=7cm
%%\subfigure[]{\picLB{cmp_BAD_01.ps}}
%%\subfigure[]{\picLB{cmp_BAD_02.ps}}
%%
%%\subfigure[]{\picLB{cmp_BAD_03.ps}}
%%\subfigure[]{\picLB{cmp_BAD_04.ps}}
%%\caption{}
%%\label{fig::cmp_BAD}
%%\end{figure}


%%\begin{figure}
%%\picW=7cm
%%\subfigure[]{\picLB{cmp_SPA_01.ps}}
%%\subfigure[]{\picLB{cmp_SPA_02.ps}}

%%\subfigure[]{\picLB{cmp_SPA_03.ps}}
%%\subfigure[]{\picLB{cmp_SPA_04.ps}}
%%\caption{}
%%\label{fig::cmp_SPA}
%%\end{figure}
