


%%\documentclass[twocolumn,aps,pre,showpacs,superscriptaddress,draft]{revtex4}

%%\documentclass[preprint,pre,showpacs,superscriptaddress]{revtex4}
\documentclass[aps,10pt,twocolumn]{revtex4}



%============================================================================
%       packages
%============================================================================
\usepackage[final]{graphics}    % for graphics
\usepackage[final]{graphicx}    % for graphics
\usepackage{subfigure}      % allow subfigure
\usepackage{amssymb}        % font
\usepackage{amstext}        % ams latex package
\usepackage{amsmath}        % font
\usepackage{amsfonts}       % font
\usepackage{amsbsy}     % font
\usepackage{hhline}     % nice lines is tables
\usepackage{xspace}     % add space at the end of macros
\usepackage{color}

%============================================================================
%        page style and size
%============================================================================
%\oddsidemargin = 0.6in          % 40mm left margin
%\oddsidemargin = 0.2in
%\textwidth = 5.75in         % 20mm right margin
%\textwidth = 5.95in
%\textheight = 9.1in         % 20mm from page number to bottom
%\voffset = -0.5in           % lift up
%\parindent = 0pc            %no indentation
%\parskip 12pt


%============================================================================
%       new commands
%============================================================================
\newcommand{\remark}[1]{\textbf{\textcolor{red}{#1}}}
\newcommand{\remove}[1]{}

\newcommand{\linespacing}[1]{\renewcommand{\baselinestretch}{#1}\large\normalsize}
\newcommand{\mrm}[1]{\ensuremath{\mathrm{#1}}}
\newcommand{\vect}[1]{ \mathbf{#1} }
\newcommand{\vecth}[1]{ \mathbf{\hat{#1} } }

\newcommand{\gvect}[1]{\forcebold{#1}}
\newcommand{\gvecth}[1]{\forcebold{\hat{#1}}}

\newcommand{\dotproduct}[2]{\vect{#1} \cdot \vect{#2}   }


\newcommand{\lp}{\left(}
\newcommand{\rp}{\right)}


%============================================================================
%       Contact distance related macros
%============================================================================

\newcommand{\so}{\sigma_0}
\newcommand{\sel}{\sigma_\ell}

\newcommand{\rij}{\vecth{r}_{ij}}
\newcommand{\ui}{\vecth{u}_i}
\newcommand{\uj}{\vecth{u}_j}
\newcommand{\sijr}{\sigma(\ui,\uj,\rij)}
\newcommand{\dotProdP}[2]{ \left( #1 \cdot #2 \right) }
\newcommand{\dotProd}[2]{ #1 \cdot #2 }

\newcommand{\sir}{\sigma(\ui,\vect{r}_{ij})}

\newcommand{\ijr}{\ui,\uj,\rij}
\newcommand{\kSp}{k_S^{\prime}}

\newcommand{\etal}{\emph{et al.}\@\xspace}
\newcommand{\ie}{\emph{i.e.}\@\xspace}
\newcommand{\eg}{\emph{e.g.}\@\xspace}
\newcommand{\etc}{etc.\@\xspace}


%============================================================================
%       Macros - figures
%============================================================================
\pagebreak
\newlength{\picH}   % picture height
\newlength{\picW}   % picture width
\newcommand{\picA}{270} % picture angle

\picW = 10cm
\newcommand{\pic}[1]{\includegraphics[width=\picW]{#1}}
\newcommand{\picL}[1]{\includegraphics[height=\picW, angle=\picA]{#1}}



%=======================================================================================
%           BEGIN DOCUMENT
%=======================================================================================
\begin{document}
\graphicspath
{
{../imgs/}
}

%%\title{Surface interaction potential effect on\\ the anchoring behaviour of liquid crystals}
%%\title{Partial substrate penetrability as a control mechanism for surface induced anchoring}

\title{Using particle shape to induce tilted and bistable liquid crystal anchoring}

\author{F. Barmes}
\affiliation{Centre Europ\'een de Calcul Atomique et Mol\'eculaire, 46, All\'ee d'Italie, 69007 Lyon, France}

\author{D.J. Cleaver}
\affiliation{Materials and Engineering Research Institute, Sheffield Hallam University, Sheffield, S1 1WB, United
Kingdom}


\pacs{61.30.-v, 64.70Md,  61.30.Cz, 68.08.-p}


%%61.30.-v : liquid crystals
%%64.70Md : transitions in liquid crystals
%%61.30.Cz : molecular and microscopic models and theories of liquid crystals
%%68.08.-p : liquid-solid interfaces


\date{\today}

%=======================================================================================
%           ABSTRACT
%=======================================================================================

%%\begin{abstract}
%%We investigate the use of partial surface penetrability as a mean of controlling
%%the surface induced anchoring orientations with systems of hard Gaussian overlaps confined in an
%%infinitely wide slab with symmetric anchoring. Using the HGO-sphere potential for interaction
%%with the planar substrates, we demonstrate that the combination of purely steric interactions
%%and surface penetrability triggers the formation of tilted as well as homeotropic phases. Upon
%%extending the HGO-sphere to the HGO-surface potential which induces a  more restrictive
%%absorption behaviour, the tilted arrangement becomes planar. In addition, this second potential
%%allows the observation of a wide and strong region of bistability where both planar and
%%homeotropic arrangements are equally stable. As a result, besides further explaining
%%the origin of tilted and planar phases, the HGO-surface potentiel opens  up the prospects of
%%studying systems where surface bistability is of critical importance.
%%\end{abstract}

\begin{abstract}
We use Monte Carlo simulations of hard Gaussian overlap (HGO) particles symmetrically confined
in slab geometry to investigate the role of particle-substrate interactions on liquid
crystalline anchoring. Despite the restriction here to purely steric interactions and smooth
substrates, a range of behaviours are captured, including tilted anchoring and
homeotropic-planar bistability. These macroscopic behaviours are all achieved through
appropriate tuning of the microscopics of the HGO-substrate interaction, based upon non-additive
descriptions for the HGO-substrate shape parameter.
\end{abstract}

%============================================================================
%           Title
%============================================================================
\maketitle
%=======================================================================================
%           INTRODUCTION
%=======================================================================================
%%\newpage
\section{Introduction}
The term \emph{surface anchoring} refers to the means by which a preferred orientation (or set of orientations) is
imposed on a liquid crystal by a confining substrate~\cite{Jerome91}. The mechanisms underlying surface anchoring
are fundamental to the operation of virtually all liquid crystal display cells, since the field-off states
utilised in such devices are usually surface-aligned~\cite{Shanks82,Geelhaar98}. Indeed, surface anchoring is
particularly important in the latest generation of bistable devices~\cite{ZBD,DennistonYeomans,DavidsonMottram02}
in which the display cells possess two optically distinct surface-stabilised arrangements.

Experimental studies of liquid crystal anchoring (see J\'{e}r\^ome~\cite{Jerome91} for a review) have identified
three classes of alignment characterised by $\alpha$, the angle between the average director tilt and the
substrate normal. These alignments are homeotropic, tilted and planar with, respectively, $\alpha=0$, $0 < \alpha
<\pi/2$ and $\alpha = \pi/2$. The anchoring properties of adsorbed liquid crystalline systems have also been the
subject of several theoretical investigations performed, in the main, using mean
field~\cite{TjiptoMargoSullivan88,DelRioTeloDaGamma95} and density
functional~\cite{OsipovHess93,Teixeira97,Allen99} approaches.

Despite this range of previous studies, molecular-level understanding of the mechanisms driving anchoring remains
limited and the methods used to control surface anchoring in current devices are largely empirical. For instance,
it has long been known that rubbed substrates can be used to create planar surface alignment~\cite{LuDeng98}, but
the mechanisms underlying this result have been the subject of an extended debate~\cite{ChenFeller89}. If the
surface is a polymer film, soft rubbing has the effect of aligning the polymer chains in the rubbing direction.
This, in turn, aligns the liquid crystal molecules thus highlighting a chemical mechanism coupling the nematic
director in the interfacial region with the polymer chain orientation~\cite{Clark85,GearyGoodby87}. If, however,
the substrate is scratched by the rubbing, creating a grooved surface, it has been argued that a steric mechanism
can generate the same effect~\cite{Berreman72}.

While treatments such as substrate rubbing offer surface pretilt and azimuthal control over the anchoring
direction, they do not represent the only routes to controllable liquid crystal alignment. This has been
illustrated by a series of computer simulation studies performed over the last decade, which have given direct
insight into the relationship between molecular adsorption and liquid crystal anchoring. The most common
arrangement found in such studies is planar anchoring; this has been found at flat substrates for
hard-particle~\cite{VanRoijDijkstra00,DijkstraVanRoij01,Chrzanowska_Teixera_01,BarmesCleaver04a},
Gay-Berne~\cite{WallCleaver03} and all-atom~\cite{BingerHanna01} models (though note that planar alignment of the
adsorbed molecules does \emph{not} always result in planar anchoring \cite{PalermoBiscarini98}). Homeotropic
anchoring has been achieved using hard-particle systems employing non-additive wall-particle interactions at
perfectly flat walls~\cite{Allen99,Cleaver_Teixeira_01,Chrzanowska_Teixera_01,LangeSchmid02,BarmesCleaver04a} and
full interactions at walls with tethered flexible chains~\cite{LangeSchmid02,LangeSchmid02a,LangeSchmid02c} and
rigid rods~\cite{DowntonAllen04}. While homeotropic anchoring has been seen in simulations of Gay-Berne particles
confined by smooth substrates~\cite{WallCleaver03}, and could certainly be forced using the well-depth anisotropy
tuning approach employed in~\cite{AntypovCleaver04}, the majority of such systems have yielded tilted
alignments~\cite{ZhangChakrabarti96,WallCleaver97,TeixeiraChrzanowska01,WebsterCleaver03}. Up to now, the tilt
observed in these systems has been ascribed to competition between the particle-particle and particle-wall
attractive interactions. However, by investigating the equivalent hard-particle system, we show here that this
tilt actually has an \emph{entropic} origin. Tilted anchoring in hard-particle systems has previously been seen
only when the substrates have been made rough through the tethering of
chains~\cite{LangeSchmid02,LangeSchmid02a,LangeSchmid02c} or rods~\cite{DowntonAllen04}. Indeed, entropy-driven
tilt of cylindrically symmetric particles at smooth walls has not, to our knowledge, been seen or even considered
in any previous simulation or theoretical study.

In this study we extend previous work~\cite{BarmesCleaver04a} on the anchoring behaviour of generic hard-particle
liquid crystal models by studying the effect of changing the particle-substrate contact function. Specifically, we
use Monte Carlo simulations to study the anchoring behaviour of hard Gaussian overlap (HGO) particles confined in
a slab geometry using two particle-surface potentials - the HGO-sphere and HGO-surface potentials. As well as
investigating the intrinsic anchoring properties of these two surfaces, we study their behaviours for varying
degrees of substrate penetrability, in order to identify the conditions under which the stable anchoring condition
changes. This is done with the aim of developing and characterising a surface potential capable of exhibiting both
homeotropic and planar anchoring alignments, i.e. bistable anchoring. A narrow region of bistability was
identified in our previous work based on the simple hard needle-wall (HNW) surface
potential~\cite{BarmesCleaver04a} and found to be explained by the non-additive nature of this potential.

The remainder of this paper is organised as follows: in Section~\ref{s:RSP} we describe the HGO-sphere potential
and its induced phase behaviour. Following this, in Section~\ref{s:RSUP} we show equivalent work performed with
the HGO-surface  model. Finally, in Section~\ref{s:CCL}, we present a discussion and the conclusions deduced from
this work and propose some directions for future work.


%=======================================================================================
%       ROD - SPHERE
%=======================================================================================
\section{The HGO-sphere surface potential}
\label{s:RSP} In this Section, surface induced structural changes are studied using Monte Carlo simulations of
rod-shaped particles that interact with one another through the HGO potential~\cite{Rigby89} and with the
confining substrates via the HGO-sphere potential. The HGO model is a steric model in which the contact distance
is the shape parameter determined by Berne and Pechukas~\cite{BernePechukas72} when they considered the overlap of
two ellipsoidal Gaussians.  Thus, the interaction potential $\mathcal{V}^HGO$ between two particles $i$ and
$j$ with respective orientations $\ui$ and $\uj$ and intermolecular vector $\vect{r}_{ij} = r_{ij}\rij$ is defined
as
%%
%%
\begin{equation}
    \mathcal{V}^\mrm{HGO} = \left\{  %}
    \begin{array}{ccc}
    0   &\mrm{if}   &r_{ij} \geq \sijr  \\
    \infty  &\mrm{if}   &r_{ij} < \sijr
    \end{array}
    \right.
\end{equation}
where $\sijr$ is the contact distance, or shape parameter,
%%
%%----------------------------------------------------------------
%%	TWO COLUMNS VERSION OF THE EQUATION
%%----------------------------------------------------------------
%\begin{multline}
%\label{eqn:HGO_sigma}
%    \sijr = \so\left\{
%    1 - \frac{1}{2}\chi\left[
%    \frac{ \lp\dotProd{\rij}{\ui} + \dotProd{\rij}{\uj}\rp^2 }{1 + \chi(\dotProd{\ui}{\uj})}
%    \right. \right. \\
%    %%
%    %%
%    %%
%    \left. \left.
%    + \frac{ \lp\dotProd{\rij}{\ui} - \dotProd{\rij}{\uj}\rp^2 }{1 - \chi(\dotProd{\ui}{\uj})}
%    \right] \right\}^{-\frac{1}{2}}.
%\end{multline}
%%
%%----------------------------------------------------------------
%%	ONE COLUMN VERSION OF THE EQUATION
%%----------------------------------------------------------------
%%
\begin{equation}
\label{eqn:HGO_sigma}
    \sijr = \so\left\{
    1 - \frac{1}{2}\chi\left[
    \frac{ \lp\dotProd{\rij}{\ui} + \dotProd{\rij}{\uj}\rp^2 }{1 + \chi(\dotProd{\ui}{\uj})}
    %%
    + \frac{ \lp\dotProd{\rij}{\ui} - \dotProd{\rij}{\uj}\rp^2 }{1 - \chi(\dotProd{\ui}{\uj})}
    \right] \right\}^{-\frac{1}{2}}.
\end{equation}
%%
%%
Here $\so$, the particle width, sets the unit of distance for this model and the shape anisotropy parameter $\chi
= (k^2-1)/(k^2+1)$, where $k= \sel/\so$, is the particle length to breadth ratio.

The HGO model is the hard-particle equivalent of the much-studied Gay-Berne model~\cite{GayBerne81}. The phase
behaviour of the HGO model is density driven and fairly simple, comprising only two non-crystalline phases;
isotropic and (for $k\gtrsim 3$) nematic fluids at, respectively, low and high number densities $\rho^{*}$. The
isotropic-nematic phase-coexistence densities have been located for various particle elongations in a series of
previous simulation studies~\cite{PadillaVelasco97,DeMiguelDelRio01,DeMiguelDelRio03}; for the most commonly used
elongation of $k=3$, the isotropic-nematic transition occurs at $\rho^{*}\approx 0.30$ with a slight system size
dependence.

Although the HGO model was originally derived using geometrical considerations, an HGO particle cannot be
represented by a fixed solid object. Rather, it is a mathematical abstraction of the interaction surface between
two non-spherical particles~\cite{Rigby89}. For moderate elongations, however, the properties of HGO particles are
similar to those of an the equivalent hard ellipsoid of revolution~\cite{Rigby89}. Simulation studies
~\cite{DeMiguelDelRio03} have borne this out, showing that the equation of state of the HGO fluid is qualitatively
equivalent to, but consistently displaced from, that of the hard ellipsoid fluid.

The shape parameter $\sijr$ given by eqn.(\ref{eqn:HGO_sigma}) has too low a symmetry to be appropriate for
describing the interaction between an HGO particle and a featureless, planar substrate. However, a simple function
with the appropriate symmetry {\em is} obtained in the limit that one of the particles is made spherical. For a
sphere of diameter $\sigma_j$, the contact distance for this interaction is given by eqn.(4)
of~\cite{BernePechukas72}~:
%
\begin{equation}
    \sigma^{\rm HGO-sphere}(\ui,\rij) = \sqrt{
    \frac{\so^2 + \sigma^2_j}
    {2(1 - \chi(\dotproduct{u_i}{r_{ij}})^2 )}
    }.
    \label{eqn:sw_RSP_BP}
\end{equation}
%
In Refs.~\cite{ZhangChakrabarti96,WallCleaver97,TeixeiraChrzanowska01,WebsterCleaver03}, this contact function was
used as the basis for the particle-substrate interaction: the surface, as viewed by any particle, was taken to be
represented by a sphere located in the surface plane but with the same $x$- and $y$-coordinates as those of the
particle. In this Section, we adopt this same HGO-sphere approximation for the particle-substrate contact
function. We also investigate the dependence of the system's anchoring on the penetrability of the substrate. This
is achieved by mediating the interaction between each particle and the substrate using a second ``inner" particle
of breadth $\so$ and length $\sel^{\prime} \leq \sel$. Thus, when the inner particle is made short, the HGO
becomes able to embed its ends into the substrate. This results in an interaction
%
\begin{equation}
%%\label{eqn:sw_RSP_BP}
    \mathcal{V}^{\rm HGO-sphere} = \left\{
    \begin{array}{ccc}
        0   &\mathrm{if}    &|z_i - z_0| \geq \sigma^{\rm HGO-sphere}_w(\ui)    \\
        \infty  &\mathrm{if}    &|z_i - z_0| < \sigma^{\rm HGO-sphere}_w(\ui)
    \end{array}
    \right.
\end{equation}
%
between particle $i$ and a substrate located at $z_0$, where
%%\begin{equation}
%%    \sigma^{\rm HGO-sphere}_w(\ui) = \sigma_0\lp \frac{1}{\sqrt{1 - \chi_S\cos^2\theta_i}} - \frac{1}{2} \rp
%%    \label{eqn:sigma_w_RSP}
%%\end{equation}
%%
\begin{equation}
    \sigma^{\rm HGO-sphere}_w(\ui) = \sigma_0\left[ (1 - \chi_S\cos^2\theta_i)^{-1/2} - \frac{1}{2} \right]
    \label{eqn:sigma_w_RSP}
\end{equation}
%%
%%
is a rewriting of eqn.~(\ref{eqn:sw_RSP_BP}) in terms of $\theta_i$, the particle zenithal angle, subject to the
imposition $\sigma_j = \so $ and a shift of the surface spheres by one particle radius so as to make the substrate
surface coincide with $z_0$. $\chi_S = (k^2_S - 1)/(k^2_S + 1)$, $k_S$ being the length to breadth ratio
($\sel^{\prime}/\so$) of the inner ellipsoid. In the next Subsections, the consequences of changing this variable
will be examined. Broadly, reducing $k_S$, and so increasing the degree of surface penetrability is expected to
stabilise the homeotropic arrangement. Indeed, previous simulation studies~\cite{Allen99,BarmesCleaver04a} have
shown the homeotropic state to be stable for $k_S=0$.
%The behaviour of this system for non-zero $k_S$ is less
%clear, however; arrangements which maximise the free volume (by, e.g., expelling particle volume) are increasingly
%favoured as the system density is raised.
At the other limit, it is established that hard rods at hard walls adopt planar
alignment~\cite{DijkstraVanRoij01}. And between the two lies the possibility of substrate-induced tilt.

%---------------------------------------------------------------------------------------
\subsection{Simulation results}
\label{ss:RSPresults} The surface induced structural changes obtained using the HGO-sphere potential have been
investigated by means of Monte Carlo computer simulations in the canonical ensemble. Systems of $N=1000$ HGO
particles with elongation $k=3$ were confined in a slab geometry with fixed wall separation $L_z=4k\sigma_0$, the
walls being situated at $z_0 = \pm\frac{L_z}{2}$ and symmetric anchoring conditions imposed. Periodic boundary
conditions were applied in the $x$- and $y$-directions. 
At the time of the computation, one individual run of half a million sweeps represented about 30
hours of computation time on a Compaq dec alpha workstation. With more recent processors (intel
pentium IV with 3.0GHz clock speed) this can be reduced to 5 hours bringing the total time to
compute an anchoring map down to 300 hours of CPU time.
The relatively modest system size of $N=1000$ has been
used here in order to enable a comprehensive mapping of the relevant phase space to be achieved. From De Miguel's
study of system size effects in 3d bulk systems of Gay-Berne particles~\cite{DeMiguel92}, it is apparent that any
$N$-dependence of bulk behaviour should be negligible for $N\geq\mathcal{O}(10^3)$. This conclusion does not
transfer automatically to confined systems, however, since the surface extrapolation lengths can become comparable
with the substrate-substrate separation~\cite{Priezjev03}. For the systems studied here, in which the surface
conditions were symmetrical, we have found that doubling the slab thickness (\ie running with $N=2000$ particles)
does not have a significant effect on the anchoring behaviour observed. However, in equivalent simulations of
hybrid anchored systems, in which the two surface extrapolation length regions can promote competing effects, we
have found that the slab thickness becomes a significant simulation parameter; this is described in detail
elsewhere~\cite{BarmesCleaver04d}. Sequences of simulations were performed at constant number density $\rho^{*}$
and decreasing $k_S$ for several values of $\rho^{*}$; from these simulations, the surface induced structural
changes were studied through computation of profiles for the number density $\rho^{*}_\ell(z)$, orientational
order with respect to the substrate normal, $Q_{zz}(z)$, and slice averaged orientational order, $\langle
P_2(z)\rangle$. Full descriptions relating to the computation of these observables are given
elsewhere~\cite{WallCleaver97,BarmesCleaver04a}.

%---------------------------------------------------------------------------------------------------
\picW = 8cm
\begin{figure}
    \centering
    \pic{fig_01.ps}
    \caption{[Color online] Typical $z$-profiles for confined systems of HGO particles with
    $k=3.0$ and $k_S/k = 0.0$ using the HGO-sphere potential.
    These data are extracted from simulation series with decreasing $k_S$.}
    \label{fig:RSP_typeProf_k3_homeo}
\end{figure}
%---------------------------------------------------------------------------------------------------

%---------------------------------------------------------------------------------------------------
\begin{figure}
    \centering
    \pic{fig_02.ps}
    \caption{[Color online] Typical $z$-profiles for confined systems of HGO particles with
    $k=3.0$ and $k_S/k = 1.0$ using the HGO-sphere potential.
    These data are extracted from simulation series with decreasing $k_S$.}
    \label{fig:RSP_typeProf_k3_planar}
\end{figure}
%---------------------------------------------------------------------------------------------------

In Figs.~\ref{fig:RSP_typeProf_k3_homeo} and~\ref{fig:RSP_typeProf_k3_planar}, we show typical observable profiles
obtained at isotropic and nematic densities for systems with $k_S/k=0$ and $k_S/k=1.0$, respectively. At low
density, the system with $k_S/k = 0.0$ shows strong surface adsorption peaks adjacent to each substrate with high
orientational order perpendicular to the surfaces. As expected, the central 50\% of the system is orientationally
isotropic. At the nematic density, however, the positive $Q_{zz}(z)$ profile values effectively replicate those
obtained for $\langle P_2(z)\rangle$, indicating uniform homeotropic anchoring. Furthermore, the peak separations
of about $\sel$ in the corresponding $\rho^{*}_\ell(z)$ profiles indicate substrate-templated pseudo-smectic
layering of the type exhibited by other homeotropic systems~\cite{WallCleaver97,BarmesCleaver04a}. The equivalent
profiles obtained from the $k_S/k=1.0$ system indicate very different behaviour, however. At low densities, the
surface density peaks are shifted away from the substrates, and no structure is apparent in $\rho^{*}_\ell(z)$
apart from these first-monolayer features. Additionally, this system has low order parameter throughout, with very
little surface-enhancement compared with that shown by the homeotropic system. On compression to a nematic
density, the $\rho^{*}_\ell(z)$ profile gains secondary peaks at each substrate, but the observed peak-peak
separation distance is not appropriate for either homeotropic or planar anchoring states. Additionally, whilst the
$\langle P_2(z)\rangle$ profile clearly indicates a well ordered nematic film at this density, the central region
of the corresponding $Q_{zz}(z)$ profile adopts near-zero values. These features suggest a tilted arrangement, as
is confirmed by the configuration snapshot, Fig.~\ref{fig:RSP_snaps}(b), generated for this system at
$\rho^{*}=0.35$.

%---------------------------------------------------------------------------------------------------
\picW = 4cm
\begin{figure}
    \centering
    \subfigure[$k_S/k = 0.0$]{\pic{fig_03a.ps}}
    \subfigure[$k_S/k = 1.0$]{\pic{fig_03b.ps}}
    \caption{[Color online] Typical configuration snapshots showing the surface induced (a) homeotropic ($k_S/k=0$) and (b)
    tilted ($k_S/k=1.0$) surface induced arrangements for confined systems of $N=1000$ HGO particles
    using $\mathcal{V}^{\rm HGO-sphere}$ for surface interactions and $\rho^{*} = 0.35$.}
    \label{fig:RSP_snaps}
\end{figure}
%---------------------------------------------------------------------------------------------------

The crossover from homeotropic to tilted anchoring has been studied further through extensive simulations
performed over the $\rho^{*},k_S/k$ phase space. These are summarised in the surface and bulk-region anchoring
maps shown in Figs.~\ref{fig:RSP_phaseDia} (a) and (b), respectively, and calculated following the method given
in~\cite{BarmesCleaver04a}. These maps show the contours of $\overline{Q_{zz}}(\rho^{*},k_S/k)$, the
density-profile-weighted average of $Q_{zz}(z)$ in the interfacial and bulk regions of the cell. The difference
between the computation of these profiles and those of~\cite{BarmesCleaver04a} lies in the convention adopted to
define the location of the boundary between the interfacial and bulk regions of the cell. Here, the interfacial
region was taken to extend from the substrate to the second local maximum in $\rho^{*}_\ell(z)$ regardless of the
surface arrangement. These maps indicate a transition between homeotropic and tilted anchoring states at $k_S/k
\simeq 0.5$, but this crossover is less sharp than the homeotropic-planar anchoring transition observed with the
HNW surface model~\cite{BarmesCleaver04a}. Indeed, profiles obtained from simulation series performed at constant
density but either increasing or decreasing $k_S$ show negligible differences, indicating that the HGO-sphere
model does not exhibit bistability for $k=3$.

%---------------------------------------------------------------------------------------------------
\picW=7cm
\begin{figure}
    \centering
    \subfigure[Interfacial region]{\pic{fig_04a.ps}}
    \subfigure[Bulk region]{\pic{fig_04b.ps}}
    \caption{Anchoring maps showing the evolution of
    $\overline{Q_{zz}}(\rho^{*},k_S/k)$ as obtained from series of simulations at constant
    density and decreasing $k_S$ and using the HGO-sphere potential. Subfigures (a) and (b)
    correspond to the interfacial and bulk regions of the cell respectively.}
    \label{fig:RSP_phaseDia}
\end{figure}
%---------------------------------------------------------------------------------------------------

\subsection{Origin of the tilt}
To examine the basis of the tilted anchoring identified in the previous Subsection, we now assess the geometrical
properties of the HGO-sphere surface interaction model. As is shown in Appendix~\ref{app:A}, for an ellipsoid of
elongation $k$ and tilt $\theta$ whose closest surface-intersection-point lies a distance $d$ from the ellipsoid
centre, $V_{\rm abs}(k,\theta)$, the ellipsoid volume absorbed into the surface is given by~:
%%
%%--------------------------------------------------
%%	TWO COLUMN VERSION OF THE EQUATION
%%--------------------------------------------------
%\begin{multline}
%    V_{\rm abs}(k,d,\theta)=\frac{k\pi}{3}\lp
%    \frac{1}{2}-\frac{d}{(k^2\cos^2\theta+\sin^2\theta)^{1/2}}\rp^2  \\
%    \times \lp 1 + \frac{d}{(k^2\cos^2\theta+\sin^2\theta)^{1/2}}\rp .
%    \label{eqn:Ve_RSP}
%\end{multline}
%%
%%--------------------------------------------------
%%	ONE COLUMN VERSION OF THE EQUATION
%%--------------------------------------------------
\begin{equation}
    V_{\rm abs}(k,d,\theta)=\frac{k\pi}{3}\lp
    \frac{1}{2}-\frac{d}{(k^2\cos^2\theta+\sin^2\theta)^{1/2}}\rp^2  
    \lp 1 + \frac{d}{(k^2\cos^2\theta+\sin^2\theta)^{1/2}}\rp .
    \label{eqn:Ve_RSP}
\end{equation}
%%
Approximating this ellipsoid with the HGOs used in the simulations and $d$ with $\sigma^{\rm HGO-sphere}_w(\ui)$
of eqn.~(\ref{eqn:sigma_w_RSP}), we obtain an expression for $V_{\rm abs}(k,k_S,\theta)$. Setting $k=3$, we
present in Fig.~\ref{fig:Ve_RSP_fkS} a graphical representation of $V_{\rm abs}(k_S,\theta)$, the absorbed
particle volume as a function of both $k_S$ and $\theta$. For short $k_S$, this adsorbed volume is maximal at
$\theta = 0$, corresponding to an homeotropic anchoring state. As $k_S$ approaches $k$, however, a second maximum
develops at intermediate $\theta$, which suggests the stability of a tilted arrangement.

%---------------------------------------------------------------------------------------------------
\picW = 6cm
\begin{figure}
    \centering
    \pic{fig_05.ps}
    \caption{[Color online] Representation of $V_{\rm abs}(k_S,\theta)$ for the HGO-sphere potential and $k=3$.}
    \label{fig:Ve_RSP_fkS}
\end{figure}
%---------------------------------------------------------------------------------------------------

More insight into this result can be gained by comparing the HGO-sphere shape parameter,
eqn.~(\ref{eqn:sigma_w_RSP}), with that of the HNW potential, $\sigma^{\rm HNW}_w=0.5k_S\cos\theta$. In the case
$k_S=k$, the latter represents the distance between the substrate and the particle's centre of mass when one of
the particle's ends is in contact with the surface plane~\cite{BarmesCleaver04a}. $\sigma^{\rm HGO-sphere}_w$, in
contrast, represents the distance at which the HGO particle interacts with a sphere embedded within the substrate.
The difference between the two shape parameters (Fig.~\ref{fig:cmpHNW_RSP}) shows that, for intermediate tilt
angles, $\sigma^{\rm HGO-sphere}_w$ drops below $\sigma^{\rm HNW}_w$. At these angles, therefore, the HGO-sphere
surface potential allows the particle ends to penetrate the surface plane. The angles corresponding to this region
of reduced $\sigma^{\rm HGO-sphere}_w$ coincide with the maximum in $V_{\rm abs}(k_S,\theta)$ and appear,
therefore, to be associated with the tilt behaviour.

%---------------------------------------------------------------------------------------------------
\begin{figure}
    \centering
    \picL{fig_06.ps}
    \caption{[Color online] Comparison between $\sigma^{\rm HNW}_w$ (solid line) and $\sigma^{\rm HGO-sphere}_w$ (dashed
    line). The dotted line represents the difference between the two
    ($\sigma^{\rm HGO-sphere}_w-\sigma^{\rm HNW}_w$.)}
    \label{fig:cmpHNW_RSP}
\end{figure}
%---------------------------------------------------------------------------------------------------

The tilt angle $\theta^{\rm max}_{tilt}$ for which the absorbed volume of a single particle is maximal can be
calculated using eqn.~(\ref{eqn:Ve_RSP}) for various values of $k$ and $k_S$. For example, we show
(Fig.~\ref{fig:Ve_RSP_fk}) the surface $V_{\rm abs}(k,\theta)$ calculated in the full-particle limit $k_S=k$. For
this case, $\theta^{\rm max}_{tilt}(k)$ has been obtained by computing the contour of $\frac{d}{d\theta}V_{\rm
abs}(k,\theta)$ at level $0$, from which we have found that $\theta^{\rm max}_{tilt}$ is constant at about 0.9
radians ($\sim 50^\circ$) for $k\leq 10$. This implies that a tilt angle of about $50^\circ$ should be adopted by
full (\emph{i.e.} $k_S=k$) HGO particles adsorbed using the HGO-sphere potential. Despite both the neglect of
many-body effects in this analysis and the geometrical approximations made, the prediction $\theta^{\rm
max}_{tilt}\sim 50$ degrees matches the simulation results reasonably well. At the state point $\rho^{*} = 0.35$
and $k_S/k = 1.0$, for the $z$ location where $\rho^{*}_\ell(z)$ is maximal, the simulations give $Q_{zz} =
0.209$, which corresponds to an average tilt angle of $46.6^\circ$. Since surface-packing of particles increases
with decrease in tilt angle, the single-particle prediction for $\theta^{\rm max}_{tilt}$ can be expected to be an
be an over-estimate; the tilt angle adopted in the simulations therefore appears to represent a reasonable
compromise.

%---------------------------------------------------------------------------------------------------
\picW = 6cm
\begin{figure}
    \centering
    \pic{fig_07.ps}
    \caption{[Color online] Representation of $V_{\rm abs}(k,\theta)$ for the HGO-sphere potential and $k_S=k$.}
    \label{fig:Ve_RSP_fk}
\end{figure}
%---------------------------------------------------------------------------------------------------

The numerical and geometrical treatments described in this Section have shown that a tilted phase can be both
predicted and obtained with a purely steric model. This sheds new light on the tilted phases obtained in
refs.~\cite{ZhangChakrabarti96, WallCleaver97,TeixeiraChrzanowska01,WebsterCleaver03} in simulations of confined
Gay-Berne systems; it now appears that the tilts seen in these systems were simply the entropically favoured
arrangements for the surface potential employed, and were \emph{not} caused by competition between
particle-particle and particle-wall enthalpic contributions. The steric argument presented here is also consistent
with the change from tilted to planar surface alignment observed when Wall and
Cleaver~\cite{WallCleaver97,WallCleaver03} simulated equivalent systems but with the molecular elongation reduced
from $k=3$ to $k=2$; we now see that in the latter case, the particles were simply too short to significantly
absorb at the surface and, therefore, adopted the planar state. Evaluation of $V_{\rm abs}(k=2,\theta)$ (see
Fig.~\ref{fig:Ve_RSP_fk}) confirms this, showing that for this elongation, the absorbed volume is virtually
independent of molecular orientation. In the light of this, it seems reasonable to assume that a planar surface
arrangement would have been obtained in the simulations of Refs.~\cite{ZhangChakrabarti96,
WallCleaver97,TeixeiraChrzanowska01,WebsterCleaver03}, had the surfaces been represented by a lattice of fixed
spheres, as was done in~\cite{PalermoBiscarini98}.

Thus, we conclude that for the HGO-sphere surface potential, a previously unrecognised angle-dependent absorption
of particles into the surface leads to the formation of tilted phases, even for full particles. The anchoring
behaviour obtained with this model is found to vary continuously with $k_S/k$, such that no bistability is found
between the homeotropic and tilted anchoring states. That said, varying $k_S/k$ does not appear to be the best
route by which to control the anchoring tilt angle: changing the surface-sphere radius ($\sigma_j$ in
eqn.(\ref{eqn:sw_RSP_BP})) is a more natural approach to adopt. In the next Section, we address the behaviour
induced by an alternative potential for which no such absorption can occur, the aim being to regain the standard
homeotropic-planar anchoring behaviour obtained with the HNW potential.

%=======================================================================================
%           ROD - SURFACE
%=======================================================================================
\section{The HGO-surface potential}
\label{s:RSUP}
In this Section, we consider the behaviour of the HGO fluid when confined by a \emph{full} (but
structureless) substrate using the HGO-surface shape parameter derived in Appendix~\ref{app:B}, that is:
\begin{equation}
\label{eqn:hgo-surface}
    \sigma^{\rm HGO-surface}_w = \sigma_0\lp \sqrt{\frac{1-\chi_S\sin^2\theta}{1-\chi_S}} -
    \frac{1}{2}\rp .
\end{equation}
With this potential, each HGO particle effectively interacts with a planar continuum rather than a single sphere.
Again a shift of $\sigma_0/2$ has been introduced so as to displace the material forming the substrate from the
simulation box.

Before presenting the simulation results obtained for this system, we first consider the surface absorption
properties expected for the modified shape parameter~(\ref{eqn:hgo-surface}). Again, this is done by using the
shape parameter to calculate $V_{\rm abs}(k,k_S,\theta)$ the particle volume absorbed into the surface as a
function of its orientation and inner-particle extension. The results of this calculation for $k=3$ are shown in
Figure~\ref{fig:Ve_RSUP_fks}. From this we see that when $k_S = 0$, $V_{\rm abs}(k,k_S,\theta)$ is maximal at
$\theta = 0$, indicating a stable homeotropic state. For $k_S = k=3$, in contrast, $V_{\rm abs}(k,k_S,\theta)$ is
close to zero for all $\theta$, with only a small maximum present at $\theta = 0$. This maximum is, in fact, an
anomaly relating to the ellipsoidal approximation employed in Appendix~\ref{app:A}; by design, $\sigma_w^{\rm
HGO-surface}$ actually forbids any particle adsorption into the substrate if $k_S=k$. For this system, therefore,
we expect to regain the planar base state previously found for rod-shaped objects in contact with a hard flat
surface~\cite{DijkstraVanRoij01}.

%---------------------------------------------------------------------------------------------------
\picW = 6cm
\begin{figure}
    \centering
    \pic{fig_08.ps}
    \caption{[Color online] Representation of $V_{\rm abs}(k_s,\theta)$ for the HGO-surface potential and $k=3$.}
    \label{fig:Ve_RSUP_fks}
\end{figure}
%---------------------------------------------------------------------------------------------------

From these calculations, it is apparent that the mechanism driving any homeotropic-planar anchoring transition
with the HGO-surface potential must be qualitatively different from that seen with the HNW
potential~\cite{BarmesCleaver04a}. For the latter, the stable anchoring state could be predicted by simply
comparing the particle volume that could be absorbed into the surfaces in the homeotropic and planar arrangements:
the favourable state was always that which maximized the total absorbed volume. In the case of the HGO-surface
potential, however, there is {\em no} absorption in the planar arrangement. Here, therefore, the competition is
between the higher orientational entropy of the planar state and the volume adsorption available (for $k_S<k$) in
the homeotropic state.

%----------------------------------------------------------------------------------------------
\subsection{Simulation results}
The anchoring behaviour of the rod-surface potential has been studied using Monte Carlo simulations broadly
equivalent to those presented in Section~\ref{ss:RSPresults}. All of the simulations were performed in the
canonical ensemble on systems of $N=1000$ HGO particles with elongation $k=3$, confined in slab geometry with
symmetric anchoring conditions. At each density investigated, two series of simulations were performed with
increasing and decreasing $k_S$. The typical $z$-profiles for this model are shown in
Figs.~\ref{fig:RSUP_typeProf_k3_homeo} and~\ref{fig:RSUP_typeProf_k3_planar} for $k_S=0.0$ and $k_S = k$,
respectively.
%---------------------------------------------------------------------------------------------------
\picW = 8cm
\begin{figure}
    \centering
    \pic{fig_09.ps}
    \caption{[Color online] Typical $z$-profiles for systems of $N=1000$ HGO particles with
    $k=3.0$ and $k_S/k = 0.0$ confined using the HGO-surface potential.
    These data were obtained from a simulation series performed with decreasing $k_S$.}
    \label{fig:RSUP_typeProf_k3_homeo}
\end{figure}
%---------------------------------------------------------------------------------------------------
%%
%%

The results obtained for $k_S=0$ are virtually indistinguishable from those found for the equivalent HGO-sphere
system (recall Fig.~\ref{fig:RSP_typeProf_k3_homeo}), indicating, as expected, strong homeotropic anchoring. This
similarity between the profiles obtained with the two $k_S=0$ systems is explained by the observation that for
$\alpha \simeq 0$, the interfacial geometry is equivalent for both potentials used. No such similarity is apparent
for the two $k_S=k$ systems, however. Here, the HGO-surface system develops a multi-peak density profile with peak
separations of $\sigma_0$, and, at nematic densities, a central region with negative $Q_{zz}(z)$ values. As
predicted, therefore, this system does indeed adopt a planar state at high inner-particle elongation.

%---------------------------------------------------------------------------------------------------
\picW = 8cm
\begin{figure}
    \centering
    \pic{fig_10.ps}
    \caption{[Color online] Typical $z$-profiles for systems of $N=1000$ HGO particles with
    $k=3.0$ and $k_S/k = 1.0$ confined using the HGO-surface potential.
    These data were obtained from a simulation series performed with decreasing $k_S$.}
    \label{fig:RSUP_typeProf_k3_planar}
\end{figure}
%---------------------------------------------------------------------------------------------------

The full anchoring behaviour of this $k=3$ system, evaluated as a function of $k_S$ and $\rho^{*}$, is indicated
by the anchoring maps shown in Figure~\ref{fig:QzzWaPhaseDia_k3_RSUP}. As previously, these show contours of
$\overline{Q_{zz}}(\rho^{*},k_S/k)$, the density-profile-weighted averages of $Q_{zz}(z)$ in the interfacial and
bulk regions. Again, in calculating these, the surface region was taken to extend from the substrate to the second
maximum in $\rho^{*}_{\ell}$, regardless of the surface arrangement obtained. The anchoring maps computed for
simulation series performed with decreasing and increasing $k_S$ are given in
Figs.~\ref{fig:QzzWaPhaseDia_k3_RSUP}(a) and (b), respectively. Clear differences are apparent from these two data
sets, indicating hysteresis in the anchoring behaviour. This is quantified in the accompanying bistability maps
(Figs.~\ref{fig:QzzWaPhaseDia_k3_RSUP}(c)) obtained by simply subtracting the (b) surfaces from their
corresponding (a) surfaces. We note that the bistability indicated here covers a much wider range of both density
and $k_S/k$ than that obtained using the HNW potential~\cite{BarmesCleaver04a}, and is centred on $k_S/k \simeq
0.75$. 

%---------------------------------------------------------------------------------------------------
\picW = 4cm
\begin{figure}
    \centering
    \subfigure[Simulation series performed at constant density with decreasing $k_S$]{\pic{fig_11al.ps}\pic{fig_11ar.ps}}
    
	\subfigure[Simulation series performed at constant density with increasing $k_S$]{\pic{fig_11bl.ps}\pic{fig_11br.ps}}
    
	\subfigure[Bistability maps obtained by subtracting Figs.(b) from Figs.(a)]{\pic{fig_11cl.ps}\pic{fig_11cr.ps}}
    \caption{Anchoring maps obtained from simulations of $N=1000$
    $k=3$ HGO particles confined using the HGO-surface surface potential.
    Diagrams on the l.h.s correspond to the interfacial
    region and those on the r.h.s correspond to the bulk region.}
    \label{fig:QzzWaPhaseDia_k3_RSUP}
\end{figure}
%---------------------------------------------------------------------------------------------------


%=======================================================================================
%   CONCLUSIONS
%=======================================================================================
\section{Conclusions}
\label{s:CCL}

We have investigated, by means of Monte Carlo computer simulation, the effect of the particle-substrate shape
parameter (or contact function) on the anchoring behaviour of a generic confined liquid crystal model.
Essentially, by tuning the degree and sense of the non-additivity of this contact function, we have been able to
establish both a tilted anchoring state and a strongly first-order (i.e. bistable) planar to homeotropic anchoring
transition.

Non-additivity has been incorporated into the systems studied in two different ways. Firstly, as was shown in
Section \ref{s:RSP}, the HGO-sphere shape parameter has an intrinsic angle-dependent non-additivity; particles
approaching the substrate in either planar or homeotropic alignments `see' the full surface, whereas particles
approaching at intermediate angles are allowed to partially absorb. For systems with $k_S \approx k$, this
microscopic effect was found to control both the structure of the fluid in the near-substrate region {\em and} the
macroscopic anchoring orientation. The second use of non-additivity in this work centred on $k_S$, the
(dimensionless) particle length used to determine the particle-substrate interactions. By using $k_S$ as a model
parameter, we have been able controllably to introduce a homeotropic anchoring state into the simulated systems,
and continuously vary its relative stability. Given that particle shape is the main determinant of structure in
most liquids, it should not, perhaps, be a great surprise that the contact function used to define
particle-substrate interactions has had so dominant an effect here. That said, the utility of this approach does
not appear to be widely recognised.

Whilst non-additivity has been used here as a convenient device with which to control model systems, we stress
that this approach does not represent an abstract concept with no relevance to real systems. Indeed, for molecular
systems (in which intramolecular flexibility may be significant) adsorbed at substrates with `soft' coatings, the
relevance of a fully-additive generic model is arguable. For the specific models used in this work, an
experimental realisation of reducing the parameter $k_S$ would be to employ a substrate coating that allows some
penetration by the molecular endgroups, but repels the central part of the molecule; for mesogens, which commonly
have sub-molecular units with significantly different character, this is perfectly achievable behaviour.

We have shown that the anchoring properties of generic model mesogens adsorbed at perfectly flat walls can be
controlled by details of the mesogen-substrate interaction. Moreover, we have shown that the nature of the
interfacial region can depend markedly on the anchoring state. For example, the depth at which the substrate
profile ceases to be apparent in the liquid structure depends strongly on the anchoring orientation; since
interfacial region structure underlies mesoscopic descriptors such as anchoring coefficients and surface
viscosities, a more detailed understanding of such differences may offer a route to enhanced device control.
Similarly, orientational correlations parallel with and perpendicular to the substrate can be expected to depend
on the anchoring orientation; the systems determined here therefore represent good candidate systems with which to
explore phenomena such as nematic bridging in microconfined and/or colloid-bearing mesogenic systems.

Finally, having achieved a microscopic model capable of exhibiting anchoring bistability, we are now in a position
to examine the orientational behaviour present in more complex systems. These include other liquid crystal cell
configurations, such as the bistable hybrid aligned nematic considered by Davidson and
Mottram~\cite{DavidsonMottram02}, and more exotic liquid crystal models such the PHGO description of flexoelectric
pear-shaped particles~\cite{BarmesCleaver03}.

%A good example of the use of absorption behaviour to control anchoring is the work of Filas and Patel on systems
%with silane coated surfaces~\cite{FilasPatel87}. Here, by coating rubbed surfaces with two silane based
%surfactants which independently induced homogeneous and homeotropic anchoring, respectively, it proved possible to
%generate any tilt angle between these two limits simply by varying the relative concentration of the two
%compounds.
%
% Using surface penetrability as a control mechanism for the surface
%induced anchoring in confined liquid crystalline systems. To this end, we have used both the  and HGO-surface
%potentials, which rely on similar mechanisms but exhibit different absorption behaviours. For application in
%simulating switching between two surface aligned states in a  bistable liquid crystal device, another aim of this
%study was to build a surface potential that produces both planar and homeotropic anchoring with a strong
%bistability region.
%%%
%%% The RSPm and RSUP effects
%%%
%%% Discussions :
%%%
%%%  -> tilted phases can be explained with surface penetrability
%%%  -> RSUP good candidate for flexo switch
%%%  -> ???
%The approach used to model the surfaces is very similar to that used in a preceding
%publication~\cite{BarmesCleaver04a} but using more realistic surface interactions. The HGO-sphere potential was
%found to display a very interesting surface behaviour as with this steric potential, due to artificial
%substrate-absorption, the expected planar arrangement was replaced by a tilted structure. This shows that a tilted
%phase can result from the sole combination of purely steric interactions and  surface penetrability and sheds a
%new light on previous simulation work where the origin of such an arrangement was attributed to the competition
%between packing constraints and attractive forces. Meanwhile, the rod-surface potential, which does not allow
%surface penetrability for full particles, yields the more common homeotropic and planar anchoring which further
%proves the argument of surface absorption in explaining the origins of the tilted phases observed with the
%HGO-sphere potential. In addition, the HGO-surface potential proves a better candidate for modelling bistable
%switching between the homeotropic and planar arrangements, as not only the surface behaviour observed with the HNW
%model was recovered, but stronger and wider bistability behaviour could be observed.
%%%
%%
%% What's next
%%%
%The results obtained from this work will be subsequently used in an attempt to model bistable switching between
%two stable arrangement in confined systems of normal and flexoelectric liquid crystalline systems.


%=======================================================================================
%   Acknowledgments
%=======================================================================================
\section*{Acknowledgments}
FB wishes to acknowledge useful discussions with Dmytro Antipov and Sheffield Hallam University's Material
Research Institute for financial support

%=======================================================================================
%   APPENDIX 1
%=======================================================================================
\appendix
%%\newpage
\section{Volume of an ellipsoid absorbed at a plane}
\label{app:A}
Here, we consider the geometry of an ellipsoidal particle close to a confining surface and
interacting with it using an arbitrary potential that allows partial absorption into the substrate. The aim is to
determine an expression for $V_{\rm abs}$, the volume absorbed into the surface. To this end, we consider the
setups shown in Fig.~\ref{fig:scalingVe-Vs}. The result is first quoted for the case of a sphere of radius $a$:
the volume of the illustrated sphere which is absorbed into the surface is
\begin{equation} \label{eqn:Vs}
    V_s = \frac{\pi}{3}(a-d_1)^2(2a+d_1).
\end{equation}
As indicated in the Figure, this same solution can then be transformed to the case of an ellipsoid of elongation
$k$ simply by scaling space by a factor $k$ along the ellipsoid's symmetry axis, $\vecth{z}$, thus $V_{\rm abs} =
kV_s$. What is required, therefore, is an expression for $V_s$ in terms of the ellipsoid's co-ordinates.
%---------------------------------------------------------------------------------------------------
\picW = 8cm
\begin{figure}[h]
    \centering
    \pic{fig_12.ps}
    \caption{[Color online] Schematic representation of the geometrical configuration considered in
    Appendix~\ref{app:A} to
    calculate the absorbed volume of an ellipsoid at a planar substrate.}
    \label{fig:scalingVe-Vs}
\end{figure}
%---------------------------------------------------------------------------------------------------

The surface of an ellipsoid of semi-axes $a,a,ka$ along $\vecth{x},\vecth{y}$ and $\vecth{z}$ is given by
\begin{equation}
    \frac{x^2}{a^2} + \frac{y^2}{a^2} + \frac{z^2}{k^2a^2} = 1 .
\end{equation}
Taking the ellipsoid tilt to be confined to $\vecth{x}-\vecth{z}$ plane, the distance $d_1$ can be determined by
considering the triangle $O A_1 B_1$ in that plane. The co-ordinates of $A_1$ and $B_1$ are equal, respectively,
to those of $A$ and $B$ rescaled by $1/k$ along $\vecth{z}$. Hence
\begin{eqnarray*}
    A_1 &=& \lp x_A, 0, \frac{z_A}{k} \rp   \\
    B_1 &=& \lp x_B, 0, \frac{z_B}{k} \rp
\end{eqnarray*}
and, from Pythagoras' theorem,
\begin{equation}
    d_1 = \sqrt{a^2 - \frac{1}{4}\lp   (x_A-x_B)^2 + \frac{1}{k^2}(z_A-z_B)^2 \rp  }  .
    \label{eqn:d1Initial}
\end{equation}
$A$ and $B$, defined as the points in the $\vecth{x}-\vecth{z}$ plane where the ellipsoid intersects the
substrate, can be found by solving the simultaneous equations
\begin{eqnarray}
        x^2+\left(\frac{z}{k}\right)^2 &=  &a^2 \\
        x \sin\theta   &=  & z\cos\theta - d .
    \label{eqn:system}
\end{eqnarray}
Combining these gives
\begin{equation}
    z^2\lp k^2\cos^2\theta + \sin^2\theta \rp - 2zk^2d\cos\theta + d^2k^2 -
    a^2k^2\sin^2\theta = 0
\end{equation}
the roots of which are
\begin{equation}
   z = \frac{dk^2\cos\theta \pm \sqrt{k^2\sin^2\theta
   \left[ a^2(k^2\cos^2\theta + \sin^2\theta) -d^2\right]}}
   {k^2\cos^2\theta + \sin^2\theta}.
\end{equation}
%%
%%
Eqn.~(\ref{eqn:d1Initial}) can now be rewritten using
\begin{eqnarray*}
    (x_B-x_A)^2 &=& \frac{\cos^2\theta}{\sin^2\theta}\lp z_B-z_A \rp^2  \\
    (z_B - z_A)^2 &=& \frac{4k^2\sin^2\theta
    \left[ -d^2 + a^2\lp k^2\cos^2\theta + \sin^2\theta  \rp  \right]}{(k^2\cos^2\theta + \sin^2\theta)^2}
\end{eqnarray*}
which, after full simplification, reduces to
\begin{equation}
    d_1 = \frac{d}{\sqrt{k^2\cos^2\theta + \sin^2\theta} }.
\end{equation}
Having obtained this expression for $d_1$ purely in terms of the ellipsoid co-ordinates, we insert it into
eqn.~(\ref{eqn:Vs}) and scale by $k$ to give the absorbed volume of the ellipsoid
%%
%%-----------------------------------------------
%%	TWO COLUMN VERSION OF THE EQUATION
%%-----------------------------------------------
%%
%\begin{multline}
%    V_{\rm abs}(k,\theta) = \frac{k\pi}{3}
%    \lp  a - \frac{d}{\sqrt{k^2\cos^2\theta + \sin^2\theta} } \rp^2   \\
%    \times \lp 2a + \frac{d}{\sqrt{k^2\cos^2\theta + \sin^2\theta} } \rp .
%    \label{eqn:finalVe}
%\end{multline}
%%
%%-----------------------------------------------
%%	ONE COLUMN VERSION OF THE EQUATION
%%-----------------------------------------------
%%
\begin{equation}
    V_{\rm abs}(k,\theta) = \frac{k\pi}{3}
    \lp  a - \frac{d}{\sqrt{k^2\cos^2\theta + \sin^2\theta} } \rp^2 
    \lp 2a + \frac{d}{\sqrt{k^2\cos^2\theta + \sin^2\theta} } \rp .
    \label{eqn:finalVe}
\end{equation}


%=======================================================================================
%   APPENDIX 2
%=======================================================================================
%%\appendix
%%\newpage

\section{Determination of the Rod-Surface $\sigma^{\rm HGO-surface}_w$ }
\label{app:B} In this appendix, we give two routes to the rod-surface shape parameter employed in
Section~\ref{s:RSUP} of this paper. The first approach adopted here is to take the Gaussian-overlap (GO)
rod-sphere interaction given by Berne and Pechukas \cite{BernePechukas72} and integrate the position of the sphere
across the xy-plane. The result of this calculation is then compared with the generic GO form to allow
identification of a rod-surface shape parameter, $\sigma (\theta)$. In the second approach, this same result is
obtained by direct calculation of the minimum distance between a single GO particle and a sphere constrained to
lie in the surface plane.

As a starting point, we take eqn.(4) of \cite{BernePechukas72}, the GO interaction potential for two ellipsoidal
particles:
\begin{equation}
\label{1}
 V(\ui,\uj,\vect{r}_{ij})= \varepsilon(\ui,\uj) \exp \left[ - \frac{r_{ij}^2} {\sigma^2(\ijr)} \right]
\end{equation}
When one of the particles is made spherical, to give a rod-sphere interaction, Berne and Pechukas tell us that the
shape parameter becomes
\begin{equation}
\label{2} \sigma(\ui,\rij)= \sqrt  \frac{\sigma_{0}^2+\sigma_{j}^2} {2(1-\chi(\ui \cdot \rij)^2)}
\end{equation}
where
\begin{equation}
\label{3}
 \chi=\frac{\sigma_{\ell}^2-\sigma_{0}^2}{\sigma_{\ell}^2+\sigma_{j}^2}.
\end{equation}
On inserting (\ref{2}) into (\ref{1}), the resultant interaction between a rod and a sphere is
\begin{equation}
\label{4}
 V(\ui,\vect{r}_{ij})= \varepsilon_0 \exp \left[ - \frac{2r_{ij}^2\{1-\chi (\ui \cdot \rij)^2\}} {\sigma_{0}^2+\sigma_{j}^2}
 \right] .
\end{equation}
With the aim of extending this to calculate a rod-surface interaction, we take eqn.~(\ref{4}) and integrate the
sphere's position over the \textit{xy}-plane. To do this, we define a co-ordinate system such that
$\ui=(\sin\theta,0,\cos\theta)$ and $\vect{r}_{ij}=(x,y,z)$. With these definitions (and redefining $\varepsilon$
to have units of energy per unit area), the double integral over $V(\ui,\vect{r}_{ij})$ becomes
%%
%%
%%-----------------------------------------------
%%	TWO COLUMN VERSION OF THE EQUATION
%%-----------------------------------------------
%%
%\begin{multline}
%\label{5}
% V(\ui,z)=
%  \varepsilon_0 \int \int_{\rm xy-plane} \\ \exp
% \left[ - \frac{2(x^2+y^2+z^2-\chi (x\sin\theta+z\cos\theta)^2)} {\sigma_{0}^2+\sigma_{j}^2} \right] dx dy
%\end{multline}
%%
%%
%%-----------------------------------------------
%%	ONE COLUMN VERSION OF THE EQUATION
%%-----------------------------------------------
\begin{equation}
\label{5}
 V(\ui,z)=
  \varepsilon_0 \int \int_{\rm xy-plane}  \exp
 \left[ - \frac{2(x^2+y^2+z^2-\chi (x\sin\theta+z\cos\theta)^2)} {\sigma_{0}^2+\sigma_{j}^2} \right] dx dy
\end{equation}
%%
The \textit{y}-integral is a straightforward Gaussian, and the \textit{x}-integral is just a `complete the square'
problem. Performing these gives
\begin{equation}
\label{6}
 V(\ui,z)= \varepsilon_0  \exp
 \left[ - \frac{2z^2(1-\chi)} {(\sigma_{0}^2+\sigma_{j}^2)(1-\chi\sin^2\theta)} \right] .
\end{equation}
Comparing this with the generic GO form (i.e. eqn.~(\ref{1})), we can identify $\sigma(\theta)$  with the square
root of the terms dividing $z^2$ in the exponential term in eqn.~(\ref{6}). So, the final result is
\begin{equation}
\label{7} \sigma(\theta)=\sqrt{\frac{(\sigma_{0}^2+\sigma_{j}^2)(1-\chi\sin^2\theta)}{2(1-\chi)}} =
\sigma_0\sqrt{\frac{1-\chi\sin^2\theta}{(1-\chi)}}
\end{equation}
where the second equality requires the (usual) imposition $\sigma_{0}=\sigma_{j}$.

Interestingly, the result (\ref{7}) obtained by integrating over the gaussian containing the rod-sphere shape
parameter can alternatively be obtained in a process involving differentiation of that same rod-sphere shape
parameter. To see this, we consider an HGO particle located in the vicinity of a planar interface and seek to
calculate the distance between the interface and the closest part of the rod. Taking the rod-sphere shape
parameter to define the shape of the rod as viewed by the interface, what this then amounts to is identifying the
sphere, constrained to lie in the plane, whose location minimises the shape parameter calculation.

The $\vect{r}_{ij}$ vector corresponding to this minimum can be identified simply by differentiating an expression
for the projection of $\rij\sigma(\ui,\rij)$ along the surface normal and setting it to zero. For arbitrary
$\rij$, this projection is given by
\begin{equation}
\label{9} \sigma_z(\ui,\rij)= (\rij\cdot\vecth{z})\sqrt  \frac{\sigma_{0}^2+\sigma_{j}^2} {2(1-\chi(\ui \cdot
\rij)^2)}.
\end{equation}
Writing $\rij=(\sin\phi,0,\cos\phi)$ the turning points of (\ref{9}) are given by
\begin{equation}
\label{10} \tan(\phi_{\rm min})= \frac{\chi\cos\theta\sin\theta} {1-\chi\sin^2\theta}.
\end{equation}
This gives the orientation of $\rij$ corresponding to the point on the rod that is nearest to the surface. Using
(\ref{10}) to substitute for $\phi$ in (\ref{9}) and performing some trigonometrical manipulation then gives an
expression for $\sigma(\theta)$ which is identical to that given by (\ref{7}).


%=======================================================================================
%           Bibliography
%=======================================================================================
%\newpage
\begin{thebibliography}{43}
\expandafter\ifx\csname natexlab\endcsname\relax\def\natexlab#1{#1}\fi
\expandafter\ifx\csname bibnamefont\endcsname\relax
  \def\bibnamefont#1{#1}\fi
\expandafter\ifx\csname bibfnamefont\endcsname\relax
  \def\bibfnamefont#1{#1}\fi
\expandafter\ifx\csname citenamefont\endcsname\relax
  \def\citenamefont#1{#1}\fi
\expandafter\ifx\csname url\endcsname\relax
  \def\url#1{\texttt{#1}}\fi
\expandafter\ifx\csname urlprefix\endcsname\relax\def\urlprefix{URL }\fi
\providecommand{\bibinfo}[2]{#2}
\providecommand{\eprint}[2][]{\url{#2}}

\bibitem[{\citenamefont{{B. J\'er\^ome}}(1991)}]{Jerome91}
\bibinfo{author}{\bibnamefont{{B. J\'er\^ome}}}, \bibinfo{journal}{Phys. Rep.}
  \textbf{\bibinfo{volume}{54}}, \bibinfo{pages}{391} (\bibinfo{year}{1991}).

\bibitem[{\citenamefont{{I.A. Shanks}}(1982)}]{Shanks82}
\bibinfo{author}{\bibnamefont{{I.A. Shanks}}}, \bibinfo{journal}{Contemp.
  Phys.} \textbf{\bibinfo{volume}{23}}, \bibinfo{pages}{65}
  (\bibinfo{year}{1982}).

\bibitem[{\citenamefont{{T. Geelhaar}}(1998)}]{Geelhaar98}
\bibinfo{author}{\bibnamefont{{T. Geelhaar}}}, \bibinfo{journal}{Liq. Crysts.}
  \textbf{\bibinfo{volume}{24}}, \bibinfo{pages}{91} (\bibinfo{year}{1998}).

\bibitem[{\citenamefont{{G.P. Bryan-Brown, C.V. Brown, I.C. Sage and V.C.
  Hui}}(1998)}]{ZBD}
\bibinfo{author}{\bibnamefont{{G.P. Bryan-Brown, C.V. Brown, I.C. Sage and V.C.
  Hui}}}, \bibinfo{journal}{Nature} \textbf{\bibinfo{volume}{392}},
  \bibinfo{pages}{365} (\bibinfo{year}{1998}).

\bibitem[{\citenamefont{{C. Denniston and J.M.
  Yeomans}}(2001)}]{DennistonYeomans}
\bibinfo{author}{\bibnamefont{{C. Denniston and J.M. Yeomans}}},
  \bibinfo{journal}{Phys. Rev. E} \textbf{\bibinfo{volume}{87}},
  \bibinfo{pages}{275505} (\bibinfo{year}{2001}).

\bibitem[{\citenamefont{{A.J Davidson and N.J.
  Mottram}}(2002)}]{DavidsonMottram02}
\bibinfo{author}{\bibnamefont{{A.J Davidson and N.J. Mottram}}},
  \bibinfo{journal}{Phys. Rev. E} \textbf{\bibinfo{volume}{65}},
  \bibinfo{pages}{051710} (\bibinfo{year}{2002}).

\bibitem[{\citenamefont{{B. Tjipto-Margo and D.E.
  Sullivan}}(1988)}]{TjiptoMargoSullivan88}
\bibinfo{author}{\bibnamefont{{B. Tjipto-Margo and D.E. Sullivan}}},
  \bibinfo{journal}{J. Chem. Phys.} \textbf{\bibinfo{volume}{88}},
  \bibinfo{pages}{6620} (\bibinfo{year}{1988}).

\bibitem[{\citenamefont{{E. Mart\'{\i}n del R\'{\i}o, M.M Telo da Gama and E.
  de Miguel}}(1995)}]{DelRioTeloDaGamma95}
\bibinfo{author}{\bibnamefont{{E. Mart\'{\i}n del R\'{\i}o, M.M Telo da Gama
  and E. de Miguel}}}, \bibinfo{journal}{Phys. Rev. E}
  \textbf{\bibinfo{volume}{52}}, \bibinfo{pages}{5028} (\bibinfo{year}{1995}).

\bibitem[{\citenamefont{{M.A. Osipov and S. Hess}}(1993)}]{OsipovHess93}
\bibinfo{author}{\bibnamefont{{M.A. Osipov and S. Hess}}}, \bibinfo{journal}{J.
  Chem. Phys.} \textbf{\bibinfo{volume}{99}}, \bibinfo{pages}{4181}
  (\bibinfo{year}{1993}).

\bibitem[{\citenamefont{{P.I.C Teixeira}}(1997)}]{Teixeira97}
\bibinfo{author}{\bibnamefont{{P.I.C Teixeira}}}, \bibinfo{journal}{Phys. Rev.
  E} \textbf{\bibinfo{volume}{55}}, \bibinfo{pages}{2876}
  (\bibinfo{year}{1997}).

\bibitem[{\citenamefont{{M.P. Allen}}(1999)}]{Allen99}
\bibinfo{author}{\bibnamefont{{M.P. Allen}}}, \bibinfo{journal}{Molec. Phys.}
  \textbf{\bibinfo{volume}{96}}, \bibinfo{pages}{1391} (\bibinfo{year}{1999}).

\bibitem[{\citenamefont{{Z. Lu, H. Deng and Y. Wei}}(1998)}]{LuDeng98}
\bibinfo{author}{\bibnamefont{{Z. Lu, H. Deng and Y. Wei}}},
  \bibinfo{journal}{Supramol. Sci.} \textbf{\bibinfo{volume}{5}},
  \bibinfo{pages}{649} (\bibinfo{year}{1998}).

\bibitem[{\citenamefont{{W. Chen, M.B. Feller and Y.R.
  Shen}}(1989)}]{ChenFeller89}
\bibinfo{author}{\bibnamefont{{W. Chen, M.B. Feller and Y.R. Shen}}},
  \bibinfo{journal}{Phys. Rev. Letts.} \textbf{\bibinfo{volume}{63}},
  \bibinfo{pages}{2665} (\bibinfo{year}{1989}).

\bibitem[{\citenamefont{{N.A. Clark}}(1985)}]{Clark85}
\bibinfo{author}{\bibnamefont{{N.A. Clark}}}, \bibinfo{journal}{Phys. Rev.
  Letts.} \textbf{\bibinfo{volume}{55}}, \bibinfo{pages}{292}
  (\bibinfo{year}{1985}).

\bibitem[{\citenamefont{{J.M. Geary, J.W. Goodgy, A.R. Kmetz and J.S.
  Patel}}(1987)}]{GearyGoodby87}
\bibinfo{author}{\bibnamefont{{J.M. Geary, J.W. Goodgy, A.R. Kmetz and J.S.
  Patel}}}, \bibinfo{journal}{J. Appl. Phys.} \textbf{\bibinfo{volume}{62}},
  \bibinfo{pages}{4100} (\bibinfo{year}{1987}).

\bibitem[{\citenamefont{{D.W. Bereman}}(1972)}]{Berreman72}
\bibinfo{author}{\bibnamefont{{D.W. Bereman}}}, \bibinfo{journal}{Phys. Rev.
  Letts.} \textbf{\bibinfo{volume}{28}}, \bibinfo{pages}{1683}
  (\bibinfo{year}{1972}).

\bibitem[{\citenamefont{{R. van Roij, M. Dijkstra and R.
  Evans}}(2000)}]{VanRoijDijkstra00}
\bibinfo{author}{\bibnamefont{{R. van Roij, M. Dijkstra and R. Evans}}},
  \bibinfo{journal}{Europhys. Letts.} \textbf{\bibinfo{volume}{49}},
  \bibinfo{pages}{350} (\bibinfo{year}{2000}).

\bibitem[{\citenamefont{{M. Dijkstra, R. van Roij and R.
  Evans}}(2001)}]{DijkstraVanRoij01}
\bibinfo{author}{\bibnamefont{{M. Dijkstra, R. van Roij and R. Evans}}},
  \bibinfo{journal}{Phys. Rev. E} \textbf{\bibinfo{volume}{63}},
  \bibinfo{pages}{051703} (\bibinfo{year}{2001}).

\bibitem[{\citenamefont{{A. Chrzanowska, P.I.C. Teixeira, H. Ehrentraut and
  D.J. Cleaver}}(2001)}]{Chrzanowska_Teixera_01}
\bibinfo{author}{\bibnamefont{{A. Chrzanowska, P.I.C. Teixeira, H. Ehrentraut
  and D.J. Cleaver}}}, \bibinfo{journal}{J. Phys.: Cond. Matt.}
  \textbf{\bibinfo{volume}{13}}, \bibinfo{pages}{4715} (\bibinfo{year}{2001}).

\bibitem[{\citenamefont{{F. Barmes and D.J.
  Cleaver}}(2004{\natexlab{a}})}]{BarmesCleaver04a}
\bibinfo{author}{\bibnamefont{{F. Barmes and D.J. Cleaver}}},
  \bibinfo{journal}{Phys. Rev. E} \textbf{\bibinfo{volume}{69}},
  \bibinfo{pages}{061705} (\bibinfo{year}{2004}{\natexlab{a}}).

\bibitem[{\citenamefont{{G.D. Wall and D.J. Cleaver}}(2003)}]{WallCleaver03}
\bibinfo{author}{\bibnamefont{{G.D. Wall and D.J. Cleaver}}},
  \bibinfo{journal}{Molec. Phys.} \textbf{\bibinfo{volume}{101}},
  \bibinfo{pages}{1105} (\bibinfo{year}{2003}).

\bibitem[{\citenamefont{{D.R. Binger and S. Hanna}}(2001)}]{BingerHanna01}
\bibinfo{author}{\bibnamefont{{D.R. Binger and S. Hanna}}},
  \bibinfo{journal}{Liq. Crysts.} \textbf{\bibinfo{volume}{28}},
  \bibinfo{pages}{1215} (\bibinfo{year}{2001}).

\bibitem[{\citenamefont{{V. Palermo, F. Biscarini and C.
  Zannoni}}(1998)}]{PalermoBiscarini98}
\bibinfo{author}{\bibnamefont{{V. Palermo, F. Biscarini and C. Zannoni}}},
  \bibinfo{journal}{Phys. Rev. E} \textbf{\bibinfo{volume}{57}},
  \bibinfo{pages}{R2519} (\bibinfo{year}{1998}).

\bibitem[{\citenamefont{{D.J. Cleaver and P.I.C
  Teixeira}}(2001)}]{Cleaver_Teixeira_01}
\bibinfo{author}{\bibnamefont{{D.J. Cleaver and P.I.C Teixeira}}},
  \bibinfo{journal}{Chem. Phys. Letts.} \textbf{\bibinfo{volume}{338}},
  \bibinfo{pages}{1} (\bibinfo{year}{2001}).

\bibitem[{\citenamefont{{H. Lange and F.
  Schmid}}(2002{\natexlab{a}})}]{LangeSchmid02}
\bibinfo{author}{\bibnamefont{{H. Lange and F. Schmid}}}, \bibinfo{journal}{J.
  Chem. Phys.} \textbf{\bibinfo{volume}{111}}, \bibinfo{pages}{362}
  (\bibinfo{year}{2002}{\natexlab{a}}).

\bibitem[{\citenamefont{{H. Lange and F.
  Schmid}}(2002{\natexlab{b}})}]{LangeSchmid02a}
\bibinfo{author}{\bibnamefont{{H. Lange and F. Schmid}}},
  \bibinfo{journal}{Euro. Phys. J. E} \textbf{\bibinfo{volume}{7}},
  \bibinfo{pages}{175} (\bibinfo{year}{2002}{\natexlab{b}}).

\bibitem[{\citenamefont{{H. Lange and F.
  Schmid}}(2002{\natexlab{c}})}]{LangeSchmid02c}
\bibinfo{author}{\bibnamefont{{H. Lange and F. Schmid}}},
  \bibinfo{journal}{Comp. Phys. Comms.} \textbf{\bibinfo{volume}{147}},
  \bibinfo{pages}{276} (\bibinfo{year}{2002}{\natexlab{c}}).

\bibitem[{\citenamefont{{M.T. Downton and M.P. Allen}}(2004)}]{DowntonAllen04}
\bibinfo{author}{\bibnamefont{{M.T. Downton and M.P. Allen}}},
  \bibinfo{journal}{Europhys. Letts.} \textbf{\bibinfo{volume}{65}},
  \bibinfo{pages}{48} (\bibinfo{year}{2004}).

\bibitem[{\citenamefont{{D. Antypov and D.J.
  Cleaver}}(2004)}]{AntypovCleaver04}
\bibinfo{author}{\bibnamefont{{D. Antypov and D.J. Cleaver}}},
  \bibinfo{journal}{J. Phys.: Cond. Matt.} \textbf{\bibinfo{volume}{16}},
  \bibinfo{pages}{S1887} (\bibinfo{year}{2004}).

\bibitem[{\citenamefont{{Z. Zhang, A. Chakrabarti, O.G. Mouristen and M.J.
  Zuckermann}}(1996)}]{ZhangChakrabarti96}
\bibinfo{author}{\bibnamefont{{Z. Zhang, A. Chakrabarti, O.G. Mouristen and
  M.J. Zuckermann}}}, \bibinfo{journal}{Phys. Rev. E}
  \textbf{\bibinfo{volume}{53}}, \bibinfo{pages}{2461} (\bibinfo{year}{1996}).

\bibitem[{\citenamefont{{G.D. Wall and D.J. Cleaver}}(1997)}]{WallCleaver97}
\bibinfo{author}{\bibnamefont{{G.D. Wall and D.J. Cleaver}}},
  \bibinfo{journal}{Phys. Rev. E} \textbf{\bibinfo{volume}{56}},
  \bibinfo{pages}{4306} (\bibinfo{year}{1997}).

\bibitem[{\citenamefont{{P.I.C Teixeira, A. Chrzanowska, G.D. Wall and D.J.
  Cleaver}}(2001)}]{TeixeiraChrzanowska01}
\bibinfo{author}{\bibnamefont{{P.I.C Teixeira, A. Chrzanowska, G.D. Wall and
  D.J. Cleaver}}}, \bibinfo{journal}{Molec. Phys.}
  \textbf{\bibinfo{volume}{99}}, \bibinfo{pages}{889} (\bibinfo{year}{2001}).

\bibitem[{\citenamefont{{R.E. Webster, N.J. Mottram and D.J.
  Cleaver}}(2003)}]{WebsterCleaver03}
\bibinfo{author}{\bibnamefont{{R.E. Webster, N.J. Mottram and D.J. Cleaver}}},
  \bibinfo{journal}{Phys. Rev. E} \textbf{\bibinfo{volume}{68}},
  \bibinfo{pages}{021706} (\bibinfo{year}{2003}).

\bibitem[{\citenamefont{{M. Rigby}}(1989)}]{Rigby89}
\bibinfo{author}{\bibnamefont{{M. Rigby}}}, \bibinfo{journal}{Molec. Phys.}
  \textbf{\bibinfo{volume}{68}}, \bibinfo{pages}{687} (\bibinfo{year}{1989}).

\bibitem[{\citenamefont{{B.J. Berne and P. Pechukas}}(1972)}]{BernePechukas72}
\bibinfo{author}{\bibnamefont{{B.J. Berne and P. Pechukas}}},
  \bibinfo{journal}{J. Chem. Phys.} \textbf{\bibinfo{volume}{56}},
  \bibinfo{pages}{4213} (\bibinfo{year}{1972}).

\bibitem[{\citenamefont{{J.G. Gay and B.J. Berne}}(1981)}]{GayBerne81}
\bibinfo{author}{\bibnamefont{{J.G. Gay and B.J. Berne}}}, \bibinfo{journal}{J.
  Chem. Phys.} \textbf{\bibinfo{volume}{74}}, \bibinfo{pages}{3316}
  (\bibinfo{year}{1981}).

\bibitem[{\citenamefont{{P. Padilla and E. Velasco}}(1997)}]{PadillaVelasco97}
\bibinfo{author}{\bibnamefont{{P. Padilla and E. Velasco}}},
  \bibinfo{journal}{J. Chem. Phys.} \textbf{\bibinfo{volume}{106}},
  \bibinfo{pages}{10299} (\bibinfo{year}{1997}).

\bibitem[{\citenamefont{{E. de Miguel, E. Mart\'{\i}n del
  R\'{\i}o}}(2001)}]{DeMiguelDelRio01}
\bibinfo{author}{\bibnamefont{{E. de Miguel, E. Mart\'{\i}n del R\'{\i}o}}},
  \bibinfo{journal}{J. Chem. Phys.} \textbf{\bibinfo{volume}{115}},
  \bibinfo{pages}{9072} (\bibinfo{year}{2001}).

\bibitem[{\citenamefont{{E. de Miguel, E. Mart\'{\i}n del
  R\'{\i}o}}(2003)}]{DeMiguelDelRio03}
\bibinfo{author}{\bibnamefont{{E. de Miguel, E. Mart\'{\i}n del R\'{\i}o}}},
  \bibinfo{journal}{J. Chem. Phys.} \textbf{\bibinfo{volume}{118}},
  \bibinfo{pages}{1852} (\bibinfo{year}{2003}).

\bibitem[{\citenamefont{{E. de Miguel}}(1992)}]{DeMiguel92}
\bibinfo{author}{\bibnamefont{{E. de Miguel}}}, \bibinfo{journal}{Phys. Rev. E}
  \textbf{\bibinfo{volume}{47}}, \bibinfo{pages}{3334} (\bibinfo{year}{1992}).

\bibitem[{\citenamefont{{N.V. Priezjev, G. Ska\u{c}ej, R.A. Pelkovits and S.
  \u{Z}umer}}(2003)}]{Priezjev03}
\bibinfo{author}{\bibnamefont{{N.V. Priezjev, G. Ska\u{c}ej, R.A. Pelkovits and
  S. \u{Z}umer}}}, \bibinfo{journal}{Phys. Rev. E}
  \textbf{\bibinfo{volume}{68}}, \bibinfo{pages}{041709}
  (\bibinfo{year}{2003}).

\bibitem[{\citenamefont{{F. Barmes and D.J.
  Cleaver}}(2004{\natexlab{b}})}]{BarmesCleaver04d}
\bibinfo{author}{\bibnamefont{{F. Barmes and D.J. Cleaver}}},
  \bibinfo{journal}{in preparation}  (\bibinfo{year}{2004}{\natexlab{b}}).

\bibitem[{\citenamefont{{F. Barmes, M. Ricci, C. Zannoni and D.J.
  Cleaver}}(2003)}]{BarmesCleaver03}
\bibinfo{author}{\bibnamefont{{F. Barmes, M. Ricci, C. Zannoni and D.J.
  Cleaver}}}, \bibinfo{journal}{Phys. Rev. E} \textbf{\bibinfo{volume}{68}},
  \bibinfo{pages}{021708} (\bibinfo{year}{2003}).

\end{thebibliography}



%=======================================================================================
\end{document}
